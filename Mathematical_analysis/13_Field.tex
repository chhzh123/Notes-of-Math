% !TEX root = main.tex

\section{场论初步}
\[\mathbf{F}(x,y,z)=(P(x,y,z),Q(x,y,z),R(x,y,z))\]
梯度场
\[\grad u=\nabla u=\lrp{\pd{}{x},\pd{}{y},\pd{}{z}}u=\lrp{\pd{u}{x},\pd{u}{y},\pd{u}{z}}\]
散度场
\[\div \mathbf{F}(x,y,z)=\pd{P}{x}+\pd{Q}{y}+\pd{R}{z}\]
旋度场
\[\rot \mathbf{F}(x,y,z)=\vmat{\vi&\vj&\vk\\\pd{}{x}&\pd{}{y}&\pd{}{z}\\P&Q&R}=\nabla\times\mathbf{F}\]
高斯公式可改写为
\[\iiint_V(\div\vF)\diff V=\oiint_S\vF\cdot\diff\vS\]
斯托克斯公式可改写为
\[\iint_S\rot\vF\cdot\diff\vS=\oint_L\vF\cdot\diff\vs\]
\begin{definition}
若对$V$内任一逐段光滑曲线$L$
\begin{enumerate}
	\item 且$L$封闭,则$\oint_L\vF\cdot\diff\vs=0$,则$\vF$为有势场
	\item $\int_L\vF\cdot\diff\vs$与路径无关,则$\vF$为保守场
\end{enumerate}
\end{definition}
\begin{theorem}
设$V$为空间的单连通区域,$\vF=(P,Q,R)$在$V$内有连续偏导数,则下列四个条件等价
\begin{enumerate}
	\item $\vF$是有势场
	\item $\vF$是保守场
	\item $\vF$是梯度场
	\item $\vF$是无旋场
\end{enumerate}
\end{theorem}