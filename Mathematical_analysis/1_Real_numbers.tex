% !TEX root = main.tex
% 填坑:可数无穷 不可数无穷 阿列夫
\section{实数系基本定理}
\label{sec:real_fundamental}
\par 实数系的基本性质作为实分析的入门,有着举足轻重的作用,深挖的话会发现很多有意思或者是反直觉的结论,但我们在这里仅仅做出一些定理的阐述和证明的思路,更多的内容等待以后再进行补充. 虽然本章涉及到的内容比较抽象,但是读者应该能够脑补出相应的情形,通过直观的认知来得到证明或证伪的思路,这对其中的证明题非常有益.
\subsection{实数连续性}
\par 实数连续性涉及到数系扩充的过程. 有理数之间存在空隙,因而数学家对其进行扩充,添加了无理数后变成实数系,而实数系竟然就弥补了之前有理数系的缺陷,变得稠密且连续,这是一个相当强的结论.
\begin{definition}[戴德金(Dedekind)连续性准则]
有大小顺序稠密的数系$S$,对于其任一分划$A|B$,$\exists$唯一$c\in S:\forall a\in A,b\in B:a\leq c\leq b$(即要么$A$有最大值,要么$B$有最小值),则称$S$连续.
\end{definition}
由此准则可以推出$\mathbb{Q}$不是连续的. 需要注意:
\begin{enumerate}
	\item 分划的断点$c$不一定属于$S$(正因$c\notin S$,$\mathbb{Q}$不连续)
	\item 分划要保证\textbf{不空、不漏、不乱}
\end{enumerate}
\begin{theorem}[实数基本定理]
对$\mathbb{R}$的任一分划$A|B$,$\exists$唯一$r\in S:\forall a\in A,b\in B:a\leq r\leq b$
\end{theorem}
\par 注意实数基本定理的证明是将实数全部以小数的形式表示,通过不断添加小数位数的方法来逼近割点,并没有用到后面的任何定理,可以算是根基.
\begin{definition}[确界]
如果实数集合$A$有上界,且上界中有最小数$\beta$,则定义$A$的上确界为$\sup A:=\beta$;若$A$有下界,且下界中有最大数$\alpha$,则定义$A$的下确界为$\inf A:=\alpha$.
由定义可知,$\beta$为$A$上确界,当且仅当
\begin{enumerate}
	\item $\forall x\in A:\,x\leq\beta$
	\item $\forall\varepsilon>0,\exists x_\varepsilon\in A:\,x_\varepsilon >\beta-\varepsilon$
\end{enumerate}
\par 类似地也有下确界的充要条件.
\begin{example}
若$f(x)$在$[a,b]$上有界,定义其振幅为
\[\omega_f[a,b]=\sup_{x\in[a,b]}f(x)-\inf_{x\in[a,b]}f(x)\]
则
\[\omega_f[a,b]=\sup_{x',x''\in[a,b]}|f(x')-f(x'')|\]
\end{example}
\begin{analysis}
设$\disp\sup_{x\in[a,b]}f(x)=M,\inf_{x\in[a,b]}f(x)=m$,由确界的定义有
\[\forall x',x''\in[a,b]:\,f(x')\leq M,f(x'')\geq m\]
故$f(x')-f(x'')\leq M-m$. 又
\[\forall\varepsilon_0>0,\exists x_1,x_2:\,f(x_1)\geq M-\varepsilon_0,f(x_2)\leq m+\varepsilon_0\]
故
\[\forall\varepsilon=2\varepsilon_0>0,\exists x_1,x_2:\,f(x_1)-f(x_2)\geq M-m-\varepsilon\]
符合上确界定义,原命题得证.
\end{analysis}
\begin{example}
\label{ex:inf_sup_inequality}
设$f(x)$与$g(x)$在$D$上有界,且大于$0$,则
\[\inf_{x\in D}f(x)\cdot\inf_{x\in D}g(x)\leq\inf_{x\in D}\{f(x)g(x)\}\leq\inf_{x\in D}f(x)\cdot\sup_{x\in D}g(x)\leq\sup_{x\in D}\{f(x)g(x)\}\leq\sup_{x\in D}f(x)\cdot\sup_{x\in D}g(x)\]
\end{example}
\begin{analysis}
左右两式是显然的,下面证明中间式.\\
反证,如果$\disp\inf_{x\in D}\{f(x)g(x)\}>\inf_{x\in D}f(x)\cdot\sup_{x\in D}g(x)$,则
\[f(x)g(x)\geq\inf_{x\in D}\{f(x)g(x)\}>\inf_{x\in D}f(x)\cdot\sup_{x\in D}g(x)\]
欲导致矛盾,即证$\disp\exists x_0:\,f(x_0)g(x_0)\leq \inf_{x\in D}f(x)\cdot\sup_{x\in D}g(x)$\\
取$\disp x_0=\arg\inf_{x\in D}f(x)$,即可满足上式,故矛盾.\\
另外一条式子同理,故得证.
\end{analysis}
\end{definition}
\begin{theorem}[确界定理]
实数集内,非空有上(下)界的数集必有上(下)确界存在.
\end{theorem}
\par 戴德金分割、确界定理、单调有界定理(见定理\ref{thm:monotone_convergence})都属于实数连续性的等价表述,可以相互推导.

\subsection{实数闭区间紧致性}
\begin{definition}[覆盖]
$E$为实数开区间构成的集合,$S$为实数子集,若$\forall x\in S$,有$(a,b)\in E$使得$x\in(a,b)$,则$E$为$S$的一个覆盖,记为
\[S\subset\bigcup_{E_\alpha\in E}E_\alpha\]
若$E$为有限区间组成,则称其为有限覆盖.
\end{definition}
\begin{theorem}[有限覆盖定理(Borel)/闭区间紧致性(compactness)]
实数\textbf{闭}区间$[a,b]$的任一个覆盖$E$,必存在有限的子覆盖
\end{theorem}
\begin{definition}[区间套]
一组实数的\textbf{闭}区间序列$[a_n,b_n],n=1,2,\cdots$,满足
\begin{partlist}
	\item $[a_{n+1},b_{n+1}]\subset[a_n,b_n],n=1,2,\cdots$
	\item $\displaystyle\lim_{n\to\infty}(b_n-a_n)=0$
\end{partlist}
则称$\{[a_n,b_n]\}$构成一个区间套
\end{definition}
\begin{theorem}[区间套定理]
$\{[a_n,b_n]\}$是一个区间套,则存在唯一$\displaystyle r\in\mathbb{R},s.t.\,r\in\bigcap_{n=1}^{\infty}[a_n,b_n]$
\end{theorem}
\par 区间套的三个条件一个都不能少,否则上面的结论不成立.
\begin{theorem}[紧致性定理(Bolzano–Weierstrass)/凝聚点定理]
有界数列必有收敛子数列
\end{theorem}
\par 上面几个定理在实分析证明题中都是利器,需要妥善使用.
\par 有限覆盖定理关键处理好有限和无限的关系,常通过构造两者的矛盾来得证.
\begin{example}
用有限覆盖定理证明紧致性定理
\end{example}
\begin{analysis}
考虑有界数列$\{x_n\}$在闭区间$[a,b]$内,即$a\leq x_n\leq b$.\\
反证,若$\{x_n\}$不存在收敛子列,则对于$[a,b]$内的每一个点$x$,都存在\textbf{开}区间$I_x$只含有限个$x_n$.\\
否则,若对于某一个点$x_0$的任意开区间$I_{x_0}$都含有无穷个$x_n$,则$\{x_n\}$的某一子列收敛于$x_0$,矛盾.\\
所有的$I_x$组成一个开区间覆盖(下式右侧),满足
\[[a,b]\subset\bigcup_{x\in[a,b]}I_x\]
左右两侧同时与$\{x_n\}$取交,
\[[a,b]\cap\{x_n\}\subset\lrp{\bigcup_{x\in[a,b]}I_x}\cap\{x_n\}=\bigcup_{x\in[a,b]}\lrp{I_x\cap\{x_n\}}\]
由有限覆盖定理,右侧有限集合,而左侧无限,故矛盾.
\end{analysis}
\par 上例采用值域(竖直方向)有限覆盖的方式,下例则采用定义域(水平方向)有限覆盖的方式求解.
\begin{example}
设$f(x)$在$[a,b]$上有定义,且在每一点处函数的极限存在,则$f(x)$在$[a,b]$上有界
\end{example}
\begin{analysis}
某一点$x$的极限存在,推得$f(x)$在$x$的邻域$D_x=(x-\delta,x+\delta)$上局部有界.\\
所有的$D_x$组成一个开区间覆盖,满足
\[[a,b]\subset\bigcup_{x\in[a,b]}D_x\]
由有限覆盖定理,有限个$D_x$即可覆盖$[a,b]$,则取这有限个开区间上函数界的最大最小值即得到$f(x)$的界.
\end{analysis}
\par 由于有限覆盖定理\textbf{有限}的性质,导致可以取一些数的最大最小值,而这个值是有限且确定的,这一点在证明一致连续性(见定理\ref{thm:cantor})上也会用到.
\par 区间套定理则更加灵活,往往采用\textbf{二分}的方法,注意关注最后的唯一点$r$具有什么性质(有点像戴德金分割),特别是涉及极限的证明.
\begin{example}
设$f(x)$在$[0,+\infty)$上连续且有界,$\forall a\in(-\infty,+\infty)$,$f(x)=a$在$[0,+\infty)$上只有有限个根或无根,求证$\disp\lim_{x\to\infty}f(x)$存在
\end{example}
\begin{analysis}
设上下界为$[u,v]$,不断二等分,取含无限长$f(x)$的一侧(因$f(x)=\dfrac{u+v}{2}$只有有限个根或无根,故可以将这些根全部枚举出来后,剩下的部分就只能在$y=\dfrac{u+v}{2}$的一侧了)\\
最终确定唯一数$r$,要证明$r=\disp\lim_{x\to\infty}f(x)$.
而由区间套的性质,当$n$充分大时,
\[f(x)\in[u_n,v_n]\subset(r-\eps,r+\eps)\]
进而得证.
\end{analysis}
\par 用区间套定理证明的其他例子.
\begin{example}
\begin{enumerate}
	\item 紧致性定理\\
	二分,取含有无穷项$\{x_n\}$的一侧
	\item 单调有界定理\\
	二分,取含上界的一侧,$r$为上确界,也为$\{x_n\}$的极限
	\item 连续函数介值定理(见\ref{sec:sub:continuous_function_properties}节)\\
	二分,取有正有负的一侧,$r$即为所求的点
	\item 有界性定理\\
	反证,二分,取无界一侧
	\item 最值定理\\
	二分,总有一侧含有一个点必另外一侧的所有点都大,取这一侧
	\item 康托定理\\
	反证,三等分,取违反一致收敛定义的一侧;用有限覆盖定理证明较为简单
\end{enumerate}
\end{example}
\par 本节中出现的有限覆盖定理、区间套定理、紧致性定理都属于实数闭区间紧致性的等价表述,可以相互推导.

\subsection{实数完备性}
\begin{theorem}[柯西(Cauchy)收敛原理]
\label{thm:cauchy_convergence}
在实数系$\rr$中,数列$\{x_n\}$有极限存在的充分必要条件是
\[\forall\varepsilon>0,\exists N,n>N,m>N:\,|x_n-x_m|<\varepsilon\]
$\{x_n\}$称为$\rr$的基本列,或柯西列.
\end{theorem}
\par 用柯西收敛原理证明数列的收敛性其实非常方便,因为只需考虑两项,作差后解一个不等式即可.
\par 证明数列发散一般通过找两个子列收敛至不同的数即可,或者用柯西收敛原理的否命题找到$\eps_0$,使得两项作差后大于$\eps_0$.
\begin{theorem}[完备性定理(Cauchy)]
实数系$\rr$是完备的,即$\rr$中每个基本列都在$\rr$中有极限存在,也即\textbf{对极限运算封闭}.
\end{theorem}
\par 注意关于实数完备性的刻画,并没有涉及实数的顺序,这与前文的连续性不同.