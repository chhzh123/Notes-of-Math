% !TEX root = main.tex

\section{实数系基本定理}
\subsection{实数连续性}
\begin{definition}[戴德金(Dedekind)连续性准则]
有大小顺序稠密的数系$S$,对于其任一分划$A|B$,$\exists$唯一$c\in S:\forall a\in A,b\in B:a\leq c\leq b$(即要么$A$有最大值,要么$B$有最小值),则称$S$连续. 
\end{definition}
由此准则可以推出$\mathbb{Q}$不是连续的. 需要注意:
\begin{enumerate}
	\itemsep -3pt
	\item 分划的断点$c$不一定属于$S$(正因$c\notin S$,$\mathbb{Q}$不连续)
	\item 分划要保证\textbf{不空、不漏、不乱}
\end{enumerate}
\begin{theorem}[实数基本定理]
对$\mathbb{R}$的任一分划$A|B$,$\exists$唯一$r\in S:\forall a\in A,b\in B:a\leq r\leq b$
\end{theorem}
\begin{definition}[确界]
$\sup A$为上确界,$\inf A$为下确界
\begin{enumerate}
	\item $\forall x\in A:\,x\leq\beta$
	\item $\forall\varepsilon>0,\exists x_\varepsilon\in A:\,x_\varepsilon >\beta-\varepsilon$
\end{enumerate}
\end{definition}
\begin{theorem}[确界定理]
实数集内,非空有上(下)界的数集必有上(下)确界存在
\end{theorem}
实数连续性的等价表述
\begin{enumerate}
	\item 戴德金分割
	\item 确界定理
	\item 单调有界定理(单调上升有上界实数列必有极限存在)
\end{enumerate}

\subsection{实数闭区间紧致性}
\begin{definition}[覆盖]
$E$为实数开区间构成的集合,$S$为实数子集,若$\forall x\in S$,有$(a,b)\in E$使得$x\in(a,b)$,则$E$为$S$的一个覆盖,记为
\[S\subset\bigcup_{E_\alpha\in E}E_\alpha\]
若$E$为有限区间组成,则为有限覆盖
\end{definition}
\begin{definition}[区间套]
一组实数的\textbf{闭}区间序列$[a_n,b_n],n=1,2,\cdots$,满足
\begin{enumerate}[a.]
	\item $[a_{n+1},b_{n+1}]\subset[a_n,b_n],n=1,2,\cdots$
	\item $\displaystyle\lim_{n\to\infty}(b_n-a_n)=0$
\end{enumerate}
则称$\{[a_n,b_n]\}$构成一个区间套
\end{definition}
\begin{theorem}[区间套定理]
$\{[a_n,b_n]\}$是一个区间套,则存在唯一$\displaystyle r\in\mathbb{R},s.t.\,r\in\bigcap_{n=1}^{\infty}[a_n,b_n]$
\end{theorem}
\par连续函数介值定理、最值定理、康托定理均用到区间套定理来证明.

\subsection{实数完备性}