% !TEX root = main.tex

\subsection{敛散性判定方法总结}
\label{summary_conver}
本小节阐述对于正项级数、一般项级数、无穷限积分、广义积分、函数项级数、含参变量广义积分(\ref{sec:sub:parameter_abnormal_integral}节)中的某几个都通用的判别法. 为方便表述和记忆,用BNF范式表述,有点泛型编程(Generic Programming)的思想在里面,下面先做一些符号说明.
\[\begin{aligned}
\lrang{s}&::=u_n\mmid v_n\\
\lrang{f}&::=f(x)\mmid g(x)\mmid f(x,y)\\
\lrang{fs}&::=u_n(x)\mmid v_n(x)
\end{aligned}\]

\subsubsection{比较判别法}
比较判别法是为通用的判别法,对于上述五种级数积分均可采用,但要注意只能用于判定\textbf{正项}的级数或积分,即其\textbf{绝对收敛}性,使用时注意\textbf{加上绝对值}
\[\begin{aligned}
\lrang{T}&::=|\lrang{s}|\mmid |\lrang{f}|\mmid |\lrang{fs}|\\
\lrang{U}&::=\ssum{|\lrang{s}|}\mmid \intab{a}{+\infty}{|\lrang{f}|}\mmid \intab{a}{b}{|\lrang{f}|}(\mbox{a为瑕点})\mmid \ssum{|\lrang{fs}|}
\end{aligned}\]
\begin{itemize}
	\item 一般形式:
		\[\mbox{$T_1\leq T_2$,$U_2$收敛则$U_1$收敛,$U_1$发散则$U_2$发散}\]
	\item 极限形式:
		\[\mbox{$\displaystyle\lim_{n\to\infty}\dfrac{T_1}{T_2}=l$,$l\in[0,+\infty)$,$U_2$收敛则$U_1$收敛;$l\in(0,+\infty],U_2$发散则$U_1$发散}\]
\end{itemize}
\par 一般形式的比较判别法需要熟悉常见的放缩技巧(见\ref{sec:2.3def_pf_limit}节),如有界量($\sin x,(-1)^n$)直接取界、均值不等式、估计等.
% 2^\ln n=n^\ln 2 3^\sqrt{n}<n^3
\par 极限形式的比较判别法常常与重要极限/无穷小量代换牵连在一起,需要懂得构造.
\begin{example}
$\ssum{\left(\sqrt{1+\dfrac{1}{n}}-1\right)}$
\end{example}
\begin{analysis}
一般形式:
\[\dfrac{\dfrac{1}{n}}{1+\sqrt{1+\dfrac{1}{n}}}=\dfrac{1}{n+n\sqrt{1+\dfrac{1}{n}}}
\geq\dfrac{1}{2n\sqrt{1+\dfrac{1}{n}}}=\dfrac{1}{2\sqrt{n(n+1)}}\geq\dfrac{1}{2n+1}\]
看似平淡无奇,但分母将加号变为乘号后再用均值不等式避免反号的技巧十分高超
\end{analysis}
\begin{example}
$\ssum{n\left(1-\cos\frac{1}{n}\right)}$
\end{example}
\begin{analysis}
极限形式:
\[\limtoinf{\dfrac{n(1-\cos\dfrac{1}{n})}{\dfrac{1}{n}}}=\limtoinf{\dfrac{2\sin^2\dfrac{1}{2n}}{4\dfrac{1}{2n}\dfrac{1}{2n}}}=\dfrac{1}{2}\]
这个降角很有技巧,为使用重要极限做准备
\end{analysis}
极限形式还有比较常见的方法是直接取某一分式作为分母,如$\dfrac{1}{n}$,$\dfrac{1}{n^2}$等
\begin{example}
\[\intab{1}{+\infty}{\ln\lrp{\cos\dfrac{1}{x}+\sin\dfrac{1}{x}}}\]
\end{example}
\begin{analysis}
这里就取三角函数内的$\dfrac{1}{x}$作为比较对象
\[\begin{aligned}
&\quad\limtoinf{\dfrac{\ln\lrp{\cos\frac{1}{x}+\sin\frac{1}{x}}}{\frac{1}{x}}}\\
&=\limtoinf{\dfrac{\lrp{-\sin\frac{1}{x}}\lrp{-\frac{1}{x}}+\lrp{\cos\frac{1}{x}}\lrp{-\frac{1}{x^2}}}{\lrp{\cos\frac{1}{x}+\sin\frac{1}{x}}\lrp{-\frac{1}{x^2}}}}\\
&=\limtoinf{\dfrac{\cos\frac{1}{x}-\sin\frac{1}{x}}{\cos\frac{1}{x}+\sin\frac{1}{x}}}\\
&=1
\end{aligned}\]
\end{analysis}
\par 注意数项级数还有另一种形式:$\dfrac{u_{n+1}}{u_n}\leq\dfrac{v_{n+1}}{v_n}$,判别条件与一般形式的类似.
%\begin{analysis}
%	$\dfrac{u_n}{u_N}\leq\dfrac{v_n}{v_N}$裂项递推
%\end{analysis}

\subsubsection{正项判别法}
同样也是判定绝对收敛
\[\begin{aligned}
\lrang{T_i}&::=|\lrang{s_i}| \mmid |\lrang{fs_i}|\\
\lrang{U}&::=\ssum{|\lrang{s}|}\mmid \ssum{|\lrang{fs}|}
\end{aligned}\]
\par 注意这里要将函数项级数看成$x$为常数的数项级数,但以下几种判别法只能判定收敛,而\textbf{不能}判定一致收敛性.
\begin{enumerate}
	\item 达朗贝尔(D'Alembert)判别法:
	\[\mbox{$\displaystyle\lim_{n\to\infty}\dfrac{T_{n+1}}{T_n}=l$,$l<1$收敛,$l>1$发散}\]
	%\begin{analysis}
	%$\dfrac{u_{n+1}}{u_n}<l+\varepsilon_0=r=\dfrac{r^{(n+1)}}{r_n}$. 幂级数、阶乘
	%\end{analysis}
	\item 拉阿比(Raabe)判别法:
	\[\mbox{$\displaystyle\lim_{n\to\infty}n\left(\dfrac{T_n}{T_{n+1}}-1\right)=S$,$S>1$收敛,$S<1$发散}\]
	%\begin{analysis}
	%注意符号反. 引理:$\forall r>p>1,\exists N, n>N, 1+\dfrac{r}{n}>\left(1+\dfrac{1}{n}\right)^p$
	%\end{analysis}
	\item 柯西(Cauchy)根式判别法:
	\[\mbox{$\displaystyle\lim_{n\to\infty}\sqrt[n]{T_n}=l$,$l<1$收敛,$l>1$发散}\]
	%\begin{analysis}
	%实质是等比数列
	%\end{analysis}
	\item 高斯(Gauss)判别法:
	\[\dfrac{a_{n+1}}{a_n}=1-\dfrac{\alpha}{n}+\dfrac{\gamma_n}{n^\beta}\]
	其中$\alpha,\beta>1$都为常数,且数列$\{\gamma_n\}$有界,则当$\alpha>1$时数列收敛,$\alpha\leq 1$时数列发散(没有失效的情况!)
\end{enumerate}
\par 达朗贝尔和拉阿比判别法(以及上述高斯判别法)常用在前后项可以相消的情况,如阶乘、指数.
\par 柯西根式判别法则用在明显的指数形式,如整个$T$的$n$次幂等.
\begin{example}
\[\ssum{n!\lrp{\dfrac{x}{n}}^n}\]
\end{example}
\begin{analysis}
指数阶乘采用达朗贝尔判别法,注意加绝对值
\[\limtoinf{\lrabs{\dfrac{u_{n+1}}{u_n}}}=\limtoinf\lrp{1-\dfrac{1}{n+1}}^n|x|=\dfrac{|x|}{e}\]
故$|x|<e$时原级数绝对收敛;$|x|>e$时,原级数发散;当$|x|=e$时,达朗贝尔判别法失效.\\
但由\textbf{斯特林(Stirling)公式}
\[n! = \sqrt{2 \pi n} \; \left(\frac{n}{e}\right)^{n}e^{\lambda_n}\]
其中
\[\displaystyle {\frac {1}{12n+1}}<\lambda _{n}<{\frac {1}{12n}}\]
故
\[n!\lrp{\dfrac{e}{n}}^n=\dfrac{\sqrt{2 \pi n} \; \left(\frac{n}{e}\right)^{n}e^{\lambda_n}e^n}{n^n}=\sqrt{2 \pi n} e^{\lambda_n}\nrightarrow 0\]
由数项级数的收敛定理知$|x|=e$时发散
\end{analysis}
\begin{example}
\[\ssum{\dfrac{\alpha(\alpha+1)\cdots(\alpha+n-1)}{n!}\dfrac{1}{n^\beta}},\alpha,\beta>0\]
\end{example}
\begin{analysis}
明显的阶乘暗示,先求
\[\dfrac{u_n}{u_{n+1}}=\dfrac{n+1}{n+\alpha}\left(1+\dfrac{1}{n}\right)^\beta\]
由拉阿比判别法
\[\begin{aligned}
\limtoinf{n\left(\dfrac{u_n}{u_{n+1}}-1\right)}
&=\limtoinf{n\left[\dfrac{n+1}{n+\alpha}\left(1+\dfrac{1}{n}\right)^\beta-1\right]}\\
&=\limtoinf{n\dfrac{(n+1)\left(1+\dfrac{1}{n}\right)^\beta-n-\alpha}{n+\alpha}}\\
&=\limtoinf{n\dfrac{(n+1)\left(1+\dfrac{\beta}{n}+\mathcal{O}\left(\dfrac{1}{n}\right)\right)-n-\alpha}{1+\dfrac{\alpha}{n}}}\mbox{最好别用洛必达,用二项式展开}\\
&=1-\alpha+\beta
\end{aligned}\]
当$1-\alpha+\beta>1$,即$\alpha<\beta$时级数收敛;当$1-\alpha+\beta<1$,即$\alpha>\beta$时级数发散;当$\alpha=\beta$时拉阿比判别法失效,采用高斯判别法
\[\begin{aligned}
\dfrac{u_n}{u_{n+1}}
&=\dfrac{n+1}{n+\alpha}\left(1+\dfrac{1}{n}\right)^\alpha\\
&=\dfrac{n+1}{n+\alpha}\lrp{1+\dfrac{\alpha}{n}+\dfrac{\alpha(\alpha-1)}{2}\dfrac{1}{n^2}+\mathcal{O}\lrp{\dfrac{1}{n^2}}}\\
&=\dfrac{n+\alpha+1-\alpha}{n+\alpha}+\dfrac{\alpha(n+1)}{n(n+\alpha)}+\dfrac{(n+1)\alpha(\alpha-1)}{2n^2(n+\alpha)}+\dfrac{n+1}{n+\alpha}\mathcal{O}\lrp{\dfrac{1}{n^2}}\\
&=1+\dfrac{1}{n}+\dfrac{1}{n^2}\left[\dfrac{(n+1)\alpha(\alpha-1)}{2(n+\alpha)}+\dfrac{n+1}{n+\alpha}\mathcal{O}(1)\right]\\
&=1-\dfrac{-1}{n}+\dfrac{\gamma_n}{n^2}
\end{aligned}\]
$\gamma_n$极限存在故有界,因此$\alpha=\beta$时原级数发散
\end{analysis}
柯西根式判别法可以用在一些奇奇怪怪的数列上,只有它有什么的$n$次方就行.
\begin{example}
\[\ssum(-1)^{n-1}\dfrac{2^n\sin^{2n}x}{n}\]
\end{example}
\begin{analysis}由柯西根式判别法
\[\limtoinf{\sqrt[n]{|u_n|}}=\limtoinf{\dfrac{2\sin^2x}{\sqrt[n]{n}}}=2\sin^2 x\]
故当$2\sin^2 x<1$即$k\pi-\dfrac{\pi}{4}<x<k\pi+\dfrac{\pi}{4},k\in\zz$时,原级数绝对收敛\\
同理可得其他情况. 综合有,原级数在$k\pi-\dfrac{\pi}{4}<x<k\pi+\dfrac{\pi}{4}$绝对收敛,在$x=k\pi\pm\dfrac{\pi}{4}$条件收敛,在$k\pi+\dfrac{\pi}{4}<x<k\pi+\dfrac{3\pi}{4}$发散,其中$k\in\zz$
\end{analysis}

\subsubsection{一般判别法}
这里要区别一致收敛和绝对收敛概念的适用范围.
\par \textbf{一致收敛}用于函数项级数或含参变量广义积分,是指其在各个点收敛的速度都差不多.
\par \textbf{绝对收敛}用于数项级数或广义积分,对于函数列的某个点$x_0$也可以讨论其绝对收敛性.
\par 至于函数列的一致收敛性,只能通过定义来证明.
\par 而对于含参变量广义积分来说,则需跟函数项级数类比,确定清楚哪一个是变化的量(对$y$积分则固定$x$),哪一个要看为常数.
\[\begin{aligned}
\lrang{T}&::=\lrang{s}\mmid \lrang{f}\mmid \lrang{fs}\\
\lrang{U}&::=\ssum{\lrang{s}}\mmid \intab{a}{+\infty}{\lrang{f}}\mmid \intab{a}{b}{\lrang{f}}(\mbox{a为瑕点})\mmid \ssum{\lrang{fs}}\\
\lrang{V}&::=\ssum{\lrang{s}\lrang{s}}\mmid \intab{a}{+\infty}{\lrang{f}\lrang{f}}\mmid \intab{a}{b}{\lrang{f}\lrang{f}}(\mbox{a为瑕点})\mmid \ssum{\lrang{fs}\lrang{fs}}
\end{aligned}\]
\begin{enumerate}
	\item 绝对收敛必收敛
	\item 柯西(Cauchy)收敛原理:注意下面的求和是广义和,即包含了积分\\
	\[\forall\varepsilon>0,\exists N,\forall n',n''>N:\;\sum_{n'}^{n''}T<\varepsilon \;\Leftrightarrow U\mbox{(一致)收敛}\]
	注意,数列中同样有柯西收敛原理,形式完全相同,只是没有求和符号,对应柯西列的概念(见定理\ref{thm:cauchy_convergence}).
	\item 狄利克雷(Dirichlet)判别法:\\
	$T_1$单调(一致)收敛于$0$,$T_2$的\textbf{部分和}(一致)有界,则$V$(一致)收敛
	\item 阿贝尔(Abel)判别法:\\
	$T_1$单调(一致)有界,$U_2$(一致)收敛,则$V$(一致)收敛
\end{enumerate}
\par 柯西收敛原理在证明题中使用简直就是大杀器,绝大多数证明收敛的题目都可以用它倒腾出来
\begin{example}
正项级数$\ssum{a_n}$收敛,证明\[\limtoinf{\dfrac{a_1+2a_2+\cdots+na_n}{n}}=0\]
\end{example}
\begin{analysis}
即证$\forall\varepsilon>0,\exists N,\forall n>N$时
\[\left|\dfrac{a_1+2a_2+\cdots+na_n}{n}\right|<\varepsilon\]
因$\ssum{a_n}$收敛,由柯西收敛原理,$\exists N_1,\forall n>N_1$都有
\[|a_{N_1+1}+a_{N_1+2}\cdots+a_n|<\dfrac{\varepsilon}{2}\]
又因$N_1$是一个确定的值,故$\exists N_2\forall n>N_2$
\[\left|\dfrac{a_1+2a_2+\cdots+N_1a_{N_1}}{n}\right|<\dfrac{\varepsilon}{2}\]
所以
\[\begin{aligned}
\left|\dfrac{a_1+2a_2+\cdots+na_n}{n}\right|
&\leq\left|\dfrac{a_1+2a_2+\cdots+N_1a_{N_1}}{n}\right|+\left|\dfrac{N_1+1}{n}a_{N_1+1}\right|+\left|\dfrac{N_1+2}{n}a_{N_1+2}\right|+\cdots+\left|\dfrac{n}{n}a_{n}\right|\\
&<\dfrac{\varepsilon}{2}+a_{N_1+1}+a_{N_1+2}+\cdots+a_n\\
&<\dfrac{\varepsilon}{2}+\dfrac{\varepsilon}{2}=\varepsilon
\end{aligned}\]
取$N=\max(N_1,N_2)$即得证
\end{analysis}
\begin{example}%P37
对数列$\{a_n\},\{b_n\}$定义$S_n=\ssumkn{a_k}$,$\Delta b_k=b_{k+1}-b_k$,求证
\begin{enumerate}
	\item 若$\nums{S_n}$有界,$\ssum|\Delta b_n|$收敛,且$\limtoinf{b_n}=0$,则$\ssum{a_nb_n}$收敛,且
	\[\ssum{a_nb_n}=-\ssum{S_n\Delta b_n}\]
	\item 若$\ssum{a_n}$与$\ssum{|\Delta b_n|}$都收敛,则$\ssum{a_nb_n}$收敛
\end{enumerate}
\end{example}
\begin{analysis}
这题相当复杂,但却是练习用定义证明的绝佳好题.
\begin{enumerate}
	\item 欲用柯西收敛原理,即$\forall\eps>0,\exists N$,当$n>N$时,$\forall p\in\zz^+$有$\disp\lrabs{\sum_{k=n+1}^{n+p}}<\eps$\\
	用阿贝尔变换
	\[\begin{aligned}
	LHS&=\lrabs{\sum_{k=n+1}^{n+p-1}(S_k-S_n)(b_k-b_{k+1})+b_{n+p}(S_{n+p}-S_n)}\\
	&\leq\sum_{k=n+1}^{n+p-1}(|S_k-S_n||\Delta b_k|+|b_{n+p}||S_{n+p}-S_n|)\qquad(\star)
	\end{aligned}\]
	因为$\nums{S_n}$有界,故
	\[\exists M>0,s.t.\forall n\in\zz^+:|S_n|\leq M\qquad(1')\]
	因为$\ssum|\Delta b_n|$收敛,故
	\[\exists N_1>0,s.t.\forall n>N_1,\forall p\in\zz^+:\lrabs{\sum_{k=n+1}^{n+p}|\Delta b_k|}<\dfrac{\eps}{4M}\qquad(2')\]
	因为$\limtoinf{b_n}=0$,故
	\[\exists N_2>0,s.t.\forall n>N_2:|b_n|<\dfrac{\eps}{4M}\qquad(3')\]
	结合$(1')(2')(3')$式,得
	\[(\star)\leq 2M\dfrac{\eps}{4M}+2M\dfrac{\eps}{4M}=\eps\]
	取$N=\max(N_1,N_2)$即可得$\ssum{a_nb_n}$收敛\\
	再由阿贝尔变换
	\[\ssumkn{a_kb_k}=\sum_{k=1}^{n-1}S_k(b_k-b_{k+1})+b_nS_n=-\sum_{k=1}^{n-1}S_k\Delta b_k+b_nS_n\]
	左右取极限$n\to\infty$有
	\[\ssum{a_nb_n}=-\ssum{S_n\Delta b_n}+\limtoinf{(b_nS_n)}\]
	又$\nums{S_n}$有界,$\limtoinf b_n=0\implies\limtoinf(b_nS_n)=0$,所以
	\[\ssum{a_nb_n}=-\ssum{S_n\Delta b_n}\]
	\item 同1理,由阿贝尔变换有
	\[\lrabs{\sum_{k=n+1}^{n+p}}\leq\sum_{k=n+1}^{n+p-1}(|S_k-S_n||\Delta b_k|+|b_{n+p}||S_{n+p}-S_n|)\qquad(\star)\]
	因为$\ssum{a_n}$收敛,故由数列极限的有界性知,其部分和数列$\nums{S_n}$有界,即
	\[\exists M_1>0,\forall n\in\zz^+:|S_n|\leq M_1\qquad(a')\]
	因为$\ssum{|\Delta b_n|}$收敛,所以$\ssum{\Delta b_n}$收敛,其部分和数列
	\[B_n=\ssumkn{\Delta b_k}=b_{n+1}-b_1\]
	极限存在,故$\{b_n\}$有极限,进而$\nums{b_n}$有界,即
	\[\exists M_2>0,\forall n\in\zz^+:|b_n|\leq M_2\qquad(b')\]
	因为$\ssum{a_n}$与$\ssum{|\Delta b_n|}$都收敛,取$M=\max(M_1,M_2)$,由柯西收敛原理
	\[\exists N_1>0,\forall n>N_1,\forall p\in\zz^+:\lrabs{\sum_{k=n+1}^{n+p}a_k}=|S_{n+p}-S_n|<\dfrac{\eps}{2M}\qquad(c')\]
	\[\exists N_2>0,\forall n>N_2,\forall p\in\zz^+:\lrabs{\sum_{k=n+1}^{n+p}\Delta b_k}<\dfrac{\eps}{4M}\qquad(d')\]
	结合$(a')(b')(c')(d')$式,取$N=\max(N_1,N_2)$,得
	\[(\star)\leq 2M\dfrac{\eps}{4M}+M\dfrac{\eps}{2M}=\eps\]
	由柯西收敛原理知$\ssum{a_nb_n}$收敛
\end{enumerate}
\end{analysis}
\par 下面的例\ref{cauchy_s1}、例\ref{cauchy_s2}是一些小结论.
\begin{example}
\label{cauchy_s1}
\begin{enumerate}
	\item 若$\ssum{a_n}$收敛,$\ssum{a_n^2}$不一定收敛
	\item 若$\ssum{a_n}(a_n\geq 0)$收敛,$\ssum{a_n^2}$收敛
	\item 若$\ssum{a_n^2}$收敛,则$\ssum{a_n^3}$绝对收敛
\end{enumerate}
\begin{analysis}
\begin{enumerate}
	\item 取$a_n=\dfrac{(-1)^n}{\sqrt{n}}$
	\item 当$n$足够大时有$a_n<1$,则$a_n^2=a_n\cdot a_n\leq a_n$,由比较判别法知收敛
	\item 同2理
\end{enumerate}
\end{analysis}
\begin{example}
\label{cauchy_s2}
关于$na_n$的一些例子
\begin{enumerate}
	\item 若$\ssum{a_n}(a_n>0)$收敛,$na_n$不一定趋于0
	\item 若$\ssum{a_n}(a_n>0)$收敛,$a_{n+1}\leq a_n$,则$\limtoinf{na_n}=0$
	\item 若$\ssum{a_n}$收敛,$\limtoinf{na_n}=0$,则$\ssum{n(a_n-a_{n+1})}=\ssum{a_n}$
	\item 若$\intab{a}{+\infty}{f(x)}$收敛,且$f(x)$在$[a,+\infty)$单调下降,则$\limtoinf{xf(x)}=0$
\end{enumerate}
\end{example}
\begin{analysis}
\begin{enumerate}
	\item 取$a_n=\dfrac{1}{n^2},n\neq k^2,k\in\zz^+$,$a_{k^2}=\dfrac{1}{k^2},k\in\zz^+$\\
分开两个$p$级数求和,当然收敛;因子列$k^2a_{k^2}\to 1$,故$na_n$不趋于$0$
	\item 由柯西收敛原理
\[(n-N)a_n=a_n+\cdots+a_n\leq a_N+\cdots+a_n<\varepsilon\]
又$\limtoinf{a_n}=0$,故
\[na_n<Na_n+\dfrac{\varepsilon}{2}<N\dfrac{\varepsilon}{2N}+\dfrac{\varepsilon}{2}=\varepsilon\]
进而$\limtoinf{na_n}=0$
	\item 由$\limtoinf{na_n}=0$及$\limtoinf{a_n}=0$有
\[\ssum{n(a_n-a_{n+1})}=\ssum{a_n}-(n+1)a_{n+1}+a_{n+1}\]
	\item 同2,但要先反证得$f(x)\geq 0$
\end{enumerate}
\end{analysis}
\end{example}
\par 下面例\ref{egdirconv1}无论是方法还是结论都相当重要,类似地可以证明$\ssum{\dfrac{\cos nx}{n}}$在$x\neq k\pi,k\in\zz$时收敛.
\begin{example}
\[\ssum{\dfrac{\sin nx}{n}}\]
\label{egdirconv1}
\end{example}
\begin{analysis}
当$x=k\pi,k\in\zz$时,级数一般项为$0$,显然绝对收敛\\
当$x\neq k\pi,k\in\zz$时,由三角函数积化和差公式,裂项相消可得
\[2\sin\dfrac{x}{2}\ssumkn{\sin kx}=\cos\dfrac{x}{2}-\cos\dfrac{2n+1}{2}x\]
故
\[\lrabs{\ssumkn{\sin kx}}\leq\dfrac{\lrabs{\cos\dfrac{x}{2}}+\lrabs{\cos\dfrac{2n+1}{2}x}}{2\lrabs{\sin\dfrac{x}{2}}}\leq\dfrac{1}{\lrabs{\sin\dfrac{x}{2}}}\]
即$\ssumkn{\sin kx}$有界,而$\{\dfrac{1}{n}\}$单调下降趋于$0$,由狄利克雷判别法知原级数收敛
\end{analysis}
\par 下面是几道变形例题.
\begin{example}
\[\ssum{\frac{\cos nx}{n^p}},x\in(0,\pi)\]
\end{example}
\begin{analysis}
\begin{enumerate}
	\item 当$p>1$时,$\lrabs{\dfrac{\cos nx}{n^p}}\leq\dfrac{1}{n^p}$,而$\ssum{\dfrac{1}{n^p}}$收敛,由比较判别法,原级数绝对收敛\par
	\item 当$0<p\leq 1$时,由例\ref{egdirconv1}类似的方法知$\ssum{\cos nx}$有界,而$\{\dfrac{1}{n^p}\}$单调下降趋于$0$,故由狄利克雷判别法,原级数收敛\par
但由于(\textbf{相当关键的一步})
\[\lrabs{\dfrac{\cos nx}{n^p}}\geq\dfrac{\cos^2 nx}{n^p}=\dfrac{1}{2n^p}+\dfrac{\cos 2nx}{2n^p}\]
$\ssum{\dfrac{1}{2n^p}}$发散,$\ssum{\dfrac{\cos 2nx}{2n^p}}$收敛(狄利克雷),故$\ssum\lrabs{\dfrac{\cos nx}{n^p}}$发散,即原级数条件收敛\par
	\item 当$p\leq 0$时,$\limtoinf{\cos nx}\neq 0$. 若不然,则
\[0=\limtoinf{\cos nx}\neq 0=\limtoinf{\dfrac{1+\cos 2nx}{2}}=\dfrac{1}{2}\]
矛盾,而$\dfrac{1}{n^p}\geq 1$,故$\limtoinf{\cos\dfrac{nx}{n^p}}\neq 0$,由数项级数收敛的必要条件知原级数发散
\end{enumerate}
综上,原级数$p>1$时绝对收敛,$p\in(0,1]$时条件收敛,$p\leq 0$时发散
\end{analysis}
\begin{example}
\label{countereg_conver_integ}
\[\intab{1}{+\infty}{\sin x^2}\]
\end{example}
\begin{analysis}
\[\intab{1}{A}{\sin x^2}=\int_{1}^{A^2}\dfrac{\sin u}{2\sqrt{u}}\,\mathrm{d}u\]
这步替换相当关键,因$\dfrac{1}{\sqrt{u}}$单调递减趋于$0$,$\sin u$积分有界,故由狄利克雷判定法,原积分收敛
\end{analysis}
\par 类似地如$\ssum(-1)^n\dfrac{\cos 2n}{n},\ssum(-1)^n\dfrac{\sin^2 n}{n}$等,都可用上述思想解决.
\begin{example}
\[\intab{2}{+\infty}{\dfrac{\ln\ln x}{\ln x}\sin x}\]
\end{example}
\begin{analysis}
\begin{enumerate}
	\item 先证明收敛性
\[\dfrac{\ln\ln x}{\ln x}\sin x=\dfrac{\sin x}{\sqrt{\ln x}}\dfrac{\ln\ln x}{\sqrt{\ln x}}\]
因为$\dfrac{1}{\sqrt{\ln x}}$单调递减趋于$0$,$\sin x$积分有界,故由狄利克雷判别法,$\intab{2}{+\infty}{\dfrac{\sin x}{\sqrt{\ln x}}}$收敛\\
而$\lim_{x\to\infty}\dfrac{\ln\ln x}{\sqrt{\ln x}}=0$,故$\dfrac{\ln\ln x}{\sqrt{\ln x}}$有界\\
又$\lrp{\dfrac{\ln\ln x}{\sqrt{\ln x}}}'=\dfrac{2 - \ln\ln x}{2 x \ln x^{\frac{3}{2}}}$,故当$x\geq e^{e^2}$时,$\dfrac{\ln\ln x}{\sqrt{\ln x}}$单调递减\\
由阿贝尔判别法,原积分收敛
\item 再证明绝对收敛性
\[\lrabs{\dfrac{\ln\ln x}{\ln x}\sin x}\geq\dfrac{\ln\ln x}{\ln x}\sin^2 x=\dfrac{|\ln\ln x}{2\ln x}-\dfrac{(\ln\ln x)\cos 2x}{2\ln x}\]
类似地,由阿贝尔判别法知,$\intab{2}{+\infty}{\dfrac{(\ln\ln x)\cos 2x}{2\ln x}}$收敛\\
而当$x$充分大时,$\dfrac{\ln\ln x}{\ln x}\geq\dfrac{1}{x}$,故由比较判别法$\intab{2}{+\infty}{\dfrac{|\ln\ln x|}{2\ln x}}$发散\\
进而$\intab{2}{+\infty}{\dfrac{\ln\ln x}{\ln x}\sin^2 x}$发散\\
由比较判别法,$\intab{2}{+\infty}{\dfrac{|\ln\ln x|}{\ln x}|\sin x|}$发散
\end{enumerate}
综上,原积分条件收敛\\
注:这题的思路需要非常清晰,否则各种判别法混着使用很容易就迷失自我.
\end{analysis}

\subsubsection{方法总结}
\begin{enumerate}
	\item 尝试求出和式或积分具体的值
	\item 用某些收敛充要(必要)条件判断不会发散
	\item 比较判别法最先考虑,绝对收敛用正项判别法,一般收敛性或一致收敛性用一般判别法
\end{enumerate}

% 判别法 criterion

%\begin{example}[实数十进制表示法]
%实数可表示成十进制形式
%\[b_mb_{m-1}\cdots b_0.a_1a_2\cdots a_n\cdots,\]
%其中$0\leq a_j,b_j\leq 9,a_j,b_j\in\mathbb{Z}$,其有何意义\footnote{南大秦理真《微积分》5.2节}
%\end{example}
%\begin{analysis}
%考虑$[0,1]$中的实数,设
%\[x=0.a_1a_2\cdots a_n\cdots=\ssum\dfrac{a_n}{10^n}\]
%已知
%\end{analysis}

% \[A=\intab{0}{\frac{\pi}{2}}{\ln(\sin x)}\]