% !TEX root = main.tex

\section{定积分}
\label{sec:definite_integration}
\subsection{黎曼积分}
\label{sub:riemann}
\begin{definition}[Riemann积分]
$f(x)$在区间$[a,b]$上有定义,用分点
\[a=x_0<x_1<x_2<\cdots<x_n=b\]
将区间任意分成$n$个小区间,每个小区间的长度为
\[\Delta x_i=x_i-x_{i-1},i=1,2,\cdots,n\]
记$\lambda = \underset{1\leq i\leq n}{\max}\{\Delta x_i\}$,在每个小区间上任取一点$\xi\in[x_{i-1},x_i]$,作和式
\[\sigma = \sum_{i=1}^{n}f(\xi_i)\Delta x_i\]
若$f(x)$在$[a,b]$可积
\[I=\lim_{\lambda\to 0}\sigma=\int_a^bf(x)\diff x\]
\end{definition}
\par\textbf{积分基本性质}
\begin{enumerate}
	\item 有界性:可积函数必有界
	\item 线性性:相同积分区间两函数
	\item 可加性:同一函数连续积分区间
	\item 单调性:$f(x),g(x)$在$[a,b]$可积,$f(x)\leq g(x),x\in[a,b]$,则$\displaystyle\int_{a}^{b}f(x)\diff x\leq\int_{a}^{b}g(x)\diff x$
	\item 绝对值不等式:$f(x)$在$[a,b]$可积,则$\displaystyle\left|\int_{a}^{b}f(x)\diff x\right|\leq\int_{a}^{b}|f(x)|\diff x$
\end{enumerate}
\begin{theorem}[闭区间连续函数可积定理]
若$f(x)$在$[a,b]$连续,则$f(x)$在$[a,b]$可积
\end{theorem}
\begin{theorem}[积分第一中值定理]
若$f(x),g(x)$在$[a,b]$连续,且$g(x)$在$[a,b]$不变号,则在$[a,b]$中存在一点$\xi$使得
\[\int_a^bf(x)g(x)\diff x=f(\xi)\int_a^bg(x)\diff x\]
特别地,$g(x)\equiv 1$时,有
\[\int_a^bf(x)\diff x=f(\xi)(b-a)\]
\end{theorem}
\par 在证明题中,绝对值不等式和积分中值定理常常一起出现,同样也是证明的利器.
\begin{theorem}
若$f(x)$在$[a,b]$可积,则变上限定积分$\displaystyle F(x)=\int_a^x f(t)dt$是$[a,b]$上的连续函数
\end{theorem}
\begin{theorem}[微积分基本定理]
$f(x)$在$[a,b]$可积,$F(x)$是$f(x)$在$[a,b]$上的任一原函数,$F'(x)=f(x)$,则$\displaystyle\int_{a}^{b}f(x)\diff x=F(b)-F(a)$
\end{theorem}
\begin{theorem}[柯西不等式(积分形式)]
$f(x),g(x)$在$[a,b]$连续,则
\[\int_a^bf^2(x)\diff x\cdot \int_a^bg^2(x)\diff x\geq\left(\int_a^bf(x)g(x)\diff x\right)^2\]
等号成立当且仅当$f(x),g(x)$线性相关
\end{theorem}
\begin{analysis}
可通过构造二次函数,令$\Delta\geq0$证明;\\
也可利用线性代数的知识,定义内积\footnote{线性代数没学好的请自行翻书}$\displaystyle	\langle u,v\rangle=\int_a^bu(x)v(x)\diff x$,通过$\|u\|\|v\|\geq|\langle u,v\rangle|$得证
\end{analysis}
\begin{theorem}[三角不等式(积分形式)]
$f(x),g(x)$在$[a,b]$连续,则
\[\sqrt{\int_a^bf^2(x)\diff x} + \sqrt{\int_a^bg^2(x)\diff x}\geq\sqrt{\int_a^b[f(x)+g(x)]^2\diff x}\]
等号成立当且仅当$g(x)=\lambda f(x),\lambda\geq 0$
\end{theorem}
\begin{analysis}
对应线性代数中$\|u\|+\|v\|\geq\|u+v\|$
\end{analysis}
\begin{example}
$f(x)$在$[a,b]$上可积,其积分为$I$,改变$[a,b]$内有限个点的$f(x)$值,使其变成另一函数$f^*(x)$,证明$f^*(x)$也在$[a,b]$上可积,且积分仍为$I$
\end{example}
\begin{analysis}
设改变了$k$个点的函数值,这些点为$c_1,c_2,\cdots,c_k$\\
构造
\[F(x)=f(x)-f^*(x)=\begin{cases}
0,x\notin\{c_1,c_2,\cdots,c_k\}\\
f(x)-f^*(x),x\in\{c_1,c_2,\cdots,c_k\}
\end{cases}\]
用分点$a=x_0<x_1<x_2<\cdots<x_n=b$将区间任意分成$n$个小区间,每个小区间的长度为$\Delta x_i=x_i-x_{i-1},i=1,2,\cdots,n$,记$\lambda = \underset{1\leq i\leq n}{\max}\{\Delta x_i\}$\\
在每个小区间上任取一点$\xi\in[x_{i-1},x_i]$,作和式$\displaystyle\sigma = \sum_{i=1}^{n}f(\xi_i)\Delta x_i$
则
\[k\lambda\underset{x\in[a,b]}{\min}F(x)\leq\sigma\leq k\lambda\underset{x\in[a,b]}{\max}F(x)\]
当$\lambda\to 0$,由极限的夹迫性知$\displaystyle\lim_{\lambda\to 0}\sigma=0$\\
故$F(x)$在$[a,b]$上可积,
\[\int_a^b F(x)\diff x=\int_a^b[f(x)-f^*(x)]\diff x=0\]
由定积分的线性性,
\[\int_a^b f^*(x)\diff x=\int_a^b f(x)\diff x-\int_a^b F(x)\diff x=I-0=I\]
因此$f^*(x)$在$[a,b]$上可积,且积分仍为$I$.
\end{analysis}
\par 学完微分积分后会涌现出很多不等式证明题,但其实很多时候不需要太多的代数技巧,直接求导即可解决(对于单变量函数).见例\ref{intinequality}法一其实就是高考题做法.
\begin{example}
\label{intinequality}
设$f'(x)$在$[a,b]$连续,且$f(a)=0$,证明
\[\left|\int_a^bf(x)\diff x\right|\leq\frac{(b-a)^2}{2}\underset{a\leq x\leq b}{\max}|f'(x)|\]
\end{example}
\begin{analysis}法一:\\
为方便计算,不妨将$f(x)$平移至$[0,t],t>0$,则$f(0)=0$,即证
\[\left|\int_0^tf(x)\diff x\right|\leq\frac{t^2}{2}\,\underset{0\leq x\leq t}{\max}|f'(x)|\]
构造
\[g(t)=\frac{t^2}{2}\,\underset{0\leq x\leq t}{\max}|f'(x)|-\left|\int_0^tf(x)\diff x\right|\quad\mbox{(注意是关于}t\mbox{的函数)}\]
则$g(0)=0$,$g'(t)=\displaystyle t\underset{0\leq x\leq t}{\max}|f'(x)|$\\
要令$g'(t)\geq 0$,即令$\displaystyle\underset{0\leq x\leq t}{\max}|f'(x)|\geq\left|\frac{f(t)}{t}\right|\quad(*)$\\
由拉格朗日中值定理,在$(0,t)$上必$\displaystyle\exists\xi>0\,s.t.\,f'(\xi)=\frac{f(t)-0}{t-0}$\\
故最大值必然大于等于区间内某一点的值$|f'(\xi)|$,即$(*)$式成立\\
进而$g'(t)\geq 0,g(t)\geq 0$,原不等式成立.
\end{analysis}
\begin{analysis}法二:\\
先证明
\[\int_a^bf(x)\diff x=-\int_a^bf'(x)(x-b)\diff x\quad(*)\]
因为
\[\begin{aligned}
-\int_a^bf'(x)(x-b)\diff x &= -\int_a^b(x-b)df(x)\\
&=-\left[f(x)(x-b)\Big|_a^b-\int_a^bf(x)\diff x\right]\qquad\mbox{分部积分}\\
&=\int_a^bf(x)\diff x
\end{aligned}\]
即证得$(*)$式.则原式\\
\[\begin{aligned}
\left|\int_a^bf(x)\diff x\right|&=\left|\int_a^bf'(x)(x-b)\diff x\right|\\
&\leq\underset{a\leq x\leq b}{\max}|f'(x)|\left|\int_a^b(x-b)\diff x\right|\\
&=\underset{a\leq x\leq b}{\max}|f'(x)|\frac{(x-b)^2}{2}\Big|_a^b\\
&=\frac{(b-a)^2}{2}\underset{a\leq x\leq b}{\max}|f'(x)|
\end{aligned}\]
\end{analysis}
同理可证下面的练习.
\begin{exercise}
设$f''(x)$在$[a,b]$连续,且$f(a)=f(b)=0$,证明:
\begin{enumerate}[(1)]
	\item $\displaystyle\int_a^bf(x)\diff x=\frac{1}{2}\int_a^bf''(x)(x-a)(x-b)\diff x$
	\item $\displaystyle\left|\int_a^bf(x)\diff x\right|\leq\frac{(b-a)^3}{12}\underset{a\leq x\leq b}{\max}|f''(x)|$
\end{enumerate}
\end{exercise}

\subsection{可积性}
\begin{definition}[达布(Darboux)和]
设$f(x)$在$[a,b]$有界,对$[a,b]$的任意分法
\[a=x_0<x_1<\cdots<x_n=b\]
记
\[M_i=\sup_{x_{i-1}\leq x\leq x_i}f(x),\,m_i=\inf_{x_{i-1}\leq x\leq x_i}f(x)\]
分别称
\[S=\sum_{i=1}^nM_i\Delta x_i,\,s=\sum_{i=1}^nm_i\Delta x_i\]
为$f(x)$对应这一分法的达布上和与达布下和
\end{definition}
\par 有以下几个定理.
\begin{theorem}\mbox{}\par
\begin{enumerate}
	\itemsep-3pt
	\item 达布上和是黎曼和的上确界,达布下和是黎曼和的下确界
	\item 若添加新的分点,达布上和不增大,达布下和不减小
	\item 任一个达布下和总不超过任一个达布上和
\end{enumerate}
\end{theorem}
\begin{theorem}[达布定理]
当$\disp\lambda=\max_{1\leq i\leq n}\{\Delta x_i\}\to 0$时,达布上和的极限为上积分,达布下和的极限为下积分
\[\lim_{\lambda\to 0}S=\overline{I},\,\lim_{\lambda\to 0}s=\underline{I}\]
且$s\leq\overline{I}\leq\underline{I}\leq S$,其中,下积分$\underline{I}$为全体下和的上确界,上积分为$\overline{I}$为全体上和的下确界.
\end{theorem}
\begin{theorem}[可积的充要条件]
$f(x)$在$[a,b]$上黎曼可积的充要条件是上积分$\overline{I}$等于下积分$\underline{I}$.
充要条件可另外表述为
\[\lim_{\lambda\to 0}(S-s)=0\,.\]
或者记
\[\omega_i=M_i-m_i=\sup_{x_{i-1}\leq x\leq x_i}f(x)-\inf_{x_{i-1}\leq x\leq x_i}f(x)=\sup_{x',x''\in[x_{i-1},x_i]}|f(x')-f(x'')|\]
则
\[\lim_{\lambda\to 0}\sum_{i=1}^n\omega_i\Delta x_i=0\]
\end{theorem}
\par 以下给出几类特殊的可积函数是可积.
\begin{enumerate}
	\itemsep -3pt
	\item 在$[a,b]$连续
	\item 在$[a,b]$单调
	\item 在$[a,b]$上有界且仅有有限个间断点的函数
\end{enumerate}
\par 注意:存在单调但有无穷个间断点且可积的函数
\par 证明函数在某个区间上可积,即证明$\disp\sum_{i=1}^n\omega_i\Delta x_i$式收敛于$0$,关键在于振幅的处理.
\begin{example}
$f(x)$在$[0,1]$上有界,不连续点为$x=\dfrac{1}{n},n\in\zz^+$,则$f(x)$在$[0,1]$上可积.
\end{example}
\begin{analysis}
关键在于将和式拆分为有限部分和无限部分考虑.\\
考虑到$f(x)$在$[0,\delta)(\delta>0)$部分包含无穷多个不连续点,而$f(x)$在$[\delta,1]$只含有有穷个不连续点.\\
故尝试将和式拆为
\[\sum_{i=1}^n\omega_i\Delta x_i=\sum_{i=1}^j\omega_i\Delta x_i+\sum_{i=j+1}^n\omega_i\Delta x_i\]
其中,区间为
\[x_0=0<x_1<x_2<\cdots<x_j<\delta<x_{j+1}<\cdots<x_n=1\]
因为$f(x)$在$[\delta,1]$有界且含有有穷个不连续点,故$f(x)$在$[\delta,1]$可积,由可积的充要条件有,
\[\forall\eps>0,\exists\delta_0>0,\lambda=\max_i\{\Delta x_i\}<\delta_0:\,\lra{\sum_{i=j+1}^n\omega_i\Delta x_i}<\dfrac{\eps}{2}\]
再考虑无穷个不连续点的部分. 因为$f(x)$在$[0,1]$有界,所以
\[\exists M>0:\,|f(x)|<M\]
进而振幅$\omega_i<2M$,有
\[\lra{\sum_{i=1}^j\omega_i\Delta x_i}\leq\sum_{i=1}^j|\omega_i||\Delta x_i|\leq 2M\sum_{i=1}^j|\Delta_i|=2Mx_j<2M\delta\]
综上,取$\delta=\dfrac{\eps}{4M}$,有
\[\lra{\sum_{i=1}^n\omega_i\Delta x_i}\leq\lra{\sum_{i=1}^j\omega_i\Delta x_i}+\lra{\sum_{i=j+1}^n\omega_i\Delta x_i}<2M\dfrac{\eps}{4M}+\dfrac{\eps}{2}=\eps\]
(注意$[0,\delta)$部分并不关心分多少份,只关心$\delta$的值,故在$[\delta,1]$部分限制$\lambda$的最大值并不影响最终结果)
进而得证.
\end{analysis}
\par 证明可积性第一步通常是对振幅$\omega_i$进行放缩,放缩成关于上确界$M_i$和下确界$m_i$的表达式.
如果条件已经给出某些函数可积,则证明的时候努力往条件的形式靠即可,类似于极限的证明题.
\begin{example}
\label{ex:f_f2_absf_relation_lemma}
若$|f(x)|$可积,则$|f(x)^\alpha|,\alpha>0$可积
\end{example}
\begin{analysis}
\begin{enumerate}
	\item[$1\degree$] 若$\alpha\in\zz^+$,有
\[\omega_i'\leq M_i^\alpha-m_i^\alpha=(M_i-m_i)(\cdots)\]
其余证明与例\ref{ex:f_f2_absf_relation}中$2\degree$的方法相同,故成立
	\item[$2\degree$] 若$\alpha\notin\zz^+$,且$\alpha\in(0,1)$,则设$\alpha=\dfrac{1}{p},p>1$\\
因为$|f(x)|$可积,所以
\[\forall\eps>0,\exists\delta>0,\lambda<\delta:\,\sum_{i=1}^m\omega_i\Delta x_i<\eps\]
设$\dfrac{1}{p}+\dfrac{1}{q}=1$,则对于$|f(x)|^\frac{1}{p}$,由赫尔德(H\"{o}lder)不等式,有
\[\begin{aligned}
\sum_{i=1}^m\omega'_i\Delta x_i&\leq\sum_{i=1}^m(M_i^{\frac{1}{p}}-m_i^{\frac{1}{p}})\Delta x_i\qquad\mbox{类似例\ref{ex:f_f2_absf_relation}中$2\degree$的方法}\\
&=\sum_{i=1}^m(M_i^{\frac{1}{p}}-m_i^{\frac{1}{p}})(\Delta x_i)^{\frac{1}{p}}(\Delta x_i)^{\frac{1}{q}}\\
&\leq\lrp{\sum_{i=1}^m(M_i^{\frac{1}{p}}-m_i^{\frac{1}{p}})^p\Delta x_i}^{\frac{1}{p}}\lrp{\sum_{i=1}^m\Delta x_i}^{\frac{1}{q}}\\
&\leq\lrp{\sum_{i=1}^m(M_i-m_i)\Delta x_i}^{\frac{1}{p}}\lrp{\sum_{i=1}^m\Delta x_i}^{\frac{1}{q}}\qquad(*)\\
&=\eps^\frac{1}{p}(x_n)^\frac{1}{q}\to 0
\end{aligned}\]
故$|f(x)|^\frac{1}{p}$可积
	\item[$3\degree$] 若$\alpha\notin\zz^+$,且$\alpha>1$
设$\alpha=[\alpha]+\dfrac{1}{p}$,因为$[\alpha]\in\zz^+$,由$1\degree$知$|g(x)|^{[\alpha]}$可积,由$2\degree$知$|g(x)|^{\frac{1}{p}}$可积,进而$|g(x)|^\alpha=|g(x)|^{[\alpha]}|g(x)|^{\frac{1}{p}}$可积\\
(因为$(a+b)^2=a^2+b^2+2ab$,由平方可积可推得乘法可积)\\
\end{enumerate}
$(*)$式补充证明:即证$(M_i^\frac{1}{p}-m_i^\frac{1}{p})^p\leq M_i-m_i$\\
由于$M_i,m_i$为$|g(x)|$在某个区间上得上确界、下确界,所以$M_i,m_i>0$\\
原式等价于
\[\lrp{\dfrac{m_i}{M_i}}^\frac{1}{p}+\lrp{1-\dfrac{m_i}{M_i}}^\frac{1}{p}\geq 1,\;0\leq\dfrac{m_i}{M_i}\leq 1\]
构造$h(x)=x^\frac{1}{p}+(1-x)^\frac{1}{p}$,求导易知当$x=\dfrac{1}{2}$时有最大值,$x=0$或$1$时有最小值$h(0)=h(1)=1$,故$(*)$式成立.
\end{analysis}
\begin{example}
\label{ex:f_f2_absf_relation}
讨论$f(x),f^2(x),|f(x)|$可积性的关系
\end{example}
\begin{analysis}
\begin{enumerate}
	\item[$1\degree$] 若$f(x)$可积,$|f(x)|$必可积\\
若$M_i,m_i>0$,则$M_i'-m_i'=M_i-m_i$\\
若$M_i,m_i<0$,则$M_i'-m_i'=M_i-m_i$\\
若$M_i>0,m_i<0$,则$M_i'-m_i'\leq M_i-m_i$\\
用$2\degree$同样的方法可证得$\disp\lim_{\lambda\to 0}\sum_{i=1}^m\omega'_i\Delta x_i=0$,故$|f(x)|$可积
	\item[$2\degree$] 若$f(x)$可积,$f^2(x)$必可积\\
若$f(x)$可积,则$f(x)$有界,即$\exists M>0:\,|f(x)|<M$\\
且$\disp\lim_{\lambda\to 0}\sum_{i=1}^m\omega_i\Delta x_i=0$\\
由极限的定义有
\[\forall\eps>0,\exists\delta>0,\lambda<\delta:\,\lra{\sum_{i=1}^m\omega_i\Delta x_i}<\dfrac{\eps}{2M}\]
对于$f^2(x)$,由例\ref{ex:inf_sup_inequality}可得
\[\begin{aligned}
\omega_i'&=\sup_{x_{i-1}\leq x\leq x_i}f(x)-\inf_{x_{i-1}\leq x\leq x_i}f(x)\\
&\leq \sup_{x_{i-1}\leq x\leq x_i}f(x)\cdot\sup_{x_{i-1}\leq x\leq x_i}f(x)-
\inf_{x_{i-1}\leq x\leq x_i}f(x)\cdot\inf_{x_{i-1}\leq x\leq x_i}f(x)\\
&=M_i^2-m_i^2
\end{aligned}\]
所以,$\forall\eps>0,\exists\delta>0$,当$\lambda<\delta$时有
\[\begin{aligned}
\lra{\sum_{i=1}^m\omega_i\Delta x_i}&\leq\lra{\sum_{i=1}^m(M_i^2-m_i^2)\Delta x_i}\\
&=\lra{\sum_{i=1}^m(M_i+m_i)(M_i-m_i)\Delta x_i}\\
&\leq\lra{\sum_{i=1}^m2M\omega_i\Delta x_i}\\
&<2M\dfrac{\eps}{2M}=\eps
\end{aligned}\]
故$\disp\lim_{\lambda\to 0}\sum_{i=1}^m\omega'_i\Delta x_i=0$,$f^2(x)$可积
	\item[$3\degree$] 若$|f(x)|$可积,$f(x)$不一定可积\\
狄利克雷函数的平移$f(x)=\begin{cases}\dfrac{1}{2}&x\in\mathbb{Q}\\
-\dfrac{1}{2}&x\in\rr/\mathbb{Q}\end{cases}$
	\item[$4\degree$] 若$|f(x)|$可积,$f^2(x)$必可积\\
同$2\degree$,将证明中所有$f(x)$替换成$|f(x)|$即可
	\item[$5\degree$] 若$f^2(x)$可积,$f(x)$不一定可积\\
同$3\degree$构造反例
	\item[$6\degree$] 若$f^2(x)$可积,$f|x|$必可积\\
令$g(x)=f^2(x)$,则$\sqrt{g(x)}=|f(x)|$.\\
由例\ref{ex:f_f2_absf_relation_lemma},令$\alpha=\dfrac{1}{2}$即可成立.
\end{enumerate}
\par 综上,除了$|f(x)|$或$f^2(x)$可积无法推出$f(x)$可积,其他都可以相互推导.
\end{analysis}

\subsection{定积分的计算}
关键掌握不定积分的求法,定积分就能迎刃而解了,最大不同只在于定积分换元时需要把上下限一起改变(出现上限小于下限的情况也是正常的).
\par 用定义计算的题则要懂得配凑出黎曼和的形式,如下面的练习\ref{limitrimsum}.
\begin{exercise}
\label{limitrimsum}
$\displaystyle\lim_{n\to\infty}\left(\frac{1}{n+1}+\frac{1}{n+2}+\cdots+\frac{1}{2n}\right)$
\end{exercise}

\subsection{定积分的近似计算}
\begin{definition}[定积分梯形公式]
\[\int_a^bf(x)\diff x\thickapprox\sum_{i=1}^n\frac{y_{i-1}+y_i}{2}\frac{b-a}{n}=\frac{b-a}{n}\left(\frac{y_0+y_n}{2}+y_1+y_2+\cdots+y_{n-1}\right)\]
若$f''(x)$在$[a,b]$上连续,且$|f''(x)|\leq M,x\in[a,b]$,则梯形公式误差可作如下估计
\[|R_n|\leq\frac{(b-a)^3}{12n^2}M\]
\end{definition}
\begin{definition}[抛物线公式/辛普森(Simpson)公式]
\[\begin{aligned}
\int_a^bf(x)\diff x &\thickapprox\frac{b-a}{6n}[(y_0+4y_1+y_2)+(y_2+4y_3+y_4)+\cdots+(y_{2n-2}+4y_{2n-1}+y_{2n})]\\
&=\frac{b-a}{6n}[(y_0+y_{2n})+2(y_2+y_4+\cdots+y_{2n-2})+4(y_1+y_3+\cdots+y_{2n-1})]\end{aligned}\]
若$f^{(4)}(x)$在$[a,b]$上连续,且$|f^{(4)}(x)|\leq M,x\in[a,b]$,则
\[|R_{2n}|\leq\frac{(b-a)^5}{180(2n)^4}M\]
\end{definition}

%\sin(x+a)\sin(x+b)