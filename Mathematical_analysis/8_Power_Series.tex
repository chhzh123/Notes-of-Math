% !TEX root = main.tex

\section{幂级数}
幂级数不过是特殊的函数项级数,因此判定定理的使用方法与无穷级数的完全一致,而本章的很多定理都是第\ref{sec:series}章定理的直接推论.
\subsection{收敛半径与收敛域}
\begin{definition}[幂级数]
$\ssumz{a_nx^n}$称为幂级数,注意它的各个项严格按照升序排列
\end{definition}
\begin{theorem}[阿贝尔(Abel)第一定理]
若幂级数在$x_1\neq 0$处收敛,则在$|x|<|x_1|$上都\textbf{绝对}收敛;若在$x_2\neq 0$处发散,则在$|x|>|x_2|$都发散.
\end{theorem}
\par 注意对于幂级数区间的两个端点$x=\pm r$要单独讨论其敛散性.
\begin{definition}
对于任意给定的幂级数,存在唯一的$r\in[0,+\infty]$,使得幂级数在$|x|<r$绝对收敛,在$|x|>r$发散,其中$r$称为收敛半径. 用达朗贝尔判别法可给出幂级数的收敛半径$r=\dfrac{1}{\rho}$,其中
\[\rho=\lim_{n\to+\infty}\dfrac{|a_{n+1}|}{|a_n|}\]
\end{definition}
\begin{theorem}[阿贝尔第二定理]
若幂级数收敛半径为$r>0$,
\begin{enumerate}
	\itemsep -3pt
	\item 则$\forall b\in(0,r)$,幂级数在$[-b,b]$\textbf{一致}收敛
	\item 且幂级数在$r$收敛,则幂级数在$[0,r]$一致收敛
	\item 且幂级数在$-r$收敛,则幂级数在$[-r,0]$一致收敛
\end{enumerate}
注意:不能说明$(-r,r)$一致收敛,如$\ssumz{x^n}$(同例\ref{ex:interesting_uniform_convergence}解释)
\end{theorem}
\par 由阿贝尔第二定理与定理\ref{thm:sum_function_analysis_properties}可直接得到以下推论.
\begin{corollary2}[幂级数和函数的分析性质]
若幂级数的收敛半径为$r>0$,则其和函数
\begin{enumerate}
	\itemsep -3pt
	\item 在$(-r,r)$连续
	\item 在收敛区间内部可以逐项微分与逐项积分,且新的幂级数收敛半径仍为$r$
	\item 在$(-r,r)$任意次可微,且$S^{(k)}$等于$\ssumz{a_nx^n}$逐项微商$k$次后得到的幂级数
\end{enumerate}
\end{corollary2}
\begin{analysis}
和函数的连续性看似与定理\ref{thm:sum_function_analysis_properties}矛盾,实则不然,下面的证明短小而意味深长,值得慢慢体会.\\
$\forall x_0\in(-r,r)$,令$b=\dfrac{r+|x_0|}{2}$,则$|x_0|<b<r$\\
由阿贝尔第二定理,幂级数在$[-b,b]$一致收敛,而$a_nx^n$在$(-\infty,+\infty)$连续,故由定理\ref{thm:sum_function_analysis_properties}知,和函数在$[-b,b]$连续,特别地在$x_0$连续.\\
由$x_0\in(-r,r)$的\textbf{任意性},和函数在$(-r,r)$连续.
\end{analysis}
\par 运用逐项微分和逐项积分的方法,可以求得一些复杂级数的和. 关键在于如何经过变换使得新的函数可以求和.
注意逐项积分求的是定积分,而不是不定积分.
也就是说$x=0$的项必须要考虑进去,因为不一定为$0$.
\begin{example}
\[\begin{aligned}
&\quad\ssum{\dfrac{(-1)^{n-1}}{n(2n-1)}x^{2n}}\qquad\mbox{收敛域$[-1,1]$}\\
&=\ssum{\int_0^x2\dfrac{(-1)^{(n-1)}}{(2n-1)}t^{2n-1}\diff t}\\
&=\int_0^x\lrp{\ssum{\int_0^x2(-1)^{n-1}t^{2n-2}\diff t}}\diff t\\
&=2\int_0^x\int_0^u\lrp{\ssum{(-t^2)^{n-1}}}\diff t\diff u\\
&=2\int_0^x\int_0^u\dfrac{1}{1+t^2}\diff t\diff u\\
&=2\int_0^x\arctan u\diff u\qquad\mbox{分部积分}\\
&=2x\arctan x-\ln(1+x^2)
\end{aligned}\]
\end{example}

\subsection{泰勒公式}
\label{sec:sub:taylor}
\subsubsection{基本性质}
泰勒(Taylor)公式在函数拟合方面有着举足轻重的作用,推导的方式也多种多样,通常通过一阶一阶逐渐逼近的方式得到,关于其通俗易懂的解释可见\footnote{怎样更好地理解并记忆泰勒展开式? - 陈二喜的回答 - 知乎 \url{http://zhihu.com/question/25627482/answer/313088784}}.
\begin{theorem}[带余项的泰勒公式]
$f(x)$在$x=a$有$n$阶微商,即$f^{(n)}(a)$存在,则
\[f(x)=f(a)+f'(a)(x-a)+\cdots+\dfrac{\dnf{n}(a)}{n!}(x-a)^n+R_n(x)\,,\]
其中,佩亚诺(Peano)余项为
\[R_n(x)=o((x-a)^n)(x\to a)\]
其仅仅表示了$x\to a$时误差的变化,更加精确的误差估计可用以下几种余项公式. 积分余项
%微积分基本定理+积分中值定理+分部积分
\[R_n(x)=\dfrac{1}{n!}\intabu{a}{x}{(x-t)^n\dnf{n+1}(t)}{t}\]
拉格朗日余项%积分第二中值
\[R_n(x)=\dfrac{\dnf{n+1}(\xi)}{(n+1)!}(x-a)^{n+1}\]
柯西余项%积分第一中值
\[R_n(x)=\dfrac{(x-a)^{n+1}}{n!}(1-\theta)^n\dnf{n+1}(a+\theta(x-a)),\theta\in(0,1),\xi\in(a,x)\]
\end{theorem}
当$a=0$时,泰勒公式即为麦克劳林(Maclaurin)公式.
\par 用泰勒公式可以求解一些复杂的极限,如下面的例子所示.
\begin{example}
$\displaystyle\lim_{x\to 0}\dfrac{1-\cos\sin x}{2\ln(1+x^2)}$
\end{example}
\begin{analysis}
由带佩亚诺余项的泰勒公式有
\[\cos\sin x=1-\dfrac{1}{2}\sin^2 x+o(\sin^2 x)=1-\dfrac{1}{2}\sin^2x+o(x^2)(x\to 0)\]
\[\ln(1+x^2)=x^2+o(x^2)\]
因而
\[\text{原式}=\lim_{x\to 0}\dfrac{\frac{1}{2}\sin^2x}{2x^2}=\dfrac{1}{4}\]
\end{analysis}
\par 当然,洛必达法则也是求解的. 但对于一些特殊的极限,只有通过等价无穷小量代换,极限才容易求得,否则易陷入求导无穷无尽的套子里.
\par 关于泰勒公式应该展开至几阶,一般遵循“分式上下同阶,加减幂次最低”原则\footnote{利用泰勒公式求极限时,如何确定泰勒公式展开到第几阶? - 摆渡人宝刀君的回答 - 知乎 \url{https://www.zhihu.com/question/52636834/answer/195479110}}.
只要保证相关级数的项齐全即可,如求
\[\lim_{x\to \infty}\lrp{x+\dfrac{1}{2}}\ln\lrp{1+\dfrac{1}{x}}\]
不能只将$\ln(1+1/x)$展开至一阶,因为乘开后会有$1/(2x)$项,但这并不涵盖所有的$x^{-1}$项,故展开至二阶才可以确保结果的准确性.
\par 下面是加减法的例子.
\begin{example}
$\disp\lim_{x\to+\infty}(\sqrt[3]{x^3+3x}-\sqrt{x^2-2x})$
\end{example}
\begin{analysis}
\[\begin{aligned}
\sqrt[3]{x^3+3x}-\sqrt{x^2-2x}&= x\lrp{\lrp{1+\dfrac{3}{x^2}}^\frac{1}{3}-\lrp{1+\dfrac{-2}{x}}^\frac{1}{2}}\\
&\thicksim x\lrp{\lrp{1+\dfrac{1}{3}\dfrac{3}{x^2}+o\lrp{\dfrac{1}{x^2}}}-\lrp{1+\dfrac{1}{2}\dfrac{-2}{x}-\dfrac{1}{8}\lrp{\dfrac{-2}{x}}^2+o\lrp{\dfrac{1}{x^2}}}}
\to 1(x\to+\infty)
\end{aligned}\]
\end{analysis}
\par 泰勒级数也可用来证明一些函数不等式,一般展开至二阶即可.
\begin{example}
$f(x)$在$[a,b]$上有二阶连续导数,且$f'(a)=f'(b)=0$,则存在$c\in(a,b)$,使得
\[|f''(c)|\geq\dfrac{4}{(b-a)^2}|f(b)-f(a)|\]
\end{example}
\begin{analysis}
将$f(x)$在$x=a$处展开,则$f(x)=f(a)+\dfrac{f''(c_1)}{2}(x-a)^2,c_1\in(a,x)$\\
将$f(x)$在$x=b$处展开,则$f(x)=f(b)+\dfrac{f''(c_2)}{2}(x-b)^2,c_2\in(x,b)$\\
分别取$x=\dfrac{a+b}{2}$代入上面两式得
\[f\lrp{\dfrac{a+b}{2}}=f(a)+\dfrac{f''(c_1)}{2}\lrp{\dfrac{a-b}{2}}^2=f(b)+\dfrac{f''(c_2)}{2}\lrp{\dfrac{a-b}{2}}^2\]
进而
\[\dfrac{4}{(b-a)^2}|f(a)-f(b)|=\dfrac{1}{2}|f''(c_2)-f''(c_1)|\leq\dfrac{1}{2}(|f''(c_2)|+|f''(c_1)|)\]
因为$f''(x)$连续,故
\[\min\{|f''(c_1)|,|f''(c_2)|\}\leq\dfrac{|f''(c_1)|+|f''(c_2)|}{2}\leq\max\{|f''(c_1)|,|f''(c_2)|\}\]
所以由连续函数介值定理,
\[\exists c\in[c_1,c_2]:\;|f''(c)|=\dfrac{|f''(c_1)|+|f''(c_2)|}{2}\]
综上,
\[\exists c\in(a,b):\;|f''(c)|\geq\dfrac{4}{(b-a)^2}|f(b)-f(a)|\]
\end{analysis}
\par 下面定理表明若函数可以展开成幂级数,则一定为泰勒级数
\begin{theorem}[唯一性]
若$f(x)$在$(x_0-r,x_0+r)$可展开为关于$x-x_0$的幂级数,则必为$f(x)$在$x_0$的泰勒级数
\end{theorem}
\par 函数无穷次可微并不是可展开成幂级数的充分条件,如$\disp f(x)=\begin{cases}e^{-\frac{1}{x^2}}&x\ne 0\\0 & x=0\end{cases}$.
\begin{theorem}[充要条件]
$f(x)$在$(-r,r)$能够展开为幂级数,当且仅当$\forall x\in(-r,r)$
\[\lim_{n\to\infty}R_n(x)=0\]
\end{theorem}
\begin{corollary2}[充分条件]
若$f(x)$的各阶微商在$(-r,r)$\textbf{一致有界},则$f(x)$在$(-r,r)$可以展开成泰勒级数
% 拉格朗日余项可证
\end{corollary2}

\subsubsection{常见公式}
\[\begin{aligned}
e^x&=1+x+\dfrac{1}{2}x^2+\cdots+\dfrac{x^n}{n!}+\cdots\\
\sin x&=x-\dfrac{x^3}{3!}+\dfrac{x^5}{5!}-\cdots+(-1)^n\dfrac{x^{2n+1}}{(2n+1)!}+\cdots\\
\cos x&=1-\dfrac{x^2}{2!}+\dfrac{x^4}{4!}-\cdots+(-1)^n\dfrac{x^{2n}}{(2n)!}+\cdots\\
(1+x)^\alpha&=1+\alpha x+\binom{\alpha}{2}x^2+\cdots+\binom{\alpha}{n}x^n+\cdots,\,x\in(-1,1)\\
\ln(1+x)&=x-\dfrac{x^2}{2}+\dfrac{x^3}{3}-\cdots+(-1)^n\dfrac{x^n}{n}+\cdots,\,x\in(-1,1]
\end{aligned}\]
\par 由这些常见的公式,我们可以快速写出一些函数的泰勒级数展开. 关键在于如何将原函数通过变换,变成我们熟悉的函数.
\begin{example}
下面是一些例子,最终的展开式不作赘述,只阐述前面变换的部分.
\begin{enumerate}
	\item $\sin^3 x=\dfrac{3\sin x-\sin 3x}{4}$
	\item $\dfrac{x}{\sqrt{1-3x}}=x(1+(3x))^{\frac{1}{2}}$
	\item $\disp\ln(x+\sqrt{1+x^2})=\int (1+x^2)^{-\frac{1}{2}}\diff x$
	\item $\dfrac{1}{1-3x+2x^2}=\dfrac{2}{1-2x}-\dfrac{1}{1-x}$
	\item $\ln(1+x+x^2)=\ln(1-x^3)-\ln(1-x)$
\end{enumerate}
\end{example}
\par 用柯西乘积(\ref{sec:subsub:series_operation}节)的方式,可求得更多级数的泰勒展开式.
\begin{example}
\[\begin{aligned}
(\arctan x)^2&=\lrp{\int_0^x\dfrac{1}{1+t^2}\diff t}^2\\
&=\lrp{\int_0^x\ssumz{(-1)^nt^{2n}}\diff t}^2\\
&=\lrp{\ssumz{\dfrac{(-1)^n x^{2 n+1}}{2 n+1}}}^2\\
&=\ssumz{\sum_{k=0}^n\dfrac{(-1)^n x^{2 n+1}}{2 n+1}\dfrac{(-1)^{n-k} x^{2 (n-k)+1}}{2 (n-k)+1}}\\
&=\ssumz{\lrp{(-1)^n\sum_{k=0}^n\dfrac{1}{(2k+1)(2(n-k)+1)}}x^{2(n+1)}},\,|x|\leq 1
\end{aligned}\]
\end{example}
\par 不仅如此,我们也可以得到一些特殊级数的和.
\begin{example}
\label{ex:square_reciprocal_sum}
$\ssum{\dfrac{1}{n^2}}$
\end{example}
\begin{analysis}
本题是著名的巴塞尔(Basel)问题\footnote{Basel Problem, \url{https://en.wikipedia.org/wiki/Basel_problem}},最先由欧拉用下面的方法解决.\\
由$\sin x$的泰勒级数展开有
\[{\displaystyle {\frac {\sin x}{x}}=1-{\frac {x^{2}}{3!}}+{\frac {x^{4}}{5!}}-{\frac {x^{6}}{7!}}+\cdots }\]
可以看出左侧的根为$x=\pm n\pi,n\in\zz^+$,由魏尔斯特拉斯分解定理\footnote{Weierstrass Factorization Theorem, \url{https://en.wikipedia.org/wiki/Weierstrass_factorization_theorem},最初欧拉没有用本定理,因而并不严谨.},可将左式写成如下形式
\[{\displaystyle {\begin{aligned}{\frac {\sin x}{x}}&=\left(1-{\frac {x}{\pi }}\right)\left(1+{\frac {x}{\pi }}\right)\left(1-{\frac {x}{2\pi }}\right)\left(1+{\frac {x}{2\pi }}\right)\left(1-{\frac {x}{3\pi }}\right)\left(1+{\frac {x}{3\pi }}\right)\cdots \\&=\left(1-{\frac {x^{2}}{\pi ^{2}}}\right)\left(1-{\frac {x^{2}}{4\pi ^{2}}}\right)\left(1-{\frac {x^{2}}{9\pi ^{2}}}\right)\cdots \end{aligned}}}\]
提取二次项得
\[{\displaystyle -\left({\frac {1}{\pi ^{2}}}+{\frac {1}{4\pi ^{2}}}+{\frac {1}{9\pi ^{2}}}+\cdots \right)=-{\frac {1}{\pi ^{2}}}\sum _{n=1}^{\infty }{\frac {1}{n^{2}}}}\]
对$\sin x/x$展开式左右两侧对比系数有
\[-{\frac {1}{\pi ^{2}}}\sum _{n=1}^{\infty }{\frac {1}{n^{2}}}=-{\frac {1}{3!}}\]
移项即得证.
\end{analysis}