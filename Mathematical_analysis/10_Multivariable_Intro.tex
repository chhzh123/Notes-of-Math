% !TEX root = main.tex

\section{多元函数简介}
\subsection{平面点集}
\begin{definition}
设$E$为平面点集,若$P_0$为
\begin{enumerate}
	\itemsep -3pt
	\item 内点:存在邻域所有点都属于$E$,$\exists\delta>0:\,O(P_0,\delta)\subset E$
	\item 外点:存在邻域所有点都不属于$E$,$\exists\delta>0:\,O(P_0,\delta)\cap E=\varnothing$
	\item 边界点:任意邻域都含有属于$E$的点也有非$E$的点,$\forall\delta>0:\,O(P_0,\delta)\cap E\ne\varnothing \land O(P_0,\delta)\backslash E\ne\varnothing$
	\item 聚点:任意\textbf{空心}邻域都含有$E$的点,$\forall\delta>0:\,O^*(P_0,\delta)\cap E\ne\varnothing$
\end{enumerate}
注:内点必为聚点,\textbf{一般}聚点集即为内点和边界点的并集
\end{definition}
上面的定义关注\textbf{点}的性质,下面的定义则关注这些点构成的\textbf{集合}的性质
\begin{definition}
几类重要的平面点集
\begin{enumerate}
	\itemsep -3pt
	\item 开集:类比开区间,所有$E$中的点都是内点
	\item 闭集:类比闭区间,所有聚点都属于$E$;或定义为\textbf{补集为开集}的集合
	\item 连通集:$E$中任两点都可用曲线相连
	\item 开区域:开集且连通
	\item 闭区域:开区域与其边界构成的集合
\end{enumerate}
需要注意以下几点:
\begin{enumerate}
	\itemsep -3pt
	\item[a.] 开集定义不是所有内点都属于$E$,因为内点必属于$E$. 开集换种说法应该是$E$中没有非内点的点
	\item[b.] 闭区域不是定义为闭集且连通,考虑点集$E=\{(x,y)\mid x^2+y^2\leq 1\lor (x-2)^2+y^2\leq 1\}$是闭集且两个圆借助点$(1,0)$连通,但是去掉边界后,两个圆不再连通,即两个圆的内部不构成开区域,进而$E$不是闭集
\end{enumerate}
\end{definition}
\begin{theorem}
开集的并是开集,闭集的交是闭集,\textbf{有限个}开集的交是开集,\textbf{有限个}闭集的并是闭集
\end{theorem}
反例如下,无限个开集的交却为闭集
\[\bigcap_{n=1}^\infty(-\frac{1}{n},\frac{1}{n})=0\]
无限个闭集的并却为开集
\[\bigcup_{n=1}^\infty[-1+\frac{1}{n},1-\frac{1}{n}]=(-1,1)\]
\begin{definition}[对角线]
$E\subset \rn$的对角线为
\[d(E):=\sup_{x_1,x_2\in E}d(x_1,x_2)\]
若$d(E)$是有限数,则称$E$有界
\end{definition}
\begin{definition}[紧集]
设$E$为平面点集,若集合$E$的任一覆盖都有有限子覆盖,则称$E$为紧集.
紧集具有以下性质
\begin{enumerate}
	\itemsep -3pt
	\item 若$E$为$\rn$的紧集,则$E$为$\rn$的闭集
	\item 任意含于$E$的闭集都为紧集
	\item $E\subset\rn$是紧集当且仅当$E$是有界闭集
\end{enumerate}
\end{definition}


\subsection{多元函数的极限}
\begin{definition}[平面点列的极限]
$\{P_n=(x_n,y_n)\}$为平面点列,若
\[\forall \eps>0,\exists N\in\zz^+,s.t. n>N:r(P_n,P_0)<\eps,\]
则记$\limtoinf P_n=P_0$
\end{definition}
\begin{definition}[有界]
若存在$M>0$使得$r(P_n,O)\leq M$,则称$\{P_n\}$有界
\end{definition}
对于实数系的各种定理,拓展到二维依然可用
\begin{theorem}
\begin{enumerate}
	\item 柯西收敛原理:$\{P_n\}$收敛充要条件为$r(P_n,P_m)<\eps$
	\item 致密性定理:$\{P_n\}$有界,则$\{P_n\}$必有收敛子列
	\item 矩形套定理:存在唯一点$P_0$含于所有矩形之中
	\item 有限覆盖定理:$E$为平面上有界闭集,$\xi$为$E$的一个覆盖,则$\xi$中存在有限个开集$G_1,G_2,\ldots,G_n$使得$E\subset \bigcup_{i=1}^n G_i$
	\item 海涅定理:$\lim_{P\to P_0}f(P)=A$的充要条件为对$O^*(P_0,\delta)$中任意满足$P_n\to P_0(n\to\infty)$的点列$\{P_n\}$,有$\limtoinf f(P_n)=A$
\end{enumerate}
\end{theorem}
\begin{definition}[二元函数的(全面)极限]
$f$在$P_0(x_0,y_0)$的某个空心邻域有定义,
\[\forall \eps>0,\exists\delta>0,s.t.\,\forall P\in O^*(P_0,\delta)\text{或}K^*(P_0,\delta):f(P)\in O(A,\eps)\]
则记$\limxy{x_0}{y_0}{f(x,y)}=A$
\end{definition}
从二元函数的极限中我们得知,要使$P$的极限为$P_0$,则$P$可以以不同方向不同路径趋近$P_0$,这是一个非常强的条件,因此也可以由此证明某些极限不存在。
如$f(x,y)=\dfrac{xy}{x^2+y^2}$,令$y$沿$kx$趋近于0,可证明$\limxy{0}{0}{f(x,y)}$不存在。
\begin{theorem}[全面极限与累次极限的关系]
若$\limxy{x_0}{y_0}f(x,y)=A$(有限或无限),当$y\ne y_0$时,$\lim_{x\to x_0}f(x,y)=\phi(y)$,则
\[\lim_{y\to y_0}\lim_{x\to x_0}f(x,y)=\lim_{y\to y_0}\phi(y)=A\]
\end{theorem}
若在某点函数的全面极限和两个累次极限都存在,则三者必相等;若两个累次极限都存在但不相等,则全面极限必不存在;注意两个累次极限是可以不相等的.

\subsection{多元函数的连续性}
\begin{definition}[连续性]
若函数$f$在点$P_0$的邻域有定义,若$\lim_{P\to P_0}f(P)=f(P_0)$,则称$f(P)$在$P_0$连续
\end{definition}
有界闭区域连续函数的性质与一元连续函数在闭区间的性质类似(见\ref{sec:sub:continuous_function_properties}节)
\begin{theorem}
$f(P)$在有界闭集$E$上连续,则
\begin{enumerate}
	\item 有界性定理:$f(P)$在$E$上有界
	\item 最值定理:$f(P)$在$E$上有最大值和最小值
	\item 一致连续性定理:$f(P)$在$E$上一致连续
	\item 介值定理:若$f(P)$在区域$G$连续,$P_1,P_2\in G,f(P_1)<f(P_2)$,则$\forall c\in (f(P_1),f(P_2)),\exists P_0\in G,\,s.t. f(P_0)=c$
\end{enumerate}
\end{theorem}