\documentclass[UTF8]{ctexbeamer}
\usepackage{latexsym}
\usepackage{amsmath,amssymb}
\usepackage{color,xcolor}
\usepackage{graphicx}
\usepackage{algorithm}
\usepackage{amsthm}

\def\diff{\,\mathrm{d}}
\newtheorem{exercise}[theorem]{练习} %*去除编号

\usetheme{AnnArbor}
\usefonttheme[onlymath]{serif}
\usecolortheme{crane}

\title{不定积分与定积分}
\subtitle{Week 3}

\author[chhzh123]{陈鸿峥}
\institute[]{\small\url{https://github.com/chhzh123/Notes-of-Math/blob/master/Mathematical_analysis/main.pdf}}
\date[Dec, 2018]{December, 2018}

\keywords{}

\begin{document}

\begin{frame}
\titlepage
\end{frame}

\begin{frame}
\tableofcontents[subsectionstyle=show]
\end{frame}

\section{基础公式}
\begin{frame}
\sectionpage
\end{frame}

\begin{frame}{基础公式}
\[\begin{aligned}
&\int\frac{1}{x}\diff x=\ln|x|+C\\
&\int\sec^2x\diff x=\tan x+C\\
&\int\csc^2x\diff x=-\cot x+C\\
&\int\frac{1}{\sqrt{1-x^2}}\diff x=\arcsin x+C\\
&\int\frac{1}{1+x^2}\diff x=\arctan x+C
\end{aligned}\]
记得加C!
\end{frame}

\section{三种基本积分方法}
\begin{frame}
\sectionpage
\end{frame}

\begin{frame}{凑微分/第一换元法}
从积分项中提取部分出来拉到微分项中
\begin{example}
\[\int\tan x\diff x\]
\end{example}
\begin{exercise}
\[\int \tan^3 x\diff x\]
\end{exercise}
\end{frame}

\begin{frame}{第二换元法}
\begin{itemize}
	\item 直接换元(令$x=g(u)$),注意$\diff x$也需要一起换.
	\item 常见于三角还原或消根式
\end{itemize}
\begin{example}[\textsection 6.2/例12]
\[\int\frac{\diff x}{x^2\sqrt{x^2-1}}\]
\end{example}
\end{frame}

\begin{frame}{分部积分法}
\begin{itemize}
	\item 先写成$\displaystyle\int u(x)\diff v(x)$的形式,然后直接交换$u(x),v(x)$即可
	\item 选取$u(x)$顺序:\textbf{对反幂三指},如求$\displaystyle\int x^2\cos x\diff x$,取$u(x)=x^2$,化为$\displaystyle\int x^2\diff\sin x$
\end{itemize}
\begin{example}
\[\int\ln x\diff x\]
\end{example}
\end{frame}

\begin{frame}{常见公式 - 三角/反三角}
\[\int \tan x\diff x=-ln|\cos x|+C\qquad\mbox{凑微分法,小心}-\mbox{号}\]
\[\int \cot x\diff x=ln|\sin x|+C\qquad\mbox{凑微分}\]
\[\int \sec x\diff x=ln|\sec x+\tan x|+C\qquad\mbox{乘}\frac{\sec x+\tan x}{\sec x+\tan x}\]
\[\int \csc x\diff x=ln|\csc x-\cot x|+C\qquad\mbox{乘}\frac{\csc x+\cot x}{\csc x+\cot x}\]
\[\int \arcsin x\diff x=x\arcsin x+\sqrt{1-x^2}+C\qquad\mbox{分部积分}\]
\[\int \sec^3 x \diff x=\frac{1}{2} (\tan x \sec x+\ln |\tan x+\sec x|)+C\qquad\mbox{分部积分}\]
\end{frame}

\begin{frame}{常见公式 - 根式/分式}
\[\int \frac{\diff x}{a^2+x^2}=\frac{1}{a} \arctan\frac{x}{a}+C\qquad\mbox{分子分母除以}a^2\]
\[\int \frac{\diff x}{\pm(x^2-a^2)}=\pm\frac{1}{2a} \ln\left|\frac{x-a}{x+a}\right|+C\qquad\mbox{裂项可得}\]
\[\int \frac{\diff x}{\sqrt{a^2-x^2}}=\arcsin\frac{x}{a}+C\qquad\mbox{将}a^2\mbox{提出来}\]
\[\int \frac{\diff x}{\sqrt{x^2\pm a^2}}=\ln|x+\sqrt{x^2\pm a^2}|+C\qquad\mbox{三角换元}\]
\[\int \sqrt{a^2-x^2}\diff x=\frac{a^2}{2}\arcsin\frac{x}{a}+\frac{x}{2}\sqrt{a^2-x^2}+C\qquad\mbox{三角换元}\]
\[\int \sqrt{x^2\pm a^2}\diff x=\pm\frac{a^2}{2} \ln\Big|x+\sqrt{x^2\pm a^2}\Big|+\frac{x}{2}\sqrt{x^2\pm a^2}+C\qquad\mbox{三角换元}\]
\end{frame}

\section{不同类型积分常见思路}
\begin{frame}
\sectionpage
\end{frame}

\subsection{有理分式}
\begin{frame}
\subsectionpage
\end{frame}

\begin{frame}{假分式}
同样,对于求积分来说,{\large\textcolor{red}{化简}}也是关键的,\textbf{假分式}先除下来变为真分式(长除法)!
\begin{example}
\[\int\frac{2x^3-4x^2-x-3}{x^2-2x-3}\diff x\]
\end{example}
\begin{exercise}[\textsection 6.1/1(5)]
\[\int\frac{3x^2}{1+x^2}\diff x\]
\end{exercise}
\begin{exercise}[\textsection 6.1/1(6)]
\[\int\frac{x+1}{\sqrt{x}}\diff x\]
\end{exercise}
\end{frame}

\begin{frame}{部分分式}
全部变为真分式后,用\textbf{部分分式}进行拆分(代数基本定理),分母全部分解为一次乘二次的形式
\[\prod_{i=1}^k(x-a_i)^{j_i}\prod_{i=1}^s(x^2+p_ix+q_i)^{l_i}\]
结合分子,即有部分分式的两种基本形式
\[\frac{A}{(x-a)^n},\quad\frac{Bx+C}{(x^2+px+q)^n}\]
如何积?
\end{frame}

\begin{frame}{部分分式}
\begin{itemize}
	\item 多项式的因式分解(首尾系数猜根)
	\[x^3+5x^2+8x+4\]
	\[x^5-x^4+2x^3-2x^2+x-1\]
	\item 线性因子掩盖法
	\item 补齐次数(对比系数解方程)
\end{itemize}
\end{frame}

\begin{frame}{部分分式}
因式分解
\begin{example}
\[\int\frac{\diff x}{1+x^4}\]
\end{example}
\end{frame}

\begin{frame}{配凑}
配凑为分母形式
\begin{example}
\[\int\frac{x\mathrm{e}^x}{(x+1)^2}\diff x\]
\end{example}
\begin{exercise}
\[\int\frac{x^2+2}{(x+1)^3}\diff x\]
\end{exercise}
\end{frame}

\begin{frame}{配凑}
配凑为导数形式
\begin{example}
\[\int\frac{x+1}{x^2+x+1}\]
\end{example}
\end{frame}

\begin{frame}{总结}
分式积分是后面三角积分和根式积分的基础,要非常熟悉,方法要点总结如下:
\begin{itemize}
	\item 假分式先除下来变为真分式,分母因式分解位一次乘二次,然后才使用部分分式
	\item 若非纯有理分式(如各种基本初等函数的组合)或分母次数太高,则将分子配凑成\textbf{分母形式}或\textbf{分母导数形式}以便分拆相加(这两种方式都十分常用)
	\item 一次式直接积出$\ln$,二次式分子配分母/分母配平方
	\item 小技巧:通过倒代换$\displaystyle x=\frac{1}{t}$降低分母次数,有$\displaystyle\frac{1}{x},\frac{1}{x^2}$等部分的可以考虑,如$\displaystyle\int\frac{\diff x}{x^{100}+x}$也可使用
\end{itemize}
\end{frame}

\subsection{三角函数}
\begin{frame}
\subsectionpage
\end{frame}

\begin{frame}{恒等变换}
\begin{itemize}
	\item 恒等式
	\item 半角
	\item 倍角(降幂升角)
	\item 积化和差
	\item 和差化积(不同角)
	\item 万能公式(弦化切)
	\item 辅助角(相同角)
	\item 关系式
\end{itemize}
\[1+\sin kx=\left(\sin\frac{kx}{2}+\cos\frac{kx}{2}\right)^2\qquad1+\cos x=2\cos^2 x\]
\[\tan^2 x+1=\sec^2 x\qquad(\tan x)'=\sec^2 x\qquad(\sec x)'=\tan x\sec x\]
\end{frame}

\begin{frame}{恒等变换}
变换就是了!
\begin{exercise}[\textsection 6.1/1(9)]
\[\int(\tan^2 x+3)\diff x\]
\end{exercise}
切化弦
\begin{example}
\[\int \tan(x+a)\tan(x+b)\diff x\]
\end{example}
\end{frame}

\begin{frame}{凑微分}
结合凑微分法努力化为同名函数
\begin{example}
\[\int\frac{\sin^3x}{1+\cos^2 x}\diff x\]
\end{example}
\begin{exercise}[\textsection 6.2/1(15)]
\[\int\cos^5x\diff x\]
\end{exercise}
\end{frame}

\begin{frame}{化关系式}
化为有关系的式子
\begin{example}
\[\int\frac{\diff x}{\sin^2 x\cos x}\]
\end{example}
\end{frame}

\begin{frame}{万能公式}
和洛必达一样,到迫不得已才使用
\begin{example}
\[\int \frac{\diff x}{2+\sin^2 x}\]
\end{example}
\begin{exercise}[\textsection 6.2/6(12)]
\[\int\frac{\sin^2x}{1+\cos^2x}\diff x\]
\end{exercise}
\end{frame}

\begin{frame}{总结}
\begin{itemize}
	\item 第一步依然是化简/变换,目的是结合凑微分法努力化为同名函数,或是有一定关系的式子
	\item 迫不得已才采用万能公式,别一上来就太暴力
	\item 最终很大几率会化为分式积分
	\item 小技巧1:通过分子分母同乘的方法强行凑平方,如求$\sec x$积分,分子分母同乘$\cos x$
	\item 小技巧2:配对偶式,如$\displaystyle\int\frac{\sin x}{a\sin x+b\cos x}\diff x$
\end{itemize}
\end{frame}

\subsection{根式}
\begin{frame}
\subsectionpage
\end{frame}

\begin{frame}{根式}
对于二次根式,采用\textbf{整块换元}或\textbf{根式内配方}后三角代换或直接用常用公式的方法
\begin{itemize}
	\item[*] 整块换元的形式
	\[\int R\left(x,\sqrt[n]{\frac{ax+b}{cx+d}}\right)\diff x,\,n>1,ad-bc\ne 0\]
	\item[*] 配方的形式
	\[\int R\left(x,\sqrt{ax^2+bx+c}\right)\diff x,\,a>0,b^2-4ac\ne 0,\mbox{或}a<0,b^2-4ac>0\]
\end{itemize}
\end{frame}

\begin{frame}{根式}
整块换元
\begin{example}
\[\int\sqrt{\frac{1-x}{1+x}}\frac{\diff x}{x^2}\]
\end{example}
配方
\begin{exercise}
\[\int \frac{x^2}{\sqrt{1+x-x^2}}\diff x\]
\end{exercise}
\end{frame}

\begin{frame}{根式}
高次复杂根式
\begin{example}[\textsection 6.2/7(9)]
\[\int \frac{\diff x}{x\sqrt[4]{1+x^4}}\]
\end{example}
\begin{exercise}[\textsection 6.2/7(3)]
\[\int \frac{\diff x}{x\sqrt{x^2+2x+2}}\]
\end{exercise}
\end{frame}

\begin{frame}{根式}
分式结合复杂根式
\begin{example}
\[\int \frac{\diff x}{(x^2+a^2)^{\frac{3}{2}}}\]
\end{example}
\end{frame}

\begin{frame}{总结}
\begin{itemize}
	\item 同样,先化简至最简根式,如$\displaystyle\frac{x}{\sqrt[4]{x^3(1-x)}}$就不算最简根式
	\item 对于二次根式,采用\textbf{整块换元};也可\textbf{根式内配方}后三角代换或直接用常用公式
	\item 对于简单高次根式的加减,用\textbf{最小公倍数}法消根号,如$\sqrt{x}$与$\sqrt[3]{x}$同时存在,令$x=t^{\,lcm(2,3)}=t^6$
	\item 对于复杂高次根式,凑微分不断换元使根式内多项式次数\textbf{降至一次},再进行\textbf{整块换元}转化成有理分式
	\item 分式与根式结合,先用分式的配凑拆分等化简
\end{itemize}
\end{frame}

\end{document}