\documentclass[UTF8]{ctexbeamer}
\usepackage{latexsym}
\usepackage{amsmath,amssymb}
\usepackage{color,xcolor}
\usepackage{graphicx}
\usepackage{algorithm}
\usepackage{amsthm}

\def\diff{\,\mathrm{d}}
\newtheorem{exercise}[theorem]{练习} %*去除编号

\usetheme{AnnArbor}
\usefonttheme[onlymath]{serif}
\usecolortheme{crane}

\title{不定积分与定积分}
\subtitle{Week 3}

\author[chhzh123]{陈鸿峥}
\institute[]{\small\url{https://github.com/chhzh123/Notes-of-Math/blob/master/Mathematical_analysis/main.pdf}}
\date[Dec, 2018]{December, 2018}

\keywords{}

\begin{document}

\begin{frame}
\titlepage
\end{frame}

\begin{frame}
\tableofcontents[subsectionstyle=show]
\end{frame}

\section{基础公式}
\begin{frame}
\sectionpage
\end{frame}

\begin{frame}{基础公式}
\[\begin{aligned}
&\int\frac{1}{x}\diff x=\ln|x|+C\\
&\int\sec^2x\diff x=\tan x+C\\
&\int\csc^2x\diff x=-\cot x+C\\
&\int\frac{1}{\sqrt{1-x^2}}\diff x=\arcsin x+C\\
&\int\frac{1}{1+x^2}\diff x=\arctan x+C
\end{aligned}\]
记得加C!
\end{frame}

\section{三种基本积分方法}
\begin{frame}
\sectionpage
\end{frame}

\begin{frame}{凑微分/第一换元法}
从积分项中提取部分出来拉到微分项中
\begin{example}
\[\int\tan x\diff x\]
\end{example}
\begin{exercise}
\[\int \tan^3 x\diff x\]
\end{exercise}
\end{frame}

\begin{frame}{第二换元法}
\begin{itemize}
	\item 直接换元(令$x=g(u)$),注意$\diff x$也需要一起换.
	\item 常见于三角还原或消根式
\end{itemize}
\begin{example}[\textsection 6.2/例12]
\[\int\frac{\diff x}{x^2\sqrt{x^2-1}}\]
\end{example}
\end{frame}

\begin{frame}{分部积分法}
\begin{itemize}
	\item 先写成$\displaystyle\int u(x)\diff v(x)$的形式,然后直接交换$u(x),v(x)$即可
	\item 选取$u(x)$顺序:\textbf{对反幂三指},如求$\displaystyle\int x^2\cos x\diff x$,取$u(x)=x^2$,化为$\displaystyle\int x^2\diff\sin x$
\end{itemize}
\begin{example}
\[\int\ln x\diff x\]
\end{example}
\end{frame}

\begin{frame}{常见公式}
见笔记
\end{frame}

\section{不同类型积分常见思路}
\begin{frame}
\sectionpage
\end{frame}

\subsection{有理分式}
\begin{frame}
\subsectionpage
\end{frame}

\begin{frame}{假分式}
同样,对于求积分来说,{\large\textcolor{red}{化简}}也是关键的,\textbf{假分式}先除下来变为真分式(长除法)!
\begin{example}
\[\int\frac{2x^3-4x^2-x-3}{x^2-2x-3}\diff x\]
\end{example}
\begin{exercise}[\textsection 6.1/1(5)]
\[\int\frac{3x^2}{1+x^2}\diff x\]
\end{exercise}
\begin{exercise}[\textsection 6.1/1(6)]
\[\int\frac{x+1}{\sqrt{x}}\diff x\]
\end{exercise}
\end{frame}

\begin{frame}{部分分式}
全部变为真分式后,用\textbf{部分分式}进行拆分(代数基本定理),分母全部分解为一次乘二次的形式
\[\prod_{i=1}^k(x-a_i)^{j_i}\prod_{i=1}^s(x^2+p_ix+q_i)^{l_i}\]
结合分子,即有部分分式的两种基本形式
\[\frac{A}{(x-a)^n},\quad\frac{Bx+C}{(x^2+px+q)^n}\]
如何积?
\end{frame}

\begin{frame}{部分分式}
\begin{itemize}
	\item 多项式的因式分解(首尾系数猜根)
	\[x^3+5x^2+8x+4\]
	\[x^5-x^4+2x^3-2x^2+x-1\]
	\item 线性因子掩盖法
	\item 补齐次数(对比系数解方程)
\end{itemize}
\end{frame}

\begin{frame}{部分分式}
\begin{example}
\[\int\frac{\diff x}{1+x^4}\]
\end{example}
\end{frame}

\begin{frame}{配凑}
配凑为分母形式
\begin{example}
\[\int\frac{x\mathbb{e}^x}{(x+1)^2}\diff x\]
\end{example}
\begin{exercise}
\[\int\frac{x^2+2}{(x+1)^3}\diff x\]
\end{exercise}
\end{frame}

\begin{frame}{配凑}
配凑为导数形式
\begin{example}
\[\int\frac{x+1}{x^2+x+1}\]
\end{example}
\end{frame}

\begin{frame}{总结}
\begin{itemize}
	\item 假分式先除下来变为真分式,然后用部分分式
	\item 若非纯有理分式(如各种基本初等函数的组合)或分母次数太高,则将分子配凑成分母形式或分母导数形式以便分拆相加(这两种方式都十分常用)
	\item 分母一定要先分解为一次乘二次的形式(虚数也可),也即分解为
		\[\prod_{i=1}^k(x-a_i)^{j_i}\prod_{i=1}^s(x^2+p_ix+q_i)^{l_i}\]
		结合分子,也就有部分分式的两种基本形式
		\[\frac{A}{(x-a)^n},\quad\frac{Bx+C}{(x^2+px+q)^n}\]
	\item 前者直接积,后者配平方
	\item 小技巧:通过倒代换$\displaystyle x=\frac{1}{t}$降低分母次数,有$\displaystyle\frac{1}{x},\frac{1}{x^2}$等部分的可以考虑,如$\displaystyle\int\frac{\diff x}{x^{100}+x}$也可使用
\end{itemize}
\end{frame}

\subsection{三角函数}
\begin{frame}
\subsectionpage
\end{frame}

\begin{frame}{恒等变换}
\begin{itemize}
	\item 半角
	\item 倍角(降幂升角)
	\item 积化和差
	\item 和差化积(不同角)
	\item 万能公式
	\item 辅助角(相同角)
\end{itemize}
\end{frame}

\begin{frame}{恒等变换}
\begin{exercise}[\textsection 6.1/1(9)]
\[\int(\tan^2 x+3)\diff x\]
\end{exercise}
\end{frame}

\begin{frame}{恒等变换}
结合凑微分法努力化为同名函数
\begin{example}
\[\int\frac{\sin^3x}{1+\cos^2 x}\diff x\]
\end{example}
化为有关系的式子
\begin{example}
\[\int\frac{\diff x}{\sin^2 x\cos x}\]
\end{example}
\end{frame}

\subsection{根式}
\begin{frame}
\subsectionpage
\end{frame}

\begin{frame}{根式}
见笔记
\end{frame}

\end{document}