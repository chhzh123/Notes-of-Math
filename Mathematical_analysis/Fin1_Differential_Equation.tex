% !TEX root = main.tex

\section{微分方程}
通过分离变量法求解一阶微分方程,注意要将微分算子看成是可以运算的(乘除).
\begin{example}
单摆的近似运动方程.
\end{example}
\begin{analysis}
由牛顿第二定律有
\[m\dfrac{\diff^2 s}{\diff t^2}=-mg\sin\theta\]
其中,$s$为扫过的弧长,因圆周角$\disp\theta=\dfrac{s}{l}$,故
\[\dfrac{\diff^2 s}{\diff t^2}=-g\sin\dfrac{s}{l}\]
该方程比较难解,我们只考虑当$\theta\to 0$时,上式变为
\[\dfrac{\diff^2 s}{\diff t^2}=-g\dfrac{s}{l}\]
令$\dfrac{\diff s}{\diff t}=p(s)$,则
\[\dfrac{\diff^2 s}{\diff t^2}=\dfrac{\diff p}{\diff t}=\dfrac{\diff p}{\diff s}\dfrac{\diff s}{\diff t}=\dfrac{\diff p}{\diff s}p\]
故可将方程变为一阶方程
\[p\dfrac{\diff p}{\diff s}=-\dfrac{g}{l}s\]
分离变量得
\[p\diff p=-\dfrac{g}{l}s\diff s\]
解得
\[p^2=-\dfrac{g}{l}s^2+C_1\]
最大偏离时弧长为$s_0$,此时速度为$0$,可得$C_1$,于是
\[\lrp{\dfrac{\diff^2 s}{\diff t^2}}^2=\dfrac{g}{l}(s_0^2-s^2)\]
开方,分离变量积分得
\[\arcsin\dfrac{s}{s_0}=\sqrt{\dfrac{g}{l}}t+C_2\]
假定开始时,摆处于铅垂线上,则
\[s=s_0\sin\sqrt{\dfrac{g}{l}}t\]
即为周期$T=2\pi\sqrt{\dfrac{l}{g}}$的简谐运动.
\end{analysis}