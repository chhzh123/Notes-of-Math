% !TEX root = main.tex

\section{含参变量积分}
\subsection{正常积分}
含参变量积分其实就对应着概率论里面的边缘概率密度,而二重积分则是联合概率密度.
\begin{theorem}
对于含参变量积分$I(x)=\disp\int_c^df(x,y)\diff y$,$f(x,y)$在$[a,b]\times [c,d]$上连续,有
\begin{enumerate}
	\item 积分与极限互换:$I(x)$在$[a,b]$连续
	\item 积分与导数互换:$f_x$也在$[a,b]\times [c,d]$连续,则$I(x)$在$[a,b]$有连续导函数
	\[I'(x)=\int_c^df_x(x,y)\diff y\]
	\item 积分交换次序:
	\[\intab{a}{b}{}\intabu{c}{d}{f(x,y)}{y}=\intabu{c}{d}{}{y}\intab{a}{b}{f(x,y)}\]
\end{enumerate}
(对于变上限积分$I(x,u)=\disp\int_c^uf(x,y)\diff y$也是一样)
\end{theorem}
\begin{example}
$\disp I(a)=\int_0^\pi\ln(1-2a\cos x+a^2)\diff x,\,|a|<1$
\end{example}
\begin{analysis}
$f(a,x)=\ln(1-2a\cos x+a^2),\,f_a(a,x)=\dfrac{2a^2-2a\cos x}{1-2a\cos x+a^2}$,由$\Delta=4\cos^2x-4<0$知$1-2a\cos x+a^2>0$恒成立,进而$f(a,x)$与$f_a(a,x)$都在$[-1,1]$连续,$I(a)$在积分号下可求导数
\[\begin{aligned}
I'(a)&=\intab{0}{\pi}{f_a(a,x)}\\
&=\frac{1}{a}\intab{0}{\pi}{\frac{2a^2-2a\cos x}{1-2a\cos x+a^2}}\\
&=\frac{1}{a}\intab{0}{\pi}{1+\frac{a^2-1}{1-2a\cos x+a^2}}\qquad\mbox{把分子的$\cos$项消去}\\
&=\frac{x}{a}\Big|_0^\pi+\frac{a^2-1}{a}\intab{0}{+\infty}{\frac{1}{a^2-2a\cos x+1}}\\
&=\frac{\pi}{a}+\frac{a^2-1}{a}\int_0^{+\infty}\frac{\frac{2}{1+t^2}\diff t}{a^2-2a\frac{1-t^2}{1+t^2}+1}\qquad\mbox{令}t=\tan\frac{x}{2}\in[0,+\infty),\text{即}x=2\arctan t\\
&=\frac{\pi}{a}+2\frac{a^2-1}{a}\int_0^{+\infty}\frac{\diff t}{t^2(a+1)^2+(a-1)^2}\\
&=\frac{\pi}{a}+2\frac{a^2-1}{a}\frac{1}{(a-1)^2}\int_0^{+\infty}\frac{\diff t}{1+\lrp{\frac{a+1}{a-1}t}^2}\\
&=\frac{\pi}{a}+\frac{2}{a}\arctan\lrp{\frac{a+1}{a-1}t}\Big|_0^{+\infty}\qquad|a|<1,\frac{a+1}{a-1}\mbox{为负数}\\
&=\frac{\pi}{a}-\frac{2}{a}\cdot\frac{\pi}{2}=0
\end{aligned}\]
故$I(a)=\disp\int 0\diff a=0$\\
另外,
\begin{itemize}
	\itemsep -3pt
	\item 对于一般的情况还需解出常数$C$
	\item 本题还可联系例\ref{eg:sum_fun_analysis}
	\item 对于$a\geq 0$的其他情况,可见\footnote{\url{https://math.stackexchange.com/questions/650513/computing-int-0-pi-ln-left1-2a-cos-xa2-right-dx}}
\end{itemize}
\end{analysis}
\begin{theorem}
设函数$f(x,y)$在$[a,b]\times[c,d]$上连续,
\begin{enumerate}
	\item $c(x),d(x)$都在$[a,b]$上连续,并且当$x\in[a,b]$时,有$c\leq c(x),d(x)\leq d$,则
	\[F(x)=\int_{c(x)}^{d(x)}f(x,y)\diff y\]
	在$[a,b]$连续
	\item $f_x$也在$[a,b]\times[c,d]$连续,又$c'(x)$和$d'(x)$在$[a,b]$存在,则变上限积分$F(x)$可导,且
	\[F'(x)=\int_{c(x)}^{d(x)}f_x(x,y)\diff y+f(x,d(x))d'(x)-f(x,c(x))c'(x)\]
\end{enumerate}
\end{theorem}
\par 关于含参变量广义积分请见\ref{sec:sub:parameter_abnormal_integral}节.


\subsection{欧拉积分}
\subsubsection{第二欧拉积分}
\[\Gamma(\alpha)=\intab{0}{+\infty}{x^{\alpha-1}\ee^{-x}},\,\alpha>0\]
分析性质:
\begin{enumerate}
	\item 在定义域$\alpha>0$内连续且有任意阶连续导数
	\[\Gamma^{(n)}(\alpha)=\intab{0}{+\infty}{x^{\alpha-1}(\ln x)^n\ee^{-x}}\]
	\item 递推公式
	\[\Gamma(\alpha+1)=\alpha\Gamma(\alpha)\]
	特别地,当$\alpha\in\zz^+$时
	\[\Gamma(\alpha+1)=\alpha!\]
	对于任意$\alpha>0$,总存在非负整数$n$使得$\alpha=n+p,\,p\in(0,1]$,则
	\[\Gamma(\alpha)=\Gamma(n+p=(n+p-1)\cdots(1+p)p\Gamma(p)\]
	\item $\Gamma(1)=1$;由概率积分可得$\Gamma(\frac{1}{2})=\sqrt{\pi}$
\end{enumerate}
概率分布中会用到的其他公式
\begin{itemize}
\item 上不完全(upper incomplete)伽马函数
\[ \Gamma(s,x) = \int_x^{\infty} t^{s-1}\ee^{-t}\diff t\]
\item 下不完全(lower incomplete)伽马函数
\[ \gamma(s,x) = \int_{0}^{x}t^{s-1}\ee^{-t}\diff t\]
\item 递推关系
\[\begin{aligned}
\Gamma(s,x)&=(s-1)\Gamma(s-1,x) + x^{s-1} \ee^{-x}\\
\gamma(s,x)&=(s-1)\gamma(s-1,x) - x^{s-1} \ee^{-x}\\
\Gamma(s) &= \Gamma(s,0)\\
\gamma (s,x)&+\Gamma (s,x)=\Gamma (s)
\end{aligned}\]
\end{itemize}

\subsubsection{第一欧拉积分}
\[B(a,b)=\intab{0}{1}{x^{a-1}(1-x)^{b-1}},\,a,b>0\]
分析性质:
\begin{enumerate}
	\item 对称性:$B(a,b)=B(b,a)$
	\item 递推公式
	\[\begin{aligned}
	B(a,b)&=\frac{b-1}{a+b-1}B(a,b-1),\,a>0,b>1\\
	&=\frac{a-1}{a+b-1}B(a-1,b),\,a>1,b>0
	\end{aligned}\]
	\item 狄利克雷公式
	\[B(a,b)=\frac{\Gamma(a)\Gamma(b)}{\Gamma(a+b)},\,a>0,b>0\]
\end{enumerate}