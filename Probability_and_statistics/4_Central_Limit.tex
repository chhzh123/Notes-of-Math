% !TEX root = main.tex

\section{大数定律}
\subsection{大数定律}
\begin{theorem}[切比雪夫(Chebyshev)不等式]
设随机变量$X$的数学期望$\E{X}=\mu$,方差$\Var{X}=\sigma^2$
\[\pr{|X-\mu|\geq\eps}\leq\dfrac{\sigma^2}{\eps^2}\]
或
\[\pr{|X-\mu|<\eps}\geq\frac{\sigma^2}{\eps^2}\]
\end{theorem}
\begin{theorem}[弱大数定律(辛钦)]
若$X_1,X_2,\ldots$为独立随机变量且同等分布(iid),$\E{X_i}=\mu,\Var{X_i}=\sigma^2$,则
\[\lim_{n\to\infty}\pr{\lra{\dfrac{1}{n}\sum_{i=1}^nX_i-\mu}\geq\eps}=0,\,\forall\eps>0\]
或
\[\lim_{n\to\infty}\pr{\lra{\dfrac{1}{n}\sum_{i=1}^nX_i-\mu}<\eps}=1,\,\forall\eps>0\]
可记成$\bar{X}\xrightarrow{P}\mu$
\end{theorem}
\begin{theorem}[实际推断原理]
概率很小的事件在一次试验中实际上几乎是不发生的
\end{theorem}
\begin{theorem}[强大数定律]
若$X_1,X_2,\ldots$为独立随机变量且同等分布,$\E{X_i}=\mu,\Var{X_i}=\sigma^2$,则
\[\pr{\lim_{n\to\infty}\dfrac{1}{n}\sum_{i=1}^nX_i=\mu}=1\]
\end{theorem}

\subsection{中心极限定理}
\begin{definition}[标准化变量]
若随机变量$X$的均值为$\mu$,方差为$\sigma^2$,则$X$的标准化变量为
\[Z=\dfrac{X-\mu}{\sigma}\]
有$\E{Z}=0$,$\Var{Z}=1$
\end{definition}
\begin{theorem}[独立同分布的中心极限定理]
若$X_1,X_2,\ldots$为独立随机变量且同等分布,且$\E{X_k}=\mu$,$\Var{X_k}=\sigma^2>0$,则随机变量之和的标准化变量
\[Y_n=\frac{\sum_{k=1}^nX_k-\E{\sum_{k=1}^nX_k}}{\sqrt{\Var{\sum_{k=1}^nX_k}}}=\frac{\sum_{k=1}^nX_k-n\mu}{\sqrt{n\sigma^2}}\]
的分布函数$F_n(x)$对任意$x$满足
\[\lim_{n\to\infty}F_n(x)=\lim_{n\to\infty}\pr{Y_n\leq x}=\intabu{-\infty}{x}{\frac{1}{2\pi}\ee^{-t^2/2}}{t}=\Phi(x)\]
也即,近似地
\[\frac{\bar{X}-\mu}{\sigma/\sqrt{n}}\thicksim N(0,1)\qquad\mbox{或}\qquad\bar{X}\thicksim N(\mu,\sigma^2/n)\]
\end{theorem}
努力将原变量转化为标准化变量形式,以使用标准正态分布解题
\begin{corollary2}[棣莫弗-拉普拉斯(De Moivre-Laplace)定理]
设随机变量$\eta_1,\eta_2,\ldots$服从参数为$n,p(0<p<1)$的二项分布,则对于任意$x$,有
\[\lim_{n\to\infty}\pr{\frac{\eta-np}{\sqrt{np(1-p)}}\leq x}=\Phi(x)\]
注:化为$n$个0-1分布可证
\end{corollary2}
\par 涉及到\textbf{数目}的,可以采用上述推论计算.
\begin{theorem}[李雅普诺夫(Lyapunov)定理]
若$X_1,X_2,\ldots$为独立随机变量且同等分布,且$\E{X_k}=\mu$,$\Var{X_k}=\sigma^2>0$,记$B_n^2=\sum_{k=1}^n\sigma_k^2$
若存在$\delta>0$,使得
\[\lim_{n\to\infty}\frac{1}{B_n^{2+\delta}}\sum_{k=1}^n\E{|X_k-\mu|^{2+\delta}}=0\]
则随机变量之和的标准化变量
\[Z_n=\frac{\sum_{k=1}^nX_k-\E{\sum_{k=1}^nX_k}}{\sqrt{\Var{\sum_{k=1}^nX_k}}}=\frac{\sum_{k=1}^nX_k-\sum_{k=1}^n\mu_k}{B_n}\]
\end{theorem}
\begin{example}
一食品店有三种蛋糕出售,售出一只蛋糕的价格是一个随机变量,取$1$元、$1.2$元、$1.5$元各个值的概率分别为$0.3$、$0.2$、$0.5$,若售出$300$只蛋糕
\begin{enumerate}
	\itemsep -3pt
	\item 求收入至少$400$元的概率
	\item 求售出价格为$1.2$元的蛋糕多于$60$只的概率
\end{enumerate}
\end{example}
\begin{analysis}
\begin{enumerate}
	\item 设$X_k,k=1,2,\ldots,300$为第$k$个蛋糕的售价
	\begin{center}
	\begin{tabular}{c|ccc}
	$X_k$ & $1$ & $1.2$ & $1.5$\\\hline
	$p_k$ & $0.3$ & $0.2$ & $0.5$
	\end{tabular}
	\end{center}
	\[\begin{aligned}
	\E{X_k}&=1.29 \qquad \Var{X_k}=0.0489\\
	\pr{\sum_{k=1}^{300}X_k\geq 400}&=1-\pr{\sum_{k=1}^{300}X_k<400}\\
	&=1-\pr{\frac{300\bar{X}-1.29\cdot 300}{\sqrt{300}{0.0489}}<\frac{400-1.29\cdot 300}{\sqrt{300}{0.0489}}}\\
	&=1-\Phi(3.39)=0.0003
	\end{aligned}\]
	\item 记$Y$为售价为$1.2$元的蛋糕数目,则$Y\thicksim b(300,0.2)$,由Laplace定理,
	\[\begin{aligned}
	\pr{Y>60}&=1-\pr{Y\leq 60}\\
	&=1-\pr{\frac{Y-300\cdot 0.2}{\sqrt{300\cdot 0.2\cdot 0.8}}\leq\frac{60-300\cdot 0.2}{\sqrt{300\cdot 0.2\cdot 0.8}}}\\
	&=1-\Phi(0)=0.5
	\end{aligned}\]
\end{enumerate}
\end{analysis}