% !TEX root = main.tex

\section{多维随机变量及其分布}
\subsection{边缘分布}
\begin{definition}
二维随机变量$(X,Y)$的分布函数/联合(joint)分布函数定义如下
\[F(x,y)=\pr{X\leq x,Y\leq y}=\disp\sum_{x_i\leq x}\sum_{y_i\leq y}p_{ij}=\intint{y}{x}{f(u,v)}{u}{v}\]
其中,$f(x,y)$为$X,Y$的联合密度函数
\end{definition}
进而,对于离散型随机变量变量有,
\[\pr{x_1<X\leq x_2,y_1<Y\leq y_2}=F(x_2,y_2)-F(x_2,y_1)+F(x_1,y_1)-F(x_1,y_2)\]
离散型随机变量
\[\pr{(X,Y)\in G}=\iint_G f(x,y)\,\diff x\,\diff y\]
\par 若$f(x,y)$在$(x,y)$连续,则$\pddxy{F(x,y)}{x}{y}=f(x,y)$

\begin{definition}[边缘(marginal)分布]
\[\begin{aligned}
F_X(x)&=\pr{X\leq x,Y<\infty}=\lim_{y\to\infty}F(x,y)=F(x,\infty)\\
&=\intab{-\infty}{\infty}{\left[\intabu{-\infty}{\infty}{f(x,y)}{y}\right]}\\
&=\intab{-\infty}{\infty}{f_X(x)}
\end{aligned}\]
\end{definition}
注意积分区间不一定是从负无穷到正无穷,而是概率有$(0,1)$之间值的区域

\begin{definition}[条件概率密度与分布函数]
\[\begin{aligned}
f_{X\mid Y}(x\mid y)&=\dfrac{f(x,y)}{f_Y(y)}\\
F_{X\mid Y}(x\mid y)&=\pr{X\leq x\mid Y=y}=\intab{-\infty}{x}{\dfrac{f(x,y)}{f_Y(y)}}
\end{aligned}\]
\end{definition}
\begin{definition}[相互独立]
\[\begin{aligned}
F(x,y)=F_X(x)F_Y(y)\\
f(x,y)=f_X(x)f_Y(y)
\end{aligned}\]
\end{definition}

\subsection{随机变量的函数的分布}
\[f_{X+Y}(z)=\intabu{-\infty}{\infty}{f(z-y,y)}{y}=\intabu{-\infty}{\infty}{f(x,z-x)}{x}\]
\[f_{Y/X}(z)=\intabu{-\infty}{\infty}{|x|f(x,xz)}{x}=f_{XY}(z)=\intinf{\dfrac{1}{|x|}f(x,\dfrac{z}{x})}\]
\[F_{\max}(z)=F_{X_1}(z)F_{X_2}(z)\cdots F_{X_n}(z)\]
\[F_{\min}(z)=1-[1-F_{X_1}(z)][1-F_{X_2}(z)]\cdots [1-F_{X_n}(z)]\]

\subsection{数字特征}
\begin{definition}[协方差]
\[\Cov{X,Y}=\E{[X-\E{X}][Y-\E{Y}]}=\E{XY}-\E{X}\E{Y}\]
\end{definition}
协方差的性质:
\[\begin{aligned}
\Cov{aX,bY}&=ab\Cov{X,Y}\\
\Cov{X_1+X_2,Y}&=\Cov{X_1,Y}+\Cov{X_2,Y}
\end{aligned}\]
\begin{definition}[相关系数]
\[\rho_{XY}=\frac{\Cov{X,Y}}{\sqrt{\Var{X}}{\Var{Y}}}\]
$\rho_{XY}\leq 1$,取等的充要条件为$\exists a,b$使得$\pr{Y=a+bX}=1$
\end{definition}
\begin{definition}[矩]
设$X,Y$为随机变量,若$\E{X^k}$存在,则称它为$X$的$k$阶(原点)矩,称$\E{[X-\E{X}]^k}$为$k$阶中心矩,称$\E{X^kY^l}$为$k+l$阶混合矩,称$\E{[X-\E{X}]^k[Y-\E{Y}]^l}$为$k+l$阶混合中心矩
\end{definition}
\begin{definition}[协方差矩阵]
设$n$维随机变量$(X_1,X_2,\ldots,X_n)$的二阶混合中心距
\[c_{ij}=\Cov{X_i,X_j}=\E{[X_i-\E{X_i}][X_j-\E{X_j}]}\]
都存在,则协方差矩阵
\[\mC=\begin{bmatrix}
c_{11}&c_{12}&\cdots&c_{1n}\\
c_{21}&c_{22}&\cdots&c_{2n}\\
\vdots&\vdots&\ddots&\vdots\\
c_{n1}&c_{n2}&\cdots&c_{nn}
\end{bmatrix}\]
又$c_{ij}=c_{ji}$,故上述矩阵是个对称矩阵
\end{definition}
\begin{theorem}
若$X_1,\ldots,X_n$独立,则
\[\begin{aligned}
\E{\prod_{i=1}^nX_i}&=\prod_{i=1}^n\E{X_i}\\
\Var{\prod_{i=1}^nX_i}&=\sum_{i=1}^n\Var{X_i}\\
\Cov{X_i,X_j}&=0,\,i\ne j
\end{aligned}\]
\end{theorem}
$Y=g(X)$的随机变量分布一般通过求分布函数后求导得到其概率密度函数