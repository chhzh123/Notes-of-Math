% !TEX root = main.tex

\section{随机变量及其分布}
\subsection{基本概念}
\begin{definition}
对于离散随机变量$X$,其概率质量函数(PMF)为$f_X(k)=\pr{X=k}$,分布函数为$\disp F_X(x)=\pr{X\leq x}=\sum_{k\leq x}f_X(k)$
\end{definition}
\begin{definition}
对于连续随机变量$X$,其累积密度函数(CDF)为$F_X(x)=\pr{X\leq x}=\intabu{-\infty}{x}{f_X(z)}{z}$
$\displaystyle \intab{-\infty}{x}{f(t)}$,其中$f_X(x)$为$X$的概率密度函数(PDF),也即$f_X(x)=\dfrac{\diff F_X(x)}{\diff x}$.
一定要注意,$f_X(x)\ne\pr{X=x}$!
\end{definition}
注意\textcolor{red}{积分区间}!注意要写\textcolor{red}{变量范围}!
\begin{definition}[期望]
设$Y$是随机变量$X$的\textbf{连续}函数$Y=g(X)$
\[\begin{aligned}
\E{X}&=\sum_x xf_X(x) \quad& \E{g(X)}&=\sum_x g(x)f_X(x)\\
\E{X}&=\intab{-\infty}{+\infty}{xf_X(x)} \quad& \E{g(X)}&=\intab{-\infty}{+\infty}{g(x)f_X(x)}
\end{aligned}\]
\end{definition}
\par 期望具有线性性,即
\[\E{X+Y}=\E{X}+\E{Y},\,\E{cX}=c\E{X}\]
\par 若$X,Y$相互独立,则
\[\E{XY}=\E{X}\E{Y}\]
但反过来不能推相互独立
\begin{definition}[方差]
\[\begin{aligned}
\Var{X}=\mathrm{Var}(X)&=\sigma^2=\E{(X-\E{X})^2}=\E{X^2}-\E{X}^2\geq 0\\
&=\sum_{k=1}^\infty[x_k-\E{X}]^2p_k\\
&=\intab{-\infty}{\infty}{[X-\E{X}]^2f(x)}
\end{aligned}\]
标准差或均方差则是$\sigma$
\end{definition}
一般通过求$\E{X}$和$\E{X^2}$来求方差\par
由方差定义和期望的线性性有
\[\Var{aX+b}=a^2\Var{X}\]
注意方差并不是线性的
\[\begin{aligned}
\Var{X+Y}&=\E{(X+Y)^2}-(\E{X}+\E{Y})^2\\
&=\E{X^2}-\E{X}^2+\E{Y^2}-\E{Y}^2+2(\E{XY}-\E{X}\E{Y})\\
&=\Var{X}+\Var{Y}+2\Cov{X,Y}
\end{aligned}\]
\begin{definition}[上$\alpha$分位点]
\[\pr{X>z_\alpha}=\alpha,\,\alpha\in(0,1)\]
\end{definition}

\subsection{随机变量的函数的分布}
\begin{theorem}
若$X$为连续型随机变量,$g$为单调递增函数(反函数存在且单调递增),且$Y=g(X)$,那么
\[f_Y(y)=f_X(g^{-1}(y))(g^{-1})'(y)\]
\end{theorem}
特别地,对于$Y=X^2$
\[f_Y(y)=\frac{1}{2\sqrt{y}}[f_X(\sqrt{y})+f_X(-\sqrt{y})],\,y>0\]
一般情况则先求$Y$的概率分布函数$F_Y(y)$,然后对$F_Y(y)$求导

\subsection{常见的离散分布}
\begin{enumerate}
	\item 伯努利/两点/0-1分布 \textbf{Bernoulli(p)}(二项分布的特殊情形)
		\[\begin{aligned}
		f_{X}(k)&=\begin{cases}
		1-p & k=0\\
		p & k=1\\
		\end{cases}\\
		\mathbb{E}(X)&=p\\
		\Var{X}&=p(1-p)
		\end{aligned}\]
	\item 二项分布 \textbf{Binomial(n,p)}
		\[\begin{aligned}
		f_{X}(k)& =\binom{n}{k}p^k(1-p)^{n-k},\,0\leq k\leq n\\
		\mathbb{E}(X)& =n\cdot p\\
		\Var{X}&=np(1-p)
		\end{aligned}\]
		e.g. 扔$n$次硬币扔到$k$次正面(做实验$n$次,记录成功的次数)
	\item 几何分布 \textbf{Geometric(p)}(负二项分布的特殊情形)
		\[\begin{aligned}
		f_{X}(k)& =(1-p)^k\cdot p,k\geq 0\\
		\mathbb{E}(X)& =\frac{1-p}{p}
		\end{aligned}\]
		e.g. 扔$k$次反面直至扔到正面(做实验直到你成功,记录失败的次数)
	\item 负二项分布 \textbf{NegetiveBinomial(r,p)}
		\[\begin{aligned}
		f_{X}(k)& =\binom{k+r-1}{r-1}p^r(1-p)^k,\,k\geq 0\\
		\mathbb{E}(X)& =r\cdot\frac{1-p}{p}
		\end{aligned}\]
		\[\begin{aligned}
		\E{X}&=\sum_{k=0}^\infty\binom{k+r-1}{r-1}p^r(1-p)^k\\
		&=\sum_{k=0}^\infty r\lrp{\binom{k+r}{r}-\binom{k+r-1}{r-1}}p^r(1-p)^k\qquad k\binom{n}{k}=n\binom{n-1}{k-1}\mbox{的变形}\\
		&=rp^r\lrp{\sum_{k=0}^\infty\binom{k+r}{r}(1-p)^k-\sum_{k=0}^\infty\binom{k+r-1}{r-1}(1-p)^k}\\
		&=rp^r\lrp{\sum_{k=0}^\infty\binom{\textcolor{red}{k+r}}{k}(1-p)^k-\sum_{k=0}^\infty\binom{k+r-1}{k}(1-p)^k}\\
		&=rp^r\lrp{\sum_{k=0}^\infty(-1)^k\binom{\textcolor{red}{-r-1}}{k}(1-p)^k-\sum_{k=0}^\infty(-1)^k\binom{-r}{k}(1-p)^k}\\
		&\quad\mbox{这步是关键,将变化的$(k+r)$转成$(-r-1)$,使得可以正常使用二项式定理}\\
		&=rp^r\lrp{\sum_{k=0}^\infty\binom{-r-1}{k}1^{-r-1-k}(p-1)^k-\sum_{k=0}^\infty\binom{-r}{k}1^{-r-k}(p-1)^k}\\
		&=rp^r\lrp{(1+(p-1))^{-r-1}-(1+(p-1))^{-r}}\qquad\mbox{牛顿二项式}\\
		&=r(p^{-1}-1)\\
		&=r\frac{1-p}{p}
		\end{aligned}\]
		补充证明:
		\[\begin{aligned}
		\binom{k+r}{k}&=\frac{(k+r)(k+r-1)\cdots(r+1)}{k(k-1)\cdots 1}\\
		&=(-1)^k\frac{(-k-r)(-k-r-1)\cdots(-r-1)}{k(k-1)\cdots 1}\\
		&=(-1)^k\frac{(-r-1)(-r-2)\cdots(-r-1-k+1)}{k(k-1)\cdots 1}\qquad\mbox{把分子各项逆过来}\\
		&=(-1)^k\binom{-r-1}{k}
		\end{aligned}\]
		e.g. 扔$k$次反面直到有$r$个正面(做实验直到你获得$r$次成功,记录失败次数)
	\item 超几何分布 \textbf{HyperGeometric(N,n,M)}
		\[\begin{aligned}
		f_{X}(k)& =\frac{\binom{M}{k}\cdot \binom{N-M}{n-k}}{\binom{N}{n}}\\
		\mathbb{E}(X)& =n\frac{M}{N}
		\end{aligned}\]
		e.g. $M$个产品中有$N$个次品,检查$n$次得到$k$个次品
	\item 泊松分布 \textbf{Poisson($\lambda$),$\lambda>0$}
		\[\begin{aligned}
		f_{X}(k)& =\frac{\lambda^k}{k!}\ee^{-\lambda},\,\lambda>0,k\geq0\\
		\mathbb{E}(X)& =\lambda
		\end{aligned}\]
		$X\thicksim B(n,p)$,若$p=\dfrac{\lambda}{n}$,且$n$非常大,则
		\[\begin{aligned}
		\pr{X=k} &= \binom{n}{k}\left(\frac{\lambda}{n}\right)^k\left(1-\frac{\lambda}{n}\right)^{n-k}\\
		&= \frac{n(n-1)\cdots(n-k+1)}{k!}\frac{\lambda^k}{n^k}\left(1+\frac{-\lambda}{n}\right)^{n-k}\\
		&\approx \frac{\lambda^k}{k!}\ee^{-\lambda}
		\end{aligned}\]
		一般$n\geq 20,p\leq 0.05$时,即可用近似
		\[\begin{aligned}
		\E{X}&=\sum_{k=0}^\infty k\frac{\lambda^k}{k!}\ee^{-\lambda}\\
		&=\lambda\ee^{-\lambda}\sum_{k=1}^\infty\frac{\lambda^{k-1}}{(k-1)!}\\
		&=\lambda\ee^{-\lambda}\ee^{\lambda}=\lambda
		\end{aligned}\]
		注意泊松分布具有无记忆性(memoryless),即
		\[\prc{X\geq a}{X\geq b}=\pr{X\geq a-b}\]
\end{enumerate}

\subsection{常见的连续分布}
\begin{enumerate}
	\item 均匀分布 \textbf{Uniform(a,b),$a<b$}
	\[\begin{aligned}
	f_X(x)&=\dfrac{1}{b-a},\,a\leq x\leq b\\
	\E{X}&=\dfrac{a+b}{2}\\
	\Var{X}&=\dfrac{(b-a)^2}{12}
	\end{aligned}\]
	\item 指数分布 \textbf{Exponential($\theta$),$\theta>0$}
	\[\begin{aligned}
	f_X(x)&=\frac{1}{\theta}\ee^{-\frac{x}{\theta}},\,x>0\\
	\E{X}&=\theta=\dfrac{1}{\lambda}
	\end{aligned}\]
	与泊松分布类似,同样具有无记忆性
	\item 正态分布 \textbf{Normal($\mu,\sigma^2$)}
	\[\begin{aligned}
	f_X(x)&=\dfrac{1}{\sqrt{2\pi}\sigma}\ee^\frac{-(x-\mu)^2}{2\sigma^2}\\
	\E{X}&=\mu
	\end{aligned}\]
	标准正态分布$N(0,1)$
	\[\begin{aligned}
	\varphi(x)&=\frac{1}{\sqrt{2\pi}}\ee^{-x^2/2}\\
	\Phi(x)&=\intabu{-\infty}{x}{\varphi(t)}{t}\\
	\Phi(-x)=1-\Phi(x)
	\end{aligned}\]
	算平方
	\[\begin{aligned}
	I^2&=\left(\intinf{\ee^\frac{-x^2}{2}}\right)^2\\
	&=\intinf{\ee^\frac{-x^2}{2}}\intinf[y]{\ee^\frac{-y^2}{2}}\\
	&=\intinf[y]{\intinf{\ee^\frac{-(x^2+y^2)}{2}}}\\
	&=\intabu{0}{2\pi}{\intabu{0}{+\infty}{r\ee^\frac{-r^2}{2}}{r}}{\theta}\\
	&\quad\mbox{重积分极坐标变量代换}\diff S=r\diff r\diff\theta\\
	&=-\intabu{0}{2\pi}{\intabu{0}{+\infty}{}{\ee^\frac{-r^2}{2}}}{\theta}\\
	&=-\intabu{0}{2\pi}{(0-1)}{\theta}\\
	&=\intabu{0}{2\pi}{1}{\theta}=2\pi
	\end{aligned}\]
	也即\textcolor{red}{概率积分$I=\sqrt{2\pi}$}\\
	若$X\thicksim N(\mu,\sigma^2)$,则$Y=aX+b\thicksim N(a\mu+b,(a\sigma)^2)$,特别地,
	\[Z=\dfrac{X-\mu}{\sigma}\thicksim N(0,1)\]
	若$X_i\thicksim N(\mu_i,\sigma_i^2)$相互独立,则它们的和
	\[Z=X_1+X_2+\cdots+X_n\thicksim N(\mu_1+\mu_2+\cdots+\mu_n,\sigma_1^2+\sigma_2^2+\cdots+\sigma_n^2)\]
	若$X\thicksim N(0,1)$,且$Y=X^2$,则
	\[f_Y(y)=\frac{1}{\sqrt{2\pi}}y^{-1/2}\ee^{-y/2}\implies Y\thicksim\Gamma(\frac{1}{2},2)\]
	\item 伽马分布 \textbf{Gamma$(\mathbf{\alpha},\mathbf{\beta})\equiv\Gamma (\alpha ,\beta )\equiv\Gamma (k,\theta)\,,k=\alpha,\theta=1/\beta$}
	\[\begin{aligned}
	f_X(x;\alpha ,\beta )&=\frac {\beta ^{\alpha }}{\Gamma (\alpha )}x^{\alpha -1}\ee^{-\beta x},\,x,\alpha ,\beta >0\\
	f_X(x;k,\theta)&=\frac {1}{\Gamma (k)\theta ^{k}}x^{k-1}\ee^{-{\frac {x}{\theta }}},\,x,k,\theta>0\\
	F(x;k,\theta )&=\int _{0}^{x}f(u;k,\theta )\diff u={\frac {\gamma \left(k,{\frac {x}{\theta }}\right)}{\Gamma (k)}}\\
	\E{X}&={\frac {\alpha }{\beta }}=k\theta\\
	\Var{X}&=\frac{\alpha}{\beta^2}=k\theta^2
	\end{aligned}\]	
	若$X_i\thicksim \Gamma(k_i,\theta)$相互独立,则它们的和
	\[Z=\sum_{i=1}^N X_i \thicksim\Gamma  \left( \sum_{i=1}^n k_i, \theta \right)\]
	注:通过$B(\alpha,\beta)=\frac{\Gamma(\alpha)\Gamma(\beta)}{\Gamma(\alpha+\beta)}$证明
\end{enumerate}