\usepackage{geometry}
\usepackage{setspace}
\usepackage{amsthm,amsmath,amsfonts,amssymb}%\because\therefore
\usepackage{mathtools}%underset
\usepackage{tikz,tikz-cd,pgfplots}
\usepackage{gensymb}
\usepackage{cancel}
\usepackage{extarrows}
\usepackage{enumerate}
\usepackage{url}
\usepackage[unicode=true,%本行非常重要 支持中文目录hyperref CJKbookmarks对二级目录没用
	colorlinks,
	linkcolor=black,
	anchorcolor=black,
	citecolor=black,
	CJKbookmarks=false]{hyperref}
\usepackage{datetime}
\newdateformat{builddatemonth}{\THEYEAR.\twodigit{\THEMONTH}}
\newdateformat{builddate}{\THEYEAR\twodigit{\THEMONTH}\twodigit{\THEDAY}}

\newtheorem{theorem}{定理}
\newtheorem{definition}{定义}
\newtheorem{proposition}{命题}
\newtheorem{example}{例}%*去除编号
\newtheorem*{analysis}{分析}
\newtheorem{corollary}{推论}
\newtheorem*{corollary2}{推论}
\newtheorem{exercise}{练习}
\geometry{top=20mm,bottom=20mm,left=20mm,right=20mm}
\pagestyle{plain}%删除页眉

\def\zz{\mathbb{Z}}
\def\rr{\mathbb{R}}
\def\diff{\,\mathrm{d}}
\def\sgn{\mathrm{sgn}\,}
\def\mmid{\enspace\Big|\enspace}
\def\eps{\varepsilon}
\def\disp{\displaystyle}
\def\delx{\Delta x}
\def\eps{\varepsilon}
\def\ee{\mathrm{e}}
\newcommand{\setenu}[2]{#1,\ldots,#2}
\newcommand{\ddx}[1][]{\dfrac{\diff #1}{\diff x}}
\newcommand{\ddxs}[1][]{\dfrac{\diff^2 #1}{\diff x^2}}
\newcommand{\ssum}[1]{\displaystyle\sum_{n=1}^{\infty}#1}
\newcommand{\ssumz}[1]{\displaystyle\sum_{n=0}^{\infty}#1}
\newcommand{\ssumk}[1]{\displaystyle\sum_{k=1}^{\infty}#1}
\newcommand{\ssumkn}[1]{\displaystyle\sum_{k=1}^{n}#1}
\newcommand{\limtoinf}[1]{\displaystyle\lim_{n\to\infty}#1}
\newcommand{\intab}[4][x]{\displaystyle\int_{#2}^{#3}{#4}\diff #1}
\newcommand{\intinf}[2][x]{\displaystyle\int_{-\infty}^{+\infty}#2\diff #1}
\newcommand{\inpro}[1]{\langle #1\rangle}
\newcommand{\lrp}[1]{\left(#1\right)}
\newcommand{\lra}[1]{\left|#1\right|}
\newcommand{\nums}[1]{\{#1\}}
\newcommand{\floor}[1]{\lfloor#1\rfloor}
\newcommand{\dnf}[1]{f^{(#1)}}
\newcommand{\ol}[1]{\mathop{\overline{#1}}}
\newcommand{\pr}[1]{\mathbb{P}\left(#1\right)}
\newcommand{\prc}[2]{\mathbb{P}\left(#1\mid #2\right)}
\newcommand{\E}[1]{\mathbb{E}\left(#1\right)}
\newcommand{\Var}[1]{\mathrm{Var}\left(#1\right)}
\newcommand{\Cov}[1]{\mathrm{Cov}\left(#1\right)}