% !TEX root = main.tex

\section{参数估计}
\subsection{样本与抽样}
\begin{definition}[简单随机样本]
在相同的条件下对总体$X$进行$n$次重复的、独立的观察,将$n$次观察结果按照实验的次序记为$X_1,X_2,\ldots,X_n$,则这些变量都\textbf{相互独立}且与$X$有\textbf{相同分布}
\end{definition}

\subsection{抽样分布}
\subsubsection{基本概念}
\begin{definition}[统计量]
样本平均
\[\bar{X}=\frac{1}{n}\sum_{i=1}^nX_i\]
样本方差(期望估计量$\E{S^2}=\sigma^2$)
\[S^2=\frac{1}{\textcolor{red}{n-1}}\sum_{i=1}^n(X_i-\bar{X})^2=\frac{1}{n-1}\lrp{\sum_{i=1}^nX_i^2-n\bar{X}^2}\]
样本$k$阶中心矩
\[A_k=\frac{1}{n}\sum_{i=1}^n(X_i-\bar{X})^k\]
经验分布函数
\[F_n(x)=\frac{1}{n}S(x)\]
其中$S(x)$是$X_1,X_2,\ldots,X_n$中不大于$x$的随机变量的个数
\end{definition}
\begin{theorem}
设总体$X$(不管服从什么分布),均值为$\mu$,方差为$\sigma^2$,$X_1,X_2,\ldots,X_n$为来自$X$的一个样本,$\bar{X}$为样本均值,$S^2$为样本方差,则
\[\E{\bar{X}}=\mu,\,\Var{\bar{X}}=\sigma^2/n=(\sigma/\sqrt{n})^2\]
\end{theorem}
\begin{theorem}
设$X_1,X_2,\ldots,X_n$是来自正态总体$N(\mu,\sigma^2)$的样本,则
\begin{enumerate}
	\item $\dfrac{\bar{X}-\mu}{\sigma/\sqrt{n}}\thicksim N(0,1)$或$\bar{X}\thicksim N(\mu,\sigma^2/n)$(中心极限定理)
	\item $\dfrac{\bar{X}-\mu}{S/\sqrt{n}}\thicksim t(n-1)$
	\item $\dfrac{(n-1)S^2}{\sigma^2}\thicksim \chi^2(n-1)$
	\item $\bar{X}$与$S^2$相互独立
\end{enumerate}
\end{theorem}
\begin{theorem}
对于两个正态总体的样本$X$和$Y$有
\begin{enumerate}
	\item $\displaystyle \frac{S_1^2/S_2^2}{\sigma_1^2/\sigma_2^2}\thicksim F(n_1-1,n_2-1)$
	\item 当$\sigma_1^2=\sigma_2^2=\sigma^2$时
	\[\frac{(\bar{X}-\bar{Y})-(\mu_1-\mu_2)}{S_w\sqrt{\frac{1}{n_1}+\frac{1}{n_2}}}\thicksim t(n_1+n_2-2)\]
	其中,
	\[S_w^2=\frac{(n_1-1)S_1^2+(n_2-1)S_2^2}{n_1+n_2-2},\;S_w=\sqrt{S_w^2}\]
	$S_w$为$\sigma^2$的无偏估计量
\end{enumerate}
\end{theorem}

\subsubsection{常见的抽样分布}
\begin{enumerate}
	\item $\chi^2$分布\\
	设$X_1,X_2,\ldots,X_n$是来自总体$N(0,1)$的样本,则统计量
	\[\chi^2=X_1^2+X_2+\cdots+X_n^2\thicksim\Gamma\lrp{\frac{n}{2},2}\]
	服从自由度为$n$的$\chi^2$分布,记为$\chi^2\thicksim\chi^2(n)$
	\[\begin{aligned}
	f(x)&=\frac{1}{2^{n/2}\Gamma(n/2)}x^{n/2-1}\ee^{-x/2},\,x>0\\
	\E{\chi^2}&=n\\
	\Var{\chi^2}&=2n
	\end{aligned}\]
	$\chi^2$具有可加性,因Gamma分布有可加性
	\[\chi_1^2(n_1)+\chi_2^2(n_2)\thicksim\chi^2(n_1+n_2)\]
	当$n$充分大时,
	\[\chi_\alpha^2\approx\frac{1}{2}(z_\alpha+\sqrt{2n-1})^2\]

	\item $t$分布/Student分布\\
	设$X\thicksim N(0,1)$,$Y\thicksim\chi^2(n)$,且$X,Y$相互独立,则
	\[t=\frac{X}{\sqrt{Y/n}}\]
	服从自由度为$n$的$t$分布,记为$t\thicksim t(n)$
	\[\begin{aligned}
	h(t)=\frac{\Gamma[(n+1)/2]}{\sqrt{\pi n}\,\Gamma(n/2)}\lrp{1+\frac{t^2}{n}}^{-(n+1)/2}\,,t\in(-\infty,+\infty)
	\end{aligned}\]
	上分位点满足$t_{1-\alpha}(n)=-t_\alpha(n)$,当$n$充分大时$t_\alpha(n)\approx z_\alpha$

	\item $F$分布\\
	设$U\thicksim\chi^2(n_1)$,$Y\thicksim\chi^2(n_2)$,且$U,V$相互独立,则
	\[F=\frac{U/n_1}{V/n_2}\]
	服从自由度$(n_1,n_2)$的$F$分布,记为$F\thicksim F(n_1,n_2)$
	\[\begin{aligned}
	\psi(x)=\frac{\Gamma[(n_1+n_2)/2](n_1/n_2)^{n_1/2}x^{(n_1/2)-1}}{\Gamma(n_1/2)\Gamma(n_2/2)[1+(n_1x/n_2)]^{(n_1+n_2)/2}}\,,x>0
	\end{aligned}\]
	若$F\thicksim F(n_1,n_2)$,则$\frac{1}{F}\thicksim F(n_2,n_1)$\\
	上分位点$F_{1-\alpha}(n_1,n_2)=\frac{1}{F_\alpha(n_2,n_1)}$
\end{enumerate}

\subsection{参数估计}
\subsubsection{估计量}
\begin{definition}[估计]
$X_1,X_2,\ldots,X_n$为独立随机变量,从有参数$\mu,\sigma,\theta,\ldots$的分布$f$中得到,参数$\theta$的估计量为$\hat{\theta}$,若
\[\E{\hat{\theta}}=\theta\]
则称$T$是期望(expected)估计/无偏估计量.
若
\[\Var{\hat{\theta}_1}\leq\Var{\hat{\theta}_2}\]
则称$\hat{\theta}_1$较$\hat{\theta}_2$有效.
若
\[\pr{|\hat{\theta}-\theta|>\eps}\to 0,\,n\to\infty\]
则称$\hat{\theta}$为相合(probable)的估计.
\end{definition}
\[\begin{aligned}
\E{S^2}&=\frac{1}{n-1}\E{\sum_{i=1}^n(X_i-\bar{X})^2}\\
&=\frac{1}{n-1}\E{\sum_{i=1^n}X_i^2-n\bar{X}^2}\\
&=\frac{1}{n-1}\sum_{i=1}^n\E{X_i^2}-n\E{\bar{X}^2}\qquad\mbox{由}\sigma^2=\E{X^2}-\mu^2\\
&=\frac{1}{n-1}n(\sigma^2+\mu^2)-n\lrp{\frac{\sigma^2}{n}+\mu^2}\qquad\mbox{由}\Var{\bar{X}}=\frac{\sigma^2}{n}=\E{\bar{X}^2}-\mu^2\\
&=\sigma^2=\Var{X}
\end{aligned}\]

\subsubsection{矩估计}
设$X$为随机变量,概率密度为$f(x;\theta_1,\theta_2,\ldots,\theta_n)$,则$X$的前$k$阶矩
\[\mu_l(\theta_1,\theta_2,\ldots,\theta_k)=\intabu{-\infty}{\infty}{x^lf(x;\theta_1,\theta_2,\ldots,\theta_k)}{x},\,l=1,2\ldots,k\]
$k$个方程组便可解得$k$个估计量$\hat{\theta_l}$

总体均值与方差的矩估计量不因不同总体分布而异
\[\hat{\mu}=\bar{X}\qquad\widehat{\sigma^2}=\frac{1}{n}\sum_{i=1}^n(X_i-\bar{X})^2\]

\subsubsection{最大似然估计法}
\[L(\theta)=L(x_1,x_2,\ldots,x_n;\theta)=\prod_{i=1}^np(x_i;\theta),\,\theta\in\Theta\]
求解对数似然方程组可得估计值
\[\pd{}{\theta_i}\ln L=0,\,i=1,2,\ldots,k\]
但最大似然函数不一定可导,或最大值不一定在驻点取到,则一般地,
\[\hat{\theta}=\arg\max_{\theta\in\Theta} L(x_1,x_2,\ldots,x_n;\theta)\]

最大似然估计具有\textbf{不变性}:若$\theta$的函数$u=u(\theta)$有单值反函数$\theta=\theta(u)$,且$\hat{\theta}$为$\theta$的最大似然估计,则$\hat{u}=u(\hat{\theta})$是$u(\theta)$的最大似然估计
\par 进而标准差的最大似然估计为
\[\hat{\sigma}=\sqrt{\widehat{\sigma^2}}=\sqrt{\frac{1}{n}\sum_{i=1}^n(X_i-\bar{X})^2}\]

常见的最大似然估计
\begin{itemize}
	\item $(a,b)$上的均匀分布:$\hat{a}=\min_i X_i,\hat{b}=\max_i X_i$
	\item 泊松分布:$\hat{\lambda}=\bar{X}$
\end{itemize}

\subsubsection{区间估计}
\begin{definition}[置信区间]
\[\pr{\underline{\theta}<\theta<\bar{\theta}}\geq 1-\alpha\]
\end{definition}
正态总体的区间估计
\begin{center}
\small
    \begin{tabular}{|c|c|c|c|}
    \hline
          待估参数 & 其他参数 & 枢轴量 & 置信区间 \bigstrut\\
    \hline
   $\mu$    & $\sigma^2$已知      & $Z=\dfrac{\bar{X}-\mu}{\sigma/\sqrt{n}}\thicksim N(0,1)$      &  $\lrp{\bar{X}\pm\dfrac{\sigma}{\sqrt{n}}z_{\alpha/2}}$ \bigstrut\\\hline
   $\mu$    &  $\sigma^2$未知     &    $t=\dfrac{\bar{X}-\mu}{S/\sqrt{n}}\thicksim t(n-1)$   & $\lrp{\bar{X}\pm\dfrac{S}{\sqrt{n}}t_{\alpha/2}(n-1)}$ \bigstrut\\\hline
   $\sigma^2$    &  $\mu$未知     &  $\chi^2=\dfrac{(n-1)S^2}{\sigma^2}\thicksim\chi^2(n-1)$     &  $\lrp{\dfrac{(n-1)S^2}{\chi^2_{\alpha/2}(n-1)},\dfrac{(n-1)S^2}{\chi^2_{1-\alpha/2}(n-1)}}$ \bigstrut\\
    \hline
  $\mu_1-\mu_2$     &  $\sigma^2_1,\sigma^2_2$已知     &  $Z=\dfrac{(\bar{X}-\bar{Y})-(\mu_1-\mu_2)}{\sqrt{\sigma_1^2/n_1+\sigma_2^2/n_2}}\thicksim N(0,1)$     & $\lrp{\bar{X}-\bar{Y}\pm z_{\alpha/2}\sqrt{\dfrac{\sigma^2_1}{n_1}+\dfrac{\sigma_2^2}{n_2}}}$ \bigstrut\\\hline
   $\mu_1-\mu_2$    &  $\sigma^2_1=\sigma^2_2=\sigma^2$未知     &  $t=\dfrac{(\bar{X}-\bar{Y})-(\mu_1-\mu_2)}{S_w\sqrt{1/n_1+1/n_2}}\thicksim t(n_1+n_2-2)$     & $\lrp{\bar{X}-\bar{Y}\pm t_{\alpha/2}(n_1+n_2-2)S_w\sqrt{\dfrac{1}{n_1}+\dfrac{1}{n_2}}}$ \bigstrut\\\hline
 $\dfrac{\sigma_1^2}{\sigma_2^2}$       &  $\mu_1,\mu_2$未知     & $F=\dfrac{S_1^2/S_2^2}{\sigma_1^2/\sigma_2^2}\thicksim F(n_1-1,n_2-1)$      & \begin{tabular}{c}$\Big(\dfrac{S_1^2}{S_2^2}\dfrac{1}{F_{\alpha/2}(n_1-1,n_2-1)},$\\
 $\dfrac{S_1^2}{S_2^2}\dfrac{1}{F_{1-\alpha/2}(n_1-1,n_2-1)}\Big)$\end{tabular} \bigstrut\\
    \hline
    \end{tabular}%
\end{center}
* 单侧的估计量$\alpha$不需除以$2$

对于0-1分布的估计
\[\frac{n\bar{X}-np}{\sqrt{np(1-p)}}\thicksim N(0,1)\implies\pr{-z_{\alpha/2}<\frac{n\bar{X}-np}{\sqrt{np(1-p)}}<z_{\alpha/2}}\approx 1-\alpha\]
解出$p$的$1-\alpha$置信区间为
\[\lrp{\frac{1}{2a}(-b\pm \sqrt{b^2-4ac})}\]
其中$a=n+z^2_{\alpha/2},b=-(2n\bar{X}+z^2_{\alpha/2}),c=n\bar{X}^2$

\subsection{可视化}
\begin{enumerate}
	\item 直方图:矩形宽度$\dfrac{f_i}{n}\Big/\Delta$,$\Delta$为组距
	\item 箱线图:最小值$\min$,第一四分位数$Q_1$,中位数$M$,第三四分位数$Q_2$,最大值$\max$
	\[q\text{分位数}\quad x_p=\begin{cases}
	x_{[np]+1} & np\notin\zz\\
	\frac{1}{2}(x_{np}+x_{np+1}) & np\in\zz
	\end{cases}\]
\end{enumerate}