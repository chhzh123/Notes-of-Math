% !TEX root = main.tex

\section{仿射空间}
由于这一章涉及的概念太多,且比较难找到对应的中文术语,故在专业术语后都会标上对应的英文名称.
\subsection{仿射组合与凸组合}
\begin{definition}[仿射组合与凸组合]\rm
$\rn$中的向量$\vv_1,\dots,\vv_p$的线性组合$c_1\vv_1+\cdots+c_p\vv_p$,若满足权重$c_1+\cdots+c_p=1$,则该线性组合称为仿射组合,集合$S$所有点仿射组合的集合称为仿射包(affine hull)或仿射生成(affine span)集,记为$\opaff S$. 在仿射组合的基础上满足权重$c_i\geq 0,i=1,\dots,p$,则该线性组合称为凸组合,集合$S$所有点凸组合的集合称为凸包(convex hull),记为$\opconv S$.
\end{definition}
由定义得知,$S=\{\vv_1\}$时,$\opaff S$和$\opconv S$均为一个点$\vv_1$;$S=\{\vv_1,\vv_2\}$时,$\opaff S$是通过$\vv_1,\vv_2$的直线(定比分点),$\opconv S$则是线段$\overline{\vv_1\vv_2}$;$S=\{\vv_1,\vv_2,\vv_3\}$时,$\opaff S$是通过$\vv_1,\vv_2,\vv_3$的平面,$\opconv S$则是三角形$\triangle\vv_1\vv_2\vv_3$;以此类推.
\begin{definition}[仿射(affine)与凸性(convex)]
对于集合$S$,若$\forall\vu,\vv\in S,\,(1-t)\vu+t\vv\in S,\,\forall t\in\mathbb{R}$,则称$S$是仿射集. 若$\forall\vu,\vv\in S,\overline{\vu\vv}\in S$,则称$S$是凸集.
\end{definition}
\begin{theorem}以下是关于仿射集和凸集的一些定理.
\begin{enumerate}
	\itemsep -3pt
	\item $S$是仿射集当且仅当$S=\opaff S$;$S$是凸集当且仅当$S=\opconv S$
	\item 仿射集的子集是仿射集,凸集的子集是凸集
\end{enumerate}
\end{theorem}
\begin{definition}
\rm$\rn$的一个集合$S$经向量$\vb{p}$的平移(translate)变为$S+\vb{p}=\{\vb{s}+\vb{p}|\vb{s}\in S\}$. $\rn$的一个面(flat)\footnote{\url{https://en.wikipedia.org/wiki/Flat_(geometry)}}是$\rn$子空间的一个平移. 若一个面是另一个的平移,则两个面平行. 一个面的维度是它对应的平行子空间的维度. $S$的维度是包含$S$最小的面的维度. $\rn$中的线是一维的面,$\rn$中的超平面(hyperplane)是$n-1$维的面.\\
注:由于子空间必须包含零点,故经过平移的子空间就不是子空间了,就称为面. 真(proper)面即不包含自己的面.
\end{definition}
\begin{definition}[齐次(homogeneous)坐标]
对于$\vv\in\rn$,其标准齐次坐标为$\tilde{\vv}=\begin{bmatrix}\vv\\1\end{bmatrix}\in\mathbb{R}^{n+1}$
\end{definition}
引入齐次坐标的好处有以下几点.
\begin{enumerate}
	\itemsep -3pt
	\item 合并矩阵加法乘法运算
	\item 引入无穷远点 等同于其他点
\end{enumerate}
\begin{definition}[仿射无关]
若存在$c_1,\dots,c_p$不全为$0$,使得$c_1+\cdots+c_p=0,c_1\vv_1+\cdots+c_p\vv_p=\vb{0}$,则称$\{\vv_1,\dots,\vv_p\}\in\rn$仿射相关,否则称为仿射无关.
\end{definition}
\begin{theorem}$S=\{\vv_1,\dots,\vv_p\}\in\rn,p\geq 2$,以下命题都等价
\begin{enumerate}
	\itemsep -3pt
	\item $S$仿射相关
	\item $S$其中一个点是其他点的仿射组合
	\item $\nums{\vv_2-\vv_1,\ldots,\vv_p-\vv_1}\in\rn$线性相关
	\item $\nums{\tilde{\vv}_1,\ldots,\tilde{\vv}_p}\in\mathbb{R}^{n+1}$线性相关
\end{enumerate}
\end{theorem}
\begin{theorem}[向量唯一表示定理(仿射组合)]
\rm $S=\nums{\vv_1,\ldots,\vv_k}$是$\rn$的仿射无关集,那么$\forall\vb{p}\in\opaff S$有唯一向量表示
\[\vb{p}=c_1\vv_1+\cdots+c_k\vv_k\quad\text{且}\quad c_1+\cdots+c_k=1\]
其中$c_1,\dots,c_p$称为$\vb{p}$的质心(barycentric)坐标或仿射坐标.
\end{theorem}
\begin{analysis}
相当于解$\tilde{\vb{p}}=c_1\tilde{\vv}_1+\cdots+c_k\tilde{\vv}_k$,通过高斯消元初等行变换可得. 广泛应用于图像渐变、透视关系.
\end{analysis}
\begin{theorem}
\begin{enumerate}
	\itemsep -3pt
	\item 一个非空子集$S$是仿射集当且仅当它是一个面
	\item $S$仿射无关,$\vb{p}\in\opaff S$,那么$\vb{p}\in\opconv S$当且仅当$\vb{p}$的质心坐标全部非负
	\item $\opconv S$是所有包含$S$的凸集的交
\end{enumerate}
\end{theorem}
\begin{theorem}[Caratheodory]
$S$是$\rn$的一个非空子集,那么$\opconv S$的每一个点都可以被表示成$S$中小于等于$n+1$个点的凸组合
\end{theorem}


\subsection{超平面}
\begin{definition}[线性泛函(linear functional)]
线性变换$f:\rn\to\mathbb{R}$是一个线性泛函. $\forall d\in\mathbb{R},[f:d]:=\{\vx\in\rn:\,f(\vx)=d\}$. 零泛函是使得$f(\vx)=0,\forall\vx\in\rn$的变换,其他均称为非零.
\end{definition}
\begin{theorem}
$\rn$的子集$H$是一个超平面当且仅当$H=[f:d]$,其中$f$是非零泛函,$d\in\mathbb{R}$. 因此,若$H$是一个超平面,则存在一个非零向量$\vb{n}$和$d\in\mathbb{R}$使得$H=\{\vx:\vb{n}\cdot\vx=d\}$
\end{theorem}
\begin{definition}[拓扑概念]
\rm $\forall\vb{p}\in\rn,\delta>0$,中心为$\vb{p}$,半径为$\delta$的开球(open ball)记为$B(\vb{p},\delta):=\{\vx:\,\|\vx-\vb{p}\|<\delta\}$. 给定集合$S\in\rn$,若$\exists\delta>0\,s.t.\,B(\vb{p},\delta)\subset S$,则称$\vb{p}$是$S$的内点(interior point). 若每一个中心在$\vb{p}$的开球都与$S$和$S$的补相交,则$\vb{p}$是$S$的边界点(boundary point). 若$S$不包含任一边界点,则称$S$是开的(open);若$S$包含所有边界点,则称$S$是闭的(closed);否则既不开也不闭. 若$\exists\delta>0\,s.t.\,S\subset B(\vb{0},\delta)$,则称$S$是有界的(bounded). $S$同时是闭的,又是有界的,则称$S$是紧的(compact).(欧氏空间上有界闭集等于紧集)
\end{definition}
\begin{theorem}
开集的凸包是开的,紧集的凸包是紧的,但闭集的凸包不一定是闭的
\end{theorem}
\begin{definition}
\rm 称超平面$H=[f:d]$将集合$A,B$分隔(seperate),若满足以下其一
\begin{enumerate}
	\itemsep -3pt
	\item $f(A)\leq d$且$f(B)\geq d$
	\item $f(A)\geq d$且$f(B)\leq d$
\end{enumerate}
\end{definition}
\begin{theorem}
$A,B$均为非空凸集,$A$是紧集,$B$是闭集,那么存在一个超平面$H$严格分割$A,B$,当且仅当$A\cap B=\varnothing$
\end{theorem}
\begin{theorem}
$A,B$均为非空紧集,那么那么存在一个超平面$H$严格分割$A,B$,当且仅当$(\opconv A)\cap (\opconv B)=\varnothing$
\end{theorem}

\subsection{多胞形}
\begin{definition}[多胞形(polytopes)]
$\rn$中的多胞形是有限集合的凸包. $\mathbb{R}^2$中,多胞体就是简单的多边形;$\mathbb{R}^3$,则是多面体.
\end{definition}
\begin{definition}
\rm $S$是$\rn$紧致的凸集,$F$是$S$的非空子集,若$F\ne S$,$\exists H=[f:d]\,s.t.\,F=S\cap H$以及$f(S)\leq d$或$f(S)\geq d$,则称$F$为$S$的面(face). $H$称为$S$的支撑超平面(supporting hyperplane). 若$\dim F=k$,则$F$称为$S$的$k$-面. 若$P$是$k$维的多胞体,称$P$为$k$-多胞体. $P$的$0$-面称为顶点(vertex/vertices),$1$-面称为边(edge),$k-1$-面是$S$的facet.
\end{definition}
\begin{definition}[端点]
\rm $S$是凸集,若$\vx,\vy\in S,\vb{p}\in\overline{\vx\vy}$,则$\vb{p}=\vx$或$\vb{p}=\vy$称为$S$的端点(extreme point),$S$的所有端点称为$S$的轮廓(profile).
\end{definition}
\begin{definition}[最小表示]
\rm 若多胞形$P=\opconv\nums{\vv_1,\ldots,\vv_k}$且$\vv_i\notin\opconv\{\vv_j:j\ne i\},\forall i=1,\dots,k$,则称$\nums{\vv_1,\ldots,\vv_k}$是$P$的最小表示(minimal representation).
\end{definition}
\begin{theorem}
$M=\nums{\vv_1,\ldots,\vv_k}$是多胞形$P$的最小表示,则以下说法等价
\begin{enumerate}
	\itemsep -3pt
	\item $\vb{p}\in M$
	\item $\vb{p}$是$P$的顶点
	\item $\vb{p}$是$P$的端点
\end{enumerate}
\end{theorem}
\begin{theorem}
$S$是非空紧致凸集,则$S$是它轮廓($S$的端点)的凸包
\end{theorem}
\begin{theorem}
$f$是一个定义在非空紧致凸集$S$上的线性泛函,则存在$S$的端点$\hat{\vv},\hat{\vb{w}}$使得
\[f(\hat{\vv})=\max_{\vv\in S}f(\vv)\quad f(\hat{\vb{w}})=\min_{\vv\in S}f(\vv)\]
注:线性规划极值点取得依据
\end{theorem}
\begin{definition}[单纯形(simplex)]
单纯形是有限仿射无关向量构成的集合的凸包. $\mathbb{R}^2$中,多胞体就是简单的多边形;$\mathbb{R}^3$,则是多面体.
\end{definition}
\[\begin{aligned}
0\text{-单纯形} S^0 &:\; \text{一个点} \{\vv_1\}\\
\vdots\\
k\text{-单纯形} S^k &:\; \opconv(S^{k-1}\cup\{\vv_{k+1}\})\text{,其中}\vv_{k+1}\notin S^{k-1}
\end{aligned}\]
\begin{definition}[超立方体(hypercube)]
\rm $I_i=\overline{\vb{0}\vb{e}_i}$,向量和(vector sum)\footnote{$A+B=\{\vb{c}:\vb{c}=\vb{a}+\vb{b},\forall\vb{a}\in A,\vb{b}\in B\}$}$C^k=I_1+I_2+\cdots+I_k$称为f$k$-维超立方体
\end{definition}
\begin{theorem}[欧拉(Euler)公式]
记$f_k(P)$为$n$-维多胞形$P$的$k$-维面的数目,则
\[\sum_{k=0}^{n-1}(-1)^kf_k(P)=1+(-1)^{n-1}\]
特别地,当$n=3$时,有$v-e+f=2$,其中$v,e,f$分别为顶点、边、面的数量.
\end{theorem}

\subsection{曲线和表面}
\begin{definition}[贝塞尔(Bézier)曲线]
三阶贝塞尔曲线
\[\vx(t)=\begin{bmatrix}\vb{p}_0&\vb{p}_1&\vb{p}_2&\vb{p}_3\end{bmatrix}\begin{bmatrix}1&-3&3&-1\\0&3&-6&3\\0&0&3&-3\\0&0&0&1\end{bmatrix}\begin{bmatrix}1\\t\\t^2\\t^3\end{bmatrix}=GM_B\vb{u}(t)\]
\rm 其中$G$为四个控制点构成的几何(geometry)矩阵,$M_B$为贝塞尔基底矩阵. 换种形式表示
\[\vx(s)=\vb{u}(s)^TM_B^TG^T=\begin{bmatrix}(1-s)^3&3s(1-s)^2&3s^2(1-s)&s^3\end{bmatrix}\begin{bmatrix}\vb{p}_0\\\vb{p}_1\\\vb{p}_2\\\vb{p}_3\end{bmatrix}\]
\end{definition}
\begin{definition}[贝塞尔表面]
\[GM_B\vb{u}(t)\begin{bmatrix}\vb{p}_{11}&\vb{p}_{12}&\vb{p}_{13}&\vb{p}_{14}\\\vb{p}_{21}&\vb{p}_{22}&\vb{p}_{23}&\vb{p}_{24}\\\vb{p}_{31}&\vb{p}_{32}&\vb{p}_{33}&\vb{p}_{34}\\\vb{p}_{41}&\vb{p}_{42}&\vb{p}_{43}&\vb{p}_{44}\end{bmatrix}\begin{bmatrix}(1-t)^3\\3t(1-t)^2\\3t^2(1-t)\\t^3\end{bmatrix}\]
进而,
\[\vb{x}(s,t)=\vb{u}(s)^TM_B^TGM_B\vb{u}(t),0\leq s,t\leq 1\]
\end{definition}