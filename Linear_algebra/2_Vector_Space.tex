% !TEX root = main.tex

\section{向量空间}
\subsection{基本定义}
\begin{definition}[向量空间/线性空间]
给定一个域$F$\footnote{参见\url{https://en.wikipedia.org/wiki/Field_(mathematics)}},若对一个非空集合$V$满足以下十条公理,则称$V$为$F$上的向量空间:
\begin{enumerate}
	\itemsep -3pt
	\item 定义两种运算 $+$ 和 $\cdot$
		$ \langle V , +, \cdot \rangle$
		加法是$V\times V$到$V$的一个映射,数乘是$F\times V$到$V$的一个映射\\
		注:在说明一个集合构成一个向量空间时,一定要指出域是什么,向量加法是怎么定义的,数乘是怎么定义的,否则一切都只是空中楼阁.
	\item 封闭性(对两种运算封闭)
	\item $V$的基本性质,设$u,v,w\in V$,$c,d$为常数
		\begin{enumerate}
			\itemsep -3pt
			\item 零元 $u+0=u$
			\item 单位元 $1\cdot u=u$
			\item 逆元 $u+(-u)=0$
			\item 加法交换律 $u+v=v+u$ (暂不定义两向量相乘)
			\item 结合律 $(u+v)+w=u+(v+w),(cd)u=c(du)$
			\item 左右分配律 $c(u+v)=cu+cv,(c+d)u=cu+du$
		\end{enumerate}
\end{enumerate}
\end{definition}
要注意,这里的向量空间实际上是很抽象的,$F,V$可以是一些很奇怪的东西.
\par如度为$n$的多项式所构成的集合$\mathbb{P}_n$,有
\[\mathbf{p}(t)=a_0+a_1t+a_2t+\cdots+a_nt^n\]
的形式(这里的$t$虽然可以取任意值,但在$\{1,t,\dots,t^n\}$这组标准基下区别不同多项式的是前面的系数,也即坐标向量. 故考虑其线性性时,不是考虑$\vb{p}(u+v)$,而是考虑$\vb{p}(t)+\vb{q}(t)$),这个集合满足以上所有公理,也被称为向量空间.
\par甚至于不用显式地表示出来的也可以算是向量空间,如定义在$[a,b]$上实值函数$f$的集合,同样是一个向量空间.

\subsection{四个基本子空间}
\begin{definition}[子空间]
称$H$是$V$的子空间,当$\vu,\vv\in H$,有:
\begin{enumerate}
	\itemsep -5pt
	\item $\mathbf{0}\in H$
	\item $\vu+\vv\in H$
	\item $c\vu\in H$
\end{enumerate}
注意:$\mathbb{R}^2$并不是$\mathbb{R}^3$的子空间,向量元素数目都不同,但可以通过补$0$来达到目的.
\end{definition}
\begin{definition}[基本子空间]
$A$为$m\times n$的矩阵,则列空间$\col A$、行空间$\row A$、零空间$\nul A$、左零空间$\nul A^T$称为$A$的基本子空间
\end{definition}
\label{nul_and_col}
我们先来探究$A$的零空间与列空间:
\renewcommand\arraystretch{1.2}
\begin{table}[!htbp]%摆放位置
\begin{center}
\begin{tabular}{|c|c|}
\hline
$\nul A$ & $\col A$ \\ \hline
$\{\mathbf{x}:\mathbf{x}\in\mathbb{R}^n,A\mathbf{x}=\mathbf{0}\}$ & $\{\mathbf{b}:\exists\,\mathbf{x}\in\mathbb{R}^n,A\mathbf{x}=\mathbf{b}\}$\\ \hline
隐性(implicit)定义 & 显性(explicit)定义 \\ \hline
$\mathbb{R}^n$的子空间 & $\mathbb{R}^m$的子空间\\ \hline
考虑有$n$个变量 & 考虑有$m$个方程\\ \hline
$\mathrm{dim}=$方程自由变量个数 & $\mathrm{dim}=$主元列数 \\ \hline
基得解出来才知道 & 基为\textbf{原矩阵}的主元列\\
\hline
\end{tabular}
\end{center}
\end{table}
\renewcommand\arraystretch{1}
\par 同时我们发现:
\begin{enumerate}
	\itemsep -3pt
	\item $\mathrm{Nul}\, A=\{\mathbf{0}\}$($A$的列线性无关)当且仅当$\mathbf{x}\mapsto A\mathbf{x}$是一一映射/单射\\
		也说明$A$的每一列都为主元列($n$个),并且$m\geq n$,无自由变元.
	\item $\mathrm{Col}\, A=\mathbb{R}^m$当且仅当$\mathbf{x}\mapsto A\mathbf{x}$将$\mathbb{R}^n$映上/满射到$\mathbb{R}^m$\\
		也说明$A$的每一行都有主元位置($m$个),并且$m\leq n$,\textbf{可能}存在自由变元.
\end{enumerate}
需要注意:
\begin{enumerate}
	\itemsep -3pt
	\item 当$A$不是方阵时,$\mathrm{Nul}\, A$与$\mathrm{Col}\, A$是两个完全不同的空间;$A$是方阵,则有可能存在非零向量同时属于这两个空间
	\item 基本行变换不会影响列之间的相关关系,但\textbf{会改变}其列空间
	\item 基本行变换\textbf{不会改变}其行空间,阶梯形的\textbf{非零行}形成其行空间的一组基,也是原矩阵行空间的一组基
\begin{analysis}
$B$是由$A$经过行变换得来的矩阵,则$B$的行是$A$的行的线性组合,因此$B$的行的线性组合也一定是$A$的行的线性组合,进而$\row B\subset\row A$;而行操作又是可逆的,相反的操作可以说明$\row A\subset\row B$,因此$\row A=\row B$.\\
若$B$是阶梯形,它的非零行一定线性无关,因为没有一个非零行是它下面非零行的线性组合(定理\ref{linear_relationship}$(c)$的逆否命题),证毕.
\end{analysis}
\end{enumerate}
\begin{theorem}
\[(\row A)^{\perp}=\nul A,(\col A)^{\perp}=\nul A^T\]
\end{theorem}
\begin{analysis}
若$\vx\in\nul A$,由矩阵乘法,$A$每一行与$\vx$都正交,而$A$的行张成$\row A$,因此$\nul A\subset(\row A)^{\perp}$.\\
若$\vx\in(\row A)^\perp$,则$\vx$与$A$的每一行都正交,进而$\vx\in\nul A$,因此$(\row A)^{\perp}\subset\nul A$.\\
故$(\row A)^{\perp}=\nul A$. 最后令$A\gets A^T$代入即可得第二个结论.
\end{analysis}
\par 关于四个基本子空间维度之间的关系,见定理\ref{rank_theo}.

\subsection{线性变换}
\begin{definition}[线性变换\protect\footnote{有些地方说线性映射(linear mapping)是从一个向量空间$V$到另一个向量空间$W$的映射且保持加法运算和数量乘法运算,而线性变换(linear transformation)是线性空间$V$到其自身的线性映射. 但从国外的课本上看,这两者并没有区分得特别清楚.}]
若$T:V\to W$ 满足
\begin{enumerate}
	\itemsep -3pt
	\item $T(\vb{u}+\vb{v})=T(\vb{u})+T(\vb{v})$
	\item $T(c\vb{u})=cT(\vb{u})$
\end{enumerate}
则称$T$为线性变换.\\
注意:这里并不需要零元的定义,即$T(\mathbf{0})=\mathbf{0}\,(*)$,因线性变换保持数乘不变性,这已经可以推出 $(*)$式
\end{definition}
\par 线性变换研究的是两个向量空间,则更加抽象. 如对一个实值连续函数求导,也可以算是一个线性变换.
\begin{theorem}[标准(standard)矩阵]
对于线性变换$T:\rn\to\mathbb{R}^m$,存在唯一矩阵$A$使得
\[T(\vx)=A\vx,\forall\vx\in\rn\]
其中$A$是一个$m\times n$的矩阵,且
\[A=\Big[T(\vb{e}_1)\quad\cdots\quad T(\vb{e}_n)\Big]\]
称$A$为$T$的标准矩阵.
\end{theorem}
\begin{analysis}
$1\degree\;$存在性:
\[\begin{aligned}T(\vx) &= T(x_1\vb{e}_1+\cdots+x_n\vb{e}_n)\\
&= x_1T(\vb{e}_1)+\cdots+x_nT(\vb{e}_n)\\
&= \Big[T(\vb{e}_1)\quad\cdots\quad T(\vb{e}_n)\Big]\begin{bmatrix}x_1\\\vdots\\x_n\end{bmatrix}\\
&= A\vx
\end{aligned}\]
$2\degree\;$唯一性:
设存在另一矩阵$B$使得$T(\vx)=B\vx,\forall\vx\in\rn$,故$A\vx=B\vx,\forall\vx\in\rn$,令$\vx=\vb{e}_i$即得$A=B$
\end{analysis}
这里需要注意几点:
\begin{enumerate}
	\itemsep -3pt
	\item 由矩阵乘向量的线性性知,所有的矩阵变换都是线性变换.
	\item 线性变换着重于映射的性质,而矩阵变换则描述了该映射是怎么实施的.
	\item 标准矩阵的唯一性是在确定了基的情况下才说的,如上面的定理是标准基,非标准基的情况请见定理\ref{linear_trans_tot}.
\end{enumerate}
\begin{definition}[单射与满射]
线性变换$T:V\to W$满足:
\begin{enumerate}
	\itemsep -3pt
	\item $\forall\vb{b}\in W$\textbf{至多}可以在$V$中找到一个原像$\vx$,则称$T$单射(即$T(\vb{u})=T(\vb{v})\implies \vb{u}=\vb{v}$)
	\item $\forall\vb{b}\in W$\textbf{至少}可以在$V$中找到一个原像$\vx$,则称$T$满射
	\item $\forall\vb{b}\in W$在$V$中都\textbf{有且只有}一个原像$\vx$,则称$T$双射/一一映射,称$V$和$W$\textbf{同构},记为$V\cong W$.
\end{enumerate}
\end{definition}
\begin{definition}[核空间与值域空间]
$T(\vx)=\vb{0}$的解为$T$的核空间,$T(\vx),\forall\vx$的所有可能值构成$T$的值域空间,分别对应着矩阵中的零空间和列空间.
\end{definition}
\begin{theorem}%4.3 T31 32
\label{lt_theo}
线性变换$T:V\to W$,$S=\{\vb{v}_1,\dots,\vb{v}_p\}\in V$,$S'=\{T(\vb{v}_1),\dots,T(\vb{v}_p)\}\in W$,$U$是$V$的子空间
\begin{enumerate}
	\itemsep -3pt
	\item \label{l1}若$S$线性相关,则$S'$线性相关;若$S'$线性无关,则$S$线性无关
	\item \label{l2}$T$是单射,若$S'$线性相关,则$S$线性相关;若$S$线性无关,则$S'$线性无关
	\item \label{l3}若$S$为$U$的一组基,则$S'$张成$T(U)$
	\item \label{l4}$V\cong W$,若$S$是$V$的线性无关组(生成集/基),则$S'$是$W$的线性无关组(生成集/基)
	\item $V\cong W\iff\dim V=\dim W$
	\item \label{subspace_range}$T(U)$是$W$的子空间
	\item $\dim T(U)\leq\dim U$,当$T$为单射时取等
	\item \textbf{核是定义域的子空间},\textbf{值域空间是对映域的子空间}
\end{enumerate}
\end{theorem}
\begin{analysis}
\begin{enumerate}
\itemsep-3pt
\item 不妨设$\vv_p=c_1\vv_1+\cdots+c_{p-1}\vv_{p-1}$,则$T(\vv_p)=T(c_1\vv_1+\cdots+c_{p-1}\vv_{p-1})=c_1T(\vv_1)+\cdots+c_{p-1}T(\vv_{p-1})$,说明$S'$也线性相关. 后者为前者的逆否命题,易知成立.
\item 不妨设$T(\vv_p)=c_1T(\vv_1)+\cdots+c_{p-1}T(\vv_{p-1})$,则$c_1T(\vv_1)+\cdots+c_{p-1}T(\vv_{p-1})=T(c_1\vv_1+\cdots+c_{p-1}\vv_{p-1})=T(\vv_p)$. 又因$T$是单射,故$\vv_p=c_1\vv_1+\cdots+c_{p-1}\vv_{p-1}$,$S$线性相关.
\item $\vb{y}\in T(U),\,\exists\,T(\vx)=\vb{y}$,因$\vx\in U$,所以$\vx=c_1\vv_1+\cdots+c_p\vv_p$,因$T$为线性,故$\vb{y}=T(\vx)=c_1T(\vb{v}_1)+\cdots+c_pT(\vb{v}_p)$,即可说明.(注意这里只能说明$T(U)$的基在$S'$内,而不能说明$S'$就是基)
\item 结合\ref{l1},\ref{l2},\ref{l3}即得证.
\item 结合\ref{l4},$S$为$V$的基,$S'$为$W$的基,都含有$p$个向量,故维度相同.
\item \begin{enumerate}
	\itemsep -3pt
	\item $\vb{0}\in U,\,T(\vb{0})=\vb{0}\in T(U)$
	\item $\forall\vv_1,\vv_2\in T(U),\,\exists\,\vu_1,\vu_2\in U\,s.t.\,\vv_1=T(\vu_1),\,\vv_2=T(\vu_2),\\
	\because \vu_1+\vu_2\in U\quad\therefore T(\vu_1+\vu_2)=T(\vu_1)+T(\vu_2)=\vv_1+\vv_2 \in T(U)$
	\item $\forall\vv\in T(U),\,\exists\,\vu\in U\,s.t.\,\vv=T(\vu),\\
	\because c\vu\in U\quad\therefore T(c\vu)=cT(\vu)=c\vv\in T(U)$
\end{enumerate}
\item 由定理\ref{basis_theo}$(d)$知$\dim T(U)\leq\dim U$. 若$S$为$U$的一组基,则$S'$张成$T(U)$,又本题的题$2$证明了$S'$线性无关,故$S'$为$T(U)$的一组基,故$\dim T(U)=p=\dim U$
\item 记$T$的核为$\ker T$,显然$\vb{0}\in\ker T$. 若$\vu,\vv\in \ker T$,即$T(\vu)=\vb{0},T(\vv)=\vb{0}$,则$T(\vu+\vv)=T(\vu)+T(\vv)=\vb{0}\implies \vu+\vv\in \ker T,T(c\vu)=cT(\vu)=\vb{0}\implies c\vu\in\ker T$,故核是定义域的子空间.\\
上面\ref{subspace_range}中令$U\gets V$,即得值域空间是对映域的子空间.
\end{enumerate}
\end{analysis}

\subsection{基、维度与秩}
\begin{definition}[基]
$H$是向量空间$V$的子空间,$\mathcal{B}=\{\vb{b}_1,\dots,\vb{b}_p\}$满足以下两个条件:
\begin{enumerate}
	\itemsep -3pt
	\item $\mathcal{B}$为线性无关集
	\item $\mathrm{Span}\{\vb{b}_1,\dots,\vb{b}_p\}=H$
\end{enumerate}
则称$\mathcal{B}$是$H$的基. 换言之,基是\textbf{极大线性无关组},也是\textbf{极小生成集}.
\end{definition}
由条件2知$\vb{b}_1,\dots,\vb{b}_p$都得在$H$中,这个显然的结论看似没用,但却是$\mathcal{B}$成为基的重要条件(在后文证明施密特正交化(算法\ref{schmidt})时也需要说明这一点).
\par如$\mathbf{v}_1=(1,0,1),\mathbf{v}_2=(0,1,1),\mathbf{v}_3=(0,1,0)$,令$H=\{(s,t,t):s,t\in \mathbb{R}\}$,有
\[
\begin{bmatrix}s \\t \\t\end{bmatrix}=s\begin{bmatrix}1 \\0 \\1\end{bmatrix}+(t-s)\begin{bmatrix}0 \\1 \\1 \end{bmatrix}+s\begin{bmatrix}0 \\1 \\0 \end{bmatrix}\]
这里$\vv_1,\vv_3\notin H$,其实$H$只是$\mathrm{Span}\{\vv_1,\vv_2,\vv_3\}$的子集,故$\{\vv_1,\vv_2,\vv_3\}$不能称为一组基.
\begin{theorem}[生成集(Spanning set)]
$S=\{\vb{v}_1,\dots,\vb{v}_p\}$是$V$中的集合,令$H=\span\{\vb{v}_1,\dots,\vb{v}_p\}$,
\begin{enumerate}[(a)]
	\itemsep -3pt
	\item 若$\vb{v}_k$是$S$中其余向量的线性组合,则将$\vb{v}_k$移除,$S$仍然张成$H$
	\item 若$H\ne\{\vb{0}\}$,则$S$的某个子集是$H$的基
\end{enumerate}
\end{theorem}
\begin{analysis}不妨设$\vb{v}_p$可以被其他向量线性表示,则
\[\vb{v}_p=a_1\vv_1+\cdots+a_{p-1}\vv_{p-1}\]
将上式代入下式,可知将$\vb{v}_p$移除后,$H$中的向量$\vx$依然可以被其余$p-1$个向量线性表示,即$S$仍然张成$H$
\[\vx=c_1\vv_1+\cdots+c_p\vv_p=(c_1+a_1c_p)\vv_1+\cdots+(c_{p-1}+a_{p-1}c_p)\vv_{p-1}\]
不断重复$(a)$的步骤,删除线性相关的向量,那么剩下的向量一定都线性无关,且可张成$H$,故形成一组基
\end{analysis}
\begin{definition}[维度]
向量空间$V$的一组基中向量的个数定义为$V$的维度,记为$\dim V$
\end{definition}
\begin{definition}[秩]
对于矩阵$A$,它的秩记为
\[\mathrm{rank}\,A=\mathrm{dim}\,\mathrm{Col}\,A\]
一般地,对于一组向量集合,它的秩是极大线性无关组中向量的数目
\end{definition}
\begin{theorem}以下是关于基和维度的一些结论. 设$S=\{\vv_1,\dots,\vv_n\}\in V,\,|S|\geq 1$,且$S$为$V$的基
\label{basis_theo}
\begin{enumerate}[(a)]
	\itemsep -1pt
	\item \label{b1}在$V$中任取$p(p>n)$个向量,一定线性相关 %P257
	\item 在$V$中任取$p(p<n)$个向量,一定不能张成$V$%P261T25
	\item $V$的所有基都包含$n$个向量
	\begin{analysis}
	由假设知$\dim V=n$,先将$S$与这$p$个向量改写成坐标向量后(见\ref{coordinate-system}节),构造$n\times p$矩阵$A$,$A$的每一列即为这$p$个向量. 由\ref{nul_and_col}节的讨论知,若$n<p$,则$A$一定存在自由变元,$A$的列一定线性相关;若$n>p$,则不能保证每一行都有主元位置,进而不能张成$\rn$,对应地也就不能张成$V$. 因此,$V$的所有基都包含$n$个向量.
	直接推论是:每一个$\mathbb{R}^n$的基都必须包含$n$个向量.
	\end{analysis}
	\item \label{b4}$H$是$V$的一个子空间,则$H$内任一线性无关集都可以被拓展为$H$的一组基,且$\dim H\leq\dim V$(基的存在性)
	\begin{analysis}
	若$H=\{\vb{0}\}$,显然$\dim H=0\leq\dim V$;\\
	否则,令$S'=\{\vu_1,\dots,\vu_k\}$是$H$的一个线性无关集. 若$S'$张成$H$,则$S'$是$H$的一组基;否则,$\exists\vu_{k+1}\in H-\span S'$,则$\{\vu_1,\dots,\vu_k,\vu_{k+1}\}$将线性无关. 继续这个过程,可以不断扩展$S'$直至其能张成$H$,也即成为$H$的一组基. 但由(\ref{b1}),$S'$中向量的数量不能超过$\dim V$. 综上,$\dim H\leq\dim V$.
	\end{analysis}
	\item \label{b5}任一$p$个元素的线性无关集或是可张成$V$的集合自动成为$V$的基
	\begin{analysis}
	由(\ref{b4}),有$p$个元素的线性无关集$S'$可以被拓展成$V$的基,但这组基一定得含$p$个元素,因为$\dim V=p$,因此$S'$已经是$V$的基. 假设$S''$有$p$个元素且可张成$V$,因为$V\ne\{\vb{0}\}$,生成集定理告诉我们$S''$的某一个子集$S^*$一定是$V$的基,又因$\dim V=p$,所以$S^*$一定包含$p$个向量,也即$S^*=S''$.
	\end{analysis}
	\item $H$是$V$的$n$维子空间,则$H=V$
	\begin{analysis}
	若$\dim V=\dim H=0$,则$V=H=\{\vb{0}\}$;\\
	若$\dim V=\dim H>0$,则$H$存在一组基$S'$包含$n$个向量,那么由(\ref{b5}),$S'$也是$V$的基,因此$H=V=\span S'$.
	\end{analysis}
	\item $\mathbb{P}$是无限维空间,$C(\mathbb{R})$为所有实值连续函数构成的空间,也是无限维的
	\begin{analysis}
	反证法,假设$\dim\mathbb{P}=k<\infty$,$\mathbb{P}_n$是$\mathbb{P}$的子空间,$\dim\mathbb{P}_{k-1}=k$,因此$\dim\mathbb{P}_{k-1}=\dim\mathbb{P}$,进而$\mathbb{P}_{k-1}=\mathbb{P}$,这显然不对,如$\vb{p}(t)=t^k$在$\mathbb{P}$内,但不在$\mathbb{P}_{k-1}$内,矛盾.\\
	因$\mathbb{P}$是$C(\mathbb{R})$的子空间,若$C(\mathbb{R})$为有限维,则由(\ref{b4}),$\mathbb{P}$也是有限维的,矛盾.
	\end{analysis}
\end{enumerate}
\end{theorem}
\begin{theorem}[秩定理]
\label{rank_theo}
\[\mathrm{rank}\,A+\mathrm{dim}\,\mathrm{Nul}\,A=n\]
\[\dim\row A+\dim \nul A^T=m\]
\end{theorem}
\begin{analysis}
由前面\ref{nul_and_col}节的讨论知道,列空间的维度为主元列数,加上零空间的维度为自由变量个数就等于总的列数.
\end{analysis}
\begin{example}
若一个非齐次线性方程组有$6$个方程$8$个未知数,已知其有一个有$2$个自由变元的解,则无论方程右侧的常数是什么,该方程组总有解.
\end{example}
\begin{analysis}
$\dim\nul A=2,\rank=8-2=6,\col A\in\mathbb{R}^6\implies\col A=\mathbb{R}^6$
\end{analysis}
\begin{example}
$A\vx=\vb{b},\forall\vb{b}\in\mathbb{R}^m$都有解$\iff A^T\vx=\vb{0}$只有平凡解
\end{example}
\begin{analysis}
$\rank A=\dim\col A=\dim\row A=m\implies\dim\nul A^T=0$
\end{analysis}
\begin{theorem}[秩的进阶定理]%P300
$A$是$m\times n$矩阵,$B$是$n\times p$矩阵,记$\rank A=r=r_1$,$\rank B=r_2$,有以下结论
\begin{enumerate}[(a)]
	\itemsep -3pt
	\item $\rank AB\leq \rank A,\rank AB\leq \rank B$
	\begin{analysis}
	设$\vb{y}\in\col AB$,则$\exists\,\vx\,s.t.\,\vb{y}=AB\vx$,而$AB\vx=A(B\vx)=\vb{y}$,因此$\vb{y}\in\col A$,故$\col AB$为$\col A$的子空间,由定理\ref{basis_theo}(\ref{b4})知$\rank AB\leq \rank A$.\\
	又由秩定理,$\rank AB=\rank(AB)^T=\rank B^TA^T\leq\rank B^T=\rank B$.
	\end{analysis}
	\item $P$是$m\times m$可逆矩阵,$Q$是$n\times n$可逆矩阵,则$\rank PA=\rank AQ=\rank A$
	\begin{analysis}
	由$(a)$有$\rank PA\leq\rank A=\rank(P^{-1}P)A=\rank P^{-1}(PA)\leq PA\implies \rank PA=\rank A$.\\
	$\rank AQ=\rank (AQ)^T=\rank Q^TA^T=\rank A^T=\rank A$.
	\end{analysis}
	\item 若$AB=O$,则$\rank A+\rank B\leq n$
	\begin{analysis}
	由$AB=O$知,$B$的每一列都在$\nul A$中,故$\col B$是$\nul A$的子空间,故$\rank B\leq \dim\nul A$.\\
	又由秩定理$n=\rank A+\dim \nul A\geq\rank A+\rank B$.
	\end{analysis}
	\item $\rank(A+B)\leq\rank A+\rank B$
	\begin{analysis}
	引理:一定存在秩分解$A=CR$,其中$C$是$m\times r$矩阵,$R$是$r\times n$矩阵.\\
	将$A,B$秩分解得$A=C_1R_1,B=C_2R_2$,构造$m\times (r_1+r_2)$矩阵$C=\begin{bmatrix}C_1&C_2\end{bmatrix}$,则
	\[A+B=C_1R_1+C_2R_2=\begin{bmatrix}C_1&C_2\end{bmatrix}\begin{bmatrix}R_1\\R_2\end{bmatrix}=CR\]
	因$C$有$r_1+r_2$列,故$\rank C\leq r_1+r_2$,同理$R$有$r_1+r_2$行,故$\rank R\leq r_1+r_2$,进而$\rank (A+B)\leq r_1+r_2=\rank A+\rank B$.
	\end{analysis}
	\item $A$的秩为$r$当且仅当$A$包含一个$r\times r$的子矩阵,且没有更大的方阵是可逆的
	\begin{analysis}
	$1\degree$ 证明$A$一定有$m\times r$且秩为$r$的子矩阵$A_1$\\
	令$A_1$包含$A$的$r$个主元列,因这些列线性无关,故$A_1$即为所求.\\
	$2\degree$ 证明$A_1$一定有$r\times r$且可逆的子矩阵$A_2$\\
	$\rank A_1=\dim\row A_1=r$,令$A_2$包含$A_1$的$r$个线性无关的行,则$A_2$即为所求,且为方阵故可逆.
	\end{analysis}
	\item 若$C=\begin{bmatrix}A&O\\O&B\end{bmatrix}$,则$\rank C=\rank A+\rank B$
	\begin{analysis}
	将$A,B$都写成阶梯形,因分块矩阵其余位置都为$0$,故$C$此时也为阶梯形,$C$的秩就等于$A,B$主元列之和.
	\end{analysis}
\end{enumerate}
\end{theorem}

\subsection{坐标系统}
\label{coordinate-system}
\begin{theorem}[向量唯一表示定理]
$\mathcal{B}=\{b_1,\cdots,b_p\}$是向量空间$V$的基,则$\forall\vx\in V$,存在唯一$c_1,\dots,c_n$使得$\vx=c_1\vb{b}_1+\cdots+c_p\vb{b}_p$
\end{theorem}
\begin{analysis}
\[0=\vx-\vx=(c_1-d_1)\vb{b}_1+\cdots+(c_p-d_p)\vb{b}_p\]
由基线性无关,知上述方程只有平凡解.
\end{analysis}
\begin{definition}[坐标向量]
$\mathcal{B}=\{b_1,\cdots,b_p\}$是$H$的基,$\forall\vx\in H,\;\vx=c_1\vb{b}_1+\cdots+c_p\vb{b}_p$,则$\vx$关于$\mathcal{B}$的坐标向量定义为
\[[\vx]_{\mathcal{B}}:=\begin{bmatrix}c_1\\ \vdots \\ c_p\end{bmatrix}\]
\end{definition}
\par坐标系统最大的用处是可以将向量空间$V$中奇奇怪怪的东西转换成$\mathbb{R}^n$中我们熟悉的向量,求解坐标的过程实质上又是解线性方程组,进而有如下定理.
\begin{theorem}[坐标变换矩阵]
对于$V$的一组基$\mathcal{B}=\{\vb{b}_1,\cdots,\vb{b}_n\}$,存在唯一$n\times n$矩阵$P_\mathcal{B}$使得
\[\mathbf{x}=P_\mathcal{B}[\mathbf{x}]_{\mathcal{B}},\,\forall \vx\in V\]
其中$P_\mathcal{B}=\bmat{\vb{b}_1}{\vb{b}_n}$称为坐标变换矩阵. 移项即可解出坐标,
\[[\mathbf{x}]_{\mathcal{B}}=P_\mathcal{B}^{-1} \mathbf{x}\]
\end{theorem}
\begin{theorem}[换基]
$\mathcal{B}=\{\vb{b}_1,\dots,\vb{b}_n\},\mathcal{C}=\{\vb{c}_1,\dots,\vb{c}_n\}$都是向量空间$V$的基,那么存在唯一的$n\times n$矩阵$\underset{\mathcal{C}\gets \mathcal{B}}{P}$,使得
\[[\vx]_{\mathcal{C}}=\underset{\mathcal{C}\gets \mathcal{B}}{P}[\vx]_{\mathcal{B}}\]
其中$\underset{\mathcal{C}\gets \mathcal{B}}{P}=\begin{bmatrix}[\vb{b}_1]_{\mathcal{C}}&[\vb{b}_2]_{\mathcal{C}}&\cdots&[\vb{b}_n]_{\mathcal{C}}\end{bmatrix}$,称为过渡矩阵.
\end{theorem}
\begin{myalgorithm}[在$\rn$中换基]
要求$\underset{\mathcal{C}\gets \mathcal{B}}{P}$,首先要将$\mathcal{B}$与$\mathcal{C}$之间的关系找到,即用$\mathcal{C}$中的向量表示$\mathcal{B}$中的向量.\\
设坐标变换矩阵$P_\mathcal{C}=\bmat{\vb{c}_1}{\vb{c}_n}$,进而$\vb{b}_i=P_{\mathcal{C}}[\vb{b}_i]_{\mathcal{C}}$,故
\[P_{\mathcal{B}}=P_{\mathcal{C}}\underset{\mathcal{C}\gets \mathcal{B}}{P}\]
(注意矩阵下标)由命题\ref{row_redu_alg},可得以下过程
\[\begin{bmatrix}P_\mathcal{C}&P_\mathcal{B}\end{bmatrix}\thicksim\begin{bmatrix}I&\underset{\mathcal{C}\gets \mathcal{B}}{P}\end{bmatrix}\]
或者$P_\mathcal{B}[\vx]_\mathcal{B}=\vx=P_\mathcal{C}[\vx]_\mathcal{C}\implies [\vx]_\mathcal{C}=P_\mathcal{C}^{-1}P_\mathcal{B}[\vx]_\mathcal{B}\implies \underset{\mathcal{C}\gets \mathcal{B}}{P}=P_{\mathcal{C}}^{-1}P_{\mathcal{B}}$.
\end{myalgorithm}
\begin{theorem}%P250 P254T23 24
关于坐标变换$T:\vx\mapsto[\vx]_{\mathcal{B}}$,其中$\mathcal{B}=\{\vb{b}_1,\dots,\vb{b}_n\}$,有如下结论.
\begin{enumerate}
	\itemsep -3pt
	\item $T$是一个双射的线性变换,$\vx$所处的空间$V$与$[\vx]_\mathcal{B}$所处的空间$\rn$同构
	\item $S=\{\vb{u}_1,\dots,\vb{u}_p\}$线性无关当且仅当$S'=\left\{[\vb{u}_1]_\mathcal{B},\dots,[\vb{u}_p]_\mathcal{B}\right\}$线性无关;$S$线性相关当且仅当$S'$线性相关
\end{enumerate}
\end{theorem}
\begin{analysis}
\begin{enumerate}
	\itemsep -3pt
	\item 若$[\vu]_\mathcal{B}=[\vb{w}]_\mathcal{B}=\begin{bmatrix}c_1\\\vdots\\c_n\end{bmatrix}$,由坐标向量定义$\vb{u}=\vb{w}=c_1\vb{b}_1+\cdots+c_n\vb{b}_n$,而$\vb{u},\vb{w}$均为$V$中任意元素,故$T$为单射. 令$\vb{y}=\begin{bmatrix}y_1\\\vdots\\y_n\end{bmatrix}$,$\vb{u}=y_1\vb{b}_1+\cdots+y_n\vb{b}_n$,则由定义$[\vb{u}]_\mathcal{B}=\vb{y}$,因$\vb{y}$为$\rn$任意向量,故$T$为满射.
	\item 由定理\ref{lt_theo}易知.
\end{enumerate}
\end{analysis}