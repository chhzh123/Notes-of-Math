% !TEX root = main.tex

\section{矩阵与线性方程组}
\subsection{基本概念}
在此仅罗列一些基本的知识点的名字,具体内容本文不作赘述.
\begin{enumerate}
	\itemsep -3pt
	\item 解线性方程组(高斯消元法)
	\item 方程组无解、唯一解、多解的判断,与主元列、主元位置的关系\\
		解的\textbf{存在性}:看每一列是否都有主元位置\\
		解的\textbf{唯一性}:看有无自由变元($A\vx=\vb{0}$是否有\textbf{非平凡解}同样看自由变元)
	\item \textbf{齐次方程}$A\vx=\vb{0}$解的参数表示\\
		\textbf{非齐次方程}$A\vx=\vb{b}$解的形式为$\vx=\vb{p}+t\vb{v},t\in\mathbb{R}$,即非齐通解$=$齐通解$+$非齐特解,但要注意首先得有解\\
		这两种方程的具体比较见\ref{nul_and_col}节
	\item 线性相关与线性无关\\
	要判定一组向量是线性相关还是线性无关,即判断$A\vx=\vb{0}$是否有非平凡解
	\item 基本矩阵运算(加减乘除、求逆、转置)
\end{enumerate}
\begin{definition}[基本行变换/初等矩阵变换]
替换、交换、数乘
\end{definition}
\begin{definition}[初等(elementary)矩阵]
对单位矩阵做一次基本行变换所得的矩阵.
\end{definition}
\par 下面是线性无关线性相关一些常见的结论.
\begin{theorem}
\label{linear_relationship}
$S=\{\vb{v}_1,\dots,\vb{v}_p\},|S|>1$
\begin{enumerate}[(a)]
	\itemsep -3pt
	\item $S$线性相关,当且仅当至少有一向量可以表示为其他向量的线性组合(注意\textbf{不是}每一个)
	\item 若$S$线性相关,$\vb{v}_1\ne 0$,那么$\exists\vb{v}_j,j>1$可以被表示为$\vb{v}_1,\dots,\vb{v}_{j-1}$的线性组合
	\item 若$S$的子集$S'$线性相关,则$S$线性相关;若$S$线性无关,则$S'$线性无关(注意关系不能颠倒)
	\item 初等行变换不改变\textbf{列}之间的线性相关关系
\end{enumerate}
\end{theorem}
\begin{analysis}
记$A=\bmat{\vb{v}_1}{\vb{v}_n}$,$B\thicksim A$为$A$的阶梯形
\begin{enumerate}[(a)]
	\item \label{enu1a}必要性:即$\exists c_1,\dots,c_n$不全为$0$,使得$c_1\vb{v}_1+\cdots+c_n\vb{v}_n=\vb{0}$.\\
	不妨设$c_1\ne 0$,则$\vb{v}_1=(-\dfrac{c_2}{c_1})\vv_2+\cdots+(-\dfrac{c_n}{c_1})\vb{v}_n$,证毕.\\
	充分性:不妨设$\vb{v}_1=c_2\vb{v}_2+\cdots+c_n\vb{v}_n$,移项得$(-1)\vb{v}_1+c_2\vb{v}_2+\cdots+c_n\vb{v}_n=\vb{0}$.
	\item 用(\ref{enu1a})的结论,设$\vb{v}_p=c_1\vb{v}_1+\cdots+c_{p-1}\vb{v}_{p-1}+c_{p+1}\vb{v}_{p+1}+\cdots+c_n\vb{v}_n,p>1$,找出最大的$j$使得$c_j\ne 0$($\vb{v}_p$的系数$c_p=-1$也计入),进而$c_1\vb{v}_1+\cdots+c_j\vb{v}_j=\vb{0}$,接下来证明方法与(\ref{enu1a})相同,$\vb{v}_j$即为所求.
	\item 因两者互为逆否命题,故只用证前者,而前者通过不断扩充$S'$并令扩充部分的权重全为$0$即可.
	\item $A\vx=\vb{0}$与$B\vx=\vb{0}$有相同的解,故线性相关关系不会改变.
\end{enumerate}
\end{analysis}

\subsection{矩阵的逆}
如无特殊说明,在本节中我们讨论的矩阵均为方阵.
\begin{definition}[逆]
$A$为$n\times n$的矩阵,若$\exists C\,s.t.\,CA=I_n,\,AC=I_n$,则称$A$可逆.\\
注意:在定义中要求左乘右乘都满足才说其可逆.
\end{definition}
\begin{theorem}
$A$是$m\times n$矩阵,若存在$n\times m$矩阵$C$和$D$使得$CA=I_n$,$AD=I_m$,则$m=n$且$C=D$
\end{theorem}
\begin{analysis}
这个定理告诉我们只有方阵才会存在逆,下面来证明.\\
$1\degree\;$证明若$CA=I_n$,则$m\geq n$\\
因为$CA\vx=I_n\vx=\vx=\vb{0}$,故$A\vx=\vb{0}$只有平凡解,没有自由变元,因此$m\geq n$.\\
$2\degree\;$证明若$AD=I_m$,则$m\leq n$\\
因为$AD\vb{b}=I_m\vb{b}=\vb{b}$,也即$\forall \vb{b}\in\mathbb{R}^m,\,A\vx=\vb{b}$都有解$\vx=D\vb{b}$,因此$m\leq n$.\\
$3\degree\;$由上分析知,$m=n$.\\
考虑乘积$CAD$,有$CAD=(CA)D=I_nD=D=C(AD)=CI_m=C$,故$C=D$.
\end{analysis}
\begin{theorem}[二阶矩阵的逆]
\[A=\begin{bmatrix}
a&b\\c&d
\end{bmatrix}\quad\implies\quad
A^{-1}=\dfrac{1}{\det A}\begin{bmatrix}
d&-b\\-c&a
\end{bmatrix},\,\det A\ne 0\]
\end{theorem}
\begin{proposition}
\label{row_redu_alg}
若$A=BC$,$B$可逆,则$\begin{bmatrix}B&A\end{bmatrix}\thicksim\begin{bmatrix}I&C\end{bmatrix}$
\end{proposition}
\begin{analysis}
\[EB=I\implies E=B^{-1}\]
\[EA=C\implies B^{-1}A=C\iff A=BC\]
\end{analysis}
\begin{myalgorithm}[求逆]
\[\begin{bmatrix}A&I\end{bmatrix}\thicksim\begin{bmatrix}I&A^{-1}\end{bmatrix}\]
\end{myalgorithm}
\begin{analysis}
在命题\ref{row_redu_alg}中令$A\gets I,\,B\gets A,\,C\gets A^{-1}$即可.
\end{analysis}
若令$A^{-1}=\bmat{\vx_1}{\vx_n}$,那么也可以将上述求逆过程看成是解下面的这个方程组
\[A\vx_1=\vb{e}_1,\quad A\vx_2=\vb{e}_2,\quad\dots,\quad A\vx_n=\vb{e}_n\]
其说明的是矩阵$A$乘上$A^{-1}$的第$i$列,将得到$I_n$的第$i$列,也即$\vb{e}_i$.
\par 虽然只是变换了视角,但对于一些问题来说这种视角的变换是重要的.
\newpage
\begin{theorem}[可逆阵定理]
对于$n\times n$矩阵$A$,下列说法均等价
\begin{multicols}{2} %multicols!
\begin{enumerate}[(a)]
	\itemsep -3pt
	\item $A$可逆
	\item $A\thicksim I_n$
	\item $A$有$n$个主元位置
	\item $A\vx=\vb{0}$只有平凡解
	\item $A$的列线性无关
	\item $\vx\mapsto A\vx$是单射
	\item $A\vx=\vb{b},\forall b\in\mathbb{R}^n$至少有一解
	\item $A$的列张成$\mathbb{R}^n$
	\item $\vx\mapsto A\vx$是满射
	\item $\exists\;n\times n\;C\,s.t.\,CA=I$
	\item $\exists\;n\times n\;D\,s.t.\,AD=I$
	\item $A^T$可逆
	\item $A$的列构成$\rn$的一组基
	\item $\col A=\rn$
	\item $\dim\col A=n$
	\item $\rank A=n$(满秩)
	\item $\nul A=\{\vb{0}\}$
	\item $\dim\nul A=0$
	\item $0$\textbf{不是}$A$的特征值
	\item $\det A\ne 0$
	\item $(\col A)^\perp=\nul A^T=\{\vb{0}\}$
	\item $(\nul A)^\perp=\row A=\rn$
	\item $\row A=\rn$
	\item $A$有$n$个非零奇异值
\end{enumerate}
\end{multicols}
\end{theorem}
\begin{myalgorithm}[LU分解]
$A$为$m\times n$的矩阵,则$A$一定可以被分解为$A=LU$,其中$L$为下三角矩阵,$U$为上三角矩阵.以下为步骤
\begin{enumerate}
	\itemsep -3pt
	\item 初等行变换(\textbf{只进行替换})将$A$变成行阶梯形$U$,并记录每次$E_1,\dots,E_p$
	\item 则$L=(E_p\cdots E_1)^{-1}$,相当于将每次初等行变换逆过来,并将矩阵直接重合即可
\end{enumerate}
\end{myalgorithm}
\par 注意到LU分解与方阵求逆的方法基本相同,但它适用于所有矩阵,算是矩阵分解里最简单的一种了. 用LU分解解方程,加法乘法计算量会降低不少(因为只需回代).
\begin{proposition}[消去律]
若$A$可逆,$AB=AC$,则$B=C$.
\end{proposition}
\begin{proposition}
若$AB$可逆,则$A$,$B$都可逆. 若$A$\textbf{或}$B$不可逆,则$AB$不可逆.
\end{proposition}
\begin{analysis}
$C=AB\implies I=BC^{-1}A\implies A^{-1}=BC^{-1}$,$B$同理. 而后面一句为前面一句的逆否命题,故证毕.
\end{analysis}
\begin{theorem}
上(下)三角矩阵的逆仍为上(下)三角
\end{theorem}
\begin{analysis}
不妨设$A$为下三角矩阵,则$a_{ij}=0,i\in\{1,\dots,n-1\},j\in\{i+1,\dots,n\}$.\\
若$j\ne n-1$,注意到$a_{(n-1)n}=0$一定被包含在$A_{ji}$中,且一定在$A_{ji}$的对角线上(最右下角);\\
若$j=n-1$,$a_{(n-1)(n-3)}=0$一定被包含在$A_{ji}$中,且也在$A_{ji}$对角线上(右下角倒数第二个元素).\\
又注意到在$a_{ij}=0,i\in\{1,\dots,n-1\},j\in\{i+1,\dots,n\}$的前提下,$A_{ji}$一定为下三角矩阵,由上分析知对角线上一定存在$0$元素,故代数余子式$C_{ji}$为$0$. 进而伴随矩阵为下三角,原矩阵的逆为下三角.
\end{analysis}

\subsection{分块矩阵}
\par 需要知道矩阵乘法另外的表示方法
\[AB=A\bmat{\vb{b}_1}{\vb{b}_n}=\bmat{A\vb{b}_1}{A\vb{b}_n}\]
类似地,分块矩阵有
\[AB=\bmat{\vb{a}_1}{\vb{a}_n}\begin{bmatrix}\vb{b}_1\\\vdots\\\vb{b}_n\end{bmatrix}=\sum_{i=1}^n\vb{a}_i\vb{b}_i\]
注意区别正常的矩阵乘法(行乘列得到某一位置上的元素),分块矩阵乘法是列乘行得到矩阵.
\par 分块矩阵的转置除了每一个矩阵转置外,大的矩阵内部也需要转,如
\[A=\begin{bmatrix}A_{11}&A_{12}\\A_{21}&A_{22}\end{bmatrix}\implies A^T=\begin{bmatrix}A_{11}^T&A_{21}^T\\A_{12}^T&A_{22}^T\end{bmatrix}\]
\par 分块矩阵求逆并不是每一块分别求逆简单合并即可,而是需要将逆设出来,乘开后通过对比系数来求. 如下面定理\ref{part_inverse}所示.
\begin{theorem}
\label{part_inverse}
\[A=\begin{bmatrix}A_{11}&A_{12}\\O&A_{22}\end{bmatrix}\implies A^{-1}=\begin{bmatrix}A_{11}^{-1}&-A_{11}^{-1}\,A_{12}\,A_{22}^{-1}\\O&A_{22}^{-1}\end{bmatrix}\]
特别地,当$A_{12}$为$0$矩阵,该分块矩阵的逆可以变成各个矩阵分别求逆. 实际上对于\textbf{对角矩阵},其$k$次方都是主对角线上的元素直接取$k$次幂,求逆看成是$-1$次幂当然也成立,见推论\ref{part_inverse_col}.
\end{theorem}
\begin{analysis}
设$B=\begin{bmatrix}B_{11}&B_{12}\\B_{21}&B_{22}\end{bmatrix}$,且
\[\begin{bmatrix}A_{11}&A_{12}\\O&A_{22}\end{bmatrix}\begin{bmatrix}B_{11}&B_{12}\\B_{21}&B_{22}\end{bmatrix}=\begin{bmatrix}I_p&O\\O&I_q\end{bmatrix}\]
乘开后对比系数有
\[\begin{aligned}
A_{11}B_{11}+A_{12}B_{21}&=I_p\\
A_{11}B_{12}+A_{12}B_{22}&=O\\
A_{22}B_{21}&=O\\
A_{22}B_{22}&=I_q\end{aligned}\]
由下往上回代即得结果.
\end{analysis}
\begin{corollary}
\label{part_inverse_col}
$A=\begin{bmatrix}B&O \\ O&C\end{bmatrix}$,$B,C$均为方阵,则$A$可逆,当且仅当$B,C$都可逆.若$A$可逆,则$A^{-1}=\begin{bmatrix}B^{-1}&O \\ O&C^{-1}\end{bmatrix}$
\end{corollary}
如求以下这个矩阵的逆就可以划分为对角矩阵后,直接对各块求逆合并.
\[\begin{bmatrix}
1&2&0&0&0\\
3&5&0&0&0\\
0&0&2&0&0\\
0&0&0&7&8\\
0&0&0&5&6
\end{bmatrix}\]
\par 当然,分块也是需要技巧的,常常需要先观察矩阵的性质,再做出合理划分,见例\ref{part_inverse_eg}.
\begin{example}%2.4T24
\label{part_inverse_eg}
\[A=\begin{bmatrix}1&0&0&\cdots&0\\1&1&0& &0\\1&1&1& &0\\\vdots& & &\ddots& &\\1&1&1&\cdots&1\end{bmatrix},\,\text{证明}B=\begin{bmatrix}1&0&0&\cdots&0\\-1&1&0& &0\\0&-1&1& &0\\\vdots& &\ddots &\ddots& &\\0& &\cdots&-1&1\end{bmatrix}\text{为$A$的逆}\]
\end{example}
\begin{analysis}
递归定义分块矩阵:$A_{k+1}=\begin{bmatrix}1&\vb{0}^T\\\vb{v}&A_k\end{bmatrix},\,B_{k+1}=\begin{bmatrix}1&\vb{0}^T\\\vb{w}&B_k\end{bmatrix}$,并用数学归纳法可证.
\end{analysis}
\begin{theorem}
两个上(下)三角矩阵的乘积仍为上(下)三角
\end{theorem}
\begin{analysis}
不妨设两个矩阵$A,B$都为下三角,且$A=[a_{ij}],B=[b_{ij}],C=AB=[c_{ij}]$,其中
\[c_{ij}=a_{i1}b_{1j}+a_{i2}b_{2j}+\cdots+a_{in}b_{nj}\]
当$i<j$时,$a_{ij}=b_{ij}=0$,故(自行画图观察可得以下事实)
\[\underwrite[0pt]{c_{ij}}{}\underwrite[0pt]{=}{}\underwrite{a_{i1}}{\ne 0}\underwrite{b_{1j}}{=0}\underwrite[0pt]{+\cdots+}{}\underwrite{a_{ii}}{\ne 0}\underwrite{b_{ij}}{=0}\underwrite[0pt]{+}{}\underwrite{a_{i(i+1)}}{=0}\underwrite{b_{(i+1)j}}{=0}\underwrite[0pt]{+\cdots+}{}\underwrite{a_{ij}}{=0}\underwrite{b_{jj}}{\ne 0}\underwrite[0pt]{+\cdots+}{}\underwrite{a_{in}}{=0}\underwrite{b_{nj}}{\ne 0}\underwrite[0pt]{= 0}{}\]
\end{analysis}
\begin{theorem}[舒尔(Schur)补]
若$A_{11}$可逆,则
\[\begin{bmatrix}A_{11}&A_{12}\\A_{21}&A_{22}\end{bmatrix}=\begin{bmatrix}I & O\\X & I\end{bmatrix}\begin{bmatrix}A_{11} & O\\O &S\end{bmatrix}\begin{bmatrix}I& Y\\O & I\end{bmatrix}\]
其中$S=A_{22}-A_{21}A_{11}^{-1}A_{12}$称为$A_{11}$的舒尔补,$X=A_{21}A_{11}^{-1}$,$Y=A_{11}^{-1}A_{12}$.
\end{theorem}
\begin{analysis}
类似定理\ref{part_inverse}的方法,乘开后通过对比系数可得.
\end{analysis}

\subsection{行列式}
\begin{proposition}[行列式的基本运算]\mbox{}\par
\begin{enumerate}
	\itemsep -1pt
	\item $\det A^T=\det A$
	\item $\det AB=(\det A)(\det B)=\det BA$
	\item \label{det3}$\det A^{-1}=\dfrac{1}{\det A}$
	\item $\det A^k=(\det A)^k$(如果把矩阵的逆看成$-1$次幂,那就是\ref{det3}式)
	\item $\det rA=r^n\det A$
\end{enumerate}
\end{proposition}
\begin{proposition}[行列式基本行变换]
三种基本行变换与矩阵相同,但替换不改变它的值,交换乘$(-1)$,数乘乘$k$.
\end{proposition}
\begin{theorem}[拉普拉斯(Laplace)展开]
$A$为$n\times n$的矩阵,称$C_{ij}=(-1)^{i+j}\det A_{ij}$为$(i,j)$的代数余子式,其中$A_{ij}$是$A$删除第$i$行和第$j$列后剩下的矩阵.\\
对第$i$行展开有
\[\det A=\sum_{j=1}^{n}a_{ij}C_{ij}\]
对第$j$列展开有
\[\det A=\sum_{i=1}^{n}a_{ij}C_{ij}\]
\end{theorem}
特别地,若$A$是三角矩阵,则$\det A=$主对角线上元素的积(因对第一行或第一列展开,形式不变).
\begin{theorem}[克莱姆(Cramer)法则]
$A\vx=\vb{b}$,则$\vx_i=\dfrac{\det A_i(\vb{b})}{\det A}$,其中\[\det A_i(\vb{b})=\Big[\vb{a}_1\quad\cdots\quad\vb{b}^{\disp\swarrow\raisebox{0.5em}{$\mathrm{col}\;i$}}\quad\cdots\quad\vb{a}_n\Big]\]
\end{theorem}
\begin{analysis}直接构造
\[AI_i(\vb{b})=A\begin{bmatrix}\vb{e}_1&\cdots&\vb{b}&\cdots&\vb{e}_n\end{bmatrix}=\begin{bmatrix}A\vb{a}_1&\cdots& A\vb{b}&\cdots& A\vb{a}_n\end{bmatrix}\]
左右两边取$\det$得
\[\det AI_i(\vb{b})=(\det A)(\det I_i(\vb{b}))=(\det A)\vx_i=\det A_i(\vb{b})\]
\end{analysis}
\begin{theorem}
若$A$可逆,则$A^{-1}=\dfrac{1}{\det A}\mathrm{adj}\,A$,其中共轭/伴随矩阵$(\mathrm{adj}\,A)_{ij}=C_{ji}$\\
注意:共轭矩阵元素下标与代数余子式下标不同.
\end{theorem}
\begin{analysis}求逆算法换个角度思考,记$B=A^{-1}$的第$j$列为$\vb{b}_j$,则$A\vb{b}_j=\vb{e}_j$,\\
又记第$i$个元素为$b_{ij}$,则由克莱姆法则,$b_{ij}=\dfrac{\det A_i(\vb{e}_j)}{\det A}$,分子
\[\det A_i(\vb{e}_j)=(-1)^{j+i}\det A_{ji}=C_{ji}\]
(对第$i$列进行展开,除了第$j$行为$1$,其余行均为$0$),得证.
\end{analysis}
\begin{theorem}[行列式的线性性]
记$T:\rn\to\mathbb{R}$,$T(\vx)=\det\begin{bmatrix}\vb{a}_1&\cdots&\vb{a}_{j-1}&\vx&\vb{a}_{j+1}&\cdots&\vb{a}_n\end{bmatrix}$,则$T$为线性变换.\\
注:这里的线性性只是针对行列式中的一列可变来说的,若整个行列式都可变,则不一定有$\det (A+B)=\det A+\det B$.
\end{theorem}
\begin{theorem}[线性变换与体积变化]
线性变换$T:\rn\to\rn$的标准矩阵为$A$,$S$是一块有限的区域,那么
\[T(S)\text{的面积或体积}=|\det A|\cdot\,S\text{的面积或体积}\]
\end{theorem}
由此定理立得椭圆面积为$\pi a b$.
\begin{theorem}[解析几何]
\begin{enumerate}
	\item 相异两点$A(x_1,y_1),B(x_2,y_2)\in\mathbb{R}^2$,则直线$AB$的方程为$\det\begin{bmatrix}1&x&y\\1&x_1&y_1\\1&x_2&y_2\end{bmatrix}=0$
	\item 过点$A(x_1,y_1)\in\mathbb{R}^2$,斜率为$k$的直线方程为$\det\begin{bmatrix}1&x&y\\1&x_1&y_1\\0&1&k\end{bmatrix}=0$
	\item $A(x_1,y_1),B(x_2,y_2),C(x_3,y_3)\in\mathbb{R}^2$,三角形$ABC$的面积为$\dfrac{1}{2}\left|\det\begin{bmatrix}x_1&y_1&1\\x_2&y_2&1\\x_3&y_3&1\end{bmatrix}\right|$
\end{enumerate}
\end{theorem}
\begin{analysis}
实际上还有很多类似的结论,在此不作赘述. 均是做基本行变换后展开即得证.
\end{analysis}
\begin{theorem}[范德蒙(Vandermonde)矩阵]
\[\displaystyle V={\begin{bmatrix}1&\alpha _{1}&\alpha _{1}^{2}&\dots &\alpha _{1}^{n-1}\\1&\alpha _{2}&\alpha _{2}^{2}&\dots &\alpha _{2}^{n-1}\\1&\alpha _{3}&\alpha _{3}^{2}&\dots &\alpha _{3}^{n-1}\\\vdots &\vdots &\vdots &\ddots &\vdots \\1&\alpha _{m}&\alpha _{m}^{2}&\dots &\alpha _{m}^{n-1}\end{bmatrix}}\]
令首行为变量,其他行为常量可得$n-1$次多项式,常被用于多项式插值. 当$m=n$时它的行列式
\[\det(V) =\prod_{1\le i<j\le n} (\alpha_j-\alpha_i)\]%\sum_{\sigma \in S_n} \sgn(\sigma) \prod_{i = 1}^n \alpha_i^{\sigma(i)-1}=
\begin{analysis}
从第$n-1$列到第$1$列,每一列乘上$-\alpha_n$后加到后一列:
\[\begin{aligned}
\det(V_n)&=\begin{vmatrix}
1 & \alpha_1-\alpha_n & \alpha_1^2-\alpha_1\alpha_n & \cdots & \alpha_1^{n-2}-\alpha_1^{n-3}\alpha_n & \alpha_1^{n-1}-\alpha_1^{n-2}\alpha_n\\
1 & \alpha_2-\alpha_n & \alpha_2^2-\alpha_2\alpha_n & \cdots & \alpha_2^{n-2}-\alpha_2^{n-3}\alpha_n & \alpha_2^{n-1}-\alpha_2^{n-2}\alpha_n\\
\vdots & \vdots & \vdots & & \vdots & \vdots\\
1 & \alpha_{n-1}-\alpha_n & \alpha_{n-1}^2-\alpha_{n-1}\alpha_n & \cdots & \alpha_1^{n-2}-\alpha_{n-1}^{n-3}\alpha_n & \alpha_{n-1}^{n-1}-\alpha_{n-1}^{n-2}\alpha_n\\
1 & 0 & 0 & \cdots & 0 & 0
\end{vmatrix}\\
&=(-1)^{n+1}\begin{vmatrix}
\alpha_1-\alpha_n & \alpha_1(\alpha_1-\alpha_n) & \cdots & \alpha_1^{n-3}(\alpha_1-\alpha_n) & \alpha_1^{n-2}(\alpha_1-\alpha_n)\\
\alpha_2-\alpha_n & \alpha_2(\alpha_2-\alpha_n) & \cdots & \alpha_2^{n-3}(\alpha_2-\alpha_n) & \alpha_2^{n-2}(\alpha_2-\alpha_n)\\
\vdots & \vdots & & \vdots& \vdots\\
\alpha_{n-1}-\alpha_n & \alpha_{n-1}(\alpha_{n-1}-\alpha_n) & \cdots & \alpha_1^{n-2}(\alpha_{n-1}-\alpha_n) & \alpha_{n-1}^{n-2}(\alpha_{n-1}-\alpha_n)
\end{vmatrix}\quad\mbox{按最后一行展开}\\
&=(-1)^{n+1}(\alpha_1-\alpha_n)(\alpha_2-\alpha_n)\cdots(\alpha_{n-1}-\alpha_n)\begin{vmatrix}1&\alpha _{1}&\alpha _{1}^{2}&\dots &\alpha _{1}^{n-2}\\1&\alpha _{2}&\alpha _{2}^{2}&\dots &\alpha _{2}^{n-2}\\1&\alpha _{3}&\alpha _{3}^{2}&\dots &\alpha _{3}^{n-2}\\\vdots &\vdots &\vdots &\ddots &\vdots \\1&\alpha _{n-1}&\alpha _{n-1}^{2}&\dots &\alpha _{n-1}^{n-2}\end{vmatrix}\\
&=(\alpha_n-\alpha_1)(\alpha_n-\alpha_2)\cdots(\alpha_{n}-\alpha_{n-1})\det(V_{n-1})=\prod_{1\le i<j\le n} (\alpha_j-\alpha_i)
\end{aligned}\]
\end{analysis}
\end{theorem}
\par 除了懂得用拉普拉斯展开外,也要懂得通过基本行变换(或列变换)来计算行列式,如例\ref{fund_row_det}.
\begin{example}
\label{fund_row_det}
\[\displaystyle A={\begin{bmatrix}1&1&1&\dots &1\\1&2&2&\dots &2\\1&2&3&\dots &3\\\vdots &\vdots &\vdots &\ddots &\vdots \\1&2&3&\dots &n\end{bmatrix}}\implies\det A=1\]
\end{example}
\begin{analysis}
将每行乘$(-1)$后加到下一行,最后只剩对角线上元素全为$1$.
\end{analysis}