% !TEX root = main.tex

\section{特征值与特征向量}
\subsection{基本定义}
\begin{definition}$A$为$n\times n$的矩阵,若$\exists\lambda,\,\mathbf{x}\ne\mathbf{0}\,s.t.\,A\mathbf{x}=\lambda\mathbf{x}$,则
\begin{enumerate}
	\itemsep -2pt
	\item $\lambda$称为$A$的\textbf{特征值},$\mathbf{x}$称为关于$\lambda$的\textbf{特征向量}
	\item $\mathrm{Nul}\,(A-\lambda I)$称为$A$关于$\lambda$的\textbf{特征空间}
	\item $\det\,(A-\lambda I)$称为$A$的\textbf{特征多项式}
	\item $\lambda$的\textbf{代数重数}是它在特征多项式中作为根的重数,\textbf{几何重数}是它对应的特征空间的维度
\end{enumerate}
\end{definition}
需要注意:
\begin{enumerate}
	\itemsep -3pt
	\item 特征向量不能为$\mathbf{0}$,但特征值可以为$0$
	\item $A$不可逆,则$0$是一个特征值(充要条件);$A-\lambda I$不可逆,$A$才有特征值
\end{enumerate}
\begin{theorem}
\label{distinct_lambda}
$\lambda_1,\cdots,\lambda_r$是$n\times n$矩阵$A$不同的特征值,则其对应的特征向量$\{\mathbf{v}_1,\cdots,\mathbf{v}_r\}$线性无关.
\end{theorem}
\begin{analysis}
反证,假设$\{\mathbf{v}_1,\cdots,\mathbf{v}_r\}$线性相关.令$p$是最小的下标使得$\vb{v}_{p+1}$是之前向量的线性组合,即
\begin{equation}\label{eq1} c_1\vb{v}_1+\cdots+c_p\vb{v}_p=\vb{v}_{p+1}\quad\end{equation}
左右同乘$A$,得
\[c_1A\vb{v}_1+\cdots+c_pA\vb{v}_p=A\vb{v}_{p+1}\]
进而
\begin{equation}\label{eq2} c_1\lambda_1\vb{v}_1+\cdots+c_p\lambda_p\vb{v}_p=\lambda_{p+1}\vb{v}_{p+1}\end{equation}
将(\ref{eq1})式同乘$\lambda_{p+1}$,并与(\ref{eq2})式相减得
\[c_1(\lambda_1-\lambda_{p+1})\vb{v}_1+\cdots+c_p(\lambda_p-\lambda_{p+1})\vb{v}_p=\vb{0}\]
由于$\{\mathbf{v}_1,\cdots,\mathbf{v}_p\}$线性无关,则它们的权重$c_i(\lambda_i-\lambda_{p+1})=0,\,i=1,\dots,p$.而$A$的特征值均不同,故$\lambda_i-\lambda_{p+1}\ne 0$,因而$c_i=0,\,i=1,\dots,p$.\\
但由(\ref{eq1})又推出$\vb{v}_{p+1}=\vb{0}$,与特征向量的定义矛盾,故假设不成立.
\end{analysis}
\begin{proposition}特征值与矩阵运算
\begin{enumerate}
	\itemsep -3pt
	\item $\lambda$为可逆矩阵$A$的特征值,则$\lambda^{-1}$为$A^{-1}$的特征值
	\item $\lambda$是$A$的特征值,当且仅当$\lambda$是$A^T$的特征值
\end{enumerate}
\begin{analysis}
前者给$A\vx=\lambda\vx$左乘$A^{-1}$后显然. 后者$\det(A-\lambda I)=\det(A-\lambda I)^T=\det(A^T-\lambda I)$特征多项式相同\end{analysis}
\end{proposition}
\begin{definition}[相似性]
矩阵$A,B$满足$A=P^{-1}BP$,则称$A,B$相似.\\
注:相似性不同于行等价,行变换\textbf{会改变}矩阵的特征值,但相似性不会变
\end{definition}

\subsection{对角化}
\begin{theorem}
\label{diagonal_iff}
$n\times n$矩阵$A$可对角化当且仅当$A$有$n$个线性无关的特征向量.
\end{theorem}
\begin{analysis}\mbox{}\\
$1\degree\,$必要性:$A$可对角化,即$A=PDP^{-1}$,$D$为对角矩阵,也即$AP=PD$.设$P=\bmat{\vb{v}_1}{\vb{v}_n}$,$D=\bmat{\lambda_1\vb{e}_1}{\lambda_n\vb{e}_n}$(注意这里$\vb{v}_i$只是$P$的列,还不是特征向量;同样$\lambda_i$也还不是$A$的特征值).则
\[AP=\bmat{A\vb{v}_1}{A\vb{v}_n}=PD=\bmat{\lambda_1\vb{v}_1}{\lambda_n\vb{v}_n}\]
进而$A\vb{v}_i=\lambda_i\vb{v}_i$,$\lambda_i$为$A$的特征值,又$\vb{v}_i\ne 0$,$\vb{v}_i$为$\lambda_i$对应的特征向量.\\
由于$P$可逆,故$P$的列,也即$A$的特征向量线性无关.\\
$2\degree\,$充分性:令$P=\bmat{\vb{v}_1}{\vb{v}_n}$,其中$\vb{v}_i$为$\lambda_i$对应的特征向量,$D=\bmat{\lambda_1\vb{e}_1}{\lambda_n\vb{e}_n}$,其中$\lambda_i$为$A$的特征向量.那么\\
\[AP=\bmat{A\vb{v}_1}{A\vb{v}_n}=\bmat{\lambda_1\vb{v}_1}{\lambda_n\vb{v}_n}=PD\]
因为$\vb{v}_i$线性无关,故$P$可逆,进而$A=PDP^{-1}$.\\
注:从以上构造可以知道$\lambda$相同并不影响对角矩阵的构造,但是$\vb{v}$相同则无法保证线性无关.
\end{analysis}
\begin{theorem}
若$n\times n$矩阵$A$有$n$个不同的特征值,则$A$可对角化.
\end{theorem}
\begin{analysis}
从定理\ref{distinct_lambda}可知,$\{\mathbf{v}_1,\cdots,\mathbf{v}_r\}$线性无关,进而由定理\ref{diagonal_iff},$A$可对角化.
\end{analysis}
\begin{myalgorithm}[对角化]\mbox{}\par
\begin{enumerate}
	\itemsep -3pt
	\item 找到$A$的特征值
	\item 找到$A$的线性无关特征向量
	\item 构造$P$,其中$P=\bmat{\mathbf{v}_1}{\mathbf{v}_n}$
	\item 构造$D$,其中$D=\bmat{\lambda_1\mathbf{e}_1}{\lambda_n\mathbf{e}_n}$
\end{enumerate}
\end{myalgorithm}
注:对角化不是唯一的,会随着特征值摆放位置、特征向量的不同而改变. 可逆与可对角化没有必然联系.
\begin{theorem}
$A$是$n\times n$矩阵,$\lambda_1,\cdots,\lambda_p$是$A$不同的特征值,则
\begin{enumerate}
	\itemsep -3pt
	\item $\forall 1\leq k\leq p$,$\lambda_k$的几何重数$\leq\lambda_k$的代数重数
	\item $A$可对角化,当且仅当不同特征值的几何重数之和为$n$,而这当且仅当对于每一个$\lambda_k$,其代数重数都等于几何重数
	\item 若$A$可对角化,$\mathcal{B}_k$是$\lambda_k$对应的特征空间的基,那么$\mathcal{B}_1,\dots,\mathcal{B}_p$形成$\rn$的一组特征向量基
\end{enumerate}
\end{theorem}
\begin{theorem}[线性变换矩阵]
\label{linear_trans_tot}
$\mathcal{B}=\{\vb{b}_1,\dots,\vb{b}_n\}$为向量空间$V$的基,$\mathcal{C}=\{\vb{c}_1,\dots,\vb{c}_m\}$为向量空间$W$的基,线性变换$T:V\to W$,存在唯一$m\times n$矩阵$M$使得
\[[T(\vx)]_\mathcal{C}=M[\vx]_\mathcal{B},\,\forall\vx\in V\]
其中
\[M=\begin{bmatrix}[T(\mathbf{b}_1)]_\mathcal{C}&[T(\mathbf{b}_2)]_\mathcal{C}&\cdots&[T(\mathbf{b}_n)]_\mathcal{C}\end{bmatrix}\]
称为$T$关于基$\mathcal{B}$和$\mathcal{C}$的矩阵 (the matrix for $T$ relative to the bases $\mathcal{B}$ and $\mathcal{C}$).\\
特别地,当线性空间和基取特殊值时,可以得到我们之前求得的一些矩阵.
\begin{enumerate}
	\itemsep -1pt
	\item $T$的标准矩阵 (standard matrix for $T$)\\
	$A=\bmat{T(\vb{e}_1)}{T(\vb{e}_n)}$,当$V=W=\rn,\mathcal{B}=\mathcal{C}=\mathcal{E}$($\rn$中的标准基)
	\item $T$的$\mathcal{B}$矩阵 (the matrix for $T$ relative to $\mathcal{B}$, or the $\mathcal{B}$-matrix for $T$)\\
	$[T]_{\mathcal{B}}=\bmat{[T(\mathbf{b}_1)]_\mathcal{B}}{[T(\mathbf{b}_n)]_\mathcal{B}}$,当$V=W,\mathcal{B}=\mathcal{C}$
	\item 坐标变换矩阵 (change-of-coordinates matrix),右乘坐标可以将坐标变换成具体的$\vx$\\
	$P_\mathcal{B}=\bmat{\vb{b}_1}{\vb{b}_n}$,当$V=W=\rn,\mathcal{C}=\mathcal{E}$,$T(\vx)=\vx$
	\item 过渡矩阵/$\mathcal{B}$到$\mathcal{C}$的坐标变换矩阵 (change-of-coordinates matrix from $\mathcal{B}$ to $\mathcal{C}$)\\
	$\underset{\mathcal{C}\gets \mathcal{B}}{P}=\bmat{[\vb{b}_1]_{\mathcal{C}}}{[\vb{b}_n]_{\mathcal{C}}}$,当$V=W$,$T(\vx)=\vx$
\end{enumerate}
\end{theorem}
\begin{analysis}
$1\degree\;$存在性:
\[\begin{aligned} [T(\vx)]_{\mathcal{C}}&= [T(x_1\vb{b}_1+\cdots+x_n\vb{b}_n)]_{\mathcal{C}}\\
&= x_1[T(\vb{b}_1)]_{\mathcal{C}}+\cdots+x_n[T(\vb{b}_n)]_{\mathcal{C}}\\
&= \bmat{[T(\vb{b}_1)]_{\mathcal{C}}}{[T(\vb{b}_n)]_{\mathcal{C}}}\begin{bmatrix}x_1\\\vdots\\x_n\end{bmatrix}\\
&= \bmat{[T(\vb{b}_1)]_{\mathcal{C}}}{[T(\vb{b}_n)]_{\mathcal{C}}}[\vx]_{\mathcal{B}}\\
&= M[\vx]_{\mathcal{B}}
\end{aligned}\]
$2\degree\;$唯一性:
设存在另一矩阵$M'$使得$[T(\vx)]_{\mathcal{C}}=M'[\vx]_{\mathcal{B}}$,故$M[\vx]_{\mathcal{B}}=M'[\vx]_{\mathcal{B}}$,令$\vx=\vb{b}_i$即得$M=M'$
\end{analysis}
\begin{proposition}
若$A=PDP^{-1}$,$D$是$n\times n$的对角矩阵,若$\mathcal{B}$是$\col P$的一组基,则$D$是线性变换$\vx\mapsto A\vx$的$\mathcal{B}$-矩阵.
\end{proposition}
\begin{analysis}
\[\begin{aligned}
[T]_{\mathcal{B}}&=\bmat{[T(\mathbf{b}_1)]_\mathcal{B}}{[T(\mathbf{b}_n)]_\mathcal{B}}\\
&=\bmat{[A\vb{b}_1]_\mathcal{B}}{[A\vb{b}_n]_\mathcal{B}}\\
&=\bmat{P^{-1}A\vb{b}_1}{P^{-1}A\vb{b}_n}\quad\mbox{坐标变换}\\
&=P^{-1}A\bmat{\vb{b}_1}{\vb{b}_n}\\
&=P^{-1}AP=D
\end{aligned}\]
求解方法$\begin{bmatrix}P&AP\end{bmatrix}\thicksim\begin{bmatrix}I&P^{-1}AP\end{bmatrix}$. 由上面分析知$D$不需是对角矩阵,只要$A$与$D$相似即可.
\end{analysis}
\par 
如下图所示\footnote{图根据课本重绘,话摘自 线性变换的矩阵为什么要强调在这组基下? - 匡世珉的回答 - 知乎
\url{https://www.zhihu.com/question/22218306/answer/88697757}},一个线性变换$T$对于标准基(或其他基)的矩阵为$A$,为了更清楚地通过矩阵看出这个线性变换的效果,就将$A$对角化:$A=PDP^{-1}$.这其实相当于先把标准基换成由特征向量组成的基($P^{-1}$的意义),于是每一个基向量在经过$T$变换后都只是乘了个常数($D$的意义),最后再把由特征向量组成的基换回标准基($P$的意义). 因此对角化其实是用一组比标准基更好的基来描述线性变换,也就是由特征向量组成的基. 至于更好的基,由特征向量组成的规范正交基(谱定理)描述的则更好.%Jordan form
\begin{table}[!htbp]%摆放位置
\begin{center}
\begin{tabular}{cp{1cm}<{\centering}p{1.4cm}<{\centering}}
$\vx$ & $\xrightarrow{\quad\text{乘}A\quad}$ & $A\vx$\\
$\xdownarrow{0.8cm}\mathllap{\scriptstyle\text{乘}P^{-1}\quad}$ &  & $\quad\mathrlap{\xuparrow{0.8cm}}{\scriptstyle\quad\text{乘}P}$\\
$[\vb{x}]_{\mathcal{B}}$ & $\xrightarrow{\quad\text{乘}D\quad}$ & $[A\vb{x}]_{\mathcal{B}}$
\end{tabular}
\end{center}
\end{table}
\begin{definition}[迹(trace)]
$A$为$n\times n$的方阵,$A$主对角线上元素之和称为$A$的迹,定义为
\[\tr A=\sum_{i=1}^na_{ii}\]
\end{definition}
\begin{theorem}
若$A,B$均为$n\times n$的方阵,则$\tr(AB)=\tr(BA)$
\end{theorem}
\begin{analysis}
\[\tr(AB)=\sum_{i=1}^n\sum_{j=1}^na_{ij}b_{ji}=\sum_{j=1}^n\sum_{i=1}^na_{ji}b_{ij}=\sum_{i=1}^n\sum_{j=1}^nb_{ij}a_{ji}=\tr(BA)\]
\end{analysis}
\begin{theorem}[相似性一些定理]已知$A$与$B$相似,
\begin{enumerate}
	\itemsep -3pt
	\item 若$A$可逆,则$A^{-1}$与$B^{-1}$相似
	\item 则$A^k$与$B^k$相似
	\item $A$与$C$相似,则$B$与$C$相似
	\item $A$可对角化,则$B$也可对角化
	\item $A=PBP^{-1}$,$\vx$是$A$关于$\lambda$的特征向量,则$P^{-1}\vx$是$B$关于$\lambda$的特征向量
	\item $\rank A=\rank B$
\end{enumerate}
\end{theorem}