% !TEX root = main.tex

\section{对称矩阵与二次型}
\subsection{谱分解}
\begin{definition}[对称矩阵]
$A^T=A$(显然得是方阵)
\end{definition}
\begin{theorem}
$A$为对称矩阵,任两个不同特征空间的特征向量一定正交
\end{theorem}
\begin{analysis}
设$\vv_1,\vv_2$为不同特征值$\lambda_1,\lambda_2$对应的特征向量
\[\begin{aligned}\lambda_1\vv_1\cdot\vv_2&=(\lambda_1\vv_1)^T\vv_2=(A\vv_1)^T\vv_2\\
&=\vv_1^TA^T\vv_2=\vv_1^T(A\vv_2)\\
&=\lambda_2\vv_1\cdot\vv_2\end{aligned}\]
进而$(\lambda_1-\lambda_2)\vv_1\cdot\vv_2=0$,而$\lambda_1,\lambda_2$不同,故$\vv_1\cdot\vv_2=0$
\end{analysis}
\begin{definition}[正交对角化]
若存在对角矩阵$D$和正交矩阵$P$使得$A=PDP^T=PDP^{-1}$,则称$A$可正交对角化
\end{definition}
\begin{myalgorithm}[正交对角化]\mbox{}\par
\begin{enumerate}
	\itemsep -3pt
	\item 先将矩阵一般对角化,暂不用求$P^{-1}$
	\item 然后将$P$的列施密特\textbf{单位}正交化,使$P$变为正交矩阵
	\item $P^T=P^{-1}$直接求得$P$的逆
\end{enumerate}
\end{myalgorithm}
\begin{theorem}[谱定理]
$A$为$n\times n$的对称矩阵,则
\begin{enumerate}
	\itemsep -3pt
	\item $A$有$n$个实特征值(包含重数)
	\item $\mathrm{dim}\;W_{\lambda_i}=\lambda_i$的代数重数
	\item 特征空间相互正交,对应不同特征值的特征向量都正交
	\item $A$可以被正交对角化(\textbf{充要条件})
\end{enumerate}
注:正交矩阵不一定可以被正交对角化,正交对角化一定会产生正交矩阵,要理清关系.
\end{theorem}
\begin{theorem}[谱分解]
$A$为对称矩阵,记正交矩阵$P$的列为$\vu_i$,则$A=\lambda_1\vu_1\vu_1^T+\lambda_2\vu_2\vu_2^T+\cdots+\lambda_n\vu_n\vu_n^T$
\end{theorem}
\begin{definition}[投影矩阵]
$A$是对称矩阵且$A^2=A$,则称其为投影矩阵
\end{definition}
\begin{proposition}
若$A=\vu\vu^T$,其中$\vu\in\rn$为\textbf{单位}向量,则
\begin{enumerate}
	\itemsep -3pt
	\item $A$是投影矩阵
	\item $A\vx=\opproj_{\vu}\vx$
	\item $\vu$是$A$的特征向量
\end{enumerate}
\end{proposition}

\subsection{奇异值分解}%7.4 奇异值分解与子空间
\begin{theorem}
$A^TA$的特征值均为非负数
\end{theorem}
\begin{analysis}
$\forall\vu_i,\,\|A\vu_i\|^2=(A\vu_i)^TA\vu_i=\vu_i^T(A^TA)\vu_i=\vu_i^T\lambda_i\vu_i=\lambda_i\vu_i\cdot\vu_i=\lambda_i\|\vu_i\|\geq 0\implies \lambda_i\geq 0$,故记$\sigma_i=\sqrt{\lambda_i}=\|A\vu_i\|$为$A$的奇异值,注意这里的$\lambda_i$都为$A^TA$的特征值.
\end{analysis}
\begin{definition}[奇异值]
$A$的奇异值是$A^TA$的特征值的算术平方根,记为$\sigma_i:=\sqrt{\lambda_i},\forall 1\leq i\leq n$,且呈递减序列.
\end{definition}
\begin{theorem}
$A$为$m\times n$矩阵,$\disp A^TA=\sum_{i=1}^n\lambda_i\vu_i\vu_i^T$,假设有$r$个非零特征值,不妨设$\lambda_1\geq\lambda_2\geq\cdots\geq\lambda_r>\lambda_{r+1}=\cdots=\lambda_n=0$,则$\{A\vu_1,\dots,A\vu_r\}$是$\opcol A$正交基,且$\oprank A=r$.
\end{theorem}
\begin{analysis}
$\|A\vu_i\|=\sqrt{\lambda_i}\ne 0,i=1,\dots,r$\\
正交性:$\forall i\ne j,\,(A\vu_i)\cdot(A\vu_j)=\vu_i^TA^TA\vu_j=\lambda_j\vu_i^T\vu_j=0$\\
任取$\vv\in\opcol A$,即$\exists\vx\in\rn\,s.t.\,A\vx=\vv$,其中$\vx=c_1\vu_1+\cdots+c_n\vu_n$,则$\vv=c_1A\vu_1+\cdots+c_nA\vu_n=c_1A\vu_1+\cdots+c_rA\vu_r+0+\cdots+0$,即说明$\vv$可由$\{A\vu_1,\dots,A\vu_r\}$线性表示,故证毕.
\end{analysis}
\begin{theorem}[奇异值分解(SVD)]
$A$是$m\times n$的矩阵,$\oprank A=r$,则存在$m\times n$矩阵$\Sigma=\begin{bmatrix}D&O\\O&O\end{bmatrix}$,其中$D$的对角线是前$r$个$A$的奇异值,且$\sigma_1\geq\sigma_2\geq\cdots\geq\sigma_r>0$,那么存在$m\times m$正交矩阵$U$和$n\times n$正交矩阵$V$使得$A=U\Sigma V^T$.
\end{theorem}
\begin{analysis}
设$\disp A^TA=\sum_{i=1}^n\lambda_i\vu_i\vu_i^T$,其有$r$个非零特征值,$\lambda_1\geq\lambda_2\geq\cdots\geq\lambda_r>0$.\\
$\{A\vu_1,\dots,A\vu_r\}$是$\opcol A$正交基,将其标准化,得$\{\vu_1,\dots,\vu_r\}$,其中$\vu_i=\dfrac{A\vv_i}{\|A\vv_i\|}=\dfrac{1}{\sigma_i}A\vv_i$,即$A\vv_i=\sigma_i\vu_i$.\\
将$\{\vu_1,\dots,\vu_r\}$拓展为$\mathbb{R}^m$的正交基$\{\vu_1,\dots,\vu_m\}$,令
\[U=\begin{bmatrix}\vu_1&\cdots&\vu_m\end{bmatrix},\,V=\begin{bmatrix}\vv_1&\cdots&\vv_n\end{bmatrix}\]
那么$U,V$均为正交矩阵,而且
\[AV=\begin{bmatrix}A\vv_1&\cdots&A\vv_r&\vb{0}\cdots\vb{0}\end{bmatrix}=\begin{bmatrix}\sigma_1\vu_1&\cdots&\sigma_r\vu_r&\vb{0}\cdots\vb{0}\end{bmatrix}\]
进而
\[\begin{aligned}U\Sigma&=\begin{bmatrix}\vu_1&\cdots&\vu_m\end{bmatrix}\begin{bmatrix}D&O\\O&O\end{bmatrix}\\
&=\begin{bmatrix}A\vv_1&\cdots&A\vv_r&\vb{0}\cdots\vb{0}\end{bmatrix}=\begin{bmatrix}\sigma_1\vu_1&\cdots&\sigma_r\vu_r&\vb{0}\cdots\vb{0}\end{bmatrix}\\
&=AV\end{aligned}\]
因为$V$为正交矩阵,故$U\Sigma V^T=AVV^T=A$.
\end{analysis}
\begin{myalgorithm}[奇异值分解]
分为以下几个步骤
\begin{enumerate}
	\itemsep -3pt
	\item 对$A^TA$正交对角化
	\item 建立起$\Sigma$和$V$
	\item 建立起$U$
\end{enumerate}
\end{myalgorithm}


\subsection{二次型}
\begin{definition}[二次型]
$Q(\vx)=\vx^TA\vx$,其中$A$是对称矩阵
\end{definition}
将二次型合并为矩阵的写法,平方项放对角线,交叉项取一半对称写. 以三元二次型为例,观察下面各个元素的去向.
\[\begin{aligned}Q(\vx)=\vx^TA\vx&=\begin{bmatrix}x_1&x_2&x_3\end{bmatrix}\begin{bmatrix}a_{11}&a_{12}&a_{13}\\a_{21}&a_{22}&a_{23}\\a_{31}&a_{32}&a_{33}\end{bmatrix}\begin{bmatrix}x_1\\x_2\\x_3\end{bmatrix}\\
&=a_{11}x_1^2+a_{22}x_2^2+a_{33}x_3^2+(a_{12}+a_{21})x_1x_2+(a_{13}+a_{31})x_1x_3+(a_{23}+a_{32})x_2x_3\end{aligned}\]
\begin{theorem}[主轴定理]
$\vx=P\vb{y}$,存在$P$将二次型$\vx^TA\vx$转为$\vb{y}^TD\vb{y}$,其中$P$的列($A$的单位特征向量)称为$\vx^TA\vx$的主轴.
\end{theorem}
\begin{myalgorithm}[标准型变换]
将矩阵$A$\textbf{正交对角化}后,得到$A=PDP^{-1}$,做变换$\vx=P\vb{y}$,可得$\vx^TA\vx=\vb{y}^TD\vb{y}$,由于$D$为对角矩阵,由上面的分析即可知变换后的二次型不存在交叉项,这种形式称为\textbf{标准型}. 又$\vx=P\vy$,令$\vy=\vb{e}_i$(变换后的主轴在坐标轴上),可得$\vx=\vu_i$,即原来的主轴为$P$的列,也即$A$的特征向量.
\end{myalgorithm}
\begin{definition}
对于二次型$Q(\vx)$
\begin{enumerate}
	\itemsep -3pt
	\item 若$Q(\vx)>0\;\forall \vx\ne\vb{0}$,正定$\iff\forall\lambda>0$;$Q(\vx)\geq0\;\forall \vx\ne\vb{0}$,半正定
	\item 若$Q(\vx)<0\;\forall \vx\ne\vb{0}$,负定$\iff\forall\lambda<0$;$Q(\vx)\leq0\;\forall \vx\ne\vb{0}$,半负定
	\item $Q(\vx)$正负值都有,不定
\end{enumerate}
\end{definition}
\begin{analysis}
对二次型进行标准型变换使其没有交叉项,$Q(\vx)=\vx^TA\vx=\vb{y}^TD\vb{y}=\lambda_1y_1^2+\cdots+\lambda_ny_n^2$,进而得证.
\end{analysis}
\begin{proposition}
\begin{enumerate}
\itemsep -3pt
\item $B$为$m\times n$矩阵,$B^TB$为半正定;若$B$为方阵且可逆,则$B^TB$正定.
\begin{analysis}
$\vx^T(B^TB)\vx=\|B\vx\|\geq 0$,可逆则$B\vx=\vb{0}$只有平凡解.
\end{analysis}
\item $A$为方阵且正定,那么存在一个正定方阵$B$使得$A=B^TB$
\end{enumerate}
\end{proposition}
\begin{proposition}
$A,B$均为方阵,且特征值均为正数,则$A+B$的特征值也全为正数.
\end{proposition}
\begin{analysis}
$\vx^TA\vx>0,\vx^TB\vx>0\implies\vx^T(A+B)\vx>0$
\end{analysis}

\subsection{带约束的最值问题}
\begin{theorem}
二次型$Q$,$\disp\min_{\|\vx\|=1}Q(\vx)=\lambda_{\min}$,$\disp\max_{\|\vx\|=1}Q(\vx)=\lambda_{\max}$
\end{theorem}
\begin{analysis}
化为标准型后直接放缩即可.
\end{analysis}
\begin{theorem}
$A$为$m\times n$矩阵,$\disp\min_{\|\vx\|=1}\|A\vx\|=\lambda_{\min}$,$\disp\max_{\|\vx\|=1}\|A\vx\|=\lambda_{\max}$
\end{theorem}
\begin{analysis}
$\|A\vx\|^2=(A\vx)^TA\vx=\vx^T(A^TA)\vx$进而转化为二次型. 实际上$\|A\vx\|\leq\lambda\|x\|$.
\end{analysis}