% !TEX root = main.tex

\section{内积空间}
\subsection{基本定义}
\begin{definition}[内积空间]
线性空间$V$上的内积是一个函数$\langle\cdot,\cdot\rangle:V\times V\to\mathbb{R}$,$\forall u,v,w\in V,c$为常数,满足
\begin{enumerate}
	\itemsep -3pt
	\item $\langle \mathbf{u},\mathbf{v} \rangle=\langle \mathbf{v},\mathbf{u} \rangle$
	\item $\langle \mathbf{u}+\mathbf{v},\mathbf{w} \rangle=\langle \mathbf{u},\mathbf{w} \rangle+\langle \mathbf{v},\mathbf{w} \rangle$
	\item $\langle c\mathbf{u},\mathbf{v} \rangle=c\langle \mathbf{u},\mathbf{v} \rangle$
	\item $\langle \mathbf{u},\mathbf{u} \rangle\geq 0$,且$\langle \mathbf{u},\mathbf{u} \rangle=0$当且仅当$u=0$
\end{enumerate}
定义了内积的向量空间称为内积空间.
\end{definition}
\begin{definition}[长度/模/范数]
$V$是一个内积空间,则对于某一向量$v\in V$的长度为$\|v\|:=\sqrt{\langle v,v\rangle}$.
\end{definition}
注:如无特殊说明,下文在谈论到内积时都是指最简单不加权的内积,即$\lrang{\vu,\vv}=\vu\cdot\vv=\vu^\T \vv$.

\subsection{正交}
\begin{definition}[正交]
若$\langle u,v\rangle=0$,则称$u,v$正交.\\
注:与高中的定义不同,这里我们可以说$\mathbf{0}$与任意向量正交.
\end{definition}
下面这些定理看似简单,因为有二维三维空间的几何直观作为基础,但拓展到高维空间这些定理还是否会成立呢?我们不得而知,因此还应从公理出发一一证明.
\begin{theorem}[毕达哥拉斯(Pythagoras)定理]
$u,v$正交当且仅当$\|\mathbf{u}\|^2+\|\mathbf{v}\|^2=\|\mathbf{u}+\mathbf{v}\|^2$
\end{theorem}
\begin{definition}
设$W$是向量空间$V$的子空间,$V$中与所有与$W$正交的向量构成的集合称为$W$的正交补,记为$W^\perp$.
\end{definition}
以下有两条性质
\begin{enumerate}
	\itemsep -3pt
	\item $\mathbf{x}\in W^\perp\iff\mathbf{x}$与张成$V$的一组向量中的每一个都正交
	\item $W^\perp$是$V$的子空间
\end{enumerate}
\begin{theorem}[正交必定线性无关]
\label{ortho_indep}
$S=\{\mathbf{u}_1,\dots,\mathbf{u}_p\}$是$V$的一个由\textbf{非零}向量构成的正交集,那么$S$线性无关,因此构成$\mathrm{Span}\;S$的基.\\
注:一定要注意前提条件!!!
\end{theorem}
\begin{analysis}
\[\vzero=c_1\vu_1+c_2\vu_2+\cdots+c_p\vu_p\]
\[\begin{aligned}0&=\lrang{\vzero,\vu_1}=\lrang{c_1\vu_1+c_2\vu_2+\cdots+c_p\vu_p,\vu_1}\\
&=\lrang{c_1\vu_1,\vu_1}+\lrang{c_2\vu_2,\vu_1}+\cdots+\lrang{c_p\vu_p,\vu_1}\\
&=c_1\lrang{\vu_1,\vu_1}+c_2\lrang{\vu_2,\vu_1}+\cdots+c_p\lrang{\vu_p,\vu_1}\\
&=c_1\lrang{\vu_1,\vu_1}
\end{aligned}\]
因为$\vu_1\ne\vzero$,所以$c_1=0$. 类似地,可以推出$c_i=0,i=2,\dots,p$,因此$S$线性无关.
\end{analysis}
\begin{theorem}[正交分解]
$\{\mathbf{u}_1,\dots,\mathbf{u}_p\}$是$V$的子空间$W$的一组正交基,对于$\vy\in V$,都可以唯一表示成$\vy=\hat{\vy}+\vz$,其中$\hat{\vy}\in W,\vz\in W^\perp$,且
\[\hat{\mathbf{y}}=c_1\mathbf{u}_1+\cdots+c_p\mathbf{u}_p\]
各项为
\[\hat{\mathbf{y}}_i=\mathrm{proj}_{u_i}\hat{\mathbf{y}}=c_i\mathbf{u}_i=\frac{\langle \hat{\mathbf{y}},\mathbf{u}_i\rangle}{\langle\mathbf{u}_i,\mathbf{u}_i\rangle}\mathbf{u}_i=\frac{\langle \hat{\mathbf{y}},\mathbf{u}_i\rangle}{\|\mathbf{u}_i\|}\frac{\mathbf{u}_i}{\|\mathbf{u}_i\|}\,,i=1,\dots,p\qquad(*)\]
注:与选的基没有关系,只和$W$有关.
\end{theorem}
\begin{analysis}
类似定理\ref{ortho_indep}的证明有$\lrang{\hat{\vy},\vu_i}=\lrang{c_1\vu_1+c_2\vu_2+\cdots+c_p\vu_p,\vu_i}=c_i\lrang{\vu_i,\vu_i}$,移项即得$(*)$式.
\[\begin{aligned}\lrang{\vz,\vu_1}=\lrang{\vy-\hat{\vy},\vu_1}&=\lrang{\vy,\vu_1}-\lrang{\frac{\langle \hat{\mathbf{y}},\mathbf{u}_1\rangle}{\langle\mathbf{u}_1,\mathbf{u}_1\rangle},\mathbf{u}_1}{\vu_1}-0-\cdots-0\\
&=\lrang{\vy,\vu_1}-\lrang{\vy,\vu_1}=0\end{aligned}\]
类似地可证明$\vz$与$W$的每一个基$\vu_i$都正交,进而$\vz\in W^\perp$.\\
为证明唯一性,不妨设存在另一表示方法$\vy=\hat{\vy}'+\vz'$,其中$\hat{\vy}'\in W$,$\vz'\in W^\perp$那么$\hat{\vy}+\vz=\hat{\vy}'+\vz'$,即$\hat{\vy}-\hat{\vy}'=\vz'-\vz$. 因为$\vz,\vz'\in W^\perp$,所以$\vz'-\vz\in W^\perp$,故$\hat{\vy}-\hat{\vy}'\in W^\perp$,而$\hat{\vy}-\hat{\vy}'\in W$,推出$\lrang{\hat{\vy}-\hat{\vy}',\hat{\vy}-\hat{\vy}'}=0$,即$\hat{\vy}=\hat{\vy}'$,同时$\vz'=\vz$.
\end{analysis}
\par 从$(*)$的最后一个等式可以看出,实际上正交分解与我们高中所学的向量知识是一致的,即$\mathbf{y}$在$\mathbf{u}_i$上的投影的长度再乘上该方向上的单位向量.
\begin{theorem}[最佳估计]
$W$为$V$的子空间,$\vy\in V$,则
\[\|\mathbf{y}-\hat{\mathbf{y}}\|\leq\|\mathbf{y}-\mathbf{v}\|,\forall v\in W\]
\end{theorem}
\begin{analysis}
\[\vy-\vv=(\vy-\hat{\vy})+(\hat{\vy}-\vv)\]
因为$\hat{\vy}-\vv\in W,\vy-\hat{\vy}\in W^\perp$,所以由毕达哥拉斯定理
\[\|\vy-\vv\|^2=\|\vy-\hat{\vy}\|^2+\|\hat{\vy}-\vv\|^2\]
即得结论.
\end{analysis}
\begin{theorem}[柯西(Cauchy)不等式]
\[\|\mathbf{u}\|\|\mathbf{v}\|\geq|\langle \mathbf{u},\mathbf{v} \rangle|\]
\end{theorem}
\begin{analysis}
\[\|\opproj_{\vu}\vv\|=\dfrac{\lrang{\vu,\vv}}{\|\vu\|}\leq\|\vv\|\]
\end{analysis}
\begin{theorem}[三角不等式]
\[\|\mathbf{u}\|+\|\mathbf{v}\|\geq\|\mathbf{u}+\mathbf{v}\|\]
\end{theorem}
\begin{analysis}
平方后用柯西不等式.
\end{analysis}
\begin{proposition}%P417
$A$是$m\times n$矩阵,$A\vx=\vzero$当且仅当$A^\T A\vx=\vzero$,进而$\opnul A=\opnul A^\T A$.\\
$A^\T A$可逆当且仅当$A$的列线性无关. $\oprank A^\T A=n-\opdim\opnul A^\T A=\opnul A=\oprank A$
\end{proposition}
\begin{analysis}
$\vx^\T A^\T A\vx=\vzero$
\end{analysis}

\subsection{正交矩阵}
\begin{theorem}
$U$是$m\times n$矩阵,则$U$每一列都\textbf{单位正交}当且仅当$U^\T U=I$
\end{theorem}
\begin{analysis}
\[U^\T U=\begin{bmatrix}\vu_1^\T \\\vdots\\\vu_n^\T \end{bmatrix}[\vu_1\quad\cdots\quad\vu_n]
=\begin{bmatrix}\vu_1^\T \vu_1 & \cdots & \vu_1^\T \vu_n\\
\vdots&\ddots&\vdots\\
\vu_n^\T \vu_1 & \cdots & \vu_n^\T \vu_n\\\end{bmatrix}
=\begin{bmatrix}1 & \cdots & 0\\
\vdots&\ddots&\vdots\\
0 & \cdots & 1\\\end{bmatrix}=I\]
\end{analysis}
\begin{theorem}
$U$是$m\times n$矩阵且每一列都单位正交,$x,y\in\mathbb{R}^n$
\begin{enumerate}
	\itemsep -3pt
	\item $\|U\mathbf{x}\|=\|\mathbf{x}\|$
	\item $(U\mathbf{x})\cdot(U\mathbf{y})=\mathbf{x}\cdot\mathbf{y}$
	\item $(U\mathbf{x})\cdot(U\mathbf{y})=0\iff\mathbf{x}\cdot\mathbf{y}=0$
\end{enumerate}
\end{theorem}
\begin{analysis}
$(U\mathbf{x})\cdot(U\mathbf{y})=(U\mathbf{x})^\T (U\mathbf{y})=\mathbf{x}^\T U^\T U\mathbf{y}=\mathbf{x}\cdot\mathbf{y}$,
$1,3$同理可证.
\end{analysis}
\begin{definition}[正交矩阵]
若$U$为\textbf{方阵}且每一列都\textbf{单位正交},则称$U$为正交矩阵.\\
注:从定义上看似乎应该叫单位正交矩阵(orthonormal matrix)比较合适,但实际上在线性代数中正交矩阵(orthogonal matrix)已经包含了单位正交的意思.
\end{definition}
由以上定理和定义可知下面几条显然的结论.
\begin{proposition}
\begin{enumerate}
	\itemsep -3pt
	\item $U$为方阵则一定可逆,因$U^\T U=I=U^{-1}U=I$
	\item $U,V$均为正交矩阵,则$UV$也为正交矩阵
	\item $U$中的列交换一下得到$V$,则$V$也为正交矩阵
\end{enumerate}
\end{proposition}
\begin{proposition}
$\{\mathbf{u}_1,\dots,\mathbf{u}_p\}$是$V$的子空间$W$的一组单位正交基,则$\opproj_{W}\vy=UU^\T \vy,\,\forall\vy\in V$
\end{proposition}
\begin{myalgorithm}[施密特(Schmidt)正交化]
\label{schmidt}
设$\{\mathbf{v}_1,\dots,\mathbf{v}_p\}$是$W$的一组基,定义
\[\begin{aligned}\mathbf {u} _{1}&=\mathbf {v} _{1}\\
\mathbf {u} _{2}&=\mathbf {v} _{2}-\mathrm {proj} _{\mathbf {u} _{1}}\vv_2\\
\vdots\\
\mathbf {u} _{p}&=\mathbf {v} _{p}-\sum _{j=1}^{p-1}\mathrm {proj} _{\mathbf {u} _{j}}\,\mathbf {v} _{p}\end{aligned}\]
则$\{\mathbf{u}_1,\dots,\mathbf{u}_p\}$是$W$的一组正交基,且
\[\mathrm{Span}\{\mathbf{v}_1,\dots,\mathbf{v}_k\}=\mathrm{Span}\{\mathbf{u}_1,\dots,\mathbf{u}_k\},\forall\;1\leq k\leq p\]
注:此过程说明了结合向量空间基的存在性可说明正交基的存在性.
\end{myalgorithm}
\begin{analysis}
设$V_k=\opspan\{\mathbf{v}_1,\dots,\mathbf{v}_k\},W_k=\opspan\{\mathbf{u}_1,\dots,\mathbf{u}_k\},1\leq k\leq p$,则$\vu_{k+1}=\vv_{k+1}-\opproj_{W_k}\vv_{k+1}$,进而$\vu_{k+1}\in W_{k+1}^\perp$,即每一$\vu_i$均与前面的$\vu_j,1\leq j\leq i-1$正交(也就保证了两两正交). 而$\vv_i\in V_i$,由向量加减法的封闭性,$\vu_i\in V_i$,故$\vu_i\in V_p$(因$V_k\subset V_{k+1}$). 又正交一定线性无关,且$V$已有一组$p$个向量的基,故$\{\mathbf{u}_1,\dots,\mathbf{u}_k\}$是$V_k$的一组正交基,进而$V_k=W_k$.\\
\end{analysis}
\begin{theorem}[QR分解]
\label{qr_fact}
若$m\times n$矩阵$A$的\textbf{列线性无关},则$A$可被分解为$A=QR$,其中$Q$是$m\times n$矩阵且它的列形成$\opcol A$的\textbf{单位正交}基,$R$是$n\times n$可逆上三角矩阵且对角线上元素全为正数.\\
注:在下面的证明可以看出$A=QR$,单位正交不是必须的,但可能因为具有某种性质,所以才要强调单位正交.
\end{theorem}
\begin{analysis}
施密特正交化将$A$的列转为正交基$\{\mathbf{u}_1,\dots,\mathbf{u}_n\}$,则$Q=\bmat{\vu_1,\ldots,\vu_n}$.\\
由算法\ref{schmidt}的分析知$\vx_k\in V_k=W_k$,故存在$r_{k1},\dots,r_{kk}$,使得
\[\vx_k=r_{k1}\vu_1+\cdots+r_{kk}\vu_k+0\cdot\vu_{k+1}+\cdots+0\cdot\vu_n\]
假设$r_{kk}\geq 0$(否则给$r_{kk}$和$\vu_k$同乘$-1$),令
\[\vr_k=\begin{bmatrix}r_{k1}\\\vdots\\r_{kk}\\0\\\vdots\\0\end{bmatrix}\]
为$R$的列,显然$R$为上三角,即可满足$\vx_k=Q\vr_k$.
\end{analysis}
\begin{myalgorithm}[QR分解]
\begin{enumerate}
	\itemsep -3pt
	\item 先将$A$的列向量单位正交化,得到$Q$
	\item 通过$R=IR=Q^\T QR=Q^\T A$计算得$R$
\end{enumerate}
\end{myalgorithm}