% !TEX root = main.tex

\section{拓展}
%4.8 微分/差分方程 5.7
\subsection{线性模型与最小二乘}%6.6 在线性模型的应用
\begin{theorem}[最小二乘]
$A$为$m\times n$的矩阵,$\mathbf{b}\in\mathbb{R}^m$,$A\mathbf{x}=\mathbf{b}$的最小二乘解是$\hat{\mathbf{x}}\in\mathbb{R}^n$使得
\[\|\vb{b}-A\hat{\vx}\|\leq\|\vb{b}-A\vx\|\]
则$\disp\hat{\vb{b}}=\mathrm{proj}_{\mathrm{Col}\;A}\vb{b}$,解与$A^TA\vx=A^T\vb{b}$相同.\\
若$A$可被$QR$分解(定理\ref{qr_fact}),则$\hat{\vx}=R^{-1}Q^T\vb{b}$.
\end{theorem}
%A^TA计算误差可能引起全局大误差,最好用QR分解,算R\vx=Q^T\vb{b}
\begin{analysis}
$\vb{a}_j\cdot(\vb{b}-A\hat{\vx})=\vb{a}_j^T(\vb{b}-A\hat{\vx})=0\implies A^T(\vb{b}-A\hat{\vx})=\vb{0}\implies A^TA\vx=A^T\vb{b}$
\end{analysis}
\begin{myalgorithm}[最小二乘估计]
直接计算$A^TA\vx=A^T\vb{b}$即可.
\end{myalgorithm}
\begin{definition}[一般线性模型]
$\vb{y}=X\bm{\beta}+\bm{\varepsilon}$,$X$为设计矩阵,$\vbb$为参数向量,$\vy$为观测向量,$\bm{\varepsilon}$为残差向量,满足这种形式的方程称为线性模型. 使$\bm{\varepsilon}$得长度最小化,即找出$X\vbb=\vy$的最小二乘解.
\end{definition}
以下是一些例子.
\begin{enumerate}
	\itemsep -3pt
	\item 直线拟合
	\[\begin{bmatrix}y_1\\y_2\\\vdots\\y_n\end{bmatrix}=\begin{bmatrix}1&x_1\\1&x_2\\\vdots&\vdots\\1&x_n\end{bmatrix}\begin{bmatrix}\beta_0\\\beta_1\end{bmatrix}+\begin{bmatrix}\varepsilon_1\\\varepsilon_2\\\vdots\\\varepsilon_n\end{bmatrix}\]
	\item 抛物线拟合
	\[\begin{bmatrix}y_1\\y_2\\\vdots\\y_n\end{bmatrix}=\begin{bmatrix}1&x_1&x_1^2\\1&x_2&x_2^2\\\vdots&\vdots&\vdots\\1&x_n&x_n^2\end{bmatrix}\begin{bmatrix}\beta_0\\\beta_1\\\beta_2\end{bmatrix}+\begin{bmatrix}\varepsilon_1\\\varepsilon_2\\\vdots\\\varepsilon_n\end{bmatrix}\]
	\item 多重回归
	\[\begin{bmatrix}y_1\\y_2\\\vdots\\y_n\end{bmatrix}=\begin{bmatrix}1&u_1&v_1\\1&u_2&v_2\\\vdots&\vdots&\vdots\\1&u_n&v_n\end{bmatrix}\begin{bmatrix}\beta_0\\\beta_1\\\beta_2\end{bmatrix}+\begin{bmatrix}\varepsilon_1\\\varepsilon_2\\\vdots\\\varepsilon_n\end{bmatrix}\]
\end{enumerate}

\subsection{马尔可夫链}%4.9 马尔科夫链
\begin{definition}[马尔可夫(Markov)链]
具有非负分量的数值且相加等于$1$的向量$\vx_0,\vx_1,\dots$称为\textbf{概率向量},各列向量均为概率向量的方阵$P$称为\textbf{随机矩阵},则\textbf{马尔可夫链}为$\vx_{k+1}=P\vx_{k},k=0,1,2,\dots$,其中$\vx_k$称为\textbf{状态向量}.
\end{definition}
\begin{theorem}[马尔可夫链收敛定理\protect\footnote{具体情况比较复杂}]
马尔可夫链$\{\vx_{k+1}\}$一定会收敛至\textbf{平衡向量}$\vb{q}$,其中$\vb{q}$满足$P\vb{q}=\vb{q}$.
\end{theorem}

\subsection{复数特征值}%5.5 复数特征值
\begin{theorem}
当$A$为实矩阵时,它的复特征值成对出现.
\end{theorem}
\begin{analysis}
$A\bar{\vx}=\bar{A}\bar{\vx}=\overline{A\vx}=\overline{\lambda\vx}=\bar{\lambda}\bar{\vx}$
\end{analysis}
\begin{theorem}
$A$为$2\times 2$实矩阵,有复特征值$\lambda=a-b\ii(b\ne 0)$及对应$\mathbb{C}^2$中的复特征向量$\vv$,则
\[A=PCP^{-1}\,\text{,其中}\,P=\begin{bmatrix}\Re\vv&\Im\vv\end{bmatrix},C=\begin{bmatrix}a&-b\\b&a\end{bmatrix}\]
\end{theorem}
\begin{analysis}
证明利用了结论:$A$是实矩阵,则$A(\Re\vx)=\Re A\vx$,$A(\Im\vx)=\Im A\vx$. $\Re\vx$与$\Im\vx$线性无关
\end{analysis}

\subsection{离散动力系统}%5.6 离散动力系统
若$A$可对角化,有$n$个线性无关的特征向量$\vv_1,\dots,\vv_n$和对应的特征值$\lambda_1,\dots,\lambda_n$,且由大到小排列. 由于$\{\vv_1,\cdots,\vv_n\}$为$\rn$的基. 故任一初始向量可唯一表示为$\vx_0=c_1\vv_1+\cdots+c_n\vv_n$,则
\[\vx_k=A^k\vx_0=c_1(\lambda_1)^k\vv_1+\cdots+c_n(\lambda_n)^k\vv_n\]
线性动力系统中,只有原点才可能是吸引子($|\lambda|<1$)或者排斥子($|\lambda|>1$),但非线性系统中可能存在多个吸引子或排斥子,其可用雅可比矩阵的特征值定义. 若特征值正负都有,则原点为鞍点.

\subsection{估计特征值}%5.8 迭代估计特征值
幂算法、逆幂法、QR算法
\par 若$A$可对角化,特征向量$\vv_1,\dots,\vv_n$是$\rn$的基,且
\[|\lambda_1|>|\lambda_2|\geq|\lambda_3|\geq\cdots\geq|\lambda_n|\]
(注意第一个符号为严格大)其中$\lambda_1$称为主特征值
\[\vx_k=A^k\vx_0=c_1(\lambda_1)^k\vv_1+\cdots+c_n(\lambda_n)^k\vv_n\]
假设$c_1\ne 0$,左右同除$\lambda_1^k$,可知$(\lambda_1)^{-k}A^k\vx\to c_1\vv_1(k\to\infty)$

\subsection{傅里叶级数}%6.8 加权最小二乘 傅里叶级数
$\forall n\geq 1,\{1,\cos t,\cos 2t,\cdots,\cos nt,\sin t,\sin 2t,\cdots,\sin nt\}$,定义内积$\inp{f}{g}=\displaystyle\int_0^{2\pi} f(t)g(t)\diff t$,知这个集合为正交集
\begin{theorem}[傅里叶(Fourier)级数]
\[f(t)=\dfrac{a_0}{2}+\sum_{k=1}^\infty(a_k\cos kt+b_k\sin kt)\]
当$k\geq 1$时,
\[a_k=\dfrac{\inp{f}{\cos kt}}{\inp{\cos kt}{\cos kt}}=\dfrac{1}{\pi}\int_0^{2\pi}f(t)\cos kt\diff t,\,b_k=\dfrac{\inp{f}{\sin kt}}{\inp{\sin kt}{\sin kt}}=\dfrac{1}{\pi}\int_0^{2\pi}f(t)\sin kt\diff t\]
常数项,
\[\dfrac{\inp{f}{1}}{\inp{1}{1}}=\dfrac{a_0}{2}\]
\end{theorem}

\subsection{统计学应用}%7.5 图像处理及统计学应用
\begin{definition}[平均值]
$p\times N$的观测矩阵$A=\bmat{\vb{X}_1}{\vb{X}_N}$(即有$N$个样本,每个样本有$p$个维度的信息),则其样本均值为$\vb{M}=\dfrac{1}{N}(\vb{X}_1+\cdots+\vb{X}_N)$. 平均偏差形式为$B=\bmat{\vb{\hat{X}}_1}{\vb{\hat{X}}_N}$,其中$\vb{\hat{X}}_k=\vb{X}_k-\vb{M}$.
\end{definition}
\begin{definition}[方差]
协方差矩阵为$S=\dfrac{1}{N-1}BB^T$($BB^T$正定,故$S$正定),其中元素$s_{ii}$称为$x_i$(某一行)的方差,$s_{ij},i\ne j$称为$x_i$与$x_j$的协方差,而总方差$\tv B=\tr S$
\end{definition}
\begin{definition}[主成分分析]
假设$A=\bmat{\vb{X}_1}{\vb{X}_N}$为平均偏差形式,找到$p\times p$正交矩阵$P=\bmat{\vu_1}{\vu_p}$使得$\vb{X}=P\vb{Y}$,其中$y_1,\dots,y_p$都线性无关且方差递减. $\vu_1,\dots,\vu_p$为数据的主成分,第一主成分是$S$最大的特征值对应的特征向量.\\
注:主成分分析等同于正交回归
\end{definition}
\begin{analysis}
$\vb{X}_k=P\vb{Y}_k\implies \vb{Y}_k=P^{-1}\vb{X}_k=P^T\vb{X}_k,k=1,\dots,N\implies S=PDP^T, P^TSP=D$\\
可以验证对于任意正交矩阵$P$,$\vb{Y},\dots\vb{Y}_N$的协方差矩阵都是$P^TSP$
\end{analysis}
变量的正交变换$\vb{X}=P\vb{Y}$不改变数据的总方差,且$\tv \vb{X}=\tv \vb{Y}=\tr D=\lambda_1+\cdots+\lambda_p$,商$\lambda_j/\tr S$表明$y_j$的占比,用于降维