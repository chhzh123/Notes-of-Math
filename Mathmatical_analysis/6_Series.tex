%\documentclass[11pt,UTF8]{ctexart}
%\usepackage{setspace}
\usepackage{geometry}
\usepackage{amsthm}
\usepackage{amsmath}
\usepackage{amsfonts}
\usepackage{amssymb}%\because\therefore
\usepackage{mathtools}%underset
\usepackage{cancel}
\usepackage{extarrows}
\usepackage{enumerate}
\usepackage{url}
\usepackage[unicode=true,%本行非常重要 支持中文目录hyperref CJKbookmarks对二级目录没用
	colorlinks,
	linkcolor=black,
	anchorcolor=black,
	citecolor=black,
	CJKbookmarks=false]{hyperref}

\newtheorem{theorem}{定理}
\newtheorem{definition}{定义}
\newtheorem{example}{例}%*去除编号
\newtheorem*{analysis}{分析}
\newtheorem{corollary}{推论}
\newtheorem*{corollary_convex}{推论}
\newtheorem{exercise}{练习}
\geometry{top=20mm,bottom=20mm,left=20mm,right=20mm}
\pagestyle{plain}%删除页眉

\def\zz{\mathbb{Z}}
\newcommand{\ssum}[1]{\displaystyle\sum_{n=1}^{\infty}#1}
%
%\setcounter{section}{6}
%\begin{document}

\section{无穷级数与广义积分}
本章中会将正项级数、一般项级数、广义积分(无穷限积分、瑕积分)、函数项级数放在一起进行比较,当然也会涉及各类级数各自的特性. 由于将内容进行整合,故最重要的各类判别法总结在~\ref{summary_conver}节才进行讲述.

\subsection{级数的基本性质}
\begin{definition}[无穷级数]
$\{u_n\}$为数列,称形式和
\[\ssum{u_n}=\lim_{n\to\infty}\sum_{k=1}^n u_k=\lim_{n\to\infty}S_n=S\]
为级数$\ssum{u_n}$的和,称其收敛到$S$.
\end{definition}
\par 可由数列极限性质直接推导出来的定理在此不再赘述.
\begin{theorem}[数项级数收敛的必要条件]
\label{series_conver}
若$\ssum{u_n}$收敛,则$\displaystyle\lim_{n\to\infty}u_n=0$
\end{theorem}
\par 注意此定理并非充分条件,如$\ssum{\ln\lrp{1+\dfrac{1}{n}}}=\ln(n+1)$发散,但其一般项趋于$0$. 并且,此定理也只适用于数项级数,广义积分并不具有这样的性质.
\par 在进行数项级数收敛性判断时,第一步就得先判定其一般项是否趋于$0$,然后再考虑其他判别方法. 如果遇到比较简单的数列,可以直接求和的,那就先求和再讨论,如下面的例~\ref{eg_series_conver1}.
\begin{example}
$\ssum{r^n\sin nx},|r|<1$
\label{eg_series_conver1}
\end{example}
\begin{analysis}
用三角函数的复数形式. 由棣莫弗公式$e^{ix}=\cos x+i\sin x$,得到$\sin x=\dfrac{e^{ix}-e^{-ix}}{2}$,其中$i^2=-1$.
\[\begin{aligned}
LHS &= \dfrac{1}{2}\ssum{r^n(e^{inx}-e^{-inx})}\\
&= \dfrac{1}{2}\lim_{n\to\infty}\left(re^{ix}\dfrac{1-(re^{ix})^n}{1-re^{ix}}-re^{-ix}\dfrac{1-(re^{-ix})^n}{1-re^{-ix}}\right)\qquad\mbox{等比数列求和}\\
&\xlongequal{z=re^{ix}} \dfrac{1}{2}\left(\dfrac{z}{1-z}-\dfrac{\bar{z}}{1-\bar{z}}\right)\qquad\mbox{因$|z|<1$故$\displaystyle\lim_{n\to\infty}z^n=0$}\\
&= \dfrac{1}{2}\dfrac{z-\bar{z}}{1-(z+\bar{z})+r^2}\qquad\mbox{$z\bar{z}=r^2$,通分}\\
&= \dfrac{r\sin x}{1-2r\cos x+r^2}
\end{aligned}\]
\par 注:当然也可以用三角函数积化和差和差化积去做,但相比起复数法会复杂很多
\end{analysis}
关于数列求和的各种技巧,如裂项求和、错位相消、分子有理化、三角恒等变形等均要熟悉.


\subsubsection{正项级数}
\begin{definition}[正项级数]
级数的每一项都\textbf{非负},则称其为正项级数. 由此知正项级数一定为实级数.
\end{definition}
\begin{theorem}[正项级数收敛的充要条件]
正项级数$\ssum{u_n}$收敛的充要条件是$\{S_n\}$有上界
\end{theorem}
\begin{analysis}
	必要性:极限有界性;充分性:单调有界原理
\end{analysis}
\begin{example}p级数
\[\ssum{\dfrac{1}{n^p}}\]
当$p\leq 1$时发散,$p>1$时收敛($p=1$时也称为调和级数)
\end{example}
\par 下面的柯西积分判别法是用连续量来研究离散量很好的例子,遇到$\ln n,\dfrac{1}{n^p}$等可考虑
\begin{theorem}[柯西积分判别法]
$f(x)$在$[0,+\infty)$连续且单调下降,$u_n=f(n)$,则$\ssum{u_n}$收敛充要条件是
\[\displaystyle\lim_{x\to+\infty}\int_1^xf(t)\diff t\]
存在
\end{theorem}
\begin{example}
\[\sum_{n=2}^{\infty}{\dfrac{1}{n^p\ln n}},p\in\rr\]
\end{example}
\begin{analysis}
\begin{enumerate}
	\item 当$p=1$时,$f(x)=\dfrac{1}{x\ln x}$在$[2,+\infty)$单调递减
	\[\limtoinf\intab{2}{n}{\dfrac{1}{x\ln x}}=\ln\ln x\Big|_2^{+\infty}=+\infty\]
	由柯西积分判别法,原级数发散
	\item 当$p\leq 1$时,$\dfrac{1}{n^p\ln n}\geq\dfrac{1}{n\ln n}$,由比较判别法,原级数发散
	\item 当$p>1$时,$\dfrac{1}{n^p\ln n}<\dfrac{1}{n^p}$,由比较判别法,原级数收敛
\end{enumerate}
\end{analysis}

\subsubsection{一般项级数}
\begin{definition}[交错级数]
$\ssum{(-1)^{n-1}u_n},u_n>0$
\end{definition}
\par 下面的莱布尼茨判别法是判断交错级数收敛简便而有效的方法.
\begin{theorem}[莱布尼茨(Leibniz)判别法]
交错级数$\ssum{(-1)^{n-1}u_n}$的一般项$\{u_n\}$单调下降趋于$0$,则交错级数收敛
\end{theorem}
\par 注意莱布尼茨判别法是判断收敛性,而不是判断绝对收敛性的,后面多种判别法一起运用时经常会搞混. 由于前面的有限项对最终的极限没有影响,因此交错级数的一般项只要从某个$n$开始单调下降趋于$0$就可以使用莱布尼茨判别法了.
\par 当然不是所有交错级数都能让你一眼望穿,直接使用判别法,如下面这个例子就需要进行拆分,再分别判断.
\begin{example}
\[\sum_{n=2}^{\infty}\dfrac{(-1)^n}{\sqrt{n}+(-1)^n}\]
\end{example}
\begin{analysis}泰勒展开
\[\dfrac{(-1)^n}{\sqrt{n}+(-1)^n}=\dfrac{(-1)^n}{\sqrt{n}}\lrp{1+\dfrac{(-1)^n}{\sqrt{n}}}^{-1}
=\dfrac{(-1)^n}{\sqrt{n}}\left[1-\dfrac{(-1)^n}{\sqrt{n}}+\mathcal{O}\lrp{\dfrac{1}{n}}\right]
=\dfrac{(-1)^n}{\sqrt{n}}-\dfrac{1}{n}+\mathcal{O}\lrp{\dfrac{1}{n^{\frac{3}{2}}}}\]
因$\nums{\dfrac{1}{\sqrt{n}}}$单调下降趋于$0$,由莱布尼茨判别法,$\displaystyle\sum_{n=2}^{\infty}\dfrac{(-1)^n}{\sqrt{n}}$收敛\\
又由$p$级数的性质知$\displaystyle\sum_{n=2}^{\infty}\dfrac{1}{n^{\frac{3}{2}}}$收敛.\\
但$\displaystyle\sum_{n=2}^\infty\dfrac{1}{n}$发散,故原级数发散.
\end{analysis}
\par 下面的阿贝尔变换和阿贝尔引理均为阿贝尔判定法的铺垫,证明均不难,记住几何意义可以现推.
\begin{theorem}[阿贝尔(Abel)变换]
两组数列$\{a_n\},\{b_n\}$,设$\{b_n\}$的部分和数列为$\{B_n\}$,则
\[\begin{aligned}
\sum_{i=1}^na_ib_i&=\sum_{i=1}^na_i(B_i-B_{i+1})\\
&=\sum_{i=1}^{n-1}(a_i-a_{i+1})B_i+a_nB_n
\end{aligned}\]
\end{theorem}
\begin{theorem}[阿贝尔引理]
设$\{a_n\}$单调,$\{B_n\}$为$\{b_n\}$的部分和有界$M$,则
\[\left|\sum_{i=1}^na_ib_i\right|\leq M(|a_1|+2|a_n|)\]
\end{theorem}

\subsubsection{代数运算}
\begin{enumerate}
	\item 结合律:\textbf{条件/绝对收敛}级数任意加括号,和不变;但不能随意去括号
	\item 交换律:\textbf{绝对收敛}成立;条件收敛适当重排,可使新级数发散或收敛到某一特定值(Riemman)
	\item 分配律:\textbf{绝对收敛}成立(Cauchy)
\end{enumerate}


\subsection{广义积分}
\begin{definition}[无穷限积分]
设$f(x)$在$[a,+\infty)$有定义,并且在任意有限区间$[a,A]$上可积,定义
\[\int_a^{+\infty}f(x)\diff x:=\lim_{A\to+\infty}\int_a^Af(x)\diff x=I\]
为$[a,+\infty)$的无穷限积分
\end{definition}
\begin{definition}[瑕积分]
设$f(x)$在$(a,b]$有定义,并且在任意区间$[a+\eta,b]$上可积($\eta>0$),在$(a,a+\eta]$无界,定义
\[\intab{a}{b}{f(x)}:=\lim_{\eta\to0^+}\intab{a+\eta}{b}{f(x)}\]
为$[a,+\infty)$的瑕积分
\end{definition}
\par 注意在无穷项积分中$\intab{a}{+\infty}{f(x)}$收敛并不能推出$f(x)\to0(x\to+\infty)$,这与数项级数不同,如例~\ref{countereg_conver_integ}说明了这一点.
\par 下面的积分第二中值定理类似于数列的阿贝尔变换,用于证明广义积分的阿贝尔判定法.
\begin{theorem}[积分第二中值定理]
设$f(x)$在$[a,b]$上可积,而$g(x)$在$[a,b]$上单调,则在$[a,b]$中存在$\xi$使得
\[\intab{a}{b}{f(x)g(x)}=g(a)\intab{a}{\xi}{f(x)}+g(b)\intab{\xi}{b}{f(x)}\]
\end{theorem}
\par 在进行广义积分的收敛性判定时,一般先尝试其是否可以进行正常的积分,即求不定积分(详细方法见第~\ref{section_integration}章),之后才考虑用各类判别法. 并且在实际运算过程中要记得分段,最好每一个区间只含一个瑕点,在端点或者内点都可.
\begin{example}
\[\intab{0}{+\infty}{\dfrac{\sqrt{x}}{1+x^2}}\]
\end{example}
\begin{analysis}
如果只是判断收敛性,则用比较判别法
\[\lim_{x\to\infty}\dfrac{\sqrt{x}}{1+x^2}\Big/\dfrac{1}{x^{\frac{3}{2}}}=\lim_{x\to\infty}\dfrac{x^2}{1+x^2}=1\]
而$\intab{0^+}{+\infty}{\dfrac{1}{x^{\frac{3}{2}}}}$收敛(注意瑕点),故原积分收敛.\\
如果要求值,则变得相当麻烦
\[\begin{aligned}
  \intab{0}{+\infty}{\dfrac{\sqrt{x}}{1+x^2}}
  &=\intabu{0}{+\infty}{\dfrac{2u^2}{1+u^4}{u}}\\
  &=\intabu{0}{+\infty}{\left(-\dfrac{\sqrt{2}/2u}{u^2+\sqrt{2}u+1}+\dfrac{\sqrt{2}/2u}{u^2-\sqrt{2}u+1}\right)}{u}\\
  &=\dfrac{\sqrt{2}}{2}\Bigg(\dfrac{1}{2}\int_0^{+\infty}\dfrac{\diff(u^2-\sqrt{2}u+1)}{u^2-\sqrt{2}u+1}+\dfrac{\sqrt{2}}{2}\int_0^{+\infty}\dfrac{\diff\left(u-\dfrac{\sqrt{2}}{2}\right)}{\left(u-\dfrac{\sqrt{2}}{2}\right)^2+\left(\dfrac{\sqrt{2}}{2}\right)^2}\\
  &\quad-\dfrac{1}{2}\int_0^{+\infty}\dfrac{\diff(u^2+\sqrt{2}u+1)}{u^2+\sqrt{2}u+1}+\dfrac{\sqrt{2}}{2}\int_0^{+\infty}\dfrac{\diff\left(u+\dfrac{\sqrt{2}}{2}\right)}{\left(u+\dfrac{\sqrt{2}}{2}\right)^2+\left(\dfrac{\sqrt{2}}{2}\right)^2}\Bigg)\\
  &=\dfrac{\sqrt{2}}{2}\left(\dfrac{1}{2}\ln\dfrac{u^2-\sqrt{2}u+1}{u^2+\sqrt{2}u+1}+\dfrac{\sqrt{2}}{2}\arctan(\sqrt{2}u-1)+\dfrac{\sqrt{2}}{2}\arctan(\sqrt{2}u+1)\right)\Bigg|_0^{+\infty}\\
  &=\dfrac{\sqrt{2}}{2}\pi                   
\end{aligned}\]
\end{analysis}

\subsection{函数项级数}
\subsubsection{函数列}
对于函数列$\{f_n(x)\}$可以将其看为二元函数$f(n,x)$,但实际上我们在研究它的性质的时候往往是针对某个点$x=x_0$(确定的$x$),令$n\to\infty$. 这样看的话,函数项级数与数项级数没有太大差别.
\begin{definition}[极限函数]
\[f(x):=\lim_{n\to\infty}f_n(x),\forall x\in X\]
称为函数列$\{f_n(x)\}$的极限函数
\end{definition}
由于在不同点函数值构成的数列收敛的快慢不一致,在快慢发生急剧变化的地方,极限函数可能会变成间断,因此引入\textbf{一致收敛}的概念,以描述函数各处收敛速度差不多的性质.
\begin{definition}[一致收敛]
$\forall \varepsilon>0,\exists N=N(\varepsilon),\forall n>N, x\in X$,有
\[|f_n(x)-f(x)|<\varepsilon\]
则称函数列$\{f_n(x)\}$在$X$一致收敛到$f(x)$. 或记
\[\rho_n=\sup_{a\leq x\leq b}|f_n(x)-f(x)|\]
$\{f_n(x)\}$在$X$一致收敛到$f(x)$当且仅当$\rho_n\to0(n\to\infty)$.
\end{definition}
\par 函数列的一致收敛性一般按照定义证明,与极限定义证明方法类似,在此不再赘述,关键在于把$x$看成常数,$n$看成变量.
\par 注意不一致收敛的定义与平时的定义改写有较大区别,在此说明.
\begin{definition}[不一致收敛]
$\exists \varepsilon_0>0,\forall N\in\zz^+,\exists n>N$与$x_n\in[a,b]$,使得
\[|f_n(x)-f(x)|\geq\varepsilon_0\]
则称函数列$\{f_n(x)\}$在$X$不一致收敛到$f(x)$
\end{definition}
\par 下面是一个关于一致收敛性很有趣的例子.
\begin{example}
\[f_n(x)=\dfrac{x^n}{1+x^n}\]
讨论在以下区间的一致收敛性
\begin{enumerate}
	\itemsep -3pt
	\item $x\in[0,b],b<1$
	\item $x\in[0,1)$
\end{enumerate}
\end{example}
\begin{analysis}易得极限函数
\[f(x)=\begin{cases}
0&x\in[0,1)\\
\dfrac{1}{2}&x=1\\
1&x>1
\end{cases}\]
\begin{enumerate}
	\item $\forall\varepsilon>0,|f_n(x)-f(x)|=\dfrac{x^n}{1+x^n}\leq x^n\leq b^n,x\in[0,b],b<1$\\
	则取$N=\floor{\log_b\varepsilon}+1$,当$n>N$时,$|f_n(x)-f(x)|\leq \varepsilon$,故一致收敛
	\item $\exists\varepsilon_0=\dfrac{1}{3},\forall N,\exists n=N+1>N$,取$x_n=\dfrac{1}{\sqrt[n]{2}}\in[0,1)$,则
	\[|f_n(x_n)-f(x_n)|=\dfrac{1/2}{1+1/2}=\dfrac{1}{3}\geq\eps_0=\dfrac{1}{3}\]
	故不一致收敛
\end{enumerate}
初看会感觉这两者很矛盾,但细想是有区别的. 前者的上确界小于$1$,而后者的上确界等于$1$,其中的极限过程已经超出了人脑能脑补的程度了.
\end{analysis}
\begin{definition}[一致有界]
\[\exists M>0, s.t. \forall x\in X,n:\;|f_n(x)|\leq M\]
则称函数列$\{f_n(x)\}$在$X$一致有界
\end{definition}
\begin{theorem}
\begin{enumerate}
	\item $f_n(x)$在$[a,b]$上有界,并且$\nums{f_n(x)}$在$[a,b]$上一致收敛到$f(x)$,则$f_n(x)$在$[a,b]$上一致有界
	\item $f_n(x)$在$[a,b]$上一致连续,并且$\nums{f_n(x)}$在$[a,b]$上一致收敛到$f(x)$,则$f(x)$在$[a,b]$上一致连续
	\item $f_n(x)$在$[a,b]$上黎曼可积,并且$\nums{f_n(x)}$在$[a,b]$上一致收敛到$f(x)$,则$f(x)$在$[a,b]$上黎曼可积
\end{enumerate}
\end{theorem}
\begin{analysis}
\begin{enumerate}
	\item 因$f_n(x)$在$[a,b]$有界,设$|f_n(x)|<M_n$\\
	因$f_n(x)\to f(x)$,取$\eps=1$,则当$n>N$时,$|f_n(x)-f(x)|<1$,故$|f(x)|<|f_{N+1}(x)|+1<M_{N+1}+1$\\
	进而$\forall n>N:|f_n(x)|<|f(x)|+1<M_{N+1}+2$,取$M=\max(M_1,M_2,\ldots,M_N,M_{N+1}+2)>0$,则$\forall n:|f_n(x)|<M$,$f_n(x)$在$[a,b]$上一致有界
	\item 最后要证$|f(x_1)-f_{N+1}(x_1)+f_{N+1}(x_1)-f_{N+1}(x_2)+f_{N+1}(x_2)-f(x_2)|<\eps$,拆分一下分别用定义即可
	\item 用定积分定义证,比较麻烦
\end{enumerate}
\end{analysis}
\begin{theorem}[函数列的分析性质]
\begin{enumerate}
	\itemsep -3pt
	\item (连续)若函数列$\{f_n(x)\}$每一项都在$[a,b]$连续,且$\{f_n(x)\}$在$[a,b]$一致收敛到$f(x)$,则$f(x)$连续
	\[\displaystyle\lim_{x\to x_0}\lim_{n\to\infty}f_n(x)=f(x_0)=\lim_{n\to\infty}\lim_{x\to x_0}f_n(x)\]
	\item (可积)若函数列$\{f_n(x)\}$每一项都在$[a,b]$连续,且$\{f_n(x)\}$在$[a,b]$一致收敛到$f(x)$,则$f(x)$可积
	\[\intab{a}{b}{f(x)}=\displaystyle\lim_{n\to+\infty}\intab{a}{b}{f_n(x)}\]
	\item (可微)若函数列$\{f_n(x)\}$每一项都在$[a,b]$有连续微商$f_n'(x)$,$\{f_n(x)\}$在$[a,b]$\textbf{逐点}收敛到$f(x)$,\\
	且$\{f_n'(x)\}$在$[a,b]$\textbf{一致}收敛到$\sigma(x)$,则$f(x)$可微,且
	\[\displaystyle f'(x)=(\lim_{n\to+\infty}f_n(x))'=\lim_{n\to+\infty}f_n'(x)=\sigma(x)\]
\end{enumerate}
\end{theorem}
\begin{example}
\begin{enumerate}
	\item 若函数列$\{f_n(x)\}$每一项都在$[a,b]$连续,且$\{f_n(x)\}$在$[a,b]$一致收敛到$f(x)$,$x_n\in[a,b]$,且$\limtoinf{x_n}=x_0$,则$\limtoinf{f_n(x_n)}=f(x_0)$
	\item $\{f_n(x)\}$在$[a,b]$一致收敛到$f(x)$,$x_0\in(a,b)$且$\disp\lim_{x\to x_0}f_n(x)=a_n$,则$\disp\limtoinf\lim_{x\to x_0}f_n(x)=\lim_{x\to x_0}\limtoinf{f_n(x)}$
\end{enumerate}
\end{example}
\begin{analysis}
\begin{enumerate}
	\item 证$|f_n(x_n)-f(x_0)|=|f_n(x_n)-f(x_n)+f(x_n)-f(x_0)|<\eps$
	\item 证$|f(x)-a|=|f(x)-f_{N+1}(x)+f_{N+1}(x)-a_{N+1}+a_{N+1}-a|<\eps$,其中会用到柯西列
\end{enumerate}
\end{analysis}

\subsubsection{和函数}
\begin{definition}[和函数]
对于函数列$\nums{u_n(x)}$,定义和函数
\[\begin{aligned}
S(x):&=\lim_{n\to\infty}S_n(x)=\lim_{n\to\infty}\sum_{k=1}^nu_k(x)\\
&=\ssumk{u_k(x)},\forall x\in X
\end{aligned}\]
\end{definition}
\par 当函数项级数可以求和时,要先把和求出来. 把$x$看成常数,函数项级数会往往变成一个等比数列,这样就可以求和了.
\begin{example}按定义讨论下面级数的一致收敛性
\[\ssum{\dfrac{(-1)^{n-1}x^2}{(1+x^2)^n}},x\in(-\infty,+\infty)\]
\end{example}
\begin{analysis}
设部分和数列为$S_n(x)$,和函数(若存在的话)为$S(x)$,则$\forall\eps>0,\exists N=N(\eps)$使得$\forall n>N$有
\[|S_n(x)-S(x)|<\eps\]
注意到
\[S_n(x)=\ssumkn{\dfrac{(-1)^{k-1}x^2}{(1+x^2)^k}}=-x^2\ssumkn{\lrp{-\dfrac{1}{1+x^2}}^k}=\dfrac{x^2}{2+x^2}\lrp{1-\lrp{-\dfrac{1}{1+x^2}}^n}\]
故
\[\limtoinf{S_n(x)}=S(x)=\dfrac{x^2}{2+x^2}\]
因此
\[\begin{aligned}
|S_n(x)-S(x)|
&=\lra{\dfrac{x^2}{2+x^2}\dfrac{(-1)^{n-1}}{(1+x^2)^n}}\\
&=\dfrac{x^2}{(1+x^2)^n(2+x^2)}\\
&\leq\dfrac{1}{2}\dfrac{x^2}{1+nx^2}\qquad\mbox{伯努利不等式}\\
&\leq\dfrac{1}{2}\dfrac{x^2}{nx^2}\\
&=\dfrac{1}{2n}<\eps
\end{aligned}\]
取$N=\floor{\dfrac{1}{2\eps}}+1$,则当$n>N$时上式成立,故一致收敛
\end{analysis}
\par 其实函数项级数就和普通的数项级数没有太大差别,用在数项级数的判别法,在函数项级数这里照样可以用,需要改动的地方不过是在一些地方加上“一致”罢了. 下面这条定理就是定理~\ref{series_conver}的函数项级数版本.
\begin{theorem}[函数项级数一致收敛必要条件]
函数项级数$X$一致收敛的必要条件是一般项构成的函数序列$\{u_n(x)\}$在$X$一致收敛于$0$
\end{theorem}
\par 而下面这个判别法则是数项级数没有的判别法,非常强大,屡用不爽.
\begin{theorem}[M判别法(Weierstrass)]
若对函数项级数$\ssumk{u_k(x)}$,存在$M_k$使得
\[|u_k(x)|\leq M_k,\forall x\in X\]
而正项级数$\ssumk{M_k}$收敛,则$\ssumk{u_k(x)}$在$X$一致收敛
\end{theorem}
\par 技巧在于,把一般项里含$x$的项都放缩去掉,只留下关于$n$的项
\begin{example}
若函数列$\{u_n(x)\}$的每一项都是$[a,b]$的单调函数,$\ssum{u_n(x)}$在$[a,b]$的端点处绝对收敛,则该级数在$[a,b]$上一致收敛
\end{example}
\begin{analysis}由函数的单调性,及M判别法
\[u_n(x)<\max(|u_n(a)|,|u_n(b)|)\]
故$\{u_n(x)\}$在$[a,b]$上一致收敛
\end{analysis}
\par 当$x$实在消不掉时,可采用一些“巧妙”的方法,如下.
\begin{example}
\[\ssum{\dfrac{nx}{1+n^4x^2}},|x|>0\]
\end{example}
\begin{analysis}
因直接对分母用均值不等式,得到$\dfrac{1}{n}$并无法判断收敛性,故换种方式\\
$\forall x_0,|x_0|>0,\exists\delta>0,\delta\in(0,|x_0|),\forall|x|\geq\delta$,有
\[\lra{\dfrac{nx}{1+n^4x^2}}\leq\dfrac{n|x|}{n^4x^2}=\dfrac{n^3|x|}\leq\dfrac{1}{n^3\delta}\]
而$\ssum{\dfrac{1}{n^3\delta}}$收敛,由M判别法,原级数在$|x|\geq\delta$一致收敛\\
又$x_0$的任意性,知原级数在$|x|>0$一致收敛
\end{analysis}
\begin{theorem}[迪尼(Dini)]
若函数列$\{u_n(x)\}$在$[a,b]$连续,$u_n(x)\geq 0$,函数项级数$\ssum{u_n(x)}$在$[a,b]$\textbf{逐点}收敛到$S(x)$,$S(x)$在$[a,b]$连续,则$\ssum{u_n(x)}$在$[a,b]$一致收敛
\end{theorem}
\begin{theorem}[和函数的分析性质]
与一般函数列的分析性质非常类似
\begin{enumerate}
	\item (连续)若函数列$\{u_n(x)\}$在$[a,b]$连续,且函数项级数$\ssum{u_n(x)}$在$[a,b]$一致收敛到$S(x)$,则$S(x)$在$[a,b]$连续
	\item (逐项积分)若函数列$\{u_n(x)\}$在$[a,b]$连续,且函数项级数$\ssum{u_n(x)}$在$[a,b]$一致收敛到$S(x)$,则$S(x)$可积
	\[\intab{a}{b}{S(x)}=\ssum{\intab{a}{b}{u_n(x)}}\]
	\item (逐项微分)若函数列$\{u_n(x)\}$在$[a,b]$有连续微商$u_n'(x)$,$\ssum{u_n(x)}$在$[a,b]$\textbf{逐点}收敛到$S(x)$,\\
	且$\ssum{f_n'(x)}$在$[a,b]$\textbf{一致}收敛到$\sigma(x)$,则$S(x)$可微,且
	\[\displaystyle S'(x)=\left(\ssum{u_n(x)}\right)'=\ssum{u_n'(x)}=\sigma(x)\]
\end{enumerate}
\end{theorem}
\begin{example}
若函数列$\{u_n(x)\}$在$[a,b]$连续,且函数项级数$\ssum{u_n(x)}$在$(a,b)$一致收敛,则$\ssum{u_n(x)}$在$[a,b]$一致收敛
\end{example}
\begin{analysis}
取一个小于$\eps$的值,对$\ssum{u_n(x)}$用柯西收敛原理后,两侧取极限$x\to a^+,x\to b^-$,即得证. (因连续性取极限会导致取等,故要取小于$\eps$的值)
\end{analysis}
\par 关于和函数分析性质的运用,可以见下例.
\begin{example}
\[\intab{-\pi}{\pi}{\dfrac{1-r^2}{1-2r\cos x+r^2}}=2\pi,|r|<1\]
\end{example}
\begin{analysis}
由例~\ref{eg_series_conver1}知
\[\dfrac{1-r^2}{1-2r\cos x+r^2}=1+2\ssum{r^n\cos nx},|r|<1\]
当$|r|<1$时,$|2r^n\cos nx|\leq 2|r|^n$,而$\ssum{2|r|^n}$收敛,由M判别法知$\ssum{r^n\cos nx}$一致收敛\\
又级数的每一项$\{r^n\cos nx\}$都在$(-\infty,+\infty)$连续,故可逐项积分,得
\[\intab{-\pi}{\pi}{\dfrac{1-r^2}{1-2r\cos x+r^2}}=\intab{-\pi}{\pi}{}+2\ssum{\intab{-\pi}{\pi}{r^n\cos nx}}=2\pi\]
\end{analysis}

\subsection{敛散性判定方法总结}
\label{summary_conver}
本小节阐述对于正项级数、一般项级数、无穷限积分、广义积分、函数项级数中的某几个都通用的判别法. 为方便表述和记忆,用BNF范式表述,有点泛型编程(Generic Programming)的思想在里面,下面先做一些符号说明.
\[\begin{aligned}
\inpro{s}&::=u_n\mmid v_n\\
\inpro{f}&::=f(x)\mmid g(x)\\
\inpro{fs}&::=u_n(x)\mmid v_n(x)
\end{aligned}\]

\subsubsection{比较判别法}
比较判别法是为通用的判别法,对于上述五种级数积分均可采用,但要注意只能用于判定\textbf{正项}的级数或积分,即其\textbf{绝对收敛}性,使用时注意\textbf{加上绝对值}
\[\begin{aligned}
\inpro{T}&::=|\inpro{s}|\mmid |\inpro{f}|\mmid |\inpro{fs}|\\
\inpro{U}&::=\ssum{|\inpro{s}|}\mmid \intab{a}{+\infty}{|\inpro{f}|}\mmid \intab{a}{b}{|\inpro{f}|}(\mbox{a为瑕点})\mmid \ssum{|\inpro{fs}|}
\end{aligned}\]
\begin{itemize}
	\item 一般形式:
		\[\mbox{$T_1\leq T_2$,$U_2$收敛则$U_1$收敛,$U_1$发散则$U_2$发散}\]
	\item 极限形式:
		\[\mbox{$\displaystyle\lim_{n\to\infty}\dfrac{T_1}{T_2}=l$,$l\in[0,+\infty)$,$U_2$收敛则$U_1$收敛;$l\in(0,+\infty],U_2$发散则$U_1$发散}\]
\end{itemize}
\par 一般形式的比较判别法需要熟悉常见的放缩技巧,如有界量($\sin x,(-1)^n$)直接取界、均值不等式、估计等.
% 2^\ln n=n^\ln 2 3^\sqrt{n}<n^3
\par 极限形式的比较判别法常常与重要极限/无穷小量代换牵连在一起,需要懂得构造.
\begin{example}
$\ssum{\left(\sqrt{1+\dfrac{1}{n}}-1\right)}$
\end{example}
\begin{analysis}
一般形式:
\[\dfrac{\dfrac{1}{n}}{1+\sqrt{1+\dfrac{1}{n}}}=\dfrac{1}{n+n\sqrt{1+\dfrac{1}{n}}}
\geq\dfrac{1}{2n\sqrt{1+\dfrac{1}{n}}}=\dfrac{1}{2\sqrt{n(n+1)}}\geq\dfrac{1}{2n+1}\]
看似平淡无奇,但分母将加号变为乘号后再用均值不等式避免反号的技巧十分高超
\end{analysis}
\begin{example}
$\ssum{n\left(1-\cos\frac{1}{n}\right)}$
\end{example}
\begin{analysis}
极限形式:
\[\limtoinf{\dfrac{n(1-\cos\dfrac{1}{n})}{\dfrac{1}{n}}}=\limtoinf{\dfrac{2\sin^2\dfrac{1}{2n}}{4\dfrac{1}{2n}\dfrac{1}{2n}}}=\dfrac{1}{2}\]
这个降角很有技巧,为使用重要极限做准备
\end{analysis}
极限形式还有比较常见的方法是直接取某一分式作为分母,如$\dfrac{1}{n}$,$\dfrac{1}{n^2}$等
\begin{example}
\[\intab{1}{+\infty}{\ln\lrp{\cos\dfrac{1}{x}+\sin\dfrac{1}{x}}}\]
\end{example}
\begin{analysis}
这里就取三角函数内的$\dfrac{1}{x}$作为比较对象
\[\begin{aligned}
&\quad\limtoinf{\dfrac{\ln\lrp{\cos\frac{1}{x}+\sin\frac{1}{x}}}{\frac{1}{x}}}\\
&=\limtoinf{\dfrac{\lrp{-\sin\frac{1}{x}}\lrp{-\frac{1}{x}}+\lrp{\cos\frac{1}{x}}\lrp{-\frac{1}{x^2}}}{\lrp{\cos\frac{1}{x}+\sin\frac{1}{x}}\lrp{-\frac{1}{x^2}}}}\\
&=\limtoinf{\dfrac{\cos\frac{1}{x}-\sin\frac{1}{x}}{\cos\frac{1}{x}+\sin\frac{1}{x}}}\\
&=1
\end{aligned}\]
\end{analysis}
\par 注意数项级数还有另一种形式:$\dfrac{u_{n+1}}{u_n}\leq\dfrac{v_{n+1}}{v_n}$,判别条件与一般形式的类似.
%\begin{analysis}
%	$\dfrac{u_n}{u_N}\leq\dfrac{v_n}{v_N}$裂项递推
%\end{analysis}

\subsubsection{正项判别法}
同样也是判定绝对收敛
\[\begin{aligned}
\inpro{T_i}&::=|\inpro{s_i}| \mmid |\inpro{fs_i}|\\
\inpro{U}&::=\ssum{|\inpro{s}|}\mmid \ssum{|\inpro{fs}|}
\end{aligned}\]
\par 注意这里要将函数项级数看成$x$为常数的数项级数,但以下几种判别法只能判定收敛,而\textbf{不能}判定一致收敛性.
\begin{enumerate}
	\item 达朗贝尔(D'Alembert)判别法:
	\[\mbox{$\displaystyle\lim_{n\to\infty}\dfrac{T_{n+1}}{T_n}=l$,$l<1$收敛,$l>1$发散}\]
	%\begin{analysis}
	%$\dfrac{u_{n+1}}{u_n}<l+\varepsilon_0=r=\dfrac{r^{(n+1)}}{r_n}$. 幂级数、阶乘
	%\end{analysis}
	\item 拉阿比(Raabe)判别法:
	\[\mbox{$\displaystyle\lim_{n\to\infty}n\left(\dfrac{T_n}{T_{n+1}}-1\right)=S$,$S>1$收敛,$S<1$发散}\]
	%\begin{analysis}
	%注意符号反. 引理:$\forall r>p>1,\exists N, n>N, 1+\dfrac{r}{n}>\left(1+\dfrac{1}{n}\right)^p$
	%\end{analysis}
	\item 柯西(Cauchy)根式判别法:
	\[\mbox{$\displaystyle\lim_{n\to\infty}\sqrt[n]{T_n}=l$,$l<1$收敛,$l>1$发散}\]
	%\begin{analysis}
	%实质是等比数列
	%\end{analysis}
	\item 高斯(Gauss)判别法:
	\[\dfrac{a_{n+1}}{a_n}=1-\dfrac{\alpha}{n}+\dfrac{\gamma_n}{n^\beta}\]
	其中$\alpha,\beta>1$都为常数,且数列$\{\gamma_n\}$有界,则当$\alpha>1$时数列收敛,$\alpha\leq 1$时数列发散(没有失效的情况!)
\end{enumerate}
\par 达朗贝尔和拉阿比判别法(以及上述高斯判别法)常用在前后项可以相消的情况,如阶乘、指数.
\par 柯西根式判别法则用在明显的指数形式,如整个$T$的$n$次幂等.
\begin{example}
\[\ssum{n!\lrp{\dfrac{x}{n}}^n}\]
\end{example}
\begin{analysis}
指数阶乘采用达朗贝尔判别法,注意加绝对值
\[\limtoinf{\lra{\dfrac{u_{n+1}}{u_n}}}=\limtoinf\lrp{1-\dfrac{1}{n+1}}^n|x|=\dfrac{|x|}{e}\]
故$|x|<e$时原级数绝对收敛;$|x|>e$时,原级数发散;当$|x|=e$时,达朗贝尔判别法失效.\\
但由\textbf{斯特林(Stirling)公式}
\[n! = \sqrt{2 \pi n} \; \left(\frac{n}{e}\right)^{n}e^{\lambda_n}\]
其中
\[\displaystyle {\frac {1}{12n+1}}<\lambda _{n}<{\frac {1}{12n}}\]
故
\[n!\lrp{\dfrac{e}{n}}^n=\dfrac{\sqrt{2 \pi n} \; \left(\frac{n}{e}\right)^{n}e^{\lambda_n}e^n}{n^n}=\sqrt{2 \pi n} e^{\lambda_n}\nrightarrow 0\]
由数项级数的收敛定理知$|x|=e$时发散
\end{analysis}
\begin{example}
\[\ssum{\dfrac{\alpha(\alpha+1)\cdots(\alpha+n-1)}{n!}\dfrac{1}{n^\beta}},\alpha,\beta>0\]
\end{example}
\begin{analysis}
明显的阶乘暗示,先求
\[\dfrac{u_n}{u_{n+1}}=\dfrac{n+1}{n+\alpha}\left(1+\dfrac{1}{n}\right)^\beta\]
由拉阿比判别法
\[\begin{aligned}
\limtoinf{n\left(\dfrac{u_n}{u_{n+1}}-1\right)}
&=\limtoinf{n\left[\dfrac{n+1}{n+\alpha}\left(1+\dfrac{1}{n}\right)^\beta-1\right]}\\
&=\limtoinf{n\dfrac{(n+1)\left(1+\dfrac{1}{n}\right)^\beta-n-\alpha}{n+\alpha}}\\
&=\limtoinf{n\dfrac{(n+1)\left(1+\dfrac{\beta}{n}+\mathcal{O}\left(\dfrac{1}{n}\right)\right)-n-\alpha}{1+\dfrac{\alpha}{n}}}\mbox{最好别用洛必达,用二项式展开}\\
&=1-\alpha+\beta
\end{aligned}\]
当$1-\alpha+\beta>1$,即$\alpha<\beta$时级数收敛;当$1-\alpha+\beta<1$,即$\alpha>\beta$时级数发散;当$\alpha=\beta$时拉阿比判别法失效,采用高斯判别法
\[\begin{aligned}
\dfrac{u_n}{u_{n+1}}
&=\dfrac{n+1}{n+\alpha}\left(1+\dfrac{1}{n}\right)^\alpha\\
&=\dfrac{n+1}{n+\alpha}\lrp{1+\dfrac{\alpha}{n}+\dfrac{\alpha(\alpha-1)}{2}\dfrac{1}{n^2}+\mathcal{O}\lrp{\dfrac{1}{n^2}}}\\
&=\dfrac{n+\alpha+1-\alpha}{n+\alpha}+\dfrac{\alpha(n+1)}{n(n+\alpha)}+\dfrac{(n+1)\alpha(\alpha-1)}{2n^2(n+\alpha)}+\dfrac{n+1}{n+\alpha}\mathcal{O}\lrp{\dfrac{1}{n^2}}\\
&=1+\dfrac{1}{n}+\dfrac{1}{n^2}\left[\dfrac{(n+1)\alpha(\alpha-1)}{2(n+\alpha)}+\dfrac{n+1}{n+\alpha}\mathcal{O}(1)\right]\\
&=1-\dfrac{-1}{n}+\dfrac{\gamma_n}{n^2}
\end{aligned}\]
$\gamma_n$极限存在故有界,因此$\alpha=\beta$时原级数发散
\end{analysis}
柯西根式判别法可以用在一些奇奇怪怪的数列上,只有它有什么的$n$次方就行.
\begin{example}
\[\ssum(-1)^{n-1}\dfrac{2^n\sin^{2n}x}{n}\]
\end{example}
\begin{analysis}由柯西根式判别法
\[\limtoinf{\sqrt[n]{|u_n|}}=\limtoinf{\dfrac{2\sin^2x}{\sqrt[n]{n}}}=2\sin^2 x\]
故当$2\sin^2 x<1$即$k\pi-\dfrac{\pi}{4}<x<k\pi+\dfrac{\pi}{4},k\in\zz$时,原级数绝对收敛\\
同理可得其他情况. 综合有,原级数在$k\pi-\dfrac{\pi}{4}<x<k\pi+\dfrac{\pi}{4}$绝对收敛,在$x=k\pi\pm\dfrac{\pi}{4}$条件收敛,在$k\pi+\dfrac{\pi}{4}<x<k\pi+\dfrac{3\pi}{4}$发散,其中$k\in\zz$
\end{analysis}

\subsubsection{一般判别法}
这里要区别一致收敛和绝对收敛概念的适用范围.
\par \textbf{一致收敛}用于函数项级数,是指其在各个点收敛的速度都差不多.
\par \textbf{绝对收敛}用于数项级数或广义积分,对于函数列的某个点$x_0$也可以讨论其绝对收敛性.
\par 至于函数列的一致收敛性,只能通过定义来证明.
\[\begin{aligned}
\inpro{T}&::=\inpro{s}\mmid \inpro{f}\mmid \inpro{fs}\\
\inpro{U}&::=\ssum{\inpro{s}}\mmid \intab{a}{+\infty}{\inpro{f}}\mmid \intab{a}{b}{\inpro{f}}(\mbox{a为瑕点})\mmid \ssum{\inpro{fs}}\\
\inpro{V}&::=\ssum{\inpro{s}\inpro{s}}\mmid \intab{a}{+\infty}{\inpro{f}\inpro{f}}\mmid \intab{a}{b}{\inpro{f}\inpro{f}}(\mbox{a为瑕点})\mmid \ssum{\inpro{fs}\inpro{fs}}
\end{aligned}\]
\begin{enumerate}
	\item 绝对收敛必收敛
	\item 柯西(Cauchy)收敛原理:注意下面的求和是广义和,即包含了积分\\
	\[\forall\varepsilon>0,\exists N,\forall n',n''>N:\;\sum_{n'}^{n''}T<\varepsilon \;\Leftrightarrow U\mbox{(一致)收敛}\]
	注意,数列中同样有柯西收敛原理,形式完全相同,只是没有求和符号,对应柯西列的概念
	\item 狄利克雷(Dirichlet)判别法:\\
	$T_1$单调(一致)收敛于$0$,$T_2$的部分和(一致)有界,则$V$(一致)收敛
	\item 阿贝尔(Abel)判别法:\\
	$T_1$单调(一致)有界,$T_2$(一致)收敛,则$V$(一致)收敛
\end{enumerate}
\par 柯西收敛原理在证明题中使用简直就是大杀器,绝大多数证明收敛的题目都可以用它倒腾出来
\begin{example}
正项级数$\ssum{a_n}$收敛,证明\[\limtoinf{\dfrac{a_1+2a_2+\cdots+na_n}{n}}=0\]
\end{example}
\begin{analysis}
即证$\forall\varepsilon>0,\exists N,\forall n>N$时
\[\left|\dfrac{a_1+2a_2+\cdots+na_n}{n}\right|<\varepsilon\]
因$\ssum{a_n}$收敛,由柯西收敛原理,$\exists N_1,\forall n>N_1$都有
\[|a_{N_1+1}+a_{N_1+2}\cdots+a_n|<\dfrac{\varepsilon}{2}\]
又因$N_1$是一个确定的值,故$\exists N_2\forall n>N_2$
\[\left|\dfrac{a_1+2a_2+\cdots+N_1a_{N_1}}{n}\right|<\dfrac{\varepsilon}{2}\]
所以
\[\begin{aligned}
\left|\dfrac{a_1+2a_2+\cdots+na_n}{n}\right|
&\leq\left|\dfrac{a_1+2a_2+\cdots+N_1a_{N_1}}{n}\right|+\left|\dfrac{N_1+1}{n}a_{N_1+1}\right|+\left|\dfrac{N_1+2}{n}a_{N_1+2}\right|+\cdots+\left|\dfrac{n}{n}a_{n}\right|\\
&<\dfrac{\varepsilon}{2}+a_{N_1+1}+a_{N_1+2}+\cdots+a_n\\
&<\dfrac{\varepsilon}{2}+\dfrac{\varepsilon}{2}=\varepsilon
\end{aligned}\]
取$N=\max(N_1,N_2)$即得证
\end{analysis}
\begin{example}%P37
对数列$\{a_n\},\{b_n\}$定义$S_n=\ssumkn{a_k}$,$\Delta b_k=b_{k+1}-b_k$,求证
\begin{enumerate}
	\item 若$\nums{S_n}$有界,$\ssum|\Delta b_n|$收敛,且$\limtoinf{b_n}=0$,则$\ssum{a_nb_n}$收敛,且
	\[\ssum{a_nb_n}=-\ssum{S_n\Delta b_n}\]
	\item 若$\ssum{a_n}$与$\ssum{|\Delta b_n|}$都收敛,则$\ssum{a_nb_n}$收敛
\end{enumerate}
\end{example}
\begin{analysis}
这题相当复杂,但却是练习用定义证明的绝佳好题.
\begin{enumerate}
	\item 欲用柯西收敛原理,即$\forall\eps>0,\exists N$,当$n>N$时,$\forall p\in\zz^+$有$\disp\lra{\sum_{k=n+1}^{n+p}}<\eps$\\
	用阿贝尔变换
	\[\begin{aligned}
	LHS&=\lra{\sum_{k=n+1}^{n+p-1}(S_k-S_n)(b_k-b_{k+1})+b_{n+p}(S_{n+p}-S_n)}\\
	&\leq\sum_{k=n+1}^{n+p-1}(|S_k-S_n||\Delta b_k|+|b_{n+p}||S_{n+p}-S_n|)\qquad(\star)
	\end{aligned}\]
	因为$\nums{S_n}$有界,故
	\[\exists M>0,s.t.\forall n\in\zz^+:|S_n|\leq M\qquad(1')\]
	因为$\ssum|\Delta b_n|$收敛,故
	\[\exists N_1>0,s.t.\forall n>N_1,\forall p\in\zz^+:\lra{\sum_{k=n+1}^{n+p}|\Delta b_k|}<\dfrac{\eps}{4M}\qquad(2')\]
	因为$\limtoinf{b_n}=0$,故
	\[\exists N_2>0,s.t.\forall n>N_2:|b_n|<\dfrac{\eps}{4M}\qquad(3')\]
	结合$(1')(2')(3')$式,得
	\[(\star)\leq 2M\dfrac{\eps}{4M}+2M\dfrac{\eps}{4M}=\eps\]
	取$N=\max(N_1,N_2)$即可得$\ssum{a_nb_n}$收敛\\
	再由阿贝尔变换
	\[\ssumkn{a_kb_k}=\sum_{k=1}^{n-1}S_k(b_k-b_{k+1})+b_nS_n=-\sum_{k=1}^{n-1}S_k\Delta b_k+b_nS_n\]
	左右取极限$n\to\infty$有
	\[\ssum{a_nb_n}=-\ssum{S_n\Delta b_n}+\limtoinf{(b_nS_n)}\]
	又$\nums{S_n}$有界,$\limtoinf b_n=0\implies\limtoinf(b_nS_n)=0$,所以
	\[\ssum{a_nb_n}=-\ssum{S_n\Delta b_n}\]
	\item 同1理,由阿贝尔变换有
	\[\lra{\sum_{k=n+1}^{n+p}}\leq\sum_{k=n+1}^{n+p-1}(|S_k-S_n||\Delta b_k|+|b_{n+p}||S_{n+p}-S_n|)\qquad(\star)\]
	因为$\ssum{a_n}$收敛,故由数列极限的有界性知,其部分和数列$\nums{S_n}$有界,即
	\[\exists M_1>0,\forall n\in\zz^+:|S_n|\leq M_1\qquad(a')\]
	因为$\ssum{|\Delta b_n|}$收敛,所以$\ssum{\Delta b_n}$收敛,其部分和数列
	\[B_n=\ssumkn{\Delta b_k}=b_{n+1}-b_1\]
	极限存在,故$\{b_n\}$有极限,进而$\nums{b_n}$有界,即
	\[\exists M_2>0,\forall n\in\zz^+:|b_n|\leq M_2\qquad(b')\]
	因为$\ssum{a_n}$与$\ssum{|\Delta b_n|}$都收敛,取$M=\max(M_1,M_2)$,由柯西收敛原理
	\[\exists N_1>0,\forall n>N_1,\forall p\in\zz^+:\lra{\sum_{k=n+1}^{n+p}a_k}=|S_{n+p}-S_n|<\dfrac{\eps}{2M}\qquad(c')\]
	\[\exists N_2>0,\forall n>N_2,\forall p\in\zz^+:\lra{\sum_{k=n+1}^{n+p}\Delta b_k}<\dfrac{\eps}{4M}\qquad(d')\]
	结合$(a')(b')(c')(d')$式,取$N=\max(N_1,N_2)$,得
	\[(\star)\leq 2M\dfrac{\eps}{4M}+M\dfrac{\eps}{2M}=\eps\]
	由柯西收敛原理知$\ssum{a_nb_n}$收敛
\end{enumerate}
\end{analysis}
\par 下面的例~\ref{cauchy_s1}、例~\ref{cauchy_s2}是一些小结论.
\begin{example}
\label{cauchy_s1}
\begin{enumerate}
	\item 若$\ssum{a_n}$收敛,$\ssum{a_n^2}$不一定收敛
	\item 若$\ssum{a_n}(a_n\geq 0)$收敛,$\ssum{a_n^2}$收敛
	\item 若$\ssum{a_n^2}$收敛,则$\ssum{a_n^3}$绝对收敛
\end{enumerate}
\begin{analysis}
\begin{enumerate}
	\item 取$a_n=\dfrac{(-1)^n}{\sqrt{n}}$
	\item 当$n$足够大时有$a_n<1$,则$a_n^2=a_n\cdot a_n\leq a_n$,由比较判别法知收敛
	\item 同2理
\end{enumerate}
\end{analysis}
\begin{example}
\label{cauchy_s2}
关于$na_n$的一些例子
\begin{enumerate}
	\item 若$\ssum{a_n}(a_n>0)$收敛,$na_n$不一定趋于0
	\item 若$\ssum{a_n}(a_n>0)$收敛,$a_{n+1}\leq a_n$,则$\limtoinf{na_n}=0$
	\item 若$\ssum{a_n}$收敛,$\limtoinf{na_n}=0$,则$\ssum{n(a_n-a_{n+1})}=\ssum{a_n}$
	\item 若$\intab{a}{+\infty}{f(x)}$收敛,且$f(x)$在$[a,+\infty)$单调下降,则$\limtoinf{xf(x)}=0$
\end{enumerate}
\end{example}
\begin{analysis}
\begin{enumerate}
	\item 取$a_n=\dfrac{1}{n^2},n\neq k^2,k\in\zz^+$,$a_{k^2}=\dfrac{1}{k^2},k\in\zz^+$\\
分开两个$p$级数求和,当然收敛;因子列$k^2a_{k^2}\to 1$,故$na_n$不趋于$0$
	\item 由柯西收敛原理
\[(n-N)a_n=a_n+\cdots+a_n\leq a_N+\cdots+a_n<\varepsilon\]
又$\limtoinf{a_n}=0$,故
\[na_n<Na_n+\dfrac{\varepsilon}{2}<N\dfrac{\varepsilon}{2N}+\dfrac{\varepsilon}{2}=\varepsilon\]
进而$\limtoinf{na_n}=0$
	\item 由$\limtoinf{na_n}=0$及$\limtoinf{a_n}=0$有
\[\ssum{n(a_n-a_{n+1})}=\ssum{a_n}-(n+1)a_{n+1}+a_{n+1}\]
	\item 同2,但要先反证得$f(x)\geq 0$
\end{enumerate}
\end{analysis}
\end{example}
\par 下面例~\ref{egdirconv1}无论是方法还是结论都相当重要,类似地可以证明$\ssum{\dfrac{\cos nx}{n}}$在$x\neq k\pi,k\in\zz$时收敛.
\begin{example}
\[\ssum{\dfrac{\sin nx}{n}}\]
\label{egdirconv1}
\end{example}
\begin{analysis}
当$x=k\pi,k\in\zz$时,级数一般项为$0$,显然绝对收敛\\
当$x\neq k\pi,k\in\zz$时,由三角函数积化和差公式,裂项相消可得
\[2\sin\dfrac{x}{2}\ssumkn{\sin kx}=\cos\dfrac{x}{2}-\cos\dfrac{2n+1}{2}x\]
故
\[\lra{\ssumkn{\sin kx}}\leq\dfrac{\lra{\cos\dfrac{x}{2}}+\lra{\cos\dfrac{2n+1}{2}x}}{2\lra{\sin\dfrac{x}{2}}}\leq\dfrac{1}{\lra{\sin\dfrac{x}{2}}}\]
即$\ssumkn{\sin kx}$有界,而$\{\dfrac{1}{n}\}$单调下降趋于$0$,由狄利克雷判别法知原级数收敛
\end{analysis}
\par 下面是几道变形例题.
\begin{example}
\[\ssum{\frac{\cos nx}{n^p}},x\in(0,\pi)\]
\end{example}
\begin{analysis}
\begin{enumerate}
	\item 当$p>1$时,$\lra{\dfrac{\cos nx}{n^p}}\leq\dfrac{1}{n^p}$,而$\ssum{\dfrac{1}{n^p}}$收敛,由比较判别法,原级数绝对收敛\par
	\item 当$0<p\leq 1$时,由例~\ref{egdirconv1}类似的方法知$\ssum{\cos nx}$有界,而$\{\dfrac{1}{n^p}\}$单调下降趋于$0$,故由狄利克雷函数判别法,原级数收敛\par
但由于(\textbf{相当关键的一步})
\[\lra{\dfrac{\cos nx}{n^p}}\geq\dfrac{\cos^2 nx}{n^p}=\dfrac{1}{2n^p}+\dfrac{\cos 2nx}{2n^p}\]
$\ssum{\dfrac{1}{2n^p}}$发散,$\ssum{\dfrac{\cos 2nx}{2n^p}}$收敛(狄利克雷),故$\ssum\lra{\dfrac{\cos nx}{n^p}}$发散,即原级数条件收敛\par
	\item 当$p\leq 0$时,$\limtoinf{\cos nx}\neq 0$. 若不然,则
\[0=\limtoinf{\cos nx}\neq 0=\limtoinf{\dfrac{1+\cos 2nx}{2}}=\dfrac{1}{2}\]
矛盾,而$\dfrac{1}{n^p}\geq 1$,故$\limtoinf{\cos\dfrac{nx}{n^p}}\neq 0$,由数项级数收敛的必要条件知原级数发散
\end{enumerate}
综上,原级数$p>1$时绝对收敛,$p\in(0,1]$时条件收敛,$p\leq 0$时发散
\end{analysis}
\begin{example}
\label{countereg_conver_integ}
\[\intab{1}{+\infty}{\sin x^2}\]
\end{example}
\begin{analysis}
\[\intab{1}{A}{\sin x^2}=\int_{1}^{A^2}\dfrac{\sin u}{2\sqrt{u}}\,\mathrm{d}u\]
这步替换相当关键,因$\dfrac{1}{\sqrt{u}}$单调递减趋于$0$,$\sin u$积分有界,故由狄利克雷判定法,原积分收敛
\end{analysis}
\par 类似地如$\ssum(-1)^n\dfrac{\cos 2n}{n},\ssum(-1)^n\dfrac{\sin^2 n}{n}$等,都可用上述思想解决.
\begin{example}
\[\intab{2}{+\infty}{\dfrac{\ln\ln x}{\ln x}\sin x}\]
\end{example}
\begin{analysis}
\begin{enumerate}
	\item 先证明收敛性
\[\dfrac{\ln\ln x}{\ln x}\sin x=\dfrac{\sin x}{\sqrt{\ln x}}\dfrac{\ln\ln x}{\sqrt{\ln x}}\]
因为$\dfrac{1}{\sqrt{\ln x}}$单调递减趋于$0$,$\sin x$积分有界,故由狄利克雷判别法,$\intab{2}{+\infty}{\dfrac{\sin x}{\sqrt{\ln x}}}$收敛\\
而$\lim_{x\to\infty}\dfrac{\ln\ln x}{\sqrt{\ln x}}=0$,故$\dfrac{\ln\ln x}{\sqrt{\ln x}}$有界\\
又$\lrp{\dfrac{\ln\ln x}{\sqrt{\ln x}}}'=\dfrac{2 - \ln\ln x}{2 x \ln x^{\frac{3}{2}}}$,故当$x\geq e^{e^2}$时,$\dfrac{\ln\ln x}{\sqrt{\ln x}}$单调递减\\
由阿贝尔判别法,原积分收敛
\item 再证明绝对收敛性
\[\lra{\dfrac{\ln\ln x}{\ln x}\sin x}\geq\dfrac{\ln\ln x}{\ln x}\sin^2 x=\dfrac{|\ln\ln x}{2\ln x}-\dfrac{(\ln\ln x)\cos 2x}{2\ln x}\]
类似地,由阿贝尔判别法知,$\intab{2}{+\infty}{\dfrac{(\ln\ln x)\cos 2x}{2\ln x}}$收敛\\
而当$x$充分大时,$\dfrac{\ln\ln x}{\ln x}\geq\dfrac{1}{x}$,故由比较判别法$\intab{2}{+\infty}{\dfrac{|\ln\ln x|}{2\ln x}}$发散\\
进而$\intab{2}{+\infty}{\dfrac{\ln\ln x}{\ln x}\sin^2 x}$发散\\
由比较判别法,$\intab{2}{+\infty}{\dfrac{|\ln\ln x|}{\ln x}|\sin x|}$发散
\end{enumerate}
综上,原积分条件收敛\\
注:这题的思路需要非常清晰,否则各种判别法混着使用很容易就迷失自我.
\end{analysis}

\subsubsection{方法总结}
\begin{enumerate}
	\itemsep -3pt
	\item 尝试求出和式或积分具体的值
	\item 用某些收敛充要(必要)条件判断不会发散
	\item 比较判别法最先考虑,绝对收敛用正项判别法,一般收敛性或一致收敛性用一般判别法
\end{enumerate}


\subsection{幂级数}
\begin{definition}[幂级数]
$\ssumz{a_nx^n}$称为幂级数,注意它的各个项严格按照升序排列
\end{definition}
\begin{theorem}
对于任意给定的幂级数,比存在唯一的$r\in[0,+\infty]$,使得幂级数在$|x|<r$绝对收敛,在$|x|>r$发散,其中$r$称为收敛半径. 注意对于区间的两个端点$\pm r$要单独讨论
\end{theorem}


%\end{document}

% 判别法 criterion

%\begin{example}[实数十进制表示法]
%实数可表示成十进制形式
%\[b_mb_{m-1}\cdots b_0.a_1a_2\cdots a_n\cdots,\]
%其中$0\leq a_j,b_j\leq 9,a_j,b_j\in\mathbb{Z}$,其有何意义\footnote{南大秦理真《微积分》5.2节}
%\end{example}
%\begin{analysis}
%考虑$[0,1]$中的实数,设
%\[x=0.a_1a_2\cdots a_n\cdots=\ssum\dfrac{a_n}{10^n}\]
%已知
%\end{analysis}

% \[A=\intab{0}{\frac{\pi}{2}}{\ln(\sin x)}\]