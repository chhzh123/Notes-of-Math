\section{无穷级数}
\subsection{基本性质}
\begin{definition}[无穷级数]
$\{u_n\}$为数列,称形式和
\[\ssum{u_n}=\lim_{n\to\infty}\sum_{k=1}^n u_k=\lim_{n\to\infty}S_n=S\]
为级数$\ssum{u_n}$的和,称其收敛到$S$.
\end{definition}
可由极限直接推导出来的定理在此不再赘述.
\begin{example}
$\ssum{r^n\sin nx},|r|<1$
\end{example}
\begin{analysis}
用三角函数的复数形式,即$\sin x=\dfrac{e^{ix}-e^{-ix}}{2}$求和,其中$i^2=-1$.
\[\begin{aligned}
LHS &= \dfrac{1}{2}\ssum{r^n(e^{inx}-e^{-inx})}\\
&= \dfrac{1}{2}\lim_{n\to\infty}\left(re^{ix}\dfrac{1-(re^{ix})^n}{1-re^{ix}}-re^{-ix}\dfrac{1-(re^{-ix})^n}{1-re^{-ix}}\right)\qquad\mbox{等比数列求和}\\
&\xlongequal{z=re^{ix}} \dfrac{1}{2}\left(\dfrac{z}{1-z}-\dfrac{\bar{z}}{1-\bar{z}}\right)\qquad\mbox{因$|z|<1$故$\displaystyle\lim_{n\to\infty}z^n=0$}\\
&= \dfrac{1}{2}\dfrac{z-\bar{z}}{1-(z+\bar{z})+r^2}\qquad\mbox{$z\bar{z}=r^2$,通分}\\
&= \dfrac{r\sin x}{1-2r\cos x+r^2}
\end{aligned}\]
\end{analysis}

\subsection{正项级数}
\begin{definition}[正项级数]
级数的每一项都\textbf{非负},则称其为正项级数. 由此知正项级数一定为实级数.
\end{definition}
对于正项级数,有以下判定敛散性的方法:
\begin{enumerate}
	\itemsep -3pt
	\item $\ssum{u_n}$收敛的充要条件是$\{S_n\}$有上界
	\begin{analysis}
		必要性:极限有界性;充分性:单调有界原理
	\end{analysis}
	\item 比较判别法
	\begin{itemize}
		\itemsep -3pt
		\item 一般形式:$u_n\leq cu_n$
		\begin{analysis}
			大收敛小收敛,小发散大发散. 证明:收敛推有界,有界推收敛
		\end{analysis}
		\item 极限形式:$\displaystyle\lim_{n\to\infty}\dfrac{u_n}{v_n}=l$
		\begin{analysis}
			$l\in(0,+\infty)$敛散性相同,$l=0,l=+\infty$判定方法同比较判别法一般形式\\
			常用$1/n$或$1/n^2$卡,多项式分式函数
		\end{analysis}
		\item 另一形式:$\dfrac{u_{n+1}}{u_n}\leq\dfrac{v_{n+1}}{v_n}$
		\begin{analysis}
			$\dfrac{u_n}{u_N}\leq\dfrac{v_n}{v_N}$裂项递推
		\end{analysis}
	\end{itemize}
	\item 达朗贝尔(D'Alembert)判别法:$\displaystyle\lim_{n\to\infty}\dfrac{u_{n+1}}{u_n}=l$
	\begin{analysis}
		$l<1$收敛,$l>1$发散. 证明:$\dfrac{u_{n+1}}{u_n}<l+\varepsilon_0=r=\dfrac{r^{(n+1)}}{r_n}$. 幂级数、阶乘
	\end{analysis}
	\item 柯西(Cauchy)判别法:$\displaystyle\lim_{n\to\infty}\sqrt[n]{u_n}=l$
	\begin{analysis}
		$l<1$收敛,$l>1$发散. 实质是等比数列
	\end{analysis}
	\item 拉阿比(Raabe)判别法:$\displaystyle\lim_{n\to\infty}n\left(\dfrac{u_n}{u_{n+1}}-1\right)=S$
	\begin{analysis}
		$S>1$收敛,$S<1$发散. 注意符号反. 引理:$\forall r>p>1,\exists N, n>N, 1+\dfrac{r}{n}>\left(1+\dfrac{1}{n}\right)^p$
	\end{analysis}
	\item 柯西积分判别法:$f(x)$在$[0,+\infty)$连续且单调下降,$u_n=f(n)$,则$\ssum{u_n}$收敛充要条件是$\displaystyle\lim_{x\to+\infty}\int_1^xf(t)dt$存在
\end{enumerate}
方法总结:
\begin{enumerate}
	\itemsep -3pt
	\item 用\textbf{收敛必要条件}判断$u_n\to 0$是否成立,即是否发散\footnote{注:不是充分条件}
	\item 尝试用数列求和方法,如裂项、错位相消等对数列求部分和
	\item 按照上面给出的判定敛散性的方法,依次判定
\end{enumerate}

\subsection{一般项级数}
交错级数莱布尼茨判别法

\subsection{代数运算}
\begin{enumerate}
	\item 结合律:\textbf{条件/绝对收敛}级数任意加括号,和不变;但不能随意去括号
	\item 交换律:\textbf{绝对收敛}成立;条件收敛适当重排,可使新级数发散或收敛到某一特定值(Riemman)
	\item 分配律:\textbf{绝对收敛}成立(Cauchy)
\end{enumerate}


\section{函数项级数}
\subsection{一致收敛}

\subsection{判别法之比较总结}
\begin{table}
\centering
\begin{tabular}{|p{4cm}|p{4cm}|p{4cm}|p{4cm}|}
\hline
判别法 & 数项级数 & 广义积分 & 函数项级数\\\hline
比较判别法 & & & \\\hline
达朗贝尔判别法 & & & \\\hline
柯西判别法 & $\ssum u_n$收敛等价于$\forall \varepsilon>0,\exists N:n>N,\forall p\in\zz^+$有$|u_{n+1}+\cdots+u_{n+p}|<\varepsilon$ & & \\\hline
拉阿比判别法 & & & \\\hline
狄利克雷判别法 & & & \\\hline
阿贝尔判别法 & & & \\
\hline
\end{tabular}
\end{table}

%\begin{example}[实数十进制表示法]
%实数可表示成十进制形式
%\[b_mb_{m-1}\cdots b_0.a_1a_2\cdots a_n\cdots,\]
%其中$0\leq a_j,b_j\leq 9,a_j,b_j\in\mathbb{Z}$,其有何意义\footnote{南大秦理真《微积分》5.2节}
%\end{example}
%\begin{analysis}
%考虑$[0,1]$中的实数,设
%\[x=0.a_1a_2\cdots a_n\cdots=\ssum\dfrac{a_n}{10^n}\]
%已知
%\end{analysis}