% !TEX root = main.tex

\section{大数据中的优化问题与算法}
\subsection{并行优化}
\begin{center}
    \begin{tikzcd}
        & \text{master }g(\vx)\arrow{dl}\arrow{d}\arrow{dr} & \\
        \text{worker1 }f_1(\vx) & \text{worker2 }f_2(\vx) & \text{worker3 }f_3(\vx)
    \end{tikzcd}
\end{center}
\[\min_x\; g(\vx)+\sum_{i=1}^N f_i(\vx)\]

针对LASSO问题,每个人都有一个样本$(A_i,\vb_i)$,最小化样本之和,以及正则化项
\[\begin{cases}
    (A_1,\vb_1)\implies\frac{1}{2}\norm{A_1 \vx-\vb_1}_2^2\\
    \vdots\\
    (A_n,\vb_n)\implies\frac{1}{2}\norm{A_N \vx-\vb_N}_2^2\\
    g(\vx)=v\norm{\vx}_1
\end{cases}\]
原问题即为
\[\min_x\; v\norm{\vx}_1+\frac{c}{2}\norm{\bmat{A_1 & \cdots & A_N}\vx-\bmat{\vb_1 & \cdots & \vb_N}}_2^2\]

\subsubsection{并行梯度下降法}
\[\begin{cases}
    \iter{\vx}{k+\frac{1}{2}}=\iter{\vx}{k}-\alpha\sum_{i=1}^N\nabla f_i(\iter{\vx}{k})\\
    \iter{\vx}{k+1}=\arg\min g(\vx)+\frac{1}{2\alpha}\norm{\vx-\iter{\vx}{k+\frac{1}{2}}}_2^2
\end{cases}\]
计算简单,只需求梯度,但所有梯度类问题都依赖于条件数。通信开销大。

\subsubsection{对偶分解法}
\begin{mini*}
    {}{\sum_{i=1}^N f_i(\vx_i)+g(\vz)}{}{}
    \addConstraint{\vx_i}{=\vz,\forall i}
\end{mini*}
\[L(\vx,\vz,\vv)=\sum_{i=1}^N f_i(\vx_i)+g(\vz)+\sum_{i=1}^N\lrang{\vv_i,\vx_i \vz}\]
\[(\iter{\vx}{k+1},\iter{\vz}{k+1})=\arg\min_{\vx,\vz}L(\vx,\vz,\iter{v}{k})\]
\[\implies\begin{cases}
    \iter{\vx_i}{k+1}=\arg\min_{\vx_i} f_i(\vx_i)+\lrang{\iter{\vv_i}{k},\vx_i}\\
    \iter{\vz}{k+1}=\arg\min_{\vz_i} g(\vz)-\sum_{i=1}^N\lrang{\iter{\vv_i}{k},\vz}\\
    \iter{\vv_i}{k+1}=\iter{\vv_i}{k}+\alpha\lrang{\iter{\vx_i}{k+1},\iter{\vz}{k+1}}
\end{cases}\]
不依赖于条件数,但每一步都需要求解一个最优解,不一定好求。通信开销小,但拉格朗日类方法收敛性差。

\subsubsection{增广拉格朗日函数}
\[\sum_{i=1}^nf_i(\vx_i)+g(\vz)+\sum_{i=1}^n\lrang{\lambda_i,\vx-\vz}+\frac{c}{2}\sum_{i=1}^n\norm{\vx_i-\vz}^2\]
正则项会产生$\vx_i$和$\vz$的交叉项,不好处理

注意到$\vx_i$之间是没有依赖的,故采用交替方向乘子法,加了增广项,可用固定步长达到最优解
\[\begin{cases}
    \iter{\vx_i}{k+1}&=\arg\min f_0(\vx_i)+\lrang{\iter{\lambda_i}{k},\vx_i}+\frac{c}{2}\norm{\vx_i-\iter{\vz}{k}}^2\\
    \iter{\vz}{k+1}&=\arg\min g(\vz)-\lrang{\sum_{i=1}^n\iter{\lambda_i}{k},\vz}+\frac{c}{2}\sum_{i=1}^n\norm{\iter{\vx_i}{k+1}-\vz}^2\\
    \iter{\lambda}{k+1}&=\iter{\lambda}{k}+c(\iter{\vx_i}{k+1}-\iter{\vz}{k+1})
\end{cases}\]
每次要多传一倍的变量,以通信量开销换性能提升

\subsection{无中心分布式优化}
\par 考虑无向图
\begin{center}
    \begin{tikzcd}
        & o\arrow{r}\arrow{ld}\arrow{rd} & o\arrow{d}\\
        o\arrow{rd} & & o\arrow{ld}\arrow{d}\\
        & o\arrow{uu} & o\arrow{l}
    \end{tikzcd}
\end{center}
每个结点自己优化,协同决策
\[\min\sum_{i=1}^n f_i(\vx)\]
梯度下降法,但由于去中心,$\iter{\vx}{k}$无处摆放
\[\iter{\vx}{k+1}=\iter{\vx}{k}-\iter{\alpha}{k}\sum_{i=1}^n\nabla f_i(\iter{\vx}{k})\]
那就每一个点分配一个本地变量$\vx_i$,对邻居的更新做一个加权平均
\[\iter{\vx}{k+1}=\sum_{i=1}^n\omega_{ij}\iter{\vx_j}{k}-\iter{\alpha}{k}\sum_{i=1}^n\omega_{ij}\nabla f_j(\iter{\vx_j}{k})\]
其中
\[\begin{cases}
    \omega_{ij}\ne 0 & (i,j)\in E, i=j\\
    \omega_{ij}=0 & \text{otherwise}
\end{cases}\]
\[W=\bmat{\omega_{ij}},W=W^\T,W\vone=\vone\]

在不可信的系统里面,存在个人隐私等信息,故更激进些,采用自己的梯度进行更新(在本地进行梯度下降),在无人机系统中非常常见
\[\iter{\vx_i}{k+1}=\sum_{i=1}\omega_{ij}\iter{\vx_j}{k}-\iter{\alpha}{k}\nabla f_i(\iter{\vx_i}{k})\]
非常糟糕的算法,如果采用固定步长,则找不到最优解
\begin{analysis}
反证法,假设$\iter{\vx_i}{k}\to \vx^\star$,将$\vx^\star$代入
\[\vx^\star=\sum_{j=1}^n\omega_{ij}\vx^\star-\alpha\nabla f_i(\vx^\star)\iff \nabla f_i(\vx^\star)=0\]
但原问题最优解
\[\sum_{i=1}^n\nabla f_i(\vx^\star)=0\]
与上面的式子不等价
\end{analysis}

改成有约束优化的形式
\begin{mini*}
    {}{\sum_{i=1}^n f_i(\vx_i)}{}{}
    \addConstraint{\vx_1=\vx_2=\cdots=\vx_n}
\end{mini*}
写出拉格朗日函数,对偶分解法
\[\sum_{i=1}^nf_i(\vx_i)+\sum_{(i,j)\in E}\lrang{\vlambda_{ij},\vx_i-\vx_j}\]
别人的东西都在对偶变量中体现,求同存异
\[\begin{cases}
    \iter{\vx_i}{k+1}=\arg\min f_i(\vx_i)+\sum_{(i,j)\in E}(\iter{\vlambda_{ij}}{k},\vx_i)-\sum_{(j,i)\in E}\lrang{\vlambda_{ij},\vx_i}\\
    \iter{\vlambda_{ij}}{k+1}=\iter{\vlambda_{ij}}{k}+\iter{\alpha}{k}(\iter{\vx_i}{k+1}-\iter{\vx_j}{k+1})-\sum_{(j,i)\in E}\lrang{\iter{\vlambda_{ij}}{k}\vx_i}
\end{cases}\]
依然要采用递减步长,才能保证收敛

\begin{center}
    \begin{tikzcd}
        \vx_i\arrow[bend left]{rr}{\vz_{ij}} & & \vx_j\arrow[bend left]{ll}{\vz_{ji}}
    \end{tikzcd}
\end{center}
\begin{mini*}
    {}{\sum_{i=1}^n f_i(\vx_i)}{}{}
    \addConstraint{\vx_i}{=\vz_{ij},\forall (i,j)\in E}
    \addConstraint{\vx_j}{=\vz_{ji},\forall (i,j)\in E}
\end{mini*}
可分线性约束,进而可以用交替方向乘子法

\[\begin{aligned}
    \sum_{i=1}^nf_i(\vx_i)&+\sum_{(i,j)\in E}\lrp{\lrang{\valpha_{ij},\vx_i-\vz_j}+\lrang{\vbeta_{ij}{k}\vx_j-\vz_{ji}}}\\
    &+\sum_{(i,j)\in E}\frac{c}{2}\lrp{\norm{\vx_i-\vz_{ij}}^2+\norm{\vx_j-\vz_{ij}}^2}
\end{aligned}\]
\[\begin{cases}
    \iter{\vx_i}{k+1}&=\arg\min f_i(\vx_i)+\sum_{(i,j)\in E}\lrang{\iter{\alpha_{ij}{k}},\vx_i}+\sum_{(i,j)\in E}\lrang{\iter{\vbeta_{ji}}{k},\vx_i}\\
    \qquad&+\frac{c}{2}\sum_{(i,j)\in E}\norm{\vx_i-\iter{\vz_j}{k}}^2+\frac{c}{2}\sum_{(i,j)\in E}\norm{\vx_i-\iter{\vz_{ji}}{k}}^2\\
    \iter{\vz_j}{k+1}&=\cdots\\
    \iter{\valpha_{ij}}{k+1}&=\cdots\\
    \iter{\vbeta_{ij}}{k+1}&=\cdots
\end{cases}\]

\subsection{有限和优化}
$n$个样本,每个样本为$f_i(\vx)$
\begin{mini*}
    {\vx}{\frac{1}{n}\sum_{i=1}^nf_i(\vx)}{}{}
\end{mini*}
等价于期望极小化问题
\begin{mini*}
    {\vx}{\E{f_i(\vx,\xi)}}{}{}
\end{mini*}

\[\iter{\vx}{k+1}=\iter{\vx}{k}-\alpha\frac{1}{n}\sum_{i=1}^n\nabla f_i(\iter{\vx}{k})\]
将$k$改为$\iter{i}{k}$,随机梯度下降(Stochastic gradient descent, SGD),取了一个无偏的估计[Bottou, NIPS 2010]
\[\iter{\vx}{k+1}=\iter{\vx}{k}-\iter{\alpha}{k}\nabla f_{\iter{i}{k}}(\iter{\vx}{k})\]
注意这里需要采用变步长,否则无法收敛到最优解
\[\begin{cases}
    \iter{\vx}{k+1}=\iter{\vx}{k}-\iter{\alpha}{k}\nabla f_{\iter{i}{k}}(\iter{\vx}{k})\\
    \vx^\star=\vx^\star-\iter{\alpha}{k}\nabla f_{\iter{i}{k}}(\vx^\star)
\end{cases}\implies\nabla f_{\iter{i}{k}}(\vx^\star)=0\]

若问题强凸,$O(\frac{1}{k})\to O(\frac{1}{k})$;
凸,$O(\frac{1}{\sqrt{k}})\to O(\frac{1}{\sqrt{k}})$

梯度噪声的问题:选的随机梯度与真正的全梯度不同

\subsection{方差消减}
方差消减(Variance Reduction):挑选样本数目增大时,方差会减小
\begin{enumerate}
\item 小批量(mini-batch)
\item SURG、SAG、SAGA
\[\iter{\vx}{k+1}=\iter{\vx}{k}-\frac{\alpha}{n}\sum_{i=1}^n\iter{y_i}{k}\]
对于每一个样本都存储一个梯度值
\[\iter{y_i}{k}=\begin{cases}
    \nabla f_i(\iter{\vx}{k}) & i=\iter{i}{k}\\
    \iter{y_i}{k-1} & i\ne\iter{i}{k}
\end{cases}\]
当时间足够长,每一个里面都存在最优梯度
\[\vx^\star=\vx^\star-\frac{\alpha}{n}\sum_{i=1}^n\nabla f_i(\vx^\star)\]
相当于用空间换时间
\end{enumerate}

\subsection{深度神经网络}
\begin{mini*}
    {}{\sum_{i=1}^S \iter{E}{i}}{}{}
\end{mini*}
其中,
\[\iter{E}{i}=\frac{1}{2}\norm{\iter{\vx_n}{i}-\iter{Y}{i}}^2\]
为损失函数,$\vx_n$为第$n$层的网络输出$f_n(\vx_{n-1},\omega_n)$,与有限和优化问题相同

反向传播算法(Back propagation):自底向上求出$E$相对于$\vx_n$和$w_n$的梯度
\[\begin{cases}
    \pd{\iter{E}{i}}{\iter{\vx_n}{i}}=\iter{\vx_n}{i}-\iter{Y}{i}\\
    \pd{\iter{E}{i}}{w_n}=\pd{\iter{E}{i}}{\iter{\vx_n}{i}}\pd{\iter{\vx_n}{i}}{w_n}
=\pd{\iter{E}{i}}{\iter{\vx_n}{i}}\pd{f_n(\iter{\vx_n}{i},w_n)}{w_n}
\end{cases}\]

\subsection{在线优化}
在线优化(Online Learning):样本不是已有的,而是依照时间给出的
\begin{mini*}
    {}{\frac{1}{T}\sum_{t=1}^T f_t(\vx)}{}{}
\end{mini*}

\[\vx_{t+1}=\vx_t-\alpha_t\nabla f_t(\vx_t)\]
Regret分析:将当前值丢进下一刻的优化函数中,如果优化效果好,说明有预测能力

\subsection{动态优化}
动态优化问题
\begin{mini*}
    {}{f_t(\vx)}{}{}
\end{mini*}
\[\vx_t=\vx_{t-1}-\alpha\nabla f_t(\vx_{t-1})\]


\subsection{Nesterov加速}
Nesterov加速$\min f(\vx)$:$O\lrp{\frac{1}{k^2}}$
\[\begin{aligned}
    \iter{\vx}{k+1}&=\iter{\vy}{k}-\frac{1}{L}\nabla f(\iter{\vy}{k})\\
    \iter{\vy}{k+1}&=(1-\iter{\vgamma}{k})\iter{\vx}{k+1}+\iter{\vgamma}{k}\iter{\vx}{k}\\
    \iter{\vbeta}{k}&=\frac{1+\sqrt{1+4(\iter{\vbeta}{k-1})^2}}{2},\;\iter{\vbeta}{0}=0\\
    \iter{\vgamma}{k}&=\frac{1-\iter{\vbeta}{k}}{\iter{\vbeta}{k+1}}
\end{aligned}\]
构造两个序列,$\vy$为辅助序列,利用问题本身\textbf{历史信息},做一个凸组合(先跳一步,从$\iter{\vy}{k}$开始做梯度下降)。
权重为$\vgamma$,不同时刻权重不同,引入$\vbeta$系数。

Trick:为避免权重趋于0($\vx$和$\vy$趋同),加速了$n$步后重新设置$\vbeta$为$0$。

梯度下降相当于对$f$做一个二阶近似,二阶Taylor展开。
\begin{center}
    \begin{tikzcd}
        \quad\arrow{r}{\iter{\vx}{k}} & f(\vx)\arrow{r} & \nabla f(\iter{\vx}{k})
    \end{tikzcd}
\end{center}

Nesterov加速是针对确定性优化问题,而机器学习是随机优化问题。