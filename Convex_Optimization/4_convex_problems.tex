% !TEX root = main.tex

\section{凸优化问题}
\subsection{标准型}
广义定义:极小化凸函数,约束为凸集
\[\begin{aligned}
& \text{minimize}& f_0(\vx)& &\\
& \text{subject to}& f_i(\vx)&\leq 0 &\quad i&=1,\ldots,m\\
&  & h_j(\vx)&=0 &\quad j&=1,\ldots,p
\end{aligned}\]
\begin{itemize}
	\item 优化变量$\vx\in\rn$
	\item 目标/损失函数$f_0:\rn\mapsto \rr$
	\item 不等式约束函数$f_i:\rn\mapsto \rr$
	\item 等式约束函数$h_j:\rn\mapsto\rr$
	\item 域$\sD=\bigcap_{i=0}^m\dom f_i\cap\bigcap_{i=1}^p\dom h_i$
	\item 可行解$\sX=\{\vz\mid f_i(\vz)\leq 0,h_j(\vz)=0,i=1,\ldots,m,j=1,\ldots,p\}$
	\item 最优值$P^\star=\inf\{f_0(\vx)\mid x\in \sX\}$
	\item 最优解$\vx^\star\iff\forall \vz\in\rn,\vz\in\sX:\;f_0(\vz)\geq f_0(\vx^\star)$
	\item 最优解集$X^\star=\{x^\star\mid f_0(\vx^\star)=P^\star,\vx^\star\in \sS\}$
	\item $\eps$-次优解集$X_\eps=\{\vx\mid f_0(\vx)\leq P^\star+\eps, \vx\in \sX\}$
	\item 局部最优$\exists R>0,f_0(x)=\inf\{f_0(\vz)\mid \vx\in\sX, \vz\in\sX,\|\vx-\vz\|\leq R\}$
	\item 局部最优解集$x_{local}=\{\vx\mid\vx\text{为局部最优}\}$
\end{itemize}

狭义定义:$f_i(x),i=0,1,\ldots$为凸函数,$h_i(x)$为仿射函数
\begin{example}
\[\begin{aligned}
& \min &f_0(x)=x_1^2+x_2^2 & \\
& \text{s.t.} & f_1(x)=\frac{x_1}{1+x_2^2}\leq 0 & \implies x_1\leq 0\\
& & h_1(x)=(x_1+x_2)^2=0 & \implies x_1+x_2=0
\end{aligned}\]
\end{example}

% 3.19
\begin{theorem}
凸问题局部最优等价于全局最优
\end{theorem}
\begin{analysis}
若$x$为局部最优
\[\exists R>0:\;f_0(x)=\inf\{f_0(z)\mid z\in\sX,x\in\sX,\|x-z\|_2\leq R\}\]
反证法,设$x$不是全局最优,$y$为全局最优,$f_0(x)>f_0(y)$\\
$z=\theta x+(1-\theta)y$,$\theta=\frac{R}{2\|y-x\|_2}$
\[\|z-x\|_2=\frac{R\|y-x\|_2}{2\|y-x\|_2}=\frac{R}{2}\]
由$\|z-x\|_2\leq R\implies f_0(x)\leq f_0(z)$,有
\[f_0(z)\leq\theta f_0(x)+(1-\theta)f_0(y)<\theta f_0(z)+(1-\theta)f_0(z)=f_0(z)\]
矛盾
\end{analysis}

可微凸目标函数

无约束$\min f_0(x),\nabla f_0^\star(x)=0$
\[\begin{aligned}
\forall x,y:&\;f_0(y)\geq f_0(x)+\lrang{\nabla f_0(x),y-x}\\
f_0(y)&\geq f_0(x^\star)+\lrang{\nabla f_0(x^\star),y-x}=f_0(x^\star)
\end{aligned}\]

有约束$\min f_0(x),\,s.t. x\in\sX$
\[x^\star\in\sX,\lrang{\nabla f_0(x^\star),y-x^\star}\geq,\forall y\in x\]

\begin{example}
等式约束 $\min f_0(x),\dom f_0\subset\rn$,$f_0$可微,使得$Ax=b$
\end{example}
\begin{analysis}
$x^\star$最优,$Ax^\star=b,\forall y Ay=b$

\[\lrang{\nabla f_0(x^\star),y-x^\star}\geq 0\]
\[\begin{cases} y=x^\star+v\\Av=0\end{cases},v\in\opnul A\]
\[\forall v\in\opnul A,\lrang{\nabla f_0(x^\star),v}\geq 0\]
\begin{enumerate}
	\item $\opnul A=\{0\}$
	\item $A$不可逆,$\nabla f_0(x^\star)\perp\opnul A$
\end{enumerate}
\end{analysis}

\begin{example}
正约束$\min f_0(x),s.t. x\geq 0$
\end{example}
\begin{analysis}
若$x^\star$最优,$\iff x^\star\geq 0,\forall y\geq 0,\lrang{\nabla f_0(x^\star),y-x^\star}\geq 0$
\[\iff \lrang{\nabla f_0(x^\star,y)}\geq\lrang{\nabla f_0(x^\star,x^\star)}\]
\begin{enumerate}
\item 若$\nabla f_0(x^\star)\ngeq 0$有矛盾(负数行乘上正无穷),故$\nabla f_0(x^\star)\geq 0$
\item 令$y=0$,有$0\geq \lrang{\nabla f_0(x^\star),x^\star}\implies\sum_{i=1}n(\nabla f_0(x^\star)_ix^\star)\leq 0$\\
前面$\geq 0$,进而互补松弛条件
\item $x^\star\geq 0$
\end{enumerate}
\end{analysis}

\subsection{线性规划}
\begin{mini*}
	{}{c^\T\vx+\vd}{}{}
	\addConstraint{G\vx}{\leq\vh}
	\addConstraint{A\vx}{\vb}
\end{mini*}
\begin{mini*}
	{}{c^\T\vx+\vd}{}{}
	\addConstraint{G\vx+\vs}{\vh}
	\addConstraint{\vs}{\geq 0}
\end{mini*}
$\vs$为松弛变量(slack variable)\\
用$\vx^+$和$\vx^-$拆分,得到$\vx=\vx^+-\vx^-,\vx^+\geq 0,\vx^-\geq 0,\vs\geq 0$

\begin{example}[食谱问题]
$m$种营养元素不小于$b_1,\ldots,b_m$,$n$种食物,单位含量$a_{1j},\ldots,a_{mj}$,食物量$x_1,\ldots,x_n$,价格$c_1,\ldots,c_n$
\begin{mini*}
	{}{\sum_{i=1}^n c_jx_j}{}{}
	\addConstraint{\sum_{j=1}^n a_{ij}x_j}{\geq b_i}
	\addConstraint{x_j}{\geq 0}
\end{mini*}
其中$i=1,\ldots,m,j=1,\ldots,n$
\end{example}

% 3.21
线性分数规划
\begin{mini*}
	{}{f_0(x)=\frac{c^\T x+d}{e^\T x+f}, \dom f=\{x \mid e^\T x + f >0 \}}{}{}
	\addConstraint{Gx}{\leq h}
	\addConstraint{Ax}{=b}
\end{mini*}
等价于
\begin{mini*}
	{}{c^\T y+dz}{}{}
	\addConstraint{Gy-hz}{\leq 0}
	\addConstraint{Ay-bz}{=0}
	\addConstraint{e^\T y+fz}{=1}
	\addConstraint{z}{\geq 0}
\end{mini*}
\begin{analysis}
	证明两个问题等价,$P_0$与$P_1$\\
	若$x$在$P_0$内可行
	\[y=\frac{x}{e^\T x+f},z=\frac{1}{e^\T x+f}\]
	若$(y,z)$在$P_1$中可行
	\[x=\frac{y}{z}(z\ne 0)\]
	若$z=0$,$x_0$为$P_0$的可行解
	\[x=x_0+ty,t\geq 0\]
	\[\lim_{t\to\infty}\frac{c^\T(x_0+ty)+d}{e^\T(x_0+ty)+f}=c^\T y\]%考试
	代入看所有条件结论都相同
\end{analysis}

二次规划(Quadratic Programming)
\begin{mini*}
	{}{\frac{1}{2}x^\T px+q^\T+r,\;p\succ 0}{}{}
	\addConstraint{Gx}{\leq h}
	\addConstraint{Ax}{=b}
\end{mini*}

二次约束二次规划(QCQP)
\begin{mini*}
	{}{\frac{1}{2}p_0x+q_0^\T x+r_0,p\succ 0}{}{}
	\addConstraint{\frac{1}{2}x^\T p_ix+q_i^\T x+r_i}{\leq 0}{,i=1,\ldots,m,p_i\succ 0}
	\addConstraint{Ax}{=b}
\end{mini*}

最小二乘问题
\begin{mini*}
	{x}{\frac{1}{2}\|Ax-b\|_2^2}{}{}
	\addConstraint{Ax+e}{=b}
\end{mini*}
\[\frac{1}{2}(x^\T A^\T Ax-2b^\T Ax+b^\T b)\]

一范数规范化最小二乘
\[\min\frac{1}{2}\|Ax-b\|_2^2+\lambda_1\|x\|_1\]
本来用零范数,但用一范数拟合

改写
\[\|x\|_1=\vone^\T\vx^++\vone^\T\vx^-\]

Basic Pursuit
\begin{mini*}
	{}{\frac{1}{2}\|Ax-b\|_2^2}{}{}
	\addConstraint{\|x\|_1}{\leq\eps_1}
\end{mini*}
原式很难平衡两者,下式只需考虑$\|x\|_1$的影响

岭回归(Ridge):所有$x$差距不要太大
\[\min\frac{1}{2}\|Ax-b\|_2^2+\frac{1}{2}\lambda_2\|x\|_2^2\]
\begin{mini*}
	{}{\frac{1}{2}\|Ax-b\|_2^2}{}{}
	\addConstraint{\|x\|_2^2}{\leq\eps_2}
\end{mini*}

投资组合问题(portfolio optimization):
初始价格$x_1,\ldots,x_n$,最终价格$P_1x_1,\ldots,P_nx_n$
\begin{maxi*}
	{}{P_1x_1+\cdots+P_nx_n}{}{}
	\addConstraint{x_1+\cdots+x_n}{=B}
	\addConstraint{}{x_1,\ldots,x_n\geq 0}
\end{maxi*}

$\bar{P}=\E{P}$已知,$\Sigma=\Var{P}$
\begin{mini*}
	{}{x^\T\Sigma x}{}{}
	\addConstraint{p^\T x}{\geq r_{\min}}
	\addConstraint{x_1+\cdots+x_n}{=B}
	\addConstraint{}{x_1,\ldots,x_n\geq 0}
\end{mini*}

% 3.26
半定规划(semi-definite programming, SDP)(矩阵意义下的线性规划问题):% 通信
$X\in\rr^{n\times n},C\in\rr^{n\times n},A_i\in\rr^{n\times n},b_i\in\rr$
\begin{mini*}
	{}{\optr(CX)}{}{}
	\addConstraint{\optr(A_i X)}{=b_i}{,i=1,\ldots,p}
	\addConstraint{X}{\succeq 0}
\end{mini*}

\begin{example}[谱范数极小化问题]
	矩阵多项式$A(x)=A_0+x_1A_1+\cdots+x_nA_n,A_i\in\rr^{p\times q}$
	\[\min_x\|A(x)\|_2\]
	谱范数代表$A(x)$的最大奇异值\footnote{谱范数是诱导范数,F-范数(Frobenias)$\|A(x)\|_F$才算是矩阵意义下的2范数}
\end{example}
\begin{mini*}
	{x,s}{S}{}{}
	\addConstraint{A^\T (x)A(x)}{\preceq SI}
\end{mini*}

\begin{example}[最快分布式线性平均]
	\[x(t) = Px(t-1)\]
	\[P=\bmat{P_{11} & \cdots & P_{1n}\\\vdots & & \vdots\\P_{n1} & \cdots & P_{nn}},P\vone=\vone\]
	其中$(i,j)\in E$或$i=j$,$P_{ij}\ne 0$;否则$P_{ij}=0$\\
	$P=P^\T,P_{ij}=P_{ji}$,$P_{ij}>0$\\
	$P\succeq 0, P_{ij}\geq 0$\\
	只要图是连通图,则一定会收敛\\
	收敛速度与第二大[特征值绝对值]有关
	\[1=\lambda_1\geq\lambda_2\geq \cdots\geq \lambda_n\geq -1\]
	即收敛速度由$\max\{|\lambda_2|,|\lambda_n|\}$决定\\
	\[\min\max\{|\lambda_2|,|\lambda_n|\}\]
	\[\max\{|\lambda_2|,|\lambda_n|\}=\norm{P-\frac{1}{n}\vone\vone^\T}\]
	\begin{mini*}
		{}{t:=\norm{P-\frac{1}{n}\vone\vone^\T}_2}{}{}
		\addConstraint{P\vone}{=\vone}
		\addConstraint{P}{=P^\T}
		\addConstraint{P}{\succeq 0}
		\addConstraint{P_{ij}}{= 0,\quad (i,j)\ne E\land i\ne j}
		\addConstraint{-tI}{\preceq P-\frac{1}{n}\vone\vone^\T\preceq tI}
	\end{mini*}
\end{example}

多目标优化问题:帕累托最优解\\
若有另一解在某个指标上更好,则必有指标更差

帕累托最优值/帕累托最优面

$\min f_{01}$与$\min f_{02}$的交点为理想点(oracle)

若$f_{01}(x),\ldots,f_{0q}(x)$为凸,$\sX$为凸
\begin{mini*}[2]
% variable function label LHS
{}{\lambda_1f_{01}(x)+\cdots+\lambda_qf_{0q}(x),\quad\lambda_1,\ldots,\lambda_q\geq 0}
{}{}
\addConstraint{x\in\sX}{}
\end{mini*}
\begin{enumerate}
	\item 能找到一个Pareto最优解
	\item 遍历$\lambda_1,\ldots,\lambda_q$,可找到全部
\end{enumerate}

岭回归的多目标优化表示
\[\begin{cases}
	\min \frac{1}{2}\norm{Ax-b}_2^2\\
	\min \frac{1}{2}\norm{x}_2^2
\end{cases}\]