% !TEX root = main.tex

\section{凸函数} % 3.5
\begin{definition}[凸函数]
\begin{enumerate}
	\item $f:\rn\to\rr$为凸$\iff\dom f$为凸且$\forall x,y\in\dom f,\theta\in[0,1]$
\[f(\theta x+(1-\theta)y)\leq\theta f(x)+(1-\theta)f(y)\]
\begin{itemize}
	\item 严格凸:$\theta\in(0,1)$,不等式不能取等
	\item 凹函数:若$-f$为凸
\end{itemize}
	\item 高维定义:
$f:\rn\mapsto\rr$为凸$\iff\dom f$为凸
\[\forall x\in\dom f,v\in\rn:g(t):=f(x+tv)\text{为凸,}\dom g=\{t\mid x+tv\in\dom f\}\]
相当于每一个剖面上的低维函数都是凸的
	\item 一阶条件(first-order condition)\footnote{$\nabla^\T f(x)=[\nabla f(x)]^\T$}
	\[f(y)\geq f(x)+\nabla^\T f(x)(y-x)\]
	\item 二阶条件:% 3.7
	$f:\rn\mapsto\rr$为凸$\iff\dom f$为凸
	\[\forall x\in\dom f:\nabla^2 f(x) \succeq 0\]
	\begin{itemize}
		\item 凹函数:$\nabla^2 f(x)\preceq 0$
		\item 严格凸:\textcolor{red}{$\impliedby$}$\nabla^2 f(x)\succ 0$,反例$f(x)=x^4$(在一个点斜率不变并不要紧)
	\end{itemize}
\end{enumerate}
\end{definition}

\begin{example}
$f(x)=a^\T x+b$
\end{example}
\begin{analysis}
有$\nabla f(x)=a$,进而
\[f(y)=a^\T y+b\geq a^\T x+b+a^\T(y-x)=a^\T y+b\]
\end{analysis}

\begin{definition}[凸函数的扩展(extended-value)]
尽管凸函数的定义域为凸,但往往不好处理,那就将其扩展到全空间。
$x\in\dom f\subset\rn, \dom\widetilde{f}=\rn$,会有
\[\widetilde{f}(x)=\begin{cases}f(x)&x\in\dom f\\+\infty&x\notin\dom f\end{cases}\]
\end{definition}

指示/示信(indicator)函数不一定是凸的
\[f(x)=\begin{cases}0&x\in C\\ +\infty & x\notin C\end{cases}\]

\begin{theorem}
若$f$为凸,可微,则$\exists x\in\dom f,\nabla f(x)=0$
\end{theorem}

\begin{example}
二次函数$f(x)=\dfrac{1}{2}x^\T Px+q^\T x+r$,$P\in S^n$(对称矩阵),$q^\T\in\rn$,$r\in\rr$
\end{example}
\begin{analysis}
$\nabla^2 f(x)=P$\\
$P\in S^n_+$凸,$P\in S_{++}^n$严格凸
\end{analysis}

\begin{example}
$f(x)=\dfrac{1}{x^2},\dom f=\lrb{x\in\rr,x\ne 0}$
\end{example}
\begin{analysis}
注意$\dom f$不是凸集
\end{analysis}
\begin{itemize}
	\item 指数函数$f(x)=\ee^{ax}$
	\item 幂函数$f(x)=x^a$
	\item 绝对值的幂函数$f(x)=|x|^p,x\in\rr,p>0$:$p\in[1,+\infty)$凸,$p\in(0,1)$既不凸又不凹
	\begin{analysis}
	\[f''(x)=\begin{cases}
	p(p-1)x^{p-2} & x<0\\
	p(p-1)(-x)^{p-2} & x<0
	\end{cases}\]
	\end{analysis}
	\item 对数函数$f(x)=\log x$
	\item 熵$f(x)=-x\log x$
	\item 极大值函数$f(x)=\max\lrb{x_1,\ldots,x_n},x\in\rn$
\end{itemize}

\begin{definition}[解析近似]
无穷阶可微
\end{definition}

极大值函数的解析近似是$f(x)=\log(\ee^{x_1}+\cdots+\ee^{x_n})$
\[\max\{x_1,\ldots,x_n\}\leq f(x)\leq\max\{x_1,\ldots,x_n\}+\log n\]
\begin{analysis}
\[\pd{f}{x_i}=\frac{\ee^{x_i}}{\ee^{x_i}+\cdots+\ee^{x_n}}\]
\[\pddxy{f}{x_i}{x_j}=\begin{cases}\frac{-\ee^{x_i}\ee^{x_i}}i=j\\i\ne j\end{cases}\]
\[z:=\bmat{\ee^{x_1}&\cdots&\ee^{x_n}}^\T\]
求Hessian矩阵
\[H=\frac{1}{(\vone^\T z)^2}(-z\cdot z^\T+(\vone^\T z)\opdiag(z))\]
将前面常量丢弃
\footnote{$H$半正定,则$\forall v\in\rn:v^\T Hv\geq 0$}
\[a_i:=v_i\sqrt{z_i}=\bmat{a_1&\cdots&a_n}^\T,b_i=\sqrt{z_i}\]
\[\begin{aligned}
v^\T Hv&=(\mathbbm{1}^\T z)v^\T \opdiag(z)v-v^\T zz^\T v\\
&=(\sum_i z_i)(\sum_i v_i^2 z_i)-(\sum_i v_i z_i)^2\\
&=(b^\T b)(a^\T a)-(a^\T b)^2\qquad\text{Cauchy}\\
&\geq 0
\end{aligned}\]
\end{analysis}

\begin{definition}[范数]
$p(x)$为范数
\begin{enumerate}
	\item $p(ax)=|a|p(x)$
	\item $p(x+y)\leq p(x)+p(y)$
	\item $p(x)=0\iff x=0$
\end{enumerate}
零范数$\|x\|_0$:非零元素数目,是伪范数(不符合第一个定义)
\end{definition}
$\rn$中的范数都是凸函数,正则化!
\begin{analysis}
\[\forall x,y,\theta\in[0,1]
p(\theta x+(1-\theta)y)\leq p(\theta x)+p((1-\theta)y)
\leq \theta p(x)+(1-\theta)p(y)\]
\end{analysis}

% 3.12
行列式的对数$f(x)=\log\det(x),\dom f=S_{++}^n$
$n=1$凹函数
证$n>1$也为凹,用高维定义
\[\begin{aligned}
g(t)&=f(z+tv)\\
&=\log\det(z+tv)\\
&=\log\det(z^{1/2}(I+tz^{1/2}vz^{-1/2})z^{1/2},\quad z^{1/2}\in S_{++}^n,z^{1/2}z^{1/2}=z\\
&=\log\det(z)+\log\det(I+tz^{1/2}vz^{-1/2})\\
&=\log\det(z)+\sum_{i=1}^n\log(1+t\lambda_i),\quad \lambda_i=z^{-1/2}vz^{1/2}\text{的特征值}
\end{aligned}\]
\[\begin{aligned}
g'(t)&=\sum_{i=1}^n\frac{\lambda_i}{1+t\lambda_i}\\
g''(t)&=\sum_{i=1}^n-\frac{\lambda_i^2}{(1+t\lambda_i)^2}
\end{aligned}\]
补充证明:对对称阵特征值分解$tz^{1/2}vz^{1/2}=tQ\Lambda Q^\T$,对角阵$\Lambda$即为
$QQ^\T=I$,Q为酉矩阵
\[I+tz^{-1/2}vz^{-1/2}=QQ^\T+tQ\Lambda Q^\T=Q(I+t\Lambda)Q^\T\]
\[\log\det(I+tz^{-1/2}vz^{-1/2})=\log\det(Q)+\log\det(I+t\Lambda)+\log\det(Q^\T)\]

保持函数凸性
\begin{itemize}
	\item 非负加权和$f_1,\ldots,f_m$为凸,定义域$\rn$
	\[f:=\sum_{i=1}^m w_if_i,w_i\geq 0\]
	\item 非负积分$f(x,y)$对$y\in A$均为凸($A$不一定为凸),$w(y)\geq 0$
	\[g(x):=\int_{y\in A}w(y)f(x,y)\diff y\]
	\item 仿射映射$f:\rn\mapsto\rr$为凸,$A\in\rr^{m\times n},b\in\rn$,$\dom g=\{x\mid Ax+b\in\dom f\}$
	\[g(x):=f(Ax+b)\]
	\begin{analysis}
	\begin{itemize}
		\item $\dom f$为凸,则$\dom g$为凸
		\item $\forall x,y\in\dom g,\forall\theta\in[0,1]$
		\[\begin{aligned}
		g(\theta x+(1-\theta)y)&=f(A(\theta x+(1-\theta)y)+b)\\
		&=f(\theta(Ax+b)+(1-\theta)(Ay+b))\\
		&\leq\theta f(Ax+b)+(1-\theta)f(Ay+b)\\
		&=\theta g(x)+(1-\theta)g(y)
		\end{aligned}\]
		\item 其实只是在定义域上改变,而不是改变值域,因而函数凸性不会改变
	\end{itemize}
	\end{analysis}
	\item 两个函数的极大值函数,$f_1,f_2$为凸
	\[f(x):=\max\{f_1(x),f_2(x)\},\dom f=\dom f_1\cap\dom f_2\]
	\item 任意个凸函数极大值函数为凸
	\[f(x)=\max\{a_1^\T x+b_1,\ldots,a_m^\T+b_m\}\]
	\item 无限个凸函数,$y\in A$,$f(x,y)$对于$x$为凸,则
	\[g(x):=\sup_{y\in A} f(x,y)\]
	\begin{example}
	点$x$到集合$\sC$的最远距离
	\[f(x)=\sup_{y\in A}\|x-y\|\]
	位移对于范数凸性不会有影响
	\end{example}
	\begin{example}
	$x\in\rn$,$x[i]$为第$i$大元素,$x[1]\geq x[2]\geq\cdots\geq x[r]\geq\cdots\geq x[n]$
	\[f(x):=\sum_{i=1}^r x[i]\]
	\begin{itemize}
	\item $r=1$:$f(x)=x[1]=\max\{x_1,\ldots,x_n\}$,每一项都是$\ve_i^\T x_i$
	\item $r>1$:$f(x)=\max\{x_{i_1}+\cdots+x_{i_r}\mid 1\leq i_1<i_2<\cdots<i_r\leq n\}$
	\end{itemize}
	\end{example}
	\item 函数的组合:$h:\rr^k\mapsto\rr,g:\rn\mapsto\rr^k$
	\[f:=h\circ g:\rn\mapsto\rr\]
	先考虑$n=k=1,\dom g=\rn,\dom h=\rr^k,\dom f=\rr$,$h,g$二阶可微
	\[\begin{aligned}
	f'(x)&=h'(g(x))\cdot g'(x)\\
	f''(x)&=h''(g(x))(g'(x))^2+h'(g(x))g''(x)>0
	\end{aligned}\]
	即当$g$为凸,$h$为凸且不降;$g$为凹,$h$为凸且不增时,$f(x)$为凸\\
	(若定义域非全空间)当$g$为凸,$h$为凸,扩展值函数$\tilde{h}$不降;$g$为凹,$h$为凸,$\tilde{h}$不增时,$f(x)$为凸
	\begin{example}
	$g$为凸,$\exp g(x)$为凸;$g$为凹,$g>0$,$\log g(x)$为凹;$g$为凸,$g>0$,$1/g(x)$为凸
	\end{example}
	\begin{example}
	$g(x)=x^2,\dom g=\rr$,$h(y)=0,\dom h=[1,2]$,$f=h\circ g$,注意$\tilde{h}$并非不降!
	\end{example}
	\item 函数透视:$P:\rr^{n+1}\mapsto\rn,\dom P\in\rn\times\rr_{++},P(z,t)=\dfrac{z}{t}$ % 3.14
	\[f:\rn\mapsto\rr,g(x,t)=tf(\frac{x}{t}),\dom g=\{(x,t)\mid\frac{x}{t}\in\dom f\},g:\rn\times\rr_{++}\mapsto\rr\]
	若$f(x)$为凸,则$g(x,t)$相对于$(x,t)$联合凸 % proof
	\begin{example}
	\begin{itemize}
		\item $f(x)=x^\T x$, $g(x,t)=x^\T x/t$
		\item $f(x)=-\log x$, $g(x,t)=t\log(t/x)$
		\item $u,v\in\rr_{++}^n$, $g(u,v)=\sum_{i=1}^nu_i\log(u_i/v_i)$,信息论常用,衡量相似性,KL散度
		\[D_{KL}:=\sum_{i=1}^n\lrp{u_i\log\frac{u_i}{v_i}-u_i+v_i}\]
	\end{itemize}
	\end{example}
\end{itemize}

\begin{definition}[$\alpha$次水平集($\alpha$-sub level set)]
$f:\rn\mapsto\rr$,$C_\alpha=\{x\in\dom f\mid f(x)\leq\alpha\}$
\end{definition}
\begin{definition}[拟凸函数(quasi-convex)]
$\alpha$次水平集为凸集$\iff$$f$为拟凸函数
\end{definition}
拟凸函数有很好的性质$\to$单模态/单峰函数

凸函数与凸集联系
\begin{itemize}
	\item 凸函数定义域为凸集
	\item 凸函数的$\alpha$次水平集为凸集
\end{itemize}