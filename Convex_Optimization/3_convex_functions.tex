% !TEX root = main.tex

\section{凸函数} % 3.5
\begin{definition}[凸函数]
	凸函数的几种基本定义如下
\begin{enumerate}
	\item $f:\rn\to\rr$为凸$\iff\dom f$为凸且$\forall x,y\in\dom f,\theta\in[0,1]$
\[f(\theta x+(1-\theta)y)\leq\theta f(x)+(1-\theta)f(y)\]
\begin{itemize}
	\item 严格凸:$\theta\in(0,1)$,不等式不能取等
	\item 凹函数:若$-f$为凸
\end{itemize}
	\item 高维定义:
$f:\rn\mapsto\rr$为凸$\iff\dom f$为凸
\[\forall \vx\in\dom f,\vv\in\rn:g(t):=f(\vx+t\vv)\text{为凸,}\dom g=\{t\mid \vx+t\vv\in\dom f\}\]
相当于每一个剖面上的低维函数都是凸的
	\item 一阶条件(first-order condition)\footnote{$\nabla^\T f(\vx)=[\nabla f(\vx)]^\T$, $\nabla^2 f(\vx)$为$f(\vx)$的Hessian矩阵}
	\[f(\vy)\geq f(\vx)+\nabla^\T f(\vx)(\vy-\vx)\]
	\item 二阶条件:% 3.7
	$f:\rn\mapsto\rr$为凸$\iff\dom f$为凸
	\[\forall \vx\in\dom f:\nabla^2 f(\vx) \succeq 0\]
	\begin{itemize}
		\item 凹函数:$\nabla^2 f(\vx)\preceq 0$
		\item 严格凸:\textcolor{red}{$\impliedby$}$\nabla^2 f(\vx)\succ 0$,反例$f(x)=x^4$(在一个点斜率不变并不要紧)
	\end{itemize}
\end{enumerate}
\end{definition}

\begin{example}
$f(\vx)=\va^\T \vx+\vb$
\end{example}
\begin{analysis}
有$\nabla f(\vx)=\va$,进而\footnote{$\lrang{\va,\vx}=\va_1\vx_1+\cdots+\va_n\vx_n\implies\nabla_\vx f(\vx)=\bmat{\pd{f(\vx)}{\vx_1}\\\vdots\\\pd{f(\vx)}{\vx_n}}=\bmat{\va_1\\\vdots\\\va_n}=\va$}
\[f(\vy)=\va^\T \vy+\vb\geq \va^\T \vx+\vb+\va^\T(\vy-\vx)=\va^\T \vy+\vb\]
\end{analysis}

\begin{definition}[凸函数的扩展(extended-value)]
尽管凸函数的定义域为凸,但往往不好处理,那就将其扩展到全空间。
$\vx\in\dom f\subset\rn, \dom\widetilde{f}=\rn$,会有
\[\widetilde{f}(\vx)=\begin{cases}f(\vx)&\vx\in\dom f\\+\infty&\vx\notin\dom f\end{cases}\]
\end{definition}

指示/示信(indicator)函数不一定是凸的
\[f(\vx)=\begin{cases}0&\vx\in \sC\\ +\infty & \vx\notin \sC\end{cases}\]

\begin{theorem}
若$f$为凸,可微,则$\exists \vx\in\dom f,\nabla f(\vx)=0$
\end{theorem}

\begin{example}
二次函数$f(\vx)=\dfrac{1}{2}\vx^\T P\vx+\vq^\T \vx+r$,$P\in \bbs^n$(对称矩阵),$\vq^\T\in\rn$,$r\in\rr$
\end{example}
\begin{analysis}
$\nabla^2 f(\vx)=P$,$P\in \bbs^n_+$凸,$P\in \bbs_{++}^n$严格凸\footnote{由二次型理论(Taylor展开),$\nabla^2 (x^\T P\vx)=P+P^\T$,又由于$P\in\bbs^n$,$\nabla^2 f(\vx)=\frac{1}{2}\cdot 2P=P$,具体推导过程可见\url{https://math.stackexchange.com/questions/239207/hessian-matrix-of-a-quadratic-form}}
\end{analysis}

\begin{example}
$f(x)=\dfrac{1}{x^2},\;\dom f=\lrb{x\in\rr,x\ne 0}$
\end{example}
\begin{analysis}
注意$\dom f$不是凸集
\end{analysis}
\begin{itemize}
	\item 指数函数$f(x)=\ee^{ax}$
	\item 幂函数$f(x)=x^a$
	\item 绝对值的幂函数$f(x)=|x|^p,x\in\rr,p>0$:$p\in[1,+\infty)$凸,$p\in(0,1)$既不凸又不凹
	\begin{analysis}
	\[f''(x)=\begin{cases}
	p(p-1)x^{p-2} & x<0\\
	p(p-1)(-x)^{p-2} & x<0
	\end{cases}\]
	\end{analysis}
	\item 对数函数$f(x)=\log x$
	\item 熵$f(x)=-x\log x$
	\item 极大值函数$f(x)=\max\lrb{x_1,\ldots,x_n},x\in\rn$
\end{itemize}

\begin{definition}[解析近似]
无穷阶可微
\end{definition}

极大值函数的解析近似是$f(x)=\log(\ee^{x_1}+\cdots+\ee^{x_n})$
\[\max\{x_1,\ldots,x_n\}\leq f(x)\leq\max\{x_1,\ldots,x_n\}+\log n\]
\begin{analysis}
\[\pd{f}{x_i}=\frac{\ee^{x_i}}{\ee^{x_1}+\cdots+\ee^{x_n}}\]
\[\pddxy{f}{x_i}{x_j}=\begin{cases}\frac{-\ee^{x_i}\ee^{x_j}}{(\ee^{x_1}+\cdots+\ee^{x_n})^2} & i\ne j\\ \frac{-\ee^{x_i}(\ee^{x_1}+\cdots+\ee^{x_{i-1}}+\ee^{x_{i+1}}+\cdots+\ee^{x_n})}{(\ee^{x_1}+\cdots+\ee^{x_n})^2} & i=j\end{cases}\]
\[\vz:=\bmat{\ee^{x_1}&\cdots&\ee^{x_n}}^\T\]
求Hessian矩阵
\[H=\frac{1}{(\vone^\T \vz)^2}(-\vz\cdot \vz^\T+(\vone^\T \vz)\opdiag(\vz))\]
将前面常量丢弃
\footnote{$H$半正定,则$\forall \vv\in\rn:\;\vv^\T H\vv\geq 0$}
\[\va_i:=\vv_i\sqrt{\vz_i}=\bmat{a_1&\cdots&a_n}^\T,b_i=\sqrt{\vz_i}\]
\[\begin{aligned}
\vv^\T H\vv&=(\mathbbm{1}^\T \vz)\vv^\T \opdiag(\vz)\vv-\vv^\T zz^\T v\\
&=(\sum_i z_i)(\sum_i v_i^2 z_i)-(\sum_i v_i z_i)^2\\
&=(\vb^\T \vb)(\va^\T \va)-(\va^\T \vb)^2\qquad\text{Cauchy}\\
&\geq 0
\end{aligned}\]
\end{analysis}

\begin{definition}[范数]
$\norm{\cdot}$为范数需要满足以下三个条件
\begin{enumerate}
	\item $\norm{a\vx}=|a|\norm{\vx}$
	\item $\norm{\vx+\vy}\leq \norm{\vx}+\norm{\vy}$
	\item $\norm{\vx}=0\iff \vx=\vzero$
\end{enumerate}
零范数$\|\vx\|_0$:非零元素数目,是伪范数(不符合第一个定义)
\end{definition}
\begin{theorem}
$\rn$中的范数都是凸函数,因而常常用来正则化!
\end{theorem}
\begin{analysis}
\[\forall \vx,\vy,\theta\in[0,1]
\norm{\theta \vx+(1-\theta)\vy}\leq \norm{\theta \vx}+\norm{(1-\theta)\vy)}\leq \theta \norm{x}+(1-\theta)\norm{y}\]
\end{analysis}

% 3.12
\begin{example}
行列式的对数$f(\vx)=\log\det(\vx),\dom f=\bbs_{++}^n$,$n=1$凹函数,证明$n>1$也为凹
\end{example}
\begin{analysis}
用高维定义
\[\begin{aligned}
g(t)&:=f(\vz+t\vv)\\
&=\log\det(\vz+t\vv)\\
&=\log\det(\vz^{1/2}(I+t\vz^{1/2}\vv\vz^{-1/2})\vz^{1/2},\quad \vz^{1/2}\in \bbs_{++}^n,\vz^{1/2}\vz^{1/2}=\vz\\
&=\log\det(\vz)+\log\det(I+t\vz^{1/2}\vv\vz^{-1/2})\\
&=\log\det(\vz)+\sum_{i=1}^n\log(1+t\lambda_i),\quad \lambda_i\text{为}\vz^{-1/2}\vv\vz^{1/2}\text{的特征值}
\end{aligned}\]
\[\begin{aligned}
g'(t)&=\sum_{i=1}^n\frac{\lambda_i}{1+t\lambda_i}\\
g''(t)&=\sum_{i=1}^n-\frac{\lambda_i^2}{(1+t\lambda_i)^2}
\end{aligned}\]
补充证明:对对称阵进行特征值分解$t\vz^{1/2}\vv\vz^{1/2}=tQ\Lambda Q^\T$,对角阵$\Lambda$即为
$QQ^\T=I$,Q为酉矩阵
\[I+t\vz^{-1/2}\vv\vz^{-1/2}=QQ^\T+tQ\Lambda Q^\T=Q(I+t\Lambda)Q^\T\]
\[\log\det(I+t\vz^{-1/2}\vv\vz^{-1/2})=\log\det(Q)+\log\det(I+t\Lambda)+\log\det(Q^\T)\]
\end{analysis}

保持函数凸性
\begin{itemize}
	\item 非负加权和$f_1,\ldots,f_m$为凸,定义域$\rn$
	\[f:=\sum_{i=1}^m w_if_i,w_i\geq 0\]
	\item 非负积分$f(x,y)$对$y\in A$均为凸($A$不一定为凸),$w(y)\geq 0$
	\[g(x):=\int_{y\in A}w(y)f(x,y)\diff y\]
	\item 仿射映射$f:\rn\mapsto\rr$为凸,$A\in\rr^{m\times n},\vb\in\rn$,$\dom g=\{\vx\mid A\vx+\vb\in\dom f\}$
	\[g(\vx):=f(A\vx+\vb)\]
	\begin{analysis}
	\begin{itemize}
		\item $\dom f$为凸,则$\dom g$为凸
		\item $\forall \vx,\vy\in\dom g,\forall\theta\in[0,1]$
		\[\begin{aligned}
		g(\theta \vx+(1-\theta)\vy)&=f(A(\theta \vx+(1-\theta)\vy)+\vb)\\
		&=f(\theta(A\vx+\vb)+(1-\theta)(A\vy+\vb))\\
		&\leq\theta f(A\vx+\vb)+(1-\theta)f(A\vy+\vb)\\
		&=\theta g(\vx)+(1-\theta)g(\vy)
		\end{aligned}\]
		\item 其实只是在定义域上改变,而不是改变值域,因而函数凸性不会改变
	\end{itemize}
	\end{analysis}
	\item 两个函数的极大值函数,$f_1,f_2$为凸
	\[f(x):=\max\{f_1(x),f_2(x)\},\dom f=\dom f_1\cap\dom f_2\]
	\item 任意个凸函数极大值函数为凸
	\[f(x)=\max\{a_1^\T x+b_1,\ldots,a_m^\T+b_m\}\]
	\item 无限个凸函数,$y\in A$,$f(x,y)$对于$x$为凸,则
	\[g(x):=\sup_{y\in A} f(x,y)\]
	\begin{example}
	点$x$到集合$\sC$的最远距离
	\[f(x)=\sup_{y\in A}\|x-y\|\]
	位移对于范数凸性不会有影响
	\end{example}
	\begin{example}
	$x\in\rn$,$x[i]$为第$i$大元素,$x[1]\geq x[2]\geq\cdots\geq x[r]\geq\cdots\geq x[n]$
	\[f(x):=\sum_{i=1}^r x[i]\]
	\begin{itemize}
	\item $r=1$:$f(x)=x[1]=\max\{x_1,\ldots,x_n\}$,每一项都是$\ve_i^\T x_i$
	\item $r>1$:$f(x)=\max\{x_{i_1}+\cdots+x_{i_r}\mid 1\leq i_1<i_2<\cdots<i_r\leq n\}$
	\end{itemize}
	\end{example}
	\item 函数的组合:$h:\rr^k\mapsto\rr,g:\rn\mapsto\rr^k$
	\[f:=h\circ g:\rn\mapsto\rr\]
	先考虑$n=k=1,\dom g=\rn,\dom h=\rr^k,\dom f=\rr$,$h,g$二阶可微
	\[\begin{aligned}
	f'(x)&=h'(g(x))\cdot g'(x)\\
	f''(x)&=h''(g(x))(g'(x))^2+h'(g(x))g''(x)>0
	\end{aligned}\]
	即当$g$为凸,$h$为凸且不降;$g$为凹,$h$为凸且不增时,$f(x)$为凸\\
	(若定义域非全空间)当$g$为凸,$h$为凸,扩展值函数$\tilde{h}$不降;$g$为凹,$h$为凸,$\tilde{h}$不增时,$f(x)$为凸
	\begin{example}
	$g$为凸,$\exp g(x)$为凸;$g$为凹,$g>0$,$\log g(x)$为凹;$g$为凸,$g>0$,$1/g(x)$为凸
	\end{example}
	\begin{example}
	$g(x)=x^2,\dom g=\rr$,$h(y)=0,\dom h=[1,2]$,$f=h\circ g$,注意$\tilde{h}$并非不降!
	\end{example}
	\item 函数透视:$P:\rr^{n+1}\mapsto\rn,\dom P\in\rn\times\rr_{++},P(z,t)=\dfrac{z}{t}$ % 3.14
	\[f:\rn\mapsto\rr,g(x,t)=tf(\frac{x}{t}),\dom g=\{(x,t)\mid\frac{x}{t}\in\dom f\},g:\rn\times\rr_{++}\mapsto\rr\]
	若$f(x)$为凸,则$g(x,t)$相对于$(x,t)$联合凸 % proof
	\begin{example}
	\begin{itemize}
		\item $f(x)=x^\T x$, $g(x,t)=x^\T x/t$
		\item $f(x)=-\log x$, $g(x,t)=t\log(t/x)$
		\item $u,v\in\rr_{++}^n$, $g(u,v)=\sum_{i=1}^nu_i\log(u_i/v_i)$,信息论常用,衡量相似性,KL散度
		\[D_{KL}:=\sum_{i=1}^n\lrp{u_i\log\frac{u_i}{v_i}-u_i+v_i}\]
	\end{itemize}
	\end{example}
\end{itemize}

\begin{definition}[$\alpha$次水平集($\alpha$-sub level set)]
$f:\rn\mapsto\rr$,$C_\alpha=\{x\in\dom f\mid f(x)\leq\alpha\}$
\end{definition}
\begin{definition}[拟凸函数(quasi-convex)]
$\alpha$次水平集为凸集$\iff$$f$为拟凸函数
\end{definition}
拟凸函数有很好的性质$\to$单模态/单峰函数

凸函数与凸集联系
\begin{itemize}
	\item 凸函数定义域为凸集
	\item 凸函数的$\alpha$次水平集为凸集
\end{itemize}