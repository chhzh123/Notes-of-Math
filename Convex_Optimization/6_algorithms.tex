% !TEX root = main.tex

\section{优化算法}
\subsection{简介}
\begin{mini*}
    {}{f_0(x)}{}{}
    \addConstraint{A\vx}{=\vb}
\end{mini*}
罚函数法
\[\min f_0(x)+\frac{\lambda}{2}\norm{A\vx-\vb}_2^2\]
\[\tilde{x}=\arg\min_x F\]
\[\nabla f_0(\tilde{\lambda})+\lambda A^\T(A\tilde{\vx}-\vb)=0\]
\[\begin{aligned}
    L(x,v)&=f_0(x)+v^\T(A\vx-\vb)\\
    \implies g(v)&=\inf_x f_0(x)+v^\T(A\vx-\vb)\\
    v&=\lambda(A\tilde{\vx}-\vb)\\
    \implies & g(\lambda(A\tilde{x}-\vb)=\inf_x f_0(x)+\lambda(A\tilde{x}-b)^\T(A\vx-\vb)
\end{aligned}\]
\[\nabla f_0(x)+\lambda A^\T(A\tilde{x}-\vb)=0\]

\begin{mini*}
    {}{f_0(x)}{}{}
    \addConstraint{A\vx}{\geq\vb}
\end{mini*}
log-barrier
\[\min f_0(x)+\sum_{i=1}^m u_i\log(a_i^\T \vx-b_i)\]


$\min f_0(x)$可微,凸,无约束
\begin{enumerate}
    \item 所有算法都是迭代的
    \[\iter{x}{k+1}=\iter{x}{k}+\iter{\alpha}{k}\iter{d}{k}\]
    $\alpha\geq 0$为步长,$d$为方向,所有算法本质上都是选择方向与步长的问题
    \item 如何选择步长$\iter{\alpha}{k}$
    \[\begin{cases}
    \text{确定步长} & \begin{cases}\text{固定步长}\\\text{变化步长(递减步长)}\end{cases}\\
    \text{搜索步长}
    \end{cases}\]
    最优步长:线搜索问题
    \[\iter{\alpha}{k}=\arg\min_{\alpha\geq 0} f_0(\iter{x}{k}+\alpha\iter{d}{k})\]
    \item 关键问题是选方向
\end{enumerate}
黄金分割法(0.618法)/优选法求解线搜索问题:这样做的采样复杂度很低,之前算过的点很容易被再用!

不精确线搜索(Armijo Rule):一阶泰勒展开
% \begin{algorithm}[H]
%     \begin{algorithmic}[1]
%         \State{$\iter{\alpha}{k}=\alpha_\max$}
%         \If{$f_0(\iter{x}{k}+\iter{\alpha}{k}\iter{d}{k})>f_0(\iter{x}{k}+\mu\iter{\alpha}{k}\lrang{\nabla f_0(\iter{x}{k},\iter{d}{k})})$}
%         \State $\iter{\alpha}{k}\gets\iter{\alpha}{k}\beta,\beta\in(0,1)$
%         \Else
%         \State Stop
%         \EndIf
%     \end{algorithmic}
% \end{algorithm}

实际上没有必要求最优步长,在该方向上的差异并没有太大

\subsection{梯度下降法}
$\iter{d}{k}=-\nabla f_0(\iter{x}{k})$
\begin{itemize}
    \item 能否收敛
    \item 收敛到哪里
    \item 收敛速度
\end{itemize}

假设
\begin{itemize}
    \item[0.] 基本假设:$f$为可微的凸函数,
    \[x^\star=\arg\min_x f_0(x)\]
    存在且有限,$f_0(x^\star)$有限
    \item[1.] Lipschitz连续梯度
    \[\exists L>0,\norm{\nabla f_0(x)-\nabla f_0(y)}\leq L\norm{x-y},\forall x,y\]
    \item[2.] 强凸性(strong convexity)
    \[\exists\mu>0:\;f_0(y)\geq f_0(x)+\lrang{\nabla f_0(x),y-x}+\frac{\mu}{2}\norm{x-y}_2^2,\forall x,y\]
\end{itemize}