% !TEX root = main.tex

\section{优化算法}
\subsection{简介}
\begin{mini*}
    {}{f_0(x)}{}{}
    \addConstraint{A\vx}{=\vb}
\end{mini*}
罚函数法
\[\min f_0(x)+\frac{\lambda}{2}\norm{A\vx-\vb}_2^2\]
\[\tilde{x}=\arg\min_x F\]
\[\nabla f_0(\tilde{\lambda})+\lambda A^\T(A\tilde{\vx}-\vb)=0\]
\[\begin{aligned}
    L(x,v)&=f_0(x)+v^\T(A\vx-\vb)\\
    \implies g(v)&=\inf_x f_0(x)+v^\T(A\vx-\vb)\\
    v&=\lambda(A\tilde{\vx}-\vb)\\
    \implies & g(\lambda(A\tilde{x}-\vb)=\inf_x f_0(x)+\lambda(A\tilde{x}-b)^\T(A\vx-\vb)
\end{aligned}\]
\[\nabla f_0(x)+\lambda A^\T(A\tilde{x}-\vb)=0\]

\begin{mini*}
    {}{f_0(x)}{}{}
    \addConstraint{A\vx}{\geq\vb}
\end{mini*}
log-barrier
\[\min f_0(x)+\sum_{i=1}^m u_i\log(a_i^\T \vx-b_i)\]


$\min f_0(x)$可微,凸,无约束
\begin{enumerate}
    \item 所有算法都是迭代的
    \[\iter{x}{k+1}=\iter{x}{k}+\iter{\alpha}{k}\iter{d}{k}\]
    $\alpha\geq 0$为步长,$d$为方向,所有算法本质上都是选择方向与步长的问题
    \item 如何选择步长$\iter{\alpha}{k}$
    \[\begin{cases}
    \text{确定步长} & \begin{cases}\text{固定步长}\\\text{变化步长(递减步长)}\end{cases}\\
    \text{搜索步长}
    \end{cases}\]
    最优步长:线搜索问题
    \[\iter{\alpha}{k}=\arg\min_{\alpha\geq 0} f_0(\iter{x}{k}+\alpha\iter{d}{k})\]
    \item 关键问题是选方向
\end{enumerate}
黄金分割法(0.618法)/优选法求解线搜索问题:这样做的采样复杂度很低,之前算过的点很容易被再用!

不精确线搜索(Armijo Rule):一阶泰勒展开
% \begin{algorithm}[H]
%     \begin{algorithmic}[1]
%         \State{$\iter{\alpha}{k}=\alpha_\max$}
%         \If{$f_0(\iter{x}{k}+\iter{\alpha}{k}\iter{d}{k})>f_0(\iter{x}{k}+\mu\iter{\alpha}{k}\lrang{\nabla f_0(\iter{x}{k},\iter{d}{k})})$}
%         \State $\iter{\alpha}{k}\gets\iter{\alpha}{k}\beta,\beta\in(0,1)$
%         \Else
%         \State Stop
%         \EndIf
%     \end{algorithmic}
% \end{algorithm}

实际上没有必要求最优步长,在该方向上的差异并没有太大

\subsection{梯度下降法}
$\iter{d}{k}=-\nabla f_0(\iter{x}{k})$
\begin{itemize}
    \item 能否收敛
    \item 收敛到哪里
    \item 收敛速度
\end{itemize}

假设
\begin{itemize}
    \item[0.] 基本假设:$f$为可微的凸函数,
    \[x^\star=\arg\min_x f_0(x)\]
    存在且有限,$f_0(x^\star)$有限
    \item[1.] Lipschitz连续梯度
    \[\exists L\geq 0,\norm{\nabla f_0(x)-\nabla f_0(y)}\leq L\norm{x-y},\forall x,y\]
    等价定义:
    \begin{itemize}
        \item[a.]若$f_0(x)$二阶可微
    \[\nabla^2f_0(x)\preceq LI,\forall x\]
        \item[b.] 下界
    \[\lrang{\nabla f_0(x)-\nabla f_0(y),x-y}\geq\frac{1}{L}\norm{\nabla f_0(x)-\nabla f_0(y)}^2\]
        \item[c.] 上界
    \[\lrang{\nabla f_0(x)-\nabla f_0(y),x-y}\leq L\norm{x-y}\]
        \item[d.] 当函数为凸时
    \[0\leq f_0(y)-f_0(x)-\lrang{\nabla,y-x}\leq\frac{L}{2}\norm{x-y}_2^2\]
    \end{itemize}
    \item[2.] 强凸性(strong convexity)
    \[\exists\mu>0:\;f_0(y)\geq f_0(x)+\lrang{\nabla f_0(x),y-x}+\frac{\mu}{2}\norm{x-y}_2^2,\forall x,y\]
    二阶可微情况下的等价定义
    \[\nabla^2 f(x)\succeq \mu I\]
\end{itemize}
\begin{example}
    \[\begin{aligned}
        f_0(x) &= \vone^\T x\qquad & L &= 0 & \text{\xmark}\\
        f_0(x) &= \frac{1}{2}\norm{x}_2^2 & L &=1 & \mu=1\\
        f_0(x) &= \frac{1}{4}\norm{x}_2^4 & \text{\xmark} & &\text{\xmark}
    \end{aligned}\]
\end{example}

区别于严格凸(strictly convex),强凸一定是严格凸
\begin{theorem}
    严格凸函数只有一个最小值点
\end{theorem}
\begin{analysis}
    反证法,假设$x,y$均为最小值点,且$x\ne y$
    \[f_0(y)>f_0(x)+\lrang{\nabla f_0(x),x-y}=f_0(x)\]
\end{analysis}

\begin{theorem}
    若$f_0(x)$有Lipschitz连续梯度,常数$L$,若$\alpha\in(0,\frac{2}{L})$,则有
    \[\begin{aligned}
        f_0(\iter{x}{k})-f_0(x^\star)&\leq 2(f_0(x^0)-f_0(x^\star))\norm{x^0-x^\star}^2\\
        &=2\norm{x^0-x^\star}^2+k\alpha(2-L\alpha)(f_0(x^0)-f_0(x^\star)),\forall x^\star
    \end{aligned}\]
    即以$O(\frac{1}{k})$速度收敛
\end{theorem}
\begin{analysis}
    $1\degree$ 点的单调性:与任意$x^\star$的距离在缩小
    \[\norm{\iter{x}{k+1}-x^\star}^2\leq\norm{\iter{x}{k}-x^\star}^2,\forall x^\star\]
    \[\begin{aligned}
        LHS &= \norm{\iter{x}{k}-x^\star-\alpha\nabla f_0(x^k)}^2\\
        &=\norm{x^k-x^\star}^2-2\alpha\lrang{x^k-x^\star,\nabla f_0(x^k)}+\alpha^2\norm{\nabla f_0(x^k)}\\
        &\leq \norm{\iter{x}{k}-x^\star}^2+\alpha(\alpha-\frac{2}{L})\norm{\nabla f_0(\iter{x}{k})}^2\qquad\text{注意到$\nabla f_0(x^\star)=0$,利用Lipschitz连续梯度}\\
        &\leq \norm{\iter{x}{k}-x^\star}^2
    \end{aligned}\]
    $2\degree$ 函数值的单调性:$f_0(\iter{x}{k+1})\leq f_0(\iter{x}{k})$(注意下降可能非常缓慢,并不一定收敛)
    \[\begin{aligned}
        f_0(\iter{x}{k+1})&\leq f_0(\iter{x}{k})+\lrang{\nabla f_0(\iter{x}{k}),\iter{x}{k+1}-\iter{x}{k}}+\frac{L}{2}\norm{\iter{x}{k+1}-\iter{x}{k}}^2\\
        &= f_0(\iter{x}{k})-\alpha(1-\frac{L\alpha}{2})\norm{\nabla f_0(\iter{x}{k})}^2\\
        &\leq f_0(\iter{x}{k})
    \end{aligned}\]
    $3\degree$ 函数值的充分下降(即证明收敛性)
    \[f_0(\iter{x}{k+1})-f_0(x^\star)\leq f_0(\iter{x}{k})-f_0(x^\star)-\omega\norm{\nabla f_0(\iter{x}{k})}^2\]
    \[\begin{aligned}
        f_0(\iter{x}{k})-f_0(x^\star)\&leq \lrang{f_0(\iter{x}{k},\iter{x}{k}-x^\star)}\\
        &=\lrang{\nabla f_0(\iter{x}{k})-\nabla f_0(x^\star),\iter{x}{k}-x^\star}\\
        &\leq\norm{\nabla f_0(\iter{x}{k})-\nabla f_0(x^\star)}\norm{\iter{x}{k}-x^\star}\\
        &\leq \norm{\nabla f_0(\iter{x}{k})}\norm{\iter{x}{k}-x^\star}
    \end{aligned}\]
    \[\iter{\Delta}{k+1}\leq \iter{\Delta}{k}-\frac{\omega}{\norm{x^0-x^\star}^2}(\iter{\Delta}{k})^2\]
    \[\frac{1}{\iter{\Delta}{k+1}}\leq\frac{1}{\iter{\Delta}{k+1}}-\frac{\omega}{\norm{x^0-x^\star}^2}\frac{\iter{\Delta}{k}}{\iter{\Delta}{k+1}}\]
    错位相消可得结论$O(\frac{1}{k})$收敛速度
\end{analysis}