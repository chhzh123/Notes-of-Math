\documentclass{note}
\usepackage{mypackage}

\renewcommand{\thefootnote}{\fnsymbol{footnote}}

\title{最优化理论}
\author{陈鸿峥}
\date{{\builddatemonth\today}\protect\footnote{\text{Build \builddate\today}}}%加了build

\begin{document}

\maketitle
\renewcommand{\thefootnote}{\arabic{footnote}}
\setcounter{footnote}{0}

\setcounter{tocdepth}{2}%设置深度
\tableofcontents
\bigskip\bigskip\bigskip

\section{简介} % Lec 1 - 2.26
\subsection{优化概述}
优化(optimization):从一个\emph{可行解}的集合中寻找出\emph{最好}的元素
\begin{itemize}
	\item 优化变量$\vx\in\rn$
	\item 目标/损失函数$f_0:\rn\mapsto \rr$
	\item 不等式约束函数$f_i:\rn\mapsto \rr$
	\item 等式约束函数$h_j:\rn\mapsto\rr$
	\item 可行解$\mathcal{S}=\{\vz\mid f_i(\vz)\leq 0,h_j(\vz)=0,i=1,\ldots,m,j=1,\ldots,p\}$
	\item 最优解$\vx^\star\iff\forall \vz\in\rn,\vz\in\mathcal{S}:\;f_0(\vz)\geq f_0(\vx^\star)$
\end{itemize}
\[\begin{aligned}
& \text{minimize}& f_0(\vx)& &\\
& \text{subject to}& f_i(\vx)&\leq 0 &\quad i&=1,\ldots,m\\
&  & h_j(\vx)&=0 &\quad j&=1,\ldots,p
\end{aligned}\]

\begin{example}
\begin{itemize}
\item 最小二乘线性拟合(凸问题)
\item 深度神经网络(非凸,见下)
\[\begin{aligned}
\vx_1^{(i)}&=f_1(\vx_0^{(i)},\vw_1)\\
\cdots&=\cdots\\
\vx_n^{(i)}&=f_n(\vx_{n-1}^{(i)},\vw_n)\\
\min&\sum_{i=1}^m(\vy^{(i)}-\vx_n^{(i)})^2
\end{aligned}\]
\item 图像处理,自然图像通常都是\textbf{分块光滑}的,原图$\Phi_0$,有噪声的新图$\Phi$\\
全变参(TV, Total Variation)范数,计算图像每个像素点左侧和下侧的差异
\[\|\Phi\|_{TV}=\sum_y\sum_x\sqrt{(\Phi(x,y)-\Phi(x,y-1))^2+(\Phi(x,y)-\Phi(x-1,y))^2}\]
可得优化目标:近似自然图像,而且跟原图不能差太远
\[\min(\|\Phi\|_{TV}+\lambda\|\Phi-\Phi_0\|_F^2)\]
\item 推荐系统:Netflix问题\\
矩阵横向为用户,纵向为电影,值为评分值($1\thicksim 5$),问题是把矩阵补全,这样就可以做推荐了$\to$低秩矩阵补全\\
电影很多,但类型不多,关联关系有限$\to$\textbf{近似低秩}\footnote{$A$的秩等于非零奇异值$\sqrt{\mathop{eig}(A^\T A)}$数目}\\
低秩本来需要最小化$\vz$的非零奇异值数目$\|\vz\|_0$,但是非凸的;
转化为最小化\emph{和范数}\footnote{矩阵所有奇异值之和}$\|\vz\|_\star$
\[\begin{aligned}
\min& \;&\|\vz\|_\star&:=\|\vz\|_1\\
\text{s.t.}& \;&\vz_{ij}&=\vM_{ij},\;(i,j)\in\Omega
\end{aligned}\]
\end{itemize}
\end{example}

\subsection{分类}
\begin{itemize}
	\item 线性规划/非线性规划
	\item 凸规划/非凸规划(更好的分类)
\end{itemize}
目标函数凸函数,可行解集为凸集则是凸优化,一般容易求解

\subsection{历史}
\begin{itemize}
\item Newton-Raphson算法:求零点,等价于求$\min f^2(x)$
\item Gauss-Seidel算法:求解线性方程组$A\vx=\vb$,等价于求$\min \|A\vx-\vb\|_2^2$
\item Lagrange
\item Kantoronc:苏联,线性规划,诺贝尔经济学奖
\item Dantzig:美国,优化决策,线性规划单纯形
\item Von Neumann:线性规划问题对偶理论
\item Karmarkar:80年代,线性规划内点法
\item Nesterov:后80年代,非线性凸优化内点法
\item 现代:并行、随机算法
\end{itemize}

\section{凸集} % Lec 2 - 2.28
\begin{definition}
一些集合概念如下
\begin{itemize}
\item 仿射集(affine set)
\[\begin{aligned}
\mathcal{C}\text{为仿射集}&\iff\text{过$\sC$内任意两点的\textbf{直线}都在$\sC$内}\\
&\iff\forall x_1,x_2\in\mathcal{C},\theta\in\rr,\theta x_1+(1-\theta)x_2\in\mathcal{C}
\end{aligned}\]
\begin{example}
用定义易证线性方程组的解集$\mathcal{C}=\{\vx\mid A\vx=\vb\}$是仿射集;
反过来,每一个仿射集都可以用线性方程组的解集表示
\end{example}
\item 仿射组合
\[\forall x_1,x_2,\ldots,x_k\in\mathcal{C},\theta_1,\ldots,\theta_k\in\rr,{\color{red}{\theta_1+\cdots+\theta_k=1}}:\;\theta_1 x_1+\cdots+\theta_k x_k\in\mathcal{C}\]
\item 仿射包(hull):所有仿射组合的集合
\[\opaff\mathcal{C}:=\{\theta_1 x_1+\cdots+\theta_k x_k\mid\forall x_1,\ldots,x_k\in\mathcal{C},\theta_1+\cdots+\theta_k=1\}\]
\item 凸集(convex set)
\[\begin{aligned}
\mathcal{C}\text{为凸集}&\iff\text{过$\sC$内任意两点的\textbf{线段}都在$\sC$内}\\
&\iff\forall x_1,x_2\in\mathcal{C},\theta\in\bm{[0,1]},\theta x_1+(1-\theta)x_2\in\mathcal{C}
\end{aligned}\]
\item 凸组合
\[\forall x_1,x_2,\ldots,x_k\in\mathcal{C},{\color{red}{\theta_1,\ldots,\theta_k\in[0,1],\theta_1+\cdots+\theta_k=1}}:\;\theta_1 x_1+\cdots+\theta_k x_k\in\mathcal{C}\]
\item 凸包:最小的凸集
\[\opconv\mathcal{C}:=\{\theta_1 x_1+\cdots+\theta_k x_k\mid\forall x_1,\ldots,x_k\in\mathcal{C},\theta_1,\ldots,\theta_k\in[0,1],\theta_1+\cdots+\theta_k=1\}\]
\item 凸锥(convex cone)
\[\mathcal{C}\text{为凸锥}\iff\forall x_1,x_2\in\mathcal{C},\theta_1,\theta_2\geq 0,\theta_1 x_1+\theta_2 x_2\in\mathcal{C}\]
除了\textbf{空集}的凸锥都得包含\textbf{原点}(取$\theta_1=\theta_2=0$)
\item 凸锥组合/非负线性组合:
\[\forall x_1,x_2,\ldots,x_k\in\mathcal{C},{\color{red}{\theta_1,\ldots,\theta_k\geq 0}}:\;\theta_1 x_1+\cdots+\theta_k x_k\in\mathcal{C}\]
\item 凸锥包:类似前面定义
\end{itemize}
\end{definition}
由上面的定义易知,仿射组合/凸锥组合(强条件)一定是凸组合。

\begin{definition}[超平面(hyperplane)与半空间(halfspace)]
超平面都是比原空间低一维
\[\{\vx\mid \va^\T\vx=b,\vx,\va\in\rn,b\in\rr,\va\ne 0\}\]
超平面将空间划分为两个部分,即半空间
\[\{\vx\mid \va^\T\vx\leq b,\va\ne 0\}\]
若方程特解为$\vx_0$,则$\va\perp(\vx-\vx_0)$
\end{definition}
\begin{definition}[欧式球(Euclidean ball)]
\[B(x_c,r)=\{x\mid\|x-x_c\|_2\leq r\}\]
范数(norm)球可类似定义
\end{definition}
\begin{definition}[椭球(ellipsoid)]
\[\eps(x_c,P)=\{x\mid(x-x_c)^\T P^{-1}(x-x_c)\leq 1\},P\succ 0\]
其中$P\succ 0$代表$P$\textbf{对称}且正定($P=P^T$)
\end{definition}
\begin{analysis}
% https://ljk.imag.fr/membres/Anatoli.Iouditski/cours/convex/chapitre_3.pdf
定义内积$\lrang{x^\T P^{-1}y}$(需证满足内积条件),进而P-范数$\|x\|_P:=\sqrt{x^\T Px}$是范数,而椭球不过是P-范数意义下的球,由定理得椭球是凸的
\end{analysis}
\begin{definition}[多面体(polyhedron)]
\[P=\{\vx\mid\va_i^\T\vx\leq b_i,\vc_j^\T\vx=d_j,i=1,\ldots,m,j=1,\ldots,p\}\]
\end{definition}
\begin{example}
\begin{itemize}
	\item 空集、点、$\rn$空间均为仿射
	\item 任意直线为仿射;若过原点则为凸锥
	\item $\rn$空间的子空间\footnote{零元、加法封闭、数乘封闭}为仿射和凸锥
	\item 超平面为仿射
	\item 半空间、欧式球、椭球、多面体为凸集
\end{itemize}
\end{example}

\begin{definition}[仿射函数]
\[f:\rn\mapsto\rr^m\quad f(\vx)=A\vx+\vb,A\in\rr^{m\times n},\vb\in\rr^m\]
性质如下:
\begin{itemize}
	\item $S\subset\rn$为凸$\implies f(S)=\{f)\vx\mid\vx\in S\}$为凸
	\item $C\subset\rr^m$为凸$\implies f^{-1}(C)=\{\vx\in\rn\mid f(\vx)\in C\}$为凸
\end{itemize}
\end{definition}
\begin{example}
两个集合的和$S_1+S_2=\{x+y\mid x\in S_1,y\in S_2\}$保凸
\end{example}
\begin{analysis}
直积$S_1\times S_2=\{(x,y)\mid x\in S_1,y\in S_2\}$显然可以保凸(相当于在两个集合同时画线)\\
令$A=\bmat{I & I},\vx=\bmat{x & y}^\T,\vb=0$,由仿射函数性质知
\end{analysis}

\begin{definition}[透视(perspective)函数\footnote{$+$代表$\geq 0$,$++$代表$>0$}]
透视函数$P:\rr^{n+1}\mapsto\rn,\dom P=\rn\times\rr_{++}$定义如下
\[P(z,t)=\dfrac{z}{t},z\in\rn,t\in\rr_{++}\]
反透视函数
\[P^{-1}(c):=\lrb{(x,t)\in\rr^{n+1}\mid\frac{x}{t}\in c,t>0}\]
\end{definition}

若$c\in\dom P$为凸,则$P(c):=\{P(x),x\in c\}$为凸;反透视函数仍保持$c$的凸性。

考虑$\rr^{n+1}$内的线段,$x=(\widetilde{x}\in\rn,x_{n+1}\in\rr_{++}),y=(\widetilde{y},y_{n+1})$
则经过透视函数仍是线段
\begin{analysis}
\[P(\theta x+(1-\theta)y)=\frac{\theta\widetilde{x}+(1-\theta)\widetilde{y}}{\theta x_{n+1}+(1-\theta y_{n+1})}=\frac{\theta x_{n+1}}{\theta x_{n+1}+(1-\theta)y_{n+1}}\frac{\widetilde{x}}{x_{n+1}}+\frac{(1-\theta)y_{n+1}}{\theta x_{n+1}+(1-\theta)y_{n+1}}\frac{\widetilde{y}}{y_{n+1}}\]
\end{analysis}

\begin{definition}[线性分数函数]
仿射函数
\[g(x)=\bmat{A\\C^\T}x+\bmat{b\\d},A\in\rr^{m\times n},b\in\rm,c\in\rn,d\in\rr\]
线性分数函数$f:\rn\mapsto\rm=p\circ g$
\[f(x)=\frac{Ax+b}{c^\T x+d},\dom f=\{x\mid c^\T+d>0\}\]
\end{definition}

保凸性
\begin{itemize}
	\item 凸集的交
	\item 仿射、逆仿射
	\item 透视函数
	\item 线性分数函数
\end{itemize}

\section{凸函数} % 3.5
\begin{definition}[凸函数]
\begin{enumerate}
	\item $f:\rn\to\rr$为凸$\iff\dom f$为凸且$\forall x,y\in\dom f,\theta\in[0,1]$
\[f(\theta x+(1-\theta)y)\leq\theta f(x)+(1-\theta)f(y)\]
\begin{itemize}
	\item 严格凸:$\theta\in(0,1)$,不等式不能取等
	\item 凹函数:若$-f$为凸
\end{itemize}
	\item 高维定义:
$f:\rn\mapsto\rr$为凸$\iff\dom f$为凸
\[\forall x\in\dom f,v\in\rn:g(t):=f(x+tv)\text{为凸,}\dom g=\{t\mid x+tv\in\dom f\}\]
相当于每一个剖面上的低维函数都是凸的
	\item 一阶条件(first-order condition)\footnote{$\nabla^\T f(x)=[\nabla f(x)]^\T$}
	\[f(y)\geq f(x)+\nabla^\T f(x)(y-x)\]
	\item 二阶条件:% 3.7
	$f:\rn\mapsto\rr$为凸$\iff\dom f$为凸
	\[\forall x\in\dom f:\nabla^2 f(x) \succeq 0\]
	\begin{itemize}
		\item 凹函数:$\nabla^2 f(x)\preceq 0$
		\item 严格凸:\textcolor{red}{$\impliedby$}$\nabla^2 f(x)\succ 0$,反例$f(x)=x^4$(在一个点斜率不变并不要紧)
	\end{itemize}
\end{enumerate}
\end{definition}

\begin{example}
$f(x)=a^\T x+b$
\end{example}
\begin{analysis}
有$\nabla f(x)=a$,进而
\[f(y)=a^\T y+b\geq a^\T x+b+a^\T(y-x)=a^\T y+b\]
\end{analysis}

\begin{definition}[凸函数的扩展(extended-value)]
尽管凸函数的定义域为凸,但往往不好处理,那就将其扩展到全空间。
$x\in\dom f\subset\rn, \dom\widetilde{f}=\rn$,会有
\[\widetilde{f}(x)=\begin{cases}f(x)&x\in\dom f\\+\infty&x\notin\dom f\end{cases}\]
\end{definition}

指示/示信(indicator)函数不一定是凸的
\[f(x)=\begin{cases}0&x\in C\\ +\infty & x\notin C\end{cases}\]

\begin{theorem}
若$f$为凸,可微,则$\exists x\in\dom f,\nabla f(x)=0$
\end{theorem}

\begin{example}
二次函数$f(x)=\dfrac{1}{2}x^\T Px+q^\T x+r$,$P\in S^n$(对称矩阵),$q^\T\in\rn$,$r\in\rr$
\end{example}
\begin{analysis}
$\nabla^2 f(x)=P$\\
$P\in S^n_+$凸,$P\in S_{++}^n$严格凸
\end{analysis}

\begin{example}
$f(x)=\dfrac{1}{x^2},\dom f=\lrb{x\in\rr,x\ne 0}$
\end{example}
\begin{analysis}
注意$\dom f$不是凸集
\end{analysis}
\begin{itemize}
	\item 指数函数$f(x)=\ee^{ax}$
	\item 幂函数$f(x)=x^a$
	\item 绝对值的幂函数$f(x)=|x|^p,x\in\rr,p>0$:$p\in[1,+\infty)$凸,$p\in(0,1)$既不凸又不凹
	\begin{analysis}
	\[f''(x)=\begin{cases}
	p(p-1)x^{p-2} & x<0\\
	p(p-1)(-x)^{p-2} & x<0
	\end{cases}\]
	\end{analysis}
	\item 对数函数$f(x)=\log x$
	\item 熵$f(x)=-x\log x$
	\item 极大值函数$f(x)=\max\lrb{x_1,\ldots,x_n},x\in\rn$
\end{itemize}

\begin{definition}[解析近似]
无穷阶可微
\end{definition}

极大值函数的解析近似是$f(x)=\log(\ee^{x_1}+\cdots+\ee^{x_n})$
\[\max\{x_1,\ldots,x_n\}\leq f(x)\leq\max\{x_1,\ldots,x_n\}+\log n\]
\begin{analysis}
\[\pd{f}{x_i}=\frac{\ee^{x_i}}{\ee^{x_i}+\cdots+\ee^{x_n}}\]
\[\pddxy{f}{x_i}{x_j}=\begin{cases}\frac{-\ee^{x_i}\ee^{x_i}}i=j\\i\ne j\end{cases}\]
\[z:=\bmat{\ee^{x_1}&\cdots&\ee^{x_n}}^\T\]
求Hessian矩阵
\[H=\frac{1}{(\vone^\T z)^2}(-z\cdot z^\T+(\vone^\T z)\opdiag(z))\]
将前面常量丢弃
\footnote{$H$半正定,则$\forall v\in\rn:v^\T Hv\geq 0$}
\[a_i:=v_i\sqrt{z_i}=\bmat{a_1&\cdots&a_n}^\T,b_i=\sqrt{z_i}\]
\[\begin{aligned}
v^\T Hv&=(\mathbbm{1}^\T z)v^\T \opdiag(z)v-v^\T zz^\T v\\
&=(\sum_i z_i)(\sum_i v_i^2 z_i)-(\sum_i v_i z_i)^2\\
&=(b^\T b)(a^\T a)-(a^\T b)^2\qquad\text{Cauchy}\\
&\geq 0
\end{aligned}\]
\end{analysis}

\begin{definition}[范数]
$p(x)$为范数
\begin{enumerate}
	\item $p(ax)=|a|p(x)$
	\item $p(x+y)\leq p(x)+p(y)$
	\item $p(x)=0\iff x=0$
\end{enumerate}
零范数$\|x\|_0$:非零元素数目,是伪范数(不符合第一个定义)
\end{definition}
$\rn$中的范数都是凸函数,正则化!
\begin{analysis}
\[\forall x,y,\theta\in[0,1]
p(\theta x+(1-\theta)y)\leq p(\theta x)+p((1-\theta)y)
\leq \theta p(x)+(1-\theta)p(y)\]
\end{analysis}

% 3.12
行列式的对数$f(x)=\log\det(x),\dom f=S_{++}^n$
$n=1$凹函数
证$n>1$也为凹,用高维定义
\[\begin{aligned}
g(t)&=f(z+tv)\\
&=\log\det(z+tv)\\
&=\log\det(z^{1/2}(I+tz^{1/2}vz^{-1/2})z^{1/2},\quad z^{1/2}\in S_{++}^n,z^{1/2}z^{1/2}=z\\
&=\log\det(z)+\log\det(I+tz^{1/2}vz^{-1/2})\\
&=\log\det(z)+\sum_{i=1}^n\log(1+t\lambda_i),\quad \lambda_i=z^{-1/2}vz^{1/2}\text{的特征值}
\end{aligned}\]
\[\begin{aligned}
g'(t)&=\sum_{i=1}^n\frac{\lambda_i}{1+t\lambda_i}\\
g''(t)&=\sum_{i=1}^n-\frac{\lambda_i^2}{(1+t\lambda_i)^2}
\end{aligned}\]
补充证明:对对称阵特征值分解$tz^{1/2}vz^{1/2}=tQ\Lambda Q^\T$,对角阵$\Lambda$即为
$QQ^\T=I$,Q为酉矩阵
\[I+tz^{-1/2}vz^{-1/2}=QQ^\T+tQ\Lambda Q^\T=Q(I+t\Lambda)Q^\T\]
\[\log\det(I+tz^{-1/2}vz^{-1/2})=\log\det(Q)+\log\det(I+t\Lambda)+\log\det(Q^\T)\]

保持函数凸性
\begin{itemize}
	\item 非负加权和$f_1,\ldots,f_m$为凸,定义域$\rn$
	\[f:=\sum_{i=1}^m w_if_i,w_i\geq 0\]
	\item 非负积分$f(x,y)$对$y\in A$均为凸($A$不一定为凸),$w(y)\geq 0$
	\[g(x):=\int_{y\in A}w(y)f(x,y)\diff y\]
	\item 仿射映射$f:\rn\mapsto\rr$为凸,$A\in\rr^{m\times n},b\in\rn$,$\dom g=\{x\mid Ax+b\in\dom f\}$
	\[g(x):=f(Ax+b)\]
	\begin{analysis}
	\begin{itemize}
		\item $\dom f$为凸,则$\dom g$为凸
		\item $\forall x,y\in\dom g,\forall\theta\in[0,1]$
		\[\begin{aligned}
		g(\theta x+(1-\theta)y)&=f(A(\theta x+(1-\theta)y)+b)\\
		&=f(\theta(Ax+b)+(1-\theta)(Ay+b))\\
		&\leq\theta f(Ax+b)+(1-\theta)f(Ay+b)\\
		&=\theta g(x)+(1-\theta)g(y)
		\end{aligned}\]
		\item 其实只是在定义域上改变,而不是改变值域,因而函数凸性不会改变
	\end{itemize}
	\end{analysis}
	\item 两个函数的极大值函数,$f_1,f_2$为凸
	\[f(x):=\max\{f_1(x),f_2(x)\},\dom f=\dom f_1\cap\dom f_2\]
	\item 任意个凸函数极大值函数为凸
	\[f(x)=\max\{a_1^\T x+b_1,\ldots,a_m^\T+b_m\}\]
	\item 无限个凸函数,$y\in A$,$f(x,y)$对于$x$为凸,则
	\[g(x):=\sup_{y\in A} f(x,y)\]
	\begin{example}
	点$x$到集合$\sC$的最远距离
	\[f(x)=\sup_{y\in A}\|x-y\|\]
	位移对于范数凸性不会有影响
	\end{example}
	\begin{example}
	$x\in\rn$,$x[i]$为第$i$大元素,$x[1]\geq x[2]\geq\cdots\geq x[r]\geq\cdots\geq x[n]$
	\[f(x):=\sum_{i=1}^r x[i]\]
	\begin{itemize}
	\item $r=1$:$f(x)=x[1]=\max\{x_1,\ldots,x_n\}$,每一项都是$\ve_i^\T x_i$
	\item $r>1$:$f(x)=\max\{x_{i_1}+\cdots+x_{i_r}\mid 1\leq i_1<i_2<\cdots<i_r\leq n\}$
	\end{itemize}
	\end{example}
	\item 函数的组合:$h:\rr^k\mapsto\rr,g:\rn\mapsto\rr^k$
	\[f:=h\circ g:\rn\mapsto\rr\]
	先考虑$n=k=1,\dom g=\rn,\dom h=\rr^k,\dom f=\rr$,$h,g$二阶可微
	\[\begin{aligned}
	f'(x)&=h'(g(x))\cdot g'(x)\\
	f''(x)&=h''(g(x))(g'(x))^2+h'(g(x))g''(x)>0
	\end{aligned}\]
	即当$g$为凸,$h$为凸且不降;$g$为凹,$h$为凸且不增时,$f(x)$为凸\\
	(若定义域非全空间)当$g$为凸,$h$为凸,扩展值函数$\tilde{h}$不降;$g$为凹,$h$为凸,$\tilde{h}$不增时,$f(x)$为凸
	\begin{example}
	$g$为凸,$\exp g(x)$为凸;$g$为凹,$g>0$,$\log g(x)$为凹;$g$为凸,$g>0$,$1/g(x)$为凸
	\end{example}
	\begin{example}
	$g(x)=x^2,\dom g=\rr$,$h(y)=0,\dom h=[1,2]$,$f=h\circ g$,注意$\tilde{h}$并非不降!
	\end{example}
\end{itemize}

\end{document}
% 神经网络次微分