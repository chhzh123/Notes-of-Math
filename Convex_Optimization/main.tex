\documentclass{note}
\usepackage{mypackage}

\renewcommand{\thefootnote}{\fnsymbol{footnote}}

\title{最优化理论}
\author{陈鸿峥}
\date{{\builddatemonth\today}\protect\footnote{\text{Build \builddate\today}}}%加了build

\begin{document}

\maketitle
\renewcommand{\thefootnote}{\arabic{footnote}}
\setcounter{footnote}{0}

\setcounter{tocdepth}{2}%设置深度
\tableofcontents
\bigskip\bigskip\bigskip

\section{简介} % Lec 1 - 2.26
\subsection{优化概述}
优化(optimization):从一个\emph{可行解}的集合中寻找出\emph{最好}的元素
\begin{itemize}
	\item 优化变量$\vx\in\rn$
	\item 目标/损失函数$f_0:\rn\mapsto \rr$
	\item 不等式约束函数$f_i:\rn\mapsto \rr$
	\item 等式约束函数$h_j:\rn\mapsto\rr$
	\item 可行解$\mathcal{S}=\{\vz\mid f_i(\vz)\leq 0,h_j(\vz)=0,i=1,\ldots,m,j=1,\ldots,p\}$
	\item 最优解$\vx^\star\iff\forall \vz\in\rn,\vz\in\mathcal{S}:\;f_0(\vz)\geq f_0(\vx^\star)$
\end{itemize}
\[\begin{aligned}
& \text{minimize}& f_0(\vx)& &\\
& \text{subject to}& f_i(\vx)&\leq 0 &\quad i&=1,\ldots,m\\
&  & h_j(\vx)&=0 &\quad j&=1,\ldots,p
\end{aligned}\]

\begin{example}
\begin{itemize}
\item 最小二乘线性拟合(凸问题)
\item 深度神经网络(非凸,见下)
\[\begin{aligned}
\vx_1^{(i)}&=f_1(\vx_0^{(i)},\vw_1)\\
\cdots&=\cdots\\
\vx_n^{(i)}&=f_n(\vx_{n-1}^{(i)},\vw_n)\\
\min&\sum_{i=1}^m(\vy^{(i)}-\vx_n^{(i)})^2
\end{aligned}\]
\item 图像处理,自然图像通常都是\textbf{分块光滑}的,原图$\Phi_0$,有噪声的新图$\Phi$\\
全变参(TV, Total Variation)范数,计算图像每个像素点左侧和下侧的差异
\[\|\Phi\|_{TV}=\sum_y\sum_x\sqrt{(\Phi(x,y)-\Phi(x,y-1))^2+(\Phi(x,y)-\Phi(x-1,y))^2}\]
可得优化目标:近似自然图像,而且跟原图不能差太远
\[\min(\|\Phi\|_{TV}+\lambda\|\Phi-\Phi_0\|_F^2)\]
\item 推荐系统:Netflix问题\\
矩阵横向为用户,纵向为电影,值为评分值($1\thicksim 5$),问题是把矩阵补全,这样就可以做推荐了$\to$低秩矩阵补全\\
电影很多,但类型不多,关联关系有限$\to$\textbf{近似低秩}\footnote{$A$的秩等于非零奇异值$\sqrt{\mathop{eig}(A^\T A)}$数目}\\
低秩本来需要最小化$\vz$的非零奇异值数目$\|\vz\|_0$,但是非凸的;
转化为最小化\emph{和范数}\footnote{矩阵所有奇异值之和}$\|\vz\|_\star$
\[\begin{aligned}
\min& \;&\|\vz\|_\star&:=\|\vz\|_1\\
\text{s.t.}& \;&\vz_{ij}&=\vM_{ij},\;(i,j)\in\Omega
\end{aligned}\]
\end{itemize}
\end{example}

\subsection{分类}
\begin{itemize}
	\item 线性规划/非线性规划
	\item 凸规划/非凸规划(更好的分类)
\end{itemize}
目标函数凸函数,可行解集为凸集则是凸优化,一般容易求解

\subsection{历史}
\begin{itemize}
\item Newton-Raphson算法:求零点,等价于求$\min f^2(x)$
\item Gauss-Seidel算法:求解线性方程组$A\vx=\vb$,等价于求$\min \|A\vx-\vb\|_2^2$
\item Lagrange
\item Kantoronc:苏联,线性规划,诺贝尔经济学奖
\item Dantzig:美国,优化决策,线性规划单纯形
\item Von Neumann:线性规划问题对偶理论
\item Karmarkar:80年代,线性规划内点法
\item Nesterov:后80年代,非线性凸优化内点法
\item 现代:并行、随机算法
\end{itemize}

\section{凸集} % Lec 2 - 2.28
\begin{definition}
一些集合概念如下
\begin{itemize}
\item 仿射集(affine set)
\[\begin{aligned}
\mathcal{C}\text{为仿射集}&\iff\text{过$\sC$内任意两点的\textbf{直线}都在$\sC$内}\\
&\iff\forall x_1,x_2\in\mathcal{C},\theta\in\rr,\theta x_1+(1-\theta)x_2\in\mathcal{C}
\end{aligned}\]
\begin{example}
用定义易证线性方程组的解集$\mathcal{C}=\{\vx\mid A\vx=\vb\}$是仿射集;
反过来,每一个仿射集都可以用线性方程组的解集表示
\end{example}
\item 仿射组合
\[\forall x_1,x_2,\ldots,x_k\in\mathcal{C},\theta_1,\ldots,\theta_k\in\rr,{\color{red}{\theta_1+\cdots+\theta_k=1}}:\;\theta_1 x_1+\cdots+\theta_k x_k\in\mathcal{C}\]
\item 仿射包(hull):所有仿射组合的集合
\[\opaff\mathcal{C}:=\{\theta_1 x_1+\cdots+\theta_k x_k\mid\forall x_1,\ldots,x_k\in\mathcal{C},\theta_1+\cdots+\theta_k=1\}\]
\item 凸集(convex set)
\[\begin{aligned}
\mathcal{C}\text{为凸集}&\iff\text{过$\sC$内任意两点的\textbf{线段}都在$\sC$内}\\
&\iff\forall x_1,x_2\in\mathcal{C},\theta\in\bm{[0,1]},\theta x_1+(1-\theta)x_2\in\mathcal{C}
\end{aligned}\]
\item 凸组合
\[\forall x_1,x_2,\ldots,x_k\in\mathcal{C},{\color{red}{\theta_1,\ldots,\theta_k\in[0,1],\theta_1+\cdots+\theta_k=1}}:\;\theta_1 x_1+\cdots+\theta_k x_k\in\mathcal{C}\]
\item 凸包:最小的凸集
\[\opconv\mathcal{C}:=\{\theta_1 x_1+\cdots+\theta_k x_k\mid\forall x_1,\ldots,x_k\in\mathcal{C},\theta_1,\ldots,\theta_k\in[0,1],\theta_1+\cdots+\theta_k=1\}\]
\item 凸锥(convex cone)
\[\mathcal{C}\text{为凸锥}\iff\forall x_1,x_2\in\mathcal{C},\theta_1,\theta_2\geq 0,\theta_1 x_1+\theta_2 x_2\in\mathcal{C}\]
除了\textbf{空集}的凸锥都得包含\textbf{原点}(取$\theta_1=\theta_2=0$)
\item 凸锥组合/非负线性组合:
\[\forall x_1,x_2,\ldots,x_k\in\mathcal{C},{\color{red}{\theta_1,\ldots,\theta_k\geq 0}}:\;\theta_1 x_1+\cdots+\theta_k x_k\in\mathcal{C}\]
\item 凸锥包:类似前面定义
\end{itemize}
\end{definition}
由上面的定义易知,仿射组合/凸锥组合(强条件)一定是凸组合。

\begin{definition}[超平面(hyperplane)与半空间(halfspace)]
超平面都是比原空间低一维
\[\{\vx\mid \va^\T\vx=b,\vx,\va\in\rn,b\in\rr,\va\ne 0\}\]
超平面将空间划分为两个部分,即半空间
\[\{\vx\mid \va^\T\vx\leq b,\va\ne 0\}\]
若方程特解为$\vx_0$,则$\va\perp(\vx-\vx_0)$
\end{definition}
\begin{definition}[欧式球(Euclidean ball)]
\[B(x_c,r)=\{x\mid\|x-x_c\|_2\leq r\}\]
范数(norm)球可类似定义
\end{definition}
\begin{definition}[椭球(ellipsoid)]
\[\eps(x_c,P)=\{x\mid(x-x_c)^\T P^{-1}(x-x_c)\leq 1\},P\succ 0\]
其中$P\succ 0$代表$P$\textbf{对称}且正定($P=P^T$)
\end{definition}
\begin{analysis}
% https://ljk.imag.fr/membres/Anatoli.Iouditski/cours/convex/chapitre_3.pdf
定义内积$\lrang{x^\T P^{-1}y}$(需证满足内积条件),进而P-范数$\|x\|_P:=\sqrt{x^\T Px}$是范数,而椭球不过是P-范数意义下的球,由定理得椭球是凸的
\end{analysis}
\begin{definition}[多面体(polyhedron)]
\[P=\{\vx\mid\va_i^\T\vx\leq b_i,\vc_j^\T\vx=d_j,i=1,\ldots,m,j=1,\ldots,p\}\]
\end{definition}
\begin{example}
\begin{itemize}
	\item 空集、点、$\rn$空间均为仿射
	\item 任意直线为仿射;若过原点则为凸锥
	\item $\rn$空间的子空间\footnote{零元、加法封闭、数乘封闭}为仿射和凸锥
	\item 超平面为仿射
	\item 半空间、欧式球、椭球、多面体为凸集
\end{itemize}
\end{example}
% 凸集的交仍然为凸集

\begin{definition}[仿射函数]
\[f:\rn\mapsto\rr^m\quad f(\vx)=A\vx+\vb,A\in\rr^{m\times n},\vb\in\rr^m\]
性质如下:
\begin{itemize}
	\item $S\subset\rn$为凸$\implies f(S)=\{f)\vx\mid\vx\in S\}$为凸
	\item $C\subset\rr^m$为凸$\implies f^{-1}(C)=\{\vx\in\rn\mid f(\vx)\in C\}$为凸
\end{itemize}
\end{definition}
\begin{example}
两个集合的和$S_1+S_2=\{x+y\mid x\in S_1,y\in S_2\}$保凸
\end{example}
\begin{analysis}
直积$S_1\times S_2=\{(x,y)\mid x\in S_1,y\in S_2\}$显然可以保凸(相当于在两个集合同时画线)\\
令$A=\bmat{I & I},\vx=\bmat{x & y}^\T,\vb=0$,由仿射函数性质知
\end{analysis}

\end{document}