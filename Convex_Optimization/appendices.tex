% !TEX root = main.tex

\newpage
\appendix
\section{矩阵微积分}
\label{appendix:matrix}
\subsection{基本矩阵知识}
\begin{definition}[正定(positive definite)矩阵]
若矩阵$A$满足
\[\forall\;\vz\ne 0:\;\vz^\T A\vz>0\]
则称$A$为正定矩阵。
\end{definition}
\begin{definition}[合同(congruent)矩阵]
    若存在可逆矩阵$C$使得$C^\T AC=B$,则称$A$为$B$合同,记作$A\backsimeq B$
\end{definition}
\begin{definition}[主子式]
    从$n$阶矩阵中选取行号和列号相同的$i$列,行列交汇处的元素形成的行列式称为$n$阶矩阵的一个$i$阶主子式。如果挑选$1-i$行和$1-i$列,则成为该矩阵的$i$阶\textbf{顺序}主子式。
\end{definition}
正定矩阵$A$的等价命题如下
\begin{itemize}
    \item $A$的所有顺序主子式均为正
    \item $A$的所有主子式均为正
    \item $A$的特征值均为正
\end{itemize}

\begin{definition}[向量范数]
    $\norm{\cdot}$为范数需要满足以下三个条件
    \begin{enumerate}
        \item 数乘:$\norm{a\vx}=|a|\norm{\vx}$
        \item 零元:$\norm{\vx}=0\iff \vx=\vzero$
        \item 三角不等式:$\norm{\vx+\vy}\leq \norm{\vx}+\norm{\vy}$
    \end{enumerate}
\end{definition}
\begin{itemize}
\item 0-范数:非零元素数目,是伪范数(不符合第一个定义)
\item 1-范数:绝对值之和
\item 2-范数:欧几里得距离
\item p-范数:p次方之和的p次根
\item 无穷范数:最大值
\end{itemize}

\begin{definition}[Taylor公式]
对$f(\vx)$在点$\vx_0$处展开有
\[f(\vx)=f(\vx_0)+(\vx-\vx_0)^\T\nabla f(\vx_0)+\frac{1}{2!}(\vx-\vx_0)^\T \nabla^2(\vx)(\vx-\vx_0)+\cdots\]
\end{definition}

\subsection{实值函数对向量的导数}
设$f(\vx):\rr^n\mapsto\rr$,劈形算子$\nabla$默认对$\vx$求导,可以得到以下公式。
\begin{enumerate}
\item $\nabla (\vv^\T\vx)=\vv$
\begin{analysis}
    展开,对每一个元素讨论
    \[\pd{\vv^\T\vx}{x_i}=\pd{\sum_{i=1}^n v_i x_i}{x_i}=v_i\]
\end{analysis}
\item $\nabla\norm{\vx}_2^2=\nabla(\vx^\T\vx)=2\vx$
\begin{analysis}
    法一:考虑每一元素
    \[\pd{\vx^\T\vx}{x_i}=\pd{\sum_{i=1}^n x_i^2}{x_i}=2x_i\]
    法二:变量多次出现的求导法则,下标$c$代表视为常数
    \[\nabla(\vx^\T\vx)=\nabla(\vx_c^\T\vx)+\nabla(\vx^\T\vx_c)=2\nabla(\vx_c^\T\vx)=2\vx_c=2\vx\]
\end{analysis}
\item $\nabla(\vx^\T A\vx)=(A+A^\T)\vx$
\begin{analysis}
    变量多次出现的求导法则
    \[\begin{aligned}
        LHS&=\nabla(\vx_c^\T A\vx)+\nabla(\vx^\T A\vx_c)\\
        &=\nabla((A^\T\vx_c)^\T\vx)+\nabla((A\vx_c)^\T \vx)\\
        &=A^\T\vx_c+A\vx_c\\
        &=RHS
    \end{aligned}\]
\end{analysis}
\item $\nabla\lrp{\frac{1}{2}\norm{A\vx-\vb}_2^2}=A^\T(A\vx-\vb)$
\begin{analysis}
    法一:展开括号,逐一求导
    \[\begin{aligned}
        LHS&=\frac{1}{2}(\nabla((A\vx-\vb)^\T(A\vx-\vb)))\\
        &=\frac{1}{2}(\nabla(\vx^\T A^\T A\vx)-\nabla(\vx^\T A^\T\vb)-\nabla(\vb^\T A\vx)+\nabla(\vb^\T\vb))\\
        &=\frac{1}{2}((A^\T A+(A^\T A)^T)\vx-A^\T\vb-A^\T\vb)\\
        &=A^\T(A\vx-\vb)\\
        &=RHS
    \end{aligned}\]
    法二:线性变换的求导公式
    \[\begin{aligned}
        LHS&=\frac{1}{2}A^\T\nabla_{A\vx-\vb}\norm{A\vx-\vb}_2^2\\
        &=A^\T(A\vx-\vb)\\
        &=RHS
    \end{aligned}\]
\end{analysis}
\end{enumerate}

\subsection{向量值函数对向量的导数}
设$\vf:\rn\mapsto\rr^m$,则劈形算子
\[\nabla_{\vx}\vf=\pd{\vf}{\vx}=\left[\pd{\vf}{\vx}\right]_{ij}=\left[\pd{f_i}{x_j}\right]\]
得到一个$m\times n$的矩阵,即为雅可比(Jacobi)矩阵
\begin{enumerate}
    \item $\nabla_\vx(A\vx)=A$
    \begin{analysis}
        基于此式,由乘法法则可以推出
        \[\begin{aligned}
            \nabla\vx^T\vx&=(\nabla \vx^\T)\vx+\vx^\T(\nabla \vx)\\
            &=(\nabla (I\vx)^\T)\vx+\vx^\T(\nabla I\vx)\\
            &=I\vx+\vx^\T I\\
            &=2\vx
        \end{aligned}\]
    \end{analysis}
    \item $\nabla^2 f=\nabla(\nabla f)$:Hessian矩阵
    \begin{analysis}
        Hessian矩阵其实是$\vx$到$\nabla f$的Jacobi矩阵
        \[(\nabla^2 f)_{ij}=\pddxy{f}{x_i}{x_j}=\pddxy{f}{\vx}{\vx^\T}=\pddxy{f}{\vx^\T}{\vx}\]
    \end{analysis}
\end{enumerate}

\section{参考资料}
\begin{enumerate}
    \item \emph{Convex optimization}, Stephen Boyd and Lieven Vandenberghe
    \item 凸优化(2018年秋季-北京大学),\url{http://bicmr.pku.edu.cn/~wenzw/opt-2018-fall.html}
    \item 机器学习中的矩阵/向量求导,\url{https://zhuanlan.zhihu.com/p/25063314}
    \item 矩阵求导术,\url{https://zhuanlan.zhihu.com/p/24709748}
\end{enumerate}

% 作业:
% 2.1 2.2 2.5 2.7 2.10 2.16 2.18 2.19
% 3.1 3.2 3.5 3.13 3.18 3.21 3.33 3.36 3.43
% 4.3 4.9 4.22 4.24 4.29 4.62
% 5.5 5.20 5.27