% !TEX root = main.tex

\section{简介} % Lec 1 - 2.26
\subsection{优化概述}
优化(optimization):从一个\emph{可行解}的集合中寻找出\emph{最好}的元素
\begin{itemize}
\item 最小二乘法(凸问题)
\[\min\;\norm{A\vx-\vb}_2^2\]
\item 深度神经网络(非凸,见下)
\[\begin{aligned}
\vx_1^{(i)}&=f_1(\vx_0^{(i)},\vw_1)\\
\cdots&\quad\cdots\\
\vx_n^{(i)}&=f_n(\vx_{n-1}^{(i)},\vw_n)\\
\min&\sum_{i=1}^m(\vy^{(i)}-\vx_n^{(i)})^2
\end{aligned}\]
\item 图像处理,自然图像通常都是\textbf{分块光滑}的,原图$\Phi_0$,有噪声的新图$\Phi$\\
全变参(TV, Total Variation)范数,计算图像每个像素点左侧和下侧的差异
\[\|\Phi\|_{TV}=\sum_y\sum_x\sqrt{(\Phi(x,y)-\Phi(x,y-1))^2+(\Phi(x,y)-\Phi(x-1,y))^2}\]
可得优化目标:近似自然图像,而且跟原图不能差太远
\[\min(\|\Phi\|_{TV}+\lambda\|\Phi-\Phi_0\|_F^2)\]
\item 推荐系统:Netflix问题$\to$低秩矩阵补全\\
矩阵横向为用户,纵向为电影,值为评分值($1\thicksim 5$),问题是把矩阵补全,这样就可以做推荐了\\
电影很多,但类型不多,关联关系有限$\to$\textbf{近似低秩}\footnote{$A$的秩等于非零奇异值$\sqrt{\mathop{eig}(A^\T A)}$数目}\\
低秩本来需要最小化$\vz$的非零奇异值数目$\|\vz\|_0$,但是非凸的;
转化为最小化\textbf{和范数}\footnote{矩阵所有奇异值之和}$\|\vz\|_\star$
\begin{mini*}
{}{\norm{\vz}_\star:=\norm{\vz}_1}{}{}
\addConstraint{\vz_{ij}}{=M_{ij},\;(i,j)\in\Omega}
\end{mini*}
\end{itemize}

\subsection{分类}
\begin{itemize}
	\item 线性规划/非线性规划
	\item 凸规划/非凸规划(更好的分类)
\end{itemize}
目标函数凸函数,可行解集为凸集则是凸优化,一般容易求解

\subsection{历史}
\begin{itemize}
\item Newton-Raphson算法:求零点,等价于求$\min f^2(x)$
\item Gauss-Seidel算法:求解线性方程组$A\vx=\vb$,等价于求$\min \|A\vx-\vb\|_2^2$
\item Lagrange:法国
\item Kantoronc:苏联,线性规划,诺贝尔经济学奖
\item Dantzig:美国,优化决策,线性规划单纯形
\item Von Neumann:线性规划问题对偶理论
\item Karmarkar:80年代,线性规划内点法
\item Nesterov:后80年代,非线性凸优化内点法
\item 现代:并行、随机算法
\end{itemize}