% !TEX root = main.tex

\section{凸集} % Lec 2 - 2.28
\begin{definition}
一些集合概念如下
\begin{itemize}
\item 仿射集(affine set)
\[\begin{aligned}
\mathcal{C}\text{为仿射集}&\iff\text{过$\sC$内任意两点的\textbf{直线}都在$\sC$内}\\
&\iff\forall x_1,x_2\in\mathcal{C},\theta\in\rr,\theta x_1+(1-\theta)x_2\in\mathcal{C}
\end{aligned}\]
\begin{example}
用定义易证线性方程组的解集$\mathcal{C}=\{\vx\mid A\vx=\vb\}$是仿射集;
反过来,每一个仿射集都可以用线性方程组的解集表示
\end{example}
\item 仿射组合(多点扩展)
\[\forall x_1,x_2,\ldots,x_k\in\mathcal{C},\theta_1,\ldots,\theta_k\in\rr,{\color{red}{\theta_1+\cdots+\theta_k=1}}:\;\theta_1 x_1+\cdots+\theta_k x_k\in\mathcal{C}\]
\item 仿射包(hull):所有仿射组合的集合
\[\opaff\mathcal{C}:=\{\theta_1 x_1+\cdots+\theta_k x_k\mid\forall x_1,\ldots,x_k\in\mathcal{C},\theta_1+\cdots+\theta_k=1\}\]
\item 凸集(convex set)
\[\begin{aligned}
\mathcal{C}\text{为凸集}&\iff\text{过$\sC$内任意两点的\textbf{线段}都在$\sC$内}\\
&\iff\forall x_1,x_2\in\mathcal{C},\theta\in\bm{[0,1]},\theta x_1+(1-\theta)x_2\in\mathcal{C}
\end{aligned}\]
\item 凸组合
\[\forall x_1,x_2,\ldots,x_k\in\mathcal{C},{\color{red}{\theta_1,\ldots,\theta_k\in[0,1],\theta_1+\cdots+\theta_k=1}}:\;\theta_1 x_1+\cdots+\theta_k x_k\in\mathcal{C}\]
\item 凸包:最小的凸集
\[\opconv\mathcal{C}:=\{\theta_1 x_1+\cdots+\theta_k x_k\mid\forall x_1,\ldots,x_k\in\mathcal{C},\theta_1,\ldots,\theta_k\in[0,1],\theta_1+\cdots+\theta_k=1\}\]
\item 凸锥(convex cone)
\[\mathcal{C}\text{为凸锥}\iff\forall x_1,x_2\in\mathcal{C},\theta_1,\theta_2\geq 0,\theta_1 x_1+\theta_2 x_2\in\mathcal{C}\]
除了\textbf{空集}的凸锥都得包含\textbf{原点}(取$\theta_1=\theta_2=0$)
\item 凸锥组合/非负线性组合:
\[\forall x_1,x_2,\ldots,x_k\in\mathcal{C},{\color{red}{\theta_1,\ldots,\theta_k\geq 0}}:\;\theta_1 x_1+\cdots+\theta_k x_k\in\mathcal{C}\]
\item 凸锥包:类似前面定义
\end{itemize}
\end{definition}
由上面的定义易知,仿射组合/凸锥组合(强条件)一定是凸组合。

\begin{definition}[超平面(hyperplane)与半空间(halfspace)]
超平面都是比原空间低一维
\[\{\vx\mid \va^\T\vx=b,\vx,\va\in\rn,b\in\rr,\va\ne 0\}\]
超平面将空间划分为两个部分,即半空间
\[\{\vx\mid \va^\T\vx\leq b,\va\ne 0\}\]
若方程特解为$\vx_0$,则$\va\perp(\vx-\vx_0)$
\end{definition}
\begin{definition}[欧式球(Euclidean ball)]
\[B(\vx_c,r)=\{\vx\mid\|\vx-\vx_c\|_2\leq r\}\]
范数(norm)球可类似定义
\end{definition}
\begin{definition}[椭球(ellipsoid)]
\[\eps(\vx_c,P)=\{\vx\mid(\vx-\vx_c)^\T P^{-1}(\vx-\vx_c)\leq 1\},P\succ 0\]
其中$P\succ 0$表示$P$\textbf{对称}($P=P^T$)\textbf{且正定},或记为$P\in\bbs_{++}$
\end{definition}
\begin{analysis}
% https://ljk.imag.fr/membres/Anatoli.Iouditski/cours/convex/chapitre_3.pdf
定义内积$\lrang{x^\T, P^{-1}y}$(需证满足内积条件),进而$\|\vx\|_p:=\sqrt{\vx^\T P\vx}$是范数,而椭球不过是$p$-范数意义下的球,由定理得椭球是凸的
\end{analysis}
\begin{definition}[多面体(polyhedron)]
\[P=\{\vx\mid\va_i^\T\vx\leq b_i,\vc_j^\T\vx=d_j,i=1,\ldots,m,j=1,\ldots,p\}\]
\end{definition}
\begin{example}
\begin{itemize}
	\item 空集、点、$\rn$空间均为仿射
	\item 任意直线为仿射;若过原点则为凸锥
	\item $\rn$空间的子空间\footnote{零元、加法封闭、数乘封闭}为仿射和凸锥
	\item 超平面为仿射
	\item 半空间、欧式球、椭球、多面体为凸集
\end{itemize}
\end{example}

\begin{definition}[仿射函数]
\[f:\rn\mapsto\rr^m\quad f(\vx)=A\vx+\vb,A\in\rr^{m\times n},\vb\in\rr^m\]
性质如下:
\begin{itemize}
	\item $S\subset\rn$为凸$\implies f(S)=\{f(\vx)\mid\vx\in S\}$为凸
	\item $C\subset\rr^m$为凸$\implies f^{-1}(C)=\{\vx\in\rn\mid f(\vx)\in C\}$为凸
\end{itemize}
\end{definition}
\begin{example}
两个集合的和$S_1+S_2=\{\vx+\vy\mid \vx\in S_1,\vy\in S_2\}$保凸
\end{example}
\begin{analysis}
直积$S_1\times S_2=\{(\vx,\vy)\mid \vx\in S_1,\vy\in S_2\}$显然可以保凸(相当于在两个集合同时画线)\\
令$A\gets\bmat{I & I},\vx\gets\bmat{\vx & \vy}^\T,\vb\gets 0$,由仿射函数性质得证
\end{analysis}

\begin{definition}[透视(perspective)函数\footnote{$+$代表$\geq 0$,$++$代表$>0$}]
透视函数$P:\rr^{n+1}\mapsto\rn,\dom P=\rn\times\rr_{++}$定义如下
\[P(\vz,t)=\dfrac{\vz}{t},\vz\in\rn,t\in\rr_{++}\]
反透视函数
\[P^{-1}(c):=\lrb{(\vx,t)\in\rr^{n+1}\Big|\frac{\vx}{t}\in c,t>0}\]
凸集经过透视函数和反透视函数依然是凸集
\end{definition}
\begin{analysis}
考虑$\rr^{n+1}$内的线段,$\vx=(\widetilde{\vx}\in\rn,\vx_{n+1}\in\rr_{++}),\vy=(\widetilde{\vy},\vy_{n+1})$
则经过透视函数仍是线段
\[P(\theta \vx+(1-\theta)\vy)=\frac{\theta\widetilde{\vx}+(1-\theta)\widetilde{\vy}}{\theta \vx_{n+1}+(1-\theta) \vy_{n+1}}=\frac{\theta \vx_{n+1}}{\theta \vx_{n+1}+(1-\theta)\vy_{n+1}}\frac{\widetilde{\vx}}{\vx_{n+1}}+\frac{(1-\theta)\vy_{n+1}}{\theta \vx_{n+1}+(1-\theta)\vy_{n+1}}\frac{\widetilde{\vy}}{\vy_{n+1}}\]
\end{analysis}

\begin{definition}[线性分数函数]
仿射函数
\[g(\vx)=\bmat{A\\\vc^\T}\vx+\bmat{\vb\\\vd},A\in\rr^{m\times n},\vb\in\rm,\vc\in\rn,\vd\in\rr\]
线性分数函数$f:\rn\mapsto\rr^m=p\circ g$
\[f(\vx)=\frac{A\vx+\vb}{\vc^\T \vx+\vd},\dom f=\lrb{\vx\mid \vc^\T+\vd>0}\]
\end{definition}

保凸性
\begin{itemize}
	\item 凸集的交
	\item 仿射、逆仿射
	\item 透视函数
	\item 线性分数函数
\end{itemize}