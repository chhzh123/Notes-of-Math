% !TEX root = main.tex

\section{数值积分与数值微分}
只要$\disp\max_{a\leq x\leq b}|f(x)-P_n(x)|\leq\eps$,就有误差估计
\[\lrabs{\intab{a}{b}{f(x)}-\intab{a}{b}{P_n(x)}}\leq (b-a)\eps\]

\subsection{基本公式}
\begin{itemize}
\item 中点公式
\[\begin{aligned}
    I&=\intab{a}{b}{f(x)}\approx(b-a)f\lrp{\frac{a+b}{2}}\\
    R&=\frac{1}{24}(b-a)^3f''(\xi)
\end{aligned}\]
\item 梯形公式(一次插值)
\[\begin{aligned}
    I&=\intab{a}{b}{f(x)}\approx\frac{1}{2}(b-a)[f(a)+f(b)]\\
    R&=-\frac{1}{12}(b-a)^3f''(\xi)
\end{aligned}\]
\item 辛普森公式(二次插值)
\[\begin{aligned}
    I&=\intab{a}{b}{f(x)}\approx\frac{1}{6}(b-a)\left[f(a)+4f\lrp{\frac{a+b}{2}}+f(b)\right]\\
    R&=-\frac{(b-a)^5}{2880}f^{(4)}(\xi)
\end{aligned}\]
\end{itemize}
\begin{analysis}
    误差公式的推导通常\textbf{在某一点处展开},然后结合\textbf{积分第一中值定理}计算。
    如下例计算中点公式误差,由泰勒公式
    \[f(x)=f\lrp{\frac{a+b}{2}}+f'\lrp{\frac{a+b}{2}}\lrp{x-\frac{a+b}{2}}+\frac{1}{2}f''(\eta(x))\lrp{x-\frac{a+b}{2}}^2,\eta(x)\in(a,b)\]
    左右两边积分,连续用两次积分第一中值定理有
    \[\begin{aligned}
        \intab{a}{b}{f(x)}&=\int_a^b f\lrp{\frac{a+b}{2}}\diff x+f'\lrp{\frac{a+b}{2}}\int_a^b\lrp{x-\frac{a+b}{2}}\diff x+\frac{1}{2}\int_a^b f''(\eta(x))\lrp{x-\frac{a+b}{2}}^2\diff x,\eta(x)\in(a,b)\\
        &=(b-a)f\lrp{\frac{a+b}{2}}+0+\frac{1}{2}f''(\xi)\int_a^b \lrp{x-\frac{a+b}{2}}^2\diff x,\xi\in(a,b)\\
        &=\frac{1}{24}(b-a)^3
    \end{aligned}\]
\end{analysis}

通常记
\[I[f]=\intab{a}{b}{f(x)}\approx\sum_{i=0}^n\omega_i f(x_i)=Q[f]\]
为求积公式,其中$x_i$为求积节点,$\omega_i$为求积系数,误差/余项则记为
\[R[f]=I[f]-Q[f]\]
\begin{definition}[代数精度]
    对\textbf{所有}次数小于等于$m$的多项式$f(x)$,等式
    \[\intab{a}{b}{f(x)}=\sum_{i=0}^n\omega_i f(x_i)\]
    成立,但对于$m+1$次的某个多项式不精确成立,则称该求积公式代数精度为$m$次。
\end{definition}
对于使代数精度尽可能高的题目,都是令$f(x)=1,x,x^2,\ldots$(相当于取特殊值),然后左右两边相等解方程组。

记
\[x_{i+\frac{1}{2}}=\frac{1}{2}(x_i+x_{i+1})\]
由定积分性质
\[\intab{a}{b}{f(x)}=\sum_{i=0}^{n-1}\intab{x_i}{x_{i+1}}{f(x)}\]
进而每一个小段都可以用前面的基本公式,进而得到复合积分公式。
记$h=\frac{b-a}{n}$,则
\begin{itemize}
\item 复合中点公式
\[\begin{aligned}
    M_n&:=\sum_{i=0}^{n-1}hf\lrp{x_{i+\frac{1}{2}}}\\
    R_M&=\sum_{i=0}^{n-1}\frac{1}{24}h^3f''(\xi_i)
\end{aligned}\]
\item 复合梯形公式
\[\begin{aligned}
    T_n&:=\frac{h}{2}[f(a)+2\sum_{i=1}^{n-1}f(x_i)+f(b)]\\
    R_T&=-\frac{1}{12}(b-a)h^2f''(\eta),\eta\in(a,b)
\end{aligned}\]
\item 复合辛普森公式
\[\begin{aligned}
    S_n&:=\frac{h}{6}\left[f(a)+4\sum_{i=0}^{n-1}f(x_{i+\frac{1}{2}})+2\sum_{i=1}^{n-1}f(x_i)+f(b)\right]\\
    R_S&=-\frac{1}{2880}(b-a)h^4f^{(4)}(\eta)
\end{aligned}\]
\end{itemize}

龙贝格(Romberg)求积方法(外推加速技术)
\[T_n^{(m)}=\frac{T_{2n}^{(m-1)}-2^{-2(m-1)}T_n^{(m-1)}}{1-2^{-2(m-1)}}\]
可证$T_n^{(m)}(m\leq n)$的计算精度为$O(h^{2m})$

\subsection{牛顿-科茨公式}
直接用$n$次的拉格朗日函数代替被积函数,即得牛顿-科茨(Newton-Cotes)公式
\[\begin{aligned}
    Q&=(b-a)\sum_{i=0}^n\frac{(-1)^{n-i}}{ni!(n-i)!}\lrp{\intabu{0}{n}{\prod_{\substack{j=0\\j\ne i}}^n(t-j)}{t}}f(x_i)\\
    &=\sum_{i=0}^n(b-a)C_i^{(n)} f(x_i)\\
    &=\sum_{i=0}^n\omega_if(x_i)
\end{aligned}\]
其中$C_i^{(n)}$为科茨系数,有以下性质
\begin{itemize}
    \item $\disp\sum_{i=0}^nC_{i}^{(n)}=1$
    \item $C_i^{(n)}$对$i$有对称性:$C_i^{(n)}=C_{n-1}^{(n)}$
\begin{center}
\begin{tabular}{c|cccc}\hline
$n$ & $C_i^{(n)}$\\\hline
1 & 1/2 & 1/2 \\\hline
2 & 1/6 & 2/3 & 1/6\\\hline
3 & 1/8 & 3/8 & 3/8 & 1/8\\\hline
\end{tabular}
\end{center}
    \item $n\geq 8$科茨系数有正有负,求积公式稳定性得不到保证,故一般不采用太高阶
\end{itemize}
\begin{theorem}
    当$n$为奇数时,牛顿-科茨公式代数精度至少为$n$次;而当$n$为偶数时,代数精度至少为$n+1$次
\end{theorem}

\subsection{高斯公式}
前面的方法都基于等分节点,而高斯公式则基于\textbf{非等分}节点
\begin{definition}
    若对于节点$x_i\in[a,b]$及求积系数$\omega_i$,求积公式
    \[I[f]=\intab{a}{b}{\rho(x)f(x)}\approx\sum_{i=0}^n\omega_if(x_i)\]
    的代数精度为$2n+1$,则称节点$x_i$为高斯点,$\omega_i$为高斯系数
\end{definition}
\begin{theorem}
    $x_i$是求积公式的高斯点的充分必要条件是,多项式$\Pi(x)=\Pi_{i=0}^n(x-x_i)$与任意次数不超过$n$的多项式$q(x)$关于权函数$\rho(x)$正交,即
    \[\intab{a}{b}{\rho(x)\Pi(x)q(x)}=0\]
\end{theorem}

几种常见的高斯公式如下
\begin{enumerate}
    \item 高斯-勒让德(Legendre)多项式:$\rho\equiv 1,[a,b]=[-1,1]$
    \item 高斯-切比雪夫公式:$\rho(x)=\frac{1}{\sqrt{1-x^2}},[a,b]=[-1,1]$
    \item 高斯-拉盖尔公式:$\rho(x)=\ee^{-x},[a,b]=[0,\infty)$
    \item 高斯-埃尔米特(Hermite)公式:$\rho(x)=\ee^{-x^2},(a,b)=(-\infty,+\infty)$
\end{enumerate}
第一步都先变换积分区间,然后查表代入$x_i$和$\omega_i$

\subsection{多重积分}
记$\omega_i,\omega_j$分别为$x,y$方向的求积系数
\[I=\iint_\Omega f(x,y)\diff x\diff y\approx\sum_{i=0}^n\sum_{j=0}^m\omega_{ij}f(x_i,y_j)\]

\begin{itemize}
    \item 横向纵向分别用一维的数值积分公式
    \item 蒙特卡洛(Monte Carlo)方法
\end{itemize}

\subsection{数值微分}
利用拉格朗日函数可以做线性、二次和高次插值,得到微分近似公式

两点公式:向前差商公式和向后差商公式,误差均为$O(h)$,精度都一阶
\[\begin{cases}
    f'(x_0)=\frac{f(x_1)-f(x_0)}{h}-\frac{h}{2}f''(\xi),\xi\in(x_0,x_1)\\
    f'(x_1)=\frac{f(x_1)-f(x_0)}{h}-\frac{h}{2}f''(\xi),\xi\in(x_0,x_1)\\
\end{cases}\]

三点求导公式:其中第二条为中心差商公式,精度均为二阶
\[\begin{cases}
    f'(x_0)=\frac{1}{2h}[-3f(x_0)+4f(x_1)-f(x_2)]+\frac{h^2}{3}\iter{f}{3}(\xi)\\
    f'(x_1)=\frac{1}{2h}[-f(x_0)+f(x_2)]-\frac{h^2}{6}\iter{f}{3}(\xi)\\
    f'(x_2)=\frac{1}{2h}[f(x_0)-4f(x_1)+3f(x_2)]+\frac{h^2}{3}\iter{f}{3}(\xi)
\end{cases}\]