% !TEX root = main.tex

\section{数值积分与数值微分}
\subsection{基本公式}
中点公式
\[\begin{aligned}
    I&=\intab{a}{b}{f(x)}\approx(b-a)f\lrp{\frac{a+b}{2}}\\
    R&=\frac{1}{24}(b-a)^3f''(\xi)
\end{aligned}\]
梯形公式(一次插值)
\[\begin{aligned}
    I&=\intab{a}{b}{f(x)}\approx\frac{1}{2}(b-a)[f(a)+f(b)]\\
    R&=-\frac{1}{12}(b-a)^3f''(\xi)
\end{aligned}\]
辛普森公式(二次插值)
\[\begin{aligned}
    I&=\intab{a}{b}{f(x)}\approx\frac{1}{6}(b-a)[f(a)+4f\lrp{\frac{a+b}{2}}+f(b)]\\
    R&=-\frac{(b-a)^5}{2880}f^{(4)}(\xi)
\end{aligned}\]
通常记
\[I[f]=\intab{a}{b}{f(x)}\approx\sum_{i=0}^n\omega_i f(x_i)=Q[f]\]
为求积公式,其中$x_i$为求积节点,$\omega_i$为求积系数,误差/余项则记为
\[R[f]=I[f]-Q[f]\]
\begin{definition}[代数精度]
    对所有次数小于等于$m$的多项式$f(x)$,等式
    \[\intab{a}{b}{f(x)}=\sum_{i=0}^n\omega_i f(x_i)\]
    成立,但对于$m+1$次的某个多项式不精确成立,则称该求积公式代数精度为$m$次。
\end{definition}

记
\[x_{i+\frac{1}{2}}=\frac{1}{2}(x_i+x_{i+1})\]
由定积分性质
\[\intab{a}{b}{f(x)}=\sum_{i=0}^{n-1}\intab{x_i}{x_{i+1}}{f(x)}\]
进而每一个小段都可以用前面的基本公式,进而得到复合积分公式。

复合中点公式
\[M_n:=\sum_{i=0}^{n-1}hf\lrp{x_{i+\frac{1}{2}}}\]
复合梯形公式
\[T_n:=\frac{h}{2}[f(a)+2\sum_{i=1}^{n-1}f(x_i)+f(b)]\]
复合辛普森公式
\[S_n:=\frac{h}{6}[f(a)+4\sum_{i=0}^{n-1}f(x_{i+\frac{1}{2}})+2\sum_{i=1}^{n-1}f(x_i)+f(b)]\]

\subsection{牛顿-科茨公式}
直接用$n$次的拉格朗日函数代替被积函数,即得牛顿-科茨(Newton-Cotes)公式
\[Q=(b-a)\sum_{i=0}^n\frac{(-1)^{n-i}}{ni!(n-i)!}\lrp{\intab{0}{n}\Pi_{j=0,j\ne i}^n(t-j)\diff t}f(x_i)=\sum_{i=0}^n\omega_if(x_i)\]
科茨系数
\[C_i^{(n)}\frac{(-1)^{n-i}}{ni!(n-i)!}\intab{0}{n}\Pi_{j=0,j\ne i}^n(t-j)\diff t\]
有以下性质
\begin{itemize}
    \item \[\sum_{i=0}^nC_{i}^{(n)}=1\]
    \item $C_i^{(n)}$对$i$有对称性:$C_i^{(n)}=C_{n-1}^{(n)}$
    \item $n\geq 8$科茨系数有正有负,求积公式稳定性得不到保证,故一般不采用太高阶
\end{itemize}
\begin{theorem}
    当$n$为奇数时,牛顿-科茨公式代数精度至少为$n$次;而当$n$为偶数时,代数精度至少为$n+1$次
\end{theorem}

\subsection{高斯公式}
高斯公式非等分节点
\[I[f]=\intab{a}{b}{\rho(x)f(x)}\approx\sum_{i=0}^n\omega_if(x_i)\]
\begin{definition}
    若对于节点$x_i\in[a,b]$及求积系数$\omega_i$,上述求积公式代数精度为$2n+1$,则称节点$x_i$为高斯点,$\omega_i$为高斯系数
\end{definition}

几种常见的高斯公式如下
\begin{enumerate}
    \item 高斯-勒让德(Legendre)多项式:$\rho\equiv 1,[a,b]=[-1,1]$
    \item 高斯-切比雪夫公式:$\rho(x)=\frac{1}{\sqrt{1-x^2}},[a,b]=[-1,1]$
    \item 高斯-拉盖尔公式:$\rho(x)=\ee^{-x},[a,b]=[0,\infty)$
    \item 高斯-埃尔米特(Hermite)公式:$\rho(x)=\ee^{-x^2},(a,b)=(-\infty,+\infty)$
\end{enumerate}
第一步都先变换积分区间,然后查表代入$x_i$和$\omega_i$

\subsection{多重积分}
记$\omega_i,\omega_j$分别为$x,y$方向的求积系数
\[I=\iint_Omega f(x,y)\diff x\diff y\approx\sum_{i=0}^n\sum_{j=0}^m\omega_{ij}f(x_i,y_j)\]

\subsection{数值微分}
利用拉格朗日函数可以做线性、二次和高次插值,得到微分近似公式

两点公式:向前差商公式和向后差商公式
\[\begin{cases}
    f'(x_0)=\frac{f(x_1)-f(x_0)}{h}-\frac{h}{2}f''(\xi),\xi\in(x_0,x_1)\\
    f'(x_1)=\frac{f(x_1)-f(x_0)}{h}-\frac{h}{2}f''(\xi),\xi\in(x_0,x_1)\\
\end{cases}\]