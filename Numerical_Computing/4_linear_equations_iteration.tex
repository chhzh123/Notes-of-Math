% !TEX root = main.tex

\section{线性方程组迭代解法}
\subsection{范数与条件数}
\begin{definition}[向量的范数]
    对任意$n$维向量$\vx$,若对应非负实数$\norm{\vx}$,满足
    \begin{enumerate}
        \item $\norm{\vx}\geq 0$,当且仅当$\vx=\vzero$时等号成立
        \item 对任意$\alpha\in\rr$,$\norm{\alpha\vx}=|\alpha|\cdot\norm{\vx}$
        \item 对任意的$n$维向量$\vx$和$\vy$,满足三角不等式$\norm{\vx+\vy}\leq\norm{\vx}+\norm{\vy}$
    \end{enumerate}
    则称$\norm{\vx}$为$\vx$的范数,其中1-范数、2-范数、无穷范数定义如下
    \[\begin{aligned}
        \norm{\vx}_1&=\sum_{i=1}^n|x_i|\\
        \norm{\vx}_2&=\sqrt{\sum_{i=1}^n|x_i|^2}\\
        \norm{\vx}_\infty&=\max_{1\leq i\leq n}|x_i|
    \end{aligned}\]
\end{definition}

\begin{definition}[矩阵的范数]
    对于$n$阶方阵$A$,若对应非负实数$\norm{A}$,满足
    \begin{enumerate}
        \item $\norm{A}\geq 0$,当且仅当$A=\vzero$时等号成立
        \item 对任意$\alpha\in\rr$,$\norm{\alpha A}=|\alpha|\cdot\norm{A}$
        \item 对任意两个$n$阶方阵$A$和$B$,满足三角不等式$\norm{A+B}\leq\norm{A}+\norm{B}$
        \item 对任意两个$n$阶方阵$A$和$B$,满足矩阵乘法要求$\norm{AB}\leq\norm{A}\cdot\norm{B}$
    \end{enumerate}
    则称$\norm{A}$为方阵$A$的矩阵范数。
    记$\rho(A)=\max_{1\leq i\leq n}|\lambda_i|$为$A$的谱半径,这里$\lambda_i$为$A$的特征值,矩阵的1-范数、2-范数、无穷范数和F范数分别定义如下
    \[\begin{aligned}
        \norm{A}_1&=\max_{1\leq j\leq n}\sum_{i=1}^n|a_{ij}|\\
        \norm{A}_2&=\sqrt{\rho(A^\T A)}\\
        \norm{A}_\infty&=\max_{1\leq i\leq n}\sum_{j=1}^n|a_{ij}|\\
        \norm{A}_F&=\sqrt{\sum_{i=1}^n\sum_{j=1}^n|a_{ij}|^2}
    \end{aligned}\]
    注意矩阵的F-范数才是向量2-范数的直接推广,而矩阵的2-范数是计算$A^\T A$的谱半径,又被称为谱范数
\end{definition}
\begin{definition}[向量范数与矩阵范数相容]
向量范数与矩阵范数相容,则满足
\[\norm{A\vx}\leq\norm{A}\cdot\norm{\vx}\]
一些相容的范数如下
\begin{itemize}
    \item 矩阵1-范数与向量1-范数
    \item 矩阵2-范数与向量2-范数
    \item 矩阵无穷范数与向量无穷范数
    \item 矩阵F范数与向量2-范数
\end{itemize}
\end{definition}

\begin{definition}[条件数]
    设$A$为$n$阶非奇异矩阵,称数$\mathop{cond}(A)=\norm{A}\norm{A^{-1}}$为线性方程组$A\vx=\vb$或矩阵$A$的条件数,其对方程组的解的相对误差起到关键的控制作用\\
    如果$A$对称正定,则有
    \[\opcond_2(A)=\lambda_{\max}(A)/\lambda_{\min}(A)\]
\end{definition}
\begin{analysis}
考虑方程$A\vx=\vb$,在方程组右端有一个小扰动$\delta b$,则解$\vx$产生一个扰动$\delta\vx$,即
\[A(\vx+\delta\vx)=\vb+\delta\vb\]
于是有
\[\delta\vx=A^{-1}\delta\vb\]
两边取范数有
\[\norm{\delta\vx}\leq\norm{A^{-1}}\norm{\delta\vb}\]
在原式中取范数有
\[\norm{\vb}=\norm{A}\norm{\vx}\]
上面两条不等式相乘,整理得
\[\frac{\norm{\delta\vx}}{\norm{\vx}}\leq\norm{A}\norm{A^{-1}}\frac{\norm{\delta\vb}}{\norm{\vb}}\]
进而得到扰动变化与条件数相关
\end{analysis}

条件数满足以下四条性质
\begin{itemize}
    \item 对于任意$n$阶非奇异矩阵$A$,$\opcond(A)\geq 1$成立
    \item 对于任意$n$阶非奇异矩阵$A$及任意非零常数$c$,$\opcond(cA)=\opcond(A)$成立
    \item 对于任意正交矩阵$A$,有$\opcond_2(A)=1$
    \item 对于任意$n$阶非奇异矩阵$A$及任意$n$阶正交矩阵$P$,有$\opcond_2(PA)=\opcond_2(AP)=\opcond_2(A)$
\end{itemize}

一类十分典型的病态矩阵是希尔伯特矩阵$H_n$,其条件数$\opcond_2(H_n)\approx\ee^{3.5n}$,病态非常严重

\subsection{基本迭代法}
给定线性方程组$A\vx=\vb$,设$A=M-N$,其中$M$为非奇异矩阵,则原式变为
\[M\vx=N\vx+\vb\]
左右乘上$M^{-1}$,有
\[\vx=M^{-1}N\vx+M^{-1}\vb=B\vx+\vg\]
给定初始向量$\vx^{(0)}$,按照下式迭代
\[\vx^{(k+1)}=B\vx^{(k)}+\vg\]
最终若收敛至$\vx^\star$,则原方程得出解。
实际迭代还是用
\[M\vx^{(k+1)}=N\vx^{(k)}+\vb\]
将$A$拆分成三个矩阵之和(只是将矩阵$A$元素分块而已)
\[A=D-L-U\]
对角线阵$D$、\textbf{负的}严格下三角阵$L$和严格上三角阵$U$

\subsubsection{雅可比迭代法}
取$M=D$和$N=L+U$,即得雅可比迭代法
\[D\vx^{(k+1)}=(L+U)\vx^{(k)}+\vb\]

\begin{theorem}
    若$A$是对称正定矩阵,则雅可比迭代收敛的充要条件是$2D-A$也是对称正定矩阵
\end{theorem}

\subsubsection{高斯-赛德尔迭代法}
如果在雅可比迭代算法的第三步,将算出来的$\vx^{(k+1)}$的分量立即投入到下一个迭代方程,则得到Gauss-Seidel迭代法
\[D\vx^{(k+1)}=L\vx^{(k+1)}+U\vx^{(k)}+\vb\]

\begin{theorem}
    若$A$是严格对角占优或不可约弱对角占优矩阵,则雅可比迭代和GS迭代都收敛
\end{theorem}

\subsubsection{超松弛(SOR)迭代法}
将GS迭代改写为
\[\begin{aligned}
    \vx^{(k+1)}&=D^{-1}(L\vx^{(k+1)}+U\vx^{(k)}+\vb)\\
    &=\vx^{(k)}+D^{-1}(L\vx^{(k+1)}+U\vx^{(k)}-D\vx^{(k)}+\vb)
\end{aligned}\]
等号右侧第二项为修正量,为获得更快的收敛速度,在其前面乘系数$\omega$,即得到逐次超松弛(SOR)迭代法,其中$\omega$为松弛因子
\[\vx^{(k+1)}=(D-\omega L)^{-1}[(1-\omega)D+\omega U]\vx^{(k)}+\omega(D-\omega L)^{-1}\vb\]
此时迭代矩阵为
\[B_{SOR}=(D-\omega L)^{-1}[(1-\omega)D+\omega U]\]

\begin{theorem}
SOR迭代收敛的必要条件是$\omega\in(0,2)$
\end{theorem}
\begin{theorem}
设系数矩阵$A$对称正定,则$0<\omega<2$时SOR迭代收敛
\end{theorem}