% !TEX root = main.tex

\section{线性方程组迭代解法}
\subsection{范数与条件数}
\begin{definition}[向量的范数]
    对任意$n$维向量$\vx$,若对应非负实数$\norm{\vx}$,满足
    \begin{enumerate}
        \item $\norm{\vx}\geq 0$,当且仅当$\vx=\vzero$时等号成立
        \item 对任意$\alpha\in\rr$,$\norm{\alpha\vx}=|\alpha|\cdot\norm{\vx}$
        \item 对任意的$n$维向量$\vx$和$\vy$,满足三角不等式$\norm{\vx+\vy}\leq\norm{\vx}+\norm{\vy}$
    \end{enumerate}
    则称$\norm{\vx}$为$\vx$的范数,其中1-范数、2-范数、无穷范数定义如下
    \[\begin{aligned}
        \norm{\vx}_1&=\sum_{i=1}^n|x_i|\\
        \norm{\vx}_2&=\sqrt{\sum_{i=1}^n|x_i|^2}\\
        \norm{\vx}_\infty&=\max_{1\leq i\leq n}|x_i|
    \end{aligned}\]
\end{definition}
\begin{definition}[矩阵的范数]
    对于$n$阶方阵$\vA$,若对应非负实数$\norm{\vA}$,满足
    \begin{enumerate}
        \item $\norm{\vA}\geq 0$,当且仅当$\vA=\vzero$时等号成立
        \item 对任意$\alpha\in\rr$,$\norm{\alpha\vA}=|\alpha|\cdot\norm{\vA}$
        \item 对任意两个$n$阶方阵$\vA$和$\vB$,满足三角不等式$\norm{\vA+\vB}\leq\norm{\vA}+\norm{\vB}$
        \item 对任意两个$n$阶方阵$\vA$和$\vB$,满足矩阵乘法要求$\norm{\vA\vB}\leq\norm{\vA}\cdot\norm{\vB}$
    \end{enumerate}
    则称$\norm{\vA}$为方阵$\vA$的矩阵范数。
    记$\rho(\vA)=\max_{1\leq i\leq n}|\lambda_i|$为$\vA$的谱半径,这里$\lambda_i$为$\vA$的特征值,矩阵的1-范数、2-范数、无穷范数和F范数分别定义如下
    \[\begin{aligned}
        \norm{\vA}_1&=\max_{1\leq j\leq n}\sum_{i=1}^n|a_{ij}|\\
        \norm{\vA}_2&=\sqrt{\rho(A^\T A)}\\
        \norm{\vA}_\infty&=\max_{1\leq i\leq n}\sum_{j=1}^n|a_{ij}|\\
        \norm{\vA}_F&=\sqrt{\sum_{i=1}^n\sum_{j=1}^n|a_{ij}|^2}
    \end{aligned}\]
    注意矩阵的F-范数才是向量2-范数的直接推广,而矩阵的2-范数是计算$A^\T A$的谱半径,又被称为谱范数
\end{definition}
相容的向量范数和矩阵范数,满足
\[\norm{\vA\vx}\leq\norm{\vA}\cdot\norm{\vx}\]
一些相容的范数如下
\begin{itemize}
    \item 矩阵1-范数与向量1-范数
    \item 矩阵2-范数与向量2-范数
    \item 矩阵无穷范数与向量无穷范数
    \item 矩阵F范数与向量2-范数
\end{itemize}
\begin{definition}[条件数]
    设$\vA$为$n$阶非奇异矩阵,称数$\mathop{cond}(\vA)=\norm{\vA}\norm{\vA^{-1}}$为线性方程组$\vA\vx=\vb$或矩阵$\vA$的条件数。
    其对方程组的解的相对误差起到关键的控制作用
\end{definition}

\subsection{基本迭代法}
给定线性方程组$\vA\vx=\vb$,设$\vA=\vM-\vN$,其中$\vM$为非奇异矩阵,则原式变为
\[\vM\vx=\vN\vx+\vb\]
左右乘上$\vM^{-1}$,有
\[\vx=\vM^{-1}\vN\vx+\vM^{-1}\vb=\vB\vx+\vg\]
给定初始向量$\vx^{(0)}$,按照下式迭代
\[\vx^{(n+1)}=\vB\vx^{(n)}+\vg\]
最终若收敛至$\vx^\star$,则原方程得出解。
实际迭代还是用
\[\vM\vx^{(k+1)}=\vN\vx^{(k)}+\vb\]
将$\vA$拆分成三个矩阵之和(只是将矩阵$\vA$元素分块而已)
\[\vA=\vD-\vL-\vU\]
对角线阵$\vD$、\textbf{负的}严格下三角阵$\vL$和严格上三角阵$\vU$

\subsubsection{雅可比迭代法}
取$\vM=\vD$和$\vN=\vL+\vU$,即得雅可比迭代法
\[\vD\vx^{(k+1)}=(\vL+\vU)\vx^{(k)}+\vb\]

\subsubsection{高斯-赛德尔迭代法}
如果在雅可比迭代算法的第三步,将算出来的$\vx^{(k+1)}$的分量立即投入到下一个迭代方程,则得到Gauss-Seidel迭代法
\[\vD\vx^{(k+1)}=\vL\vx^{(k+1)}+\vU\vx^{(k)}+\vb\]

\subsubsection{超松弛(SOR)迭代法}
将GS迭代改写为
\[\begin{aligned}
    \vx^{(k+1)}&=\vD^{-1}(\vL\vx^{(k+1)}+\vU\vx^{(k)}+\vb\\
    &=\vx^{(k)}+\vD^{-1}(\vL\vx^{(k+1)}+\vU\vx^{(k)}-\vD\vx^{(k)}+\vb)
\end{aligned}\]
等号右侧第二项为修正量,为获得更快的收敛速度,在其前面乘系数$\omega$,即得到逐次超松弛(SOR)迭代法,其中$\omega$为松弛因子
\[\vx^{(k+1)}=(\vD-\omega\vL)^{-1}[(1-\omega)\vD+\omega\vU]\vx^{(k)}+\omega(\vD-\omega\vL)^{-1}\vb\]
此时迭代矩阵为
\[\vB_{SOR}=(\vD-\omega\vL)^{-1}[(1-\omega)\vD+\omega\vU]\]