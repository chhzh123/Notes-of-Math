\documentclass{note}
\usepackage{mypackage}

\renewcommand{\thefootnote}{\fnsymbol{footnote}}

\title{数值计算方法}
\author{陈鸿峥}
\date{{\builddatemonth\today}\protect\footnote{\text{Build \builddate\today}}}%加了build

\begin{document}

\maketitle
\renewcommand{\thefootnote}{\arabic{footnote}}
\setcounter{footnote}{0}

\setcounter{tocdepth}{2}%设置深度
\tableofcontents
\bigskip\bigskip\bigskip

\section{简介}
\subsection{误差}
误差来源
\begin{itemize}
	\item 原始误差:模型误差
	\item 观测误差:测量数据产生的误差
	\item 方法误差:截断误差
	\item 计算误差:舍入误差
\end{itemize}
数值计算需要注意的问题
\begin{itemize}
	\item 避免相近的数相减,如$\sqrt{1001}-\sqrt{1000}$有效数字会损失,分子有理化可使误差减小。\textbf{数学上等价的公式在计算上是不等价的!}
	\item 避免数量级相差太大的两数相除,容易溢出
	\item 避免大数和小数相加减,浮点数计算要对阶
	\item 简化计算步骤
\end{itemize}

\end{document}