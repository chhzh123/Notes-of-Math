% !TEX root = main.tex

\section{常微分方程} % 重点
\[\begin{cases}
    y'=f(x,y),&x\in[a,b]\\
    y(a)=y_0
\end{cases}\]
其中,$y_0\in\rr^m$,$f(x,y)$是定义在$m+1$维区域
\[G=\{(x,y)\mid x\in[a,b],y\in\rr^m\}\]
上的$m$维已知函数向量

\subsection{欧拉公式}
前向欧拉公式
\[y_{n+1}=y_n+hf(x_n,y_n)\]

后退欧拉公式(隐格式需要解方程)
\[y_{n+1}=y_n+hf(x_{n+1},y_{n+1})\]

梯形公式
\[y_{n+1}=y_n+\frac{h}{2}[f(x_n,y_n)+f(x_{n+1},y_{n+1})]\]

改进欧拉公式
\[\begin{cases}
    \text{预估} & \bar{y}_{n+1}=y_n+hf(x_n,y_n)\\
    \text{校正} & \disp y_{n+1}=y_n+\frac{h}{2}[f(x_n,y_n)+f(x_{n+1},\bar{y}_{n+1})]
\end{cases}\]

\begin{theorem}
若常微分初值问题的计算公式
\[y_{n+1}=y_n+hg(y_{n+1},y_n,\ldots,y_{n-r})\]
的局部截断误差为
\[\eps_{n+1}=y(x_{n+1})-\hat{y}_{n+1}=O(h^{p+1})\]
则称该公式所代表的数值方法具有$p$阶精度或称为$p$阶方法。
如欧拉公式就是一阶方法,梯形公式是二阶方法。
\end{theorem}

\begin{example}
    将下列方程化为一阶方程组
    \[\begin{cases}
        y''-3y'+2y=0 & x\in(0,1)\\
        y(0)=1,\;y'(0)=1
    \end{cases}\]
\end{example}
\begin{analysis}
    令$y'=z$,则
    \[z'=y''=3y'-2y=3z-2y\]
    因此得到方程组
    \[\begin{cases}
        \bmat{y\\z}'=\bmat{0 & 1\\-2 & 3}\bmat{y\\z}\\
        \bmat{y\\z}\Big|_{x=0}=\bmat{1\\1}
    \end{cases}\]
\end{analysis}

\subsection{龙格库塔公式}
二阶龙格库塔公式(中点公式)
\[\begin{cases}
    y_{n+1}&=y_n+k_2\\
    k_1&=hf(x_n,y_n)\\
    k_2&=hf\lrp{x_n+\frac{1}{2}h,y_n+\frac{1}{2}k_1}
\end{cases}\]

三阶三段龙格库塔公式
\[\begin{cases}
    y_{n+1}&=y_n+\frac{1}{6}(k_1+4k_2+k_3)\\
    k_1&=hf(x_n,y_n)\\
    k_2&=hf\lrp{x_n+\frac{1}{2}h,y_n+\frac{1}{2}k_1}\\
    k_3&=hf\lrp{x_n+h,y_n-k_1+2k_2}
\end{cases}\]

标准四阶龙格库塔公式
\[\begin{cases}
    \disp y_{n+1}&=y_n+\frac{1}{6}(k_1+2k_2+2k_3+k_4)\\
    \disp k_1&=hf(x_n,y_n)\\
    \disp k_2&=hf\lrp{x_n+\frac{1}{2}h,y_n+\frac{1}{2}k_1}\\
    \disp k_3&=hf\lrp{x_n+\frac{1}{2}h,y_n+\frac{1}{2}k_2}\\
    \disp k_4&=hf\lrp{x_n+h,y_n+k_3}
\end{cases}\]