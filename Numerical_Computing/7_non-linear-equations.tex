% !TEX root = main.tex

\section{非线性方程求根}
对于下式,分别取$r=1,2,3$时为线性收敛、平方收敛和立方收敛,当$r\in(1,2)$时为超线性收敛
\[\lim_{k\to\infty}\frac{|e_{k+1}|}{|e_k|^r}=C,\;C>0\]

\subsection{二分法和不动点法}
\begin{itemize}
\item 二分法:如果指定精度为$\eps$,则最多需要迭代步数为
\[k=\Big\lceil\log_2\frac{b-a}{\eps}\Big\rceil\]
\item 不动点法:$x^\star=\phi(x^\star)$
\end{itemize}
\begin{theorem}
    设一元函数$\phi(x)$在包含区间$[a,b]$的开区间上一阶连续可导,且
    \begin{itemize}
        \item $\forall x\in[a,b]:\;\phi(x)\in[a,b]$
        \item $\exists L\in[0,1),\forall x\in[a,b]:\;|\phi'(x)|\leq L$
    \end{itemize}
    则有下列结论成立:
    \begin{itemize}
        \item $\forall x_0\in[a,b]$,由$x_{k+1}=\phi(x_k)$产生的迭代序列$\{x_k\}$必然收敛于函数$\phi(x)$在区间$[a,b]$上的唯一不动点,即$x^\star=\phi(x^\star)$
        \item 序列$\{x_k\}$的收敛速度估计
        \[|x_k-x^\star|\leq\frac{1}{1-L}|x_{k+1}-x_k|\]
        和
        \[|x_k-x^\star|\leq\frac{L^k}{1-L}|x_1-x_0|\]
    \end{itemize}
\end{theorem}

迭代加速:利用导数估计$\phi'(\xi)\approx L$,得到
\[\bar{\phi(x)}=\frac{\phi(x)-Lx}{1-L}\]
为新的不动点函数

\subsection{牛顿法}
牛顿法:相当于$\phi(x)=x-f(x)/f'(x)$的不动点迭代方法
\[x_{k+1}=x_{k}-\frac{f(x_k)}{f'(x_k)}\]
\begin{theorem}[牛顿法全局收敛性]
非线性函数$f$在$[a,b]$上二阶连续可微,满足$f(a)f(b)<0$,且$\forall x\in[a,b]:\;f'(x)\ne 0,f''(x)\ne 0$。
若选定初始点$x_0\in[a,b]$满足$f(x_0)f''(x_0)>0$,则牛顿迭代法收敛于方程唯一解$x^\star$
\end{theorem}
\begin{theorem}
设在区间$[a,b]$上有二阶连续可微函数$f(x)$,$f(a)f(b)<0$,且对于所有$x\in[a,b]$,有$f'(x)\ne 0,f''(x)\ne 0$,若$a,b$两点满足
\[\max\lrp{\lrabs{\frac{f(a)}{f'(a)}},\lrabs{\frac{f(b)}{f'(b)}}}<b-a\]
则对于任何$x_0\in[a,b]$,牛顿迭代法收敛于方程的唯一根$x^\star$
\end{theorem}

牛顿下山法:相当于加了个步长
\[x_{k+1}(\lambda)=x_k-\lambda\frac{f(x_k)}{f'(x_k)}\]

割线法:
\[x_{k+1}=x_k-f(x_k)\frac{x_k-x_{k-1}}{f(x_k)-f(x_{k-1})}\]