% !TEX root = main.tex

% http://mathworld.wolfram.com/DividedDifference.html
% http://mathfaculty.fullerton.edu/mathews/n2003/Web/NewtonPolyMod/NewtonPolyMod.html
\section{插值}
\begin{theorem}[唯一性]
$n$次插值多项式存在且唯一
\end{theorem}
\begin{analysis}
设$n$次多项式$p(x)=a_0+a_1x+\cdots+a_nx^n$为函数$f(x)$在$[a,b]$上$n+1$个互异节点$x_i(i=0,1,\ldots,n)$上的插值多项式,则求$p(x)$的系数$a_i$相当于解线性方程组
\[\begin{cases}
a_0+a_1x_0+\cdots+a_nx_0^n=y_0\\
a_0+a_1x_1+\cdots+a_nx_1^n=y_1\\
\vdots\\
a_0+a_1x_n+\cdots+a_nx_n^n=y_n
\end{cases}\]
其系数行列式为范德蒙特行列式$V=\prod_{i=1}^n\prod_{j=0}^{i-1}(x_i-x_j)$,因为$x_i$互不相同,故$V\ne 0$。
进而由克莱姆法则,$a_i$解存在且唯一。
\end{analysis}

由于$p(x)$为线性空间$P_n(x)$\footnote{次数小于等于$n$的代数多项式的集合}的一个点,因而其基底/插值基函数$\phi(x)$不唯一,有多种表示方法(不同的线性组合)
\[p(x)=a_0\phi_0(x)+a_1\phi_1(x)+\cdots+a_n\phi_n(x)\]

\subsection{拉格朗日插值}
\begin{definition}[拉格朗日插值基函数]
节点$x_k$上的插值基函数$l_k(x)$满足
\[l_k(x_k)=\begin{cases}1&i=k\\0&i\ne k\end{cases}\]
\end{definition}
\par 由定义知$l_k(x)$的零点为$x_i,i\ne k$,待定系数$A_k$
\[l_k(x)=A_k(x-x_0)(x-x_1)\cdots(x-x_{k-1})(x-x_{k+1})\cdots(x-x_n)\]
可求得$A_k$,进而得到插值基函数
\[l_k(x)=\prod_{\substack{i=0\\i\ne k}}^{n}\frac{(x-x_i)}{x_k-x_i}\]
因此拉格朗日插值函数为
\[L_n(x)=\sum_{k=0}^nf(x_k)\prod_{\substack{i=0\\i\ne k}}^{n}\frac{(x-x_i)}{x_k-x_i}\]
其中$L_1(x)$称为线性插值,$L_2(x)$称为抛物线插值

误差估计
\[R_n(x)=\frac{f^{(n+1)}(\xi)}{(n+1)!}\prod_{i=0}^n(x-x_i),\;\xi=\xi(x)\in(a,b)\]

\subsection{牛顿插值}
\begin{definition}[牛顿插值基函数]
\[\begin{cases}
\varphi_0(x)=1\\
\varphi_j(x)=\displaystyle\prod_{i=0}^{j-1}(x-x_i),\;j=1,2,\ldots,n
\end{cases}\]
\end{definition}
\par 进而牛顿插值函数为
\[N_n(x)=a_0+a_1(x-x_0)+a_2(x-x_0)(x-x_1)+\cdots=a_0+\sum_{i=1}^na_i\prod_{j=0}^{i-1}(x-x_j)\]
实际上就是将$x_i$之后的项全部为$0$消掉了。
\par 由$f(x_k)=N_n(x_k)$,对比系数可得
\[\begin{aligned}
a_0 &=f(x_0)\\
a_1 &=\frac{f(x_1)-f(x_0)}{x_1-x_0}\\
a_2 &=\frac{\dfrac{f(x_2)-f(x_0)}{x_2-x_0}-\dfrac{f(x_1)-f(x_0)}{x_1-x_0}}{x_2-x_1}
\end{aligned}\]

\begin{definition}[差商]
一阶差商$f[x_i,x_k]=\dfrac{f(x_k)-f(x_i)}{x_k-x_i}$,二阶差商$f[x_i,x_j,x_k]=\dfrac{f[x_i,x_k]-f[x_j,x_k]}{x_k-x_j}$,高阶差商为
\[f[x_0,x_1,\ldots,x_{k-1},x_k]=\frac{f[x_0,x_1,\ldots,x_{k-2},x_k]-f[x_0,x_1,\ldots,x_{k-1}]}{x_k-x_{k-1}}\]
\end{definition}
由差商定义可得以下差商表(注:下面$f_i:=f(x_i)$)
\begin{center}
\begin{tabular}{ccccc}\hline
$x_k$ & $y_k$ & 一阶 & 二阶 & 三阶\\\hline
$x_0$ & $f_0$ & & &\\
$x_1$ & $f_1$ & $f[x_0,x_1]$ & &\\
$x_2$ & $f_2$ & $f[x_1,x_2]$ & $f[x_0,x_1,x_2]$ &\\
$x_3$ & $f_3$ & $f[x_2,x_3]$ & $f[x_1,x_2,x_3]$ & $f[x_0,x_1,x_2,x_3]$ \\\hline
\end{tabular}
\end{center}
并且有
\[\begin{aligned}
a_0&=f(x_0)\\
a_1&=f[x_0,x_1]\\
a_2&=f[x_0,x_1,x_2]\\
\vdots\\
a_n&=f[x_0,x_1,\ldots,x_n]
\end{aligned}\]

由误差估计的计算可得差商与导数之间的关系
\[f[x_0,x_1,\ldots,x_k]=\frac{f^{(k)}(\xi)}{k!}\]

可以看出,当增加一个节点时,牛顿插值只需增加一项,而拉格朗日插值需要全部重新算。

\begin{definition}[差分]
设已知函数$f(x)$在等距节点$x_i=x_0+ih(i=0,1,2,\ldots,n)$上的函数值为$f(x_i)=f_i$,其中$h>0$为步长,则称$\Delta f_i=f(x_i+h)-f(x_i)=f_{i+1}-f_i$为$f(x)$在$x_i$处步长为$h$的一阶向前差分。
称$\Delta^2 f_i=\Delta(\Delta f_i)=\Delta f_{i+1}-\Delta f_i$为函数$f(x)$在$x_i$处的二阶向前差分。
一般地$\Delta^n f_i=\Delta(\Delta^{n-1} f_i)$为函数$f(x)$在$x_i$处的$n$阶向前差分。
并规定,$\Delta^0 f_i=f_i$为零阶差分。
\end{definition}
\par 差分与差商的关系为
\[f[x_0,x_1,\ldots,x_k]=\frac{\Delta^k f_0}{k!h^k},\,k=1,2,\ldots,n\]

则牛顿插值公式的向前差分形式为
\[N_n(x)=N_n(x_0+th)=f_0+\frac{t}{1!}\Delta f_0+\frac{t(t-1)}{2!}\Delta^2 f_0+\cdots+\frac{t(t-1)\cdots(t-n+1)}{n!}\Delta^n f_0,\,t\in[0,1]\]
类似于泰勒展开。

\subsection{埃尔米特插值}
不仅要求插值点的函数值$H(x_i)$与原函数的值$f(x_i)$相同,还要求它们有公切线,即$H'(x_i)=f'(x_i)$。
若所有插值点都要满足上述条件,则埃尔米特(Hermite)插值多项式为$2n+1$次的。

\subsection{三次样条插值}
分多段,每段用三次函数逼近

三次样条条件
\begin{itemize}
    \item 插值条件:$s_k(x_j)=y_j,\quad j=k-1,k,\quad k=1,2,\ldots,n$,共$2n$个条件
    \item 端点导数条件:$\lim_{x\to x_k^-}s^{(p)}(x)=\lim_{x\to x^+_k}s^{(p)}(x),\quad p=1,2,\quad k=1,\ldots,n-1$,共$2n-2$个条件
    \item 边界条件:
    \begin{itemize}
        \item 第一类边界条件:$s'(x_0)=f_0',s'(x_n)=f_n'$,其中$f_0',f_n'$为给定值
        \item 第二类边界条件:$s''(x_0)=f_0'',s''(x_n)=f_n''$,其中$f_0'',f_n''$为给定值,若为$0$,则为\textbf{自然}边界条件
        \item 周期性条件:$\lim_{x\to x_0^+}s^{(p)}(x)=\lim_{x\to x_n^-}s^{(p)}(x),p=1,2$
    \end{itemize}
\end{itemize}

三转角方法,用一阶导数求解;三弯矩方法,用二阶导数求解。
\[\bmat{2 & 1 & & & \\\mu_1 & 2 & \lambda_1 & & \\ & \mu_2 & 2 & \lambda_2 & \\ & & \ddots & \ddots & \ddots \\ & & & \mu_{n-1} & 2 & \lambda_{n-1}\\&&&&1&2}\bmat{M_0\\M_1\\M_2\\\vdots\\M_{n-1}\\M_n}=\bmat{d_0\\d_1\\d_2\\\vdots\\d_{n-1}\\d_n}\]