% !TEX root = main.tex

\section{函数逼近}
用函数$f(x)$和$p(x)$的最大误差
\[\norm{p-f}_\infty=\max_{a\leq x\leq b}|p(x)-f(x)|\]
作为度量逼近函数$p(x)$对被逼近函数$f(x)$的逼近程度。
若存在一个函数序列满足$\lim_{n\to\infty}\norm{p_n(x)-f(x)}_\infty=0$,则意味着序列$\{p_n(x)\}$在区间$[a,b]$上一致收敛至$f(x)$。
序列$\{p_n(x)\}$对$f(x)$的逼近称为\textbf{一致逼近}。

也可用积分$\norm{p-f}_2=\lrp{\int_a^b(p(x)-f(x))^2\diff x}^{\frac{1}{2}}$作为度量函数。
若存在一个函数序列满足$\lim_{n\to\infty}\norm{p_n(x)-f(x)}_2=0$,则意味着序列$\{p_n(x)\}$在区间$[a,b]$上平方收敛至$f(x)$。
序列$\{p_n(x)\}$对$f(x)$的逼近称为\textbf{平方逼近}。

\subsection{内积与正交多项式}
\begin{definition}[权函数]
    设$[a,b]$为有限或无限区间,$\rho(x)$是定义在$[a,b]$上的非负函数,$\intab{a}{b}{x^k\rho(x)}$对$k=0,1,\ldots$都存在,且对非负的$f(x)\in C[a,b]$\footnote{$C[a,b]$表示区间$[a,b]$上连续函数的全体},若$\intab{a}{b}{f(x)\rho(x)}=0$,则$f(x)\equiv 0$,称$\rho(x)$为$[a,b]$上的权函数,其刻画了点$x$在$[a,b]$上的重要性
\end{definition}
\begin{definition}[内积]
    设$f(x),g(x)\in C[a,b]$,$\rho(x)$为$[a,b]$上的权函数,则函数$f(x)$和$g(x)$的内积为
    \[\lrang{f,g}:=\intab{a}{b}{\rho(x)f(x)g(x)}\]
\end{definition}
定义了内积的线性空间称为内积空间。
内积需满足
\begin{enumerate}
    \item $\lrang{f,g}=\lrang{g,f}$
    \item $\lrang{c_1f+c_2g,h}=c_1\lrang{f,h}+c_2\lrang{g,h}$, $c_1,c_2$为常数
    \item $\lrang{f,f}\geq 0\iff f\equiv 0,\lrang{f,f}=0$
\end{enumerate}

\begin{definition}[范数]
对内积空间$C[a,b]$中的每一个函数$f(x)$,都赋予一个数值
\[\norm{f}=\sqrt{\int{a}{b}{\rho(x)[f(x)]^2}}\]
\end{definition}
范数需满足
\begin{enumerate}
    \item $\norm{f}\geq 0$,当且仅当$f\equiv 0$时,$\norm{f}=0$
    \item $\norm{cf}=|c|\cdot\norm{f}$,$c$是常数
    \item $\norm{f+g}\leq\norm{f}+\norm{g}$
\end{enumerate}

正交函数系
\begin{definition}[正交]
    设$f(x),g(x)\in C[a,b]$,$\rho(x)$为$[a,b]$上的权函数,若内积
    \[\lrang{f,g}=\intab{a}{b}{\rho(x)f(x)g(x)}=0\]
    则称$f(x),g(x)$在$[a,b]$上带权$\rho(x)$正交
\end{definition}

\subsubsection{勒让德多项式}
勒让德(Legendre)多项式是区间$[-1,1]$上权函数$\rho(x)=1$的正交多项式
\[P_0(x)=1,\qquad P_n(x)=\frac{1}{2^nn!}\frac{\diff^n}{\diff x^n}(x^2-1)^n\]
勒让德多项式有许多重要的性质
\begin{enumerate}
    \item 正交性
\[\lrang{P_n,P_m}=\intab{-1}{1}{P_n(x)P_m(x)}=\begin{cases}0&m\ne n\\\frac{2}{2n+1}&m=n\end{cases}\]
    \item 递推公式
    \[(n+1)P_{n+1}(x)=(2n+1)xP_n(x)-nP_{n-1}(x),P_0(x)=1,P_1(x)=x\]
\end{enumerate}

\subsubsection{切比雪夫多项式}
切比雪夫(Chebyshev)多项式为区间$[-1,1]$上关于权函数$\rho(x)=\frac{1}{\sqrt{1-x^2}}$的正交多项式
\[T_n(x)=\cos(n\arccos x)\]
\begin{enumerate}
    \item 正交性
\[\lrang{T_n,T_m}=\intab{-1}{1}{\frac{T_n(x)P_m(x)}{\sqrt{1-x^2}}}=\begin{cases}0&m\ne n\\\frac{\pi}{2}&m=n\ne 0\\\pi&m=n=0\end{cases}\]
    \item 递推公式
\[T_{n+1}(x)=2xT_n(x)-T_{n-1}(x),T_0(x)=1,T_1(x)=x\]
    \item 奇偶性
\[T_n(-x)=(-1)^nT_n(x)\]
    \item $T_n(x)$在$(-1,1)$内的$n$个零点为$x_k=\cos\frac{2k-1}{2n}\pi$,在$[-1,1]$上有$n+1$个极值点$y_k=\cos\frac{k}{n}\pi$
    \item $T_n(x)$的最高次幂$x^n$的系数为$2^{n-1},n\geq 1$
\end{enumerate}

\subsubsection{其他正交多项式}
\begin{enumerate}
    \item 拉盖尔(Laguerre)多项式
    \item 埃尔米特(Hermite)多项式
\end{enumerate}

\subsection{最佳一致逼近}


\subsection{最佳平方逼近}
\begin{definition}[最佳平方逼近]
    设$f(x)\in C[a,b]$,如果存在$s^\star(x)\in\Phi$,使
    \[\intab{a}{b}{\rho(x)[f(x)-s^\star(x)]^2}=\min_{s(x)\in\Phi}\intab{a}{b}{\rho(x)[f(x)-s(x)]^2}\]
    则称$s^\star$为$f(x)$在集合$\Phi$中的最佳平方逼近函数。
    若$\Phi=P_n=\opspan\{1,x,\ldots,x^n\}$,则$s^\star(x)$为$f(x)$的$n$次最佳平方逼近多项式。
\end{definition}
\[G_n=\bmat{\lrang{\phi_0,\phi_0} & \lrang{\phi_0,\phi_1} & \cdots & \lrang{\phi_n,\phi_n}\\\vdots & \vdots & \ddots & \vdots\\\lrang{\phi_n,\phi_0} & \lrang{\phi_n,\phi_1} & \cdots & \lrang{\phi_n,\phi_n}}\]
希尔伯特(Hilbert)矩阵
\[H_{n+1}=\bmat{1 & \frac{1}{2} & \cdots & \frac{1}{n+1}\\\frac{1}{2} & \frac{1}{3} & \cdots & \frac{1}{n+2}\\\vdots & \vdots & \ddots & \vdots\\\frac{1}{n+1} & \frac{1}{n+2} & \cdots & \frac{1}{2n+1}}\]


\subsection{最小二乘法}
拟合函数为
\[\sum_{j=0}^na_j\phi_j(x)\]
$\phi(x)$为基函数,线性无关,求解下面法方程
\[\bmat{\lrang{\phi_0,\phi_0} & \lrang{\phi_0,\phi_1} & \cdots & \lrang{\phi_n,\phi_n}\\\vdots & \vdots & \ddots & \vdots\\\lrang{\phi_n,\phi_0} & \lrang{\phi_n,\phi_1} & \cdots & \lrang{\phi_n,\phi_n}}\bmat{a_0\\\vdots\\a_n}=\bmat{\lrang{f,\phi_0}\\\cdots\\\lrang{f,\phi_n}}\]
其中
\[\begin{cases}
    \lrang{\phi_j,\phi_k}=\sum_{i=0}^m\phi_j(x_i)\phi_k(x_i)\\
    \lrang{f,\phi_k}=\sum_{i=0}^mf_(x_i)\phi_k(x_i)
\end{cases}\]
特别地,如果用代数多项式拟合,即取
\[\{\phi_0,\phi_1,\ldots,\phi_n\}=\{1,x,\ldots,x^n\}\]
有法方程
\[\bmat{\sum_{i=1}^m 1 & \sum_{i=1}^m x_i & \cdots & \sum_{i=1}^m x_i^n\\\vdots & \vdots & \ddots & \vdots\\\sum_{i=0}^mx_i^n & \sum_{i=0}^mx_i^{n+1} & \cdots & \sum_{i=0}^mx_i^{2n}}\]