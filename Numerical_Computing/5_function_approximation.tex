% !TEX root = main.tex

\section{函数逼近}
用函数$f(x)$和$p(x)$的\textbf{最大误差}
\[\norm{p-f}_\infty=\max_{a\leq x\leq b}|p(x)-f(x)|\]
作为度量逼近函数$p(x)$对被逼近函数$f(x)$的逼近程度。
若存在一个函数序列满足
\[\lim_{n\to\infty}\norm{p_n(x)-f(x)}_\infty=0\]
则意味着序列$\{p_n(x)\}$在区间$[a,b]$上一致收敛至$f(x)$。
序列$\{p_n(x)\}$对$f(x)$的逼近称为\textbf{一致逼近}。

也可用积分
\[\norm{p-f}_2=\lrp{\int_a^b(p(x)-f(x))^2\diff x}^{\frac{1}{2}}\]
作为度量函数。
若存在一个函数序列满足
\[\lim_{n\to\infty}\norm{p_n(x)-f(x)}_2=0\]
则意味着序列$\{p_n(x)\}$在区间$[a,b]$上平方收敛至$f(x)$。
序列$\{p_n(x)\}$对$f(x)$的逼近称为\textbf{平方逼近}。

\subsection{内积与正交多项式}
\begin{definition}[权函数]
    设$[a,b]$为有限或无限区间,$\rho(x)$是定义在$[a,b]$上的非负函数,$\intab{a}{b}{x^k\rho(x)}$对$k=0,1,\ldots$都存在,且对非负的$f(x)\in C[a,b]$\footnote{$C[a,b]$表示区间$[a,b]$上连续函数的全体},若$\intab{a}{b}{f(x)\rho(x)}=0$,则$f(x)\equiv 0$,称$\rho(x)$为$[a,b]$上的权函数,其刻画了点$x$在$[a,b]$上的重要性
\end{definition}
\begin{definition}[内积]
    设$f(x),g(x)\in C[a,b]$,$\rho(x)$为$[a,b]$上的权函数,则函数$f(x)$和$g(x)$的内积为
    \[\lrang{f,g}:=\intab{a}{b}{\rho(x)f(x)g(x)}\]
\end{definition}
定义了内积的线性空间称为内积空间。
内积需满足
\begin{enumerate}
    \item $\lrang{f,g}=\lrang{g,f}$
    \item $\lrang{c_1f+c_2g,h}=c_1\lrang{f,h}+c_2\lrang{g,h}$, $c_1,c_2$为常数
    \item $\lrang{f,f}\geq 0\iff f\equiv 0,\lrang{f,f}=0$
\end{enumerate}

\begin{definition}[范数]
对内积空间$C[a,b]$中的每一个函数$f(x)$,都赋予一个数值
\[\norm{f}=\sqrt{\intab{a}{b}{\rho(x)[f(x)]^2}}\]
\end{definition}
范数需满足
\begin{enumerate}
    \item $\norm{f}\geq 0$,当且仅当$f\equiv 0$时,$\norm{f}=0$
    \item $\norm{cf}=|c|\cdot\norm{f}$,$c$是常数
    \item $\norm{f+g}\leq\norm{f}+\norm{g}$
\end{enumerate}

正交函数系
\begin{definition}[正交]
    设$f(x),g(x)\in C[a,b]$,$\rho(x)$为$[a,b]$上的权函数,若内积
    \[\lrang{f,g}=\intab{a}{b}{\rho(x)f(x)g(x)}=0\]
    则称$f(x),g(x)$在$[a,b]$上带权$\rho(x)$正交
\end{definition}

\subsubsection{勒让德多项式}
勒让德(Legendre)多项式是区间$[-1,1]$上权函数$\rho(x)=1$的正交多项式
\[P_0(x)=1,\qquad P_n(x)=\frac{1}{2^nn!}\frac{\diff^n}{\diff x^n}(x^2-1)^n\]
勒让德多项式有许多重要的性质
\begin{enumerate}
    \item 正交性
\[\lrang{P_n,P_m}=\intab{-1}{1}{P_n(x)P_m(x)}=\begin{cases}0&m\ne n\\\frac{2}{2n+1}&m=n\end{cases}\]
    \item 递推公式
    \[(n+1)P_{n+1}(x)=(2n+1)xP_n(x)-nP_{n-1}(x),P_0(x)=1,P_1(x)=x\]
\end{enumerate}

\subsubsection{切比雪夫多项式}
切比雪夫(Chebyshev)多项式为区间$[-1,1]$上关于权函数$\rho(x)=\frac{1}{\sqrt{1-x^2}}$的正交多项式
\[T_n(x)=\cos(n\arccos x)\]
\begin{enumerate}
    \item 正交性
\[\lrang{T_n,T_m}=\intab{-1}{1}{\frac{T_n(x)P_m(x)}{\sqrt{1-x^2}}}=\begin{cases}0&m\ne n\\\frac{\pi}{2}&m=n\ne 0\\\pi&m=n=0\end{cases}\]
    \begin{analysis}
        令$x=\cos\theta$,由三角函数正交性\footnote{数学分析傅里叶级数一节,三角函数系$\{1,\cos x,\sin x,\ldots,\cos nx,\sin nx,\ldots\}$中任意两个不同函数在$[-\pi,\pi]$都正交}即得结论
    \end{analysis}
    \item 递推公式
\[T_{n+1}(x)=2xT_n(x)-T_{n-1}(x),T_0(x)=1,T_1(x)=x\]
\begin{analysis}
    同样三角代换$x=\cos\theta$,并结合和差化积即得结论
\end{analysis}
    \item 奇偶性
\[T_n(-x)=(-1)^nT_n(x)\]
    \item $T_n(x)$在$(-1,1)$内的$n$个零点为$x_k=\cos\frac{2k-1}{2n}\pi$,在$[-1,1]$上有$n+1$个极值点$y_k=\cos\frac{k}{n}\pi$
    \item $T_n(x)$的最高次幂$x^n$的系数为$2^{n-1},n\geq 1$
\end{enumerate}

\subsubsection{其他正交多项式}
\begin{enumerate}
    \item 拉盖尔(Laguerre)多项式
    \[L_n(x)=\ee^x\frac{\diff^n}{\diff x^n}(x^n\ee^{-x}),\qquad n=0,1,\ldots\]
    递推公式为
    \[L_{n+1}(x)=(1+2n-x)L_n(x)-n^2L_{n-1}(x),\qquad n=1,2,\ldots\]
    \item 埃尔米特(Hermite)多项式
    \[H_n(x)=(-1)^n\ee^{x^2}\frac{\diff^n}{\diff x^n}\ee^{-x^2},\qquad n=0,1,\ldots\]
    递推公式为
    \[H_{n+1}(x)=2xH_n(x)-2nH_{n-1}(x),\qquad n=1,2,\ldots\]
\end{enumerate}

\subsection{最佳一致逼近}
\begin{definition}
    设$f(x)\in\sC[a,b]$,以及多项式序列$p_n(x)$,若$\forall\eps,\exists N, n> N$,不等式
    \[\max_{a\leq x\leq b}|p_n(x)-f(x)|<\eps\]
    成立,则称多项式$p_n(x)$在$[a,b]$上一致逼近于$f(x)$
\end{definition}
\begin{definition}
    若$\exists x_0\in[a,b]$,使得
    \[|p(x_0)-f(x_0)|=\mu=\norm{p-f}_\infty=\max_{a\leq x\leq b}|p(x)-f(x)|\]
    则称$x_0$是$p(x)$关于$f(x)$的正/负偏差点
\end{definition}
\begin{theorem}[Chebyshev]
    $p_n^\star(x)\in P_n$是$f(x)\in\sC[a,b]$的最佳一致逼近多项式的充分必要条件是:在$[a,b]$上至少有$n+2$个交替为正负的偏差点,即至少有$n+2$个点$a\leq x_1<x_2\cdots<x_{n+2}\leq b$,使得
    \[p_n^\star(x_k)-f(x_k)=(-1)^k\sigma\norm{f-p_n^\star}_\infty,\;\sigma=\pm 1,\,k=1,2,\ldots,n+2\]
\end{theorem}
上述点$\{x_k\}_1^{n+2}$称为切比雪夫交错点组

线性最佳一致逼近:设$f(x)$在$[a,b]$上有二阶导数,且$f''(x)$在$[a,b]$上不变号,$p_1(x)=a_0+a_1x$为线性最佳一致逼近,其中
\[\begin{cases}
    a_0=\dfrac{1}{2}(f(a)+f(x_2))-\dfrac{a+x_2}{2}\dfrac{f(b)-f(a)}{b-a} & f'(x_2)=a_1\\
    a_1=\dfrac{f(b)-f(a)}{b-a}
\end{cases}\]
即拟合直线的斜率与连接$a,b$两点的割线斜率相同(Lagrange切线,$x_2$为切点),且过$a$和$x_2$的中点。

\subsection{最佳平方逼近}
\begin{definition}[最佳平方逼近]
    设$f(x)\in C[a,b]$,如果存在$s^\star(x)\in\varphi$,使
    \[\intab{a}{b}{\rho(x)[f(x)-s^\star(x)]^2}=\min_{s(x)\in\varphi}\intab{a}{b}{\rho(x)[f(x)-s(x)]^2}\]
    则称$s^\star$为$f(x)$在集合$\varphi$中的最佳平方逼近函数。
    若$\varphi=P_n=\opspan\{1,x,\ldots,x^n\}$,则$s^\star(x)$为$f(x)$的$n$次最佳平方逼近多项式。
\end{definition}
\[G_n=\bmat{\lrang{\varphi_0,\varphi_0} & \lrang{\varphi_0,\varphi_1} & \cdots & \lrang{\varphi_n,\varphi_n}\\\vdots & \vdots & \ddots & \vdots\\\lrang{\varphi_n,\varphi_0} & \lrang{\varphi_n,\varphi_1} & \cdots & \lrang{\varphi_n,\varphi_n}}\]
希尔伯特(Hilbert)矩阵
\[H_{n+1}=\bmat{1 & \frac{1}{2} & \cdots & \frac{1}{n+1}\\\frac{1}{2} & \frac{1}{3} & \cdots & \frac{1}{n+2}\\\vdots & \vdots & \ddots & \vdots\\\frac{1}{n+1} & \frac{1}{n+2} & \cdots & \frac{1}{2n+1}}\]
法方程为
\[H_{n+1}\va=\vd\]
解为$\va=\va^\star$,进而得到最佳平方逼近式$p_n^\star(x)$

\subsection{最小二乘法}
拟合函数为
\[\sum_{j=0}^na_j\varphi_j(x)\]
$\varphi_i(x)$为基函数,且线性无关,求解下面法方程
\[\bmat{\lrang{\varphi_0,\varphi_0} & \lrang{\varphi_0,\varphi_1} & \cdots & \lrang{\varphi_n,\varphi_n}\\\vdots & \vdots & \ddots & \vdots\\\lrang{\varphi_n,\varphi_0} & \lrang{\varphi_n,\varphi_1} & \cdots & \lrang{\varphi_n,\varphi_n}}\bmat{a_0\\\vdots\\a_n}=\bmat{\lrang{f,\varphi_0}\\\vdots\\\lrang{f,\varphi_n}}\]
其中
\[\begin{cases}
    \lrang{\varphi_j,\varphi_k}=\disp\sum_{i=0}^m\varphi_j(x_i)\varphi_k(x_i)\\
    \lrang{f,\varphi_k}=\disp\sum_{i=0}^mf(x_i)\varphi_k(x_i)
\end{cases}\]
特别地,如果用代数多项式拟合,即取
\[\{\varphi_0,\varphi_1,\ldots,\varphi_n\}=\{1,x,\ldots,x^n\}\]
有法方程
\[\bmat{\sum_{i=1}^m 1 & \sum_{i=1}^m x_i & \cdots & \sum_{i=1}^m x_i^n\\\vdots & \vdots & \ddots & \vdots\\\sum_{i=0}^mx_i^n & \sum_{i=0}^mx_i^{n+1} & \cdots & \sum_{i=0}^mx_i^{2n}}\]

\begin{example}
    用最小二乘法求下面数据的二次拟合$y=a\theta+b\theta^2$
    \begin{center}
        \begin{tabular}{ccccc}\hline
            $\theta$ & 1 & 2 & 3 & 4\\\hline
            $f$ & 0.8 & 1.5 & 1.8 & 2.0\\\hline
        \end{tabular}
    \end{center}
\end{example}
\begin{analysis}
    注意基底分别为$\varphi_1(x)=\theta,\varphi_2(x)=\theta^2$,则
    \[\bmat{\sum_{i=1}^4\theta_i^2 & \sum_{i=1}^4\theta_i^3\\\sum_{i=1}^4\theta_i^3 & \sum_{i=1}^4\theta_i^4}\bmat{a\\b}=\bmat{\sum_{i=1}^4\theta_i f_i\\\sum_{i=1}^4\theta_i^2 f_i}\]
    代入数据可解得
    \[\begin{cases}
        a=0.9491\\
        b=-0.11274
    \end{cases}\]
\end{analysis}
