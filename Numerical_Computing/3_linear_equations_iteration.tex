% !TEX root = main.tex

\section{线性方程组迭代解法}
\subsection{范数与条件数}
\begin{definition}[向量的范数]
    对任意$n$维向量$\vx$,若对应非负实数$\norm{\vx}$,满足
    \begin{enumerate}
        \item 正定性:$\norm{\vx}\geq 0$,当且仅当$\vx=\vzero$时等号成立
        \item 数乘:对任意$\alpha\in\rr$,$\norm{\alpha\vx}=|\alpha|\cdot\norm{\vx}$
        \item 三角不等式:对任意的$n$维向量$\vx$和$\vy$,满足$\norm{\vx+\vy}\leq\norm{\vx}+\norm{\vy}$
    \end{enumerate}
    则称$\norm{\vx}$为$\vx$的范数,其中1-范数、2-范数、无穷范数定义如下
    \[\begin{aligned}
        \norm{\vx}_1&=\sum_{i=1}^n|x_i|\\
        \norm{\vx}_2&=\sqrt{\sum_{i=1}^n|x_i|^2}\\
        \norm{\vx}_\infty&=\max_{1\leq i\leq n}|x_i|
    \end{aligned}\]
\end{definition}

\begin{definition}[矩阵的范数]
    对于$n$阶方阵$A$,若对应非负实数$\norm{A}$,满足
    \begin{enumerate}
        \item $\norm{A}\geq 0$,当且仅当$A=\vzero$时等号成立
        \item 对任意$\alpha\in\rr$,$\norm{\alpha A}=|\alpha|\cdot\norm{A}$
        \item 对任意两个$n$阶方阵$A$和$B$,满足三角不等式$\norm{A+B}\leq\norm{A}+\norm{B}$
        \item 对任意两个$n$阶方阵$A$和$B$,满足矩阵乘法要求$\norm{AB}\leq\norm{A}\cdot\norm{B}$
    \end{enumerate}
    则称$\norm{A}$为方阵$A$的矩阵范数。
    记$\rho(A)=\max_{1\leq i\leq n}|\lambda_i|$为$A$的谱半径,这里$\lambda_i$为$A$的特征值,矩阵的1-范数、2-范数、无穷范数和F范数分别定义如下
    \[\begin{aligned}
        \norm{A}_1&=\max_{1\leq j\leq n}\sum_{i=1}^n|a_{ij}|\\
        \norm{A}_2&=\sqrt{\rho(A^\T A)}\\
        \norm{A}_\infty&=\max_{1\leq i\leq n}\sum_{j=1}^n|a_{ij}|\\
        \norm{A}_F&=\sqrt{\sum_{i=1}^n\sum_{j=1}^n|a_{ij}|^2}
    \end{aligned}\]
    注意矩阵的F-范数才是向量2-范数的直接推广,而矩阵的2-范数是计算$A^\T A$的谱半径,又被称为谱范数
\end{definition}
\begin{definition}[向量范数与矩阵范数相容]
向量范数与矩阵范数相容,则满足
\[\norm{A\vx}\leq\norm{A}\cdot\norm{\vx}\]
一些相容的范数如下
\begin{itemize}
    \item 矩阵1-范数与向量1-范数
    \item 矩阵2-范数与向量2-范数
    \item 矩阵无穷范数与向量无穷范数
    \item 矩阵F范数与向量2-范数
\end{itemize}
\end{definition}

\begin{example}
    证明:对于矩阵范数,如果$\norm{A}<1$,则
    \[\norm{(I+A)^{-1}}\leq\frac{1}{1-\norm{A}}\]
\end{example}
\begin{analysis}
    首先证明$I+A$为非奇异矩阵,若不然,$(I+A)\vx=\vzero$有非零解$\vx\ne\vzero$,即$\vx=-A\vx$。
    两侧取向量范数有
    \[\norm{\vx}=\norm{A\vx}\leq\norm{A}\cdot\norm{\vx}\]
    由于$\norm{\vx}\ne 0$,所以$\norm{A}\geq 1$,与题设矛盾。\\
    由$(I+A)^{-1}(I+A)=I$得到
    \[(I+A)^{-1}=I-(I+A)^{-1}A\]
    两边取矩阵范数,有
    \[\norm{(I+A)^{-1}}\leq1+\norm{(I+A)^{-1}}\cdot\norm{A}\]
    故原式成立
\end{analysis}

\begin{definition}[条件数]
    设$A$为$n$阶非奇异矩阵,称数$\opcond(A)=\norm{A}\norm{A^{-1}}$为线性方程组$A\vx=\vb$或矩阵$A$的条件数,其对方程组的解的相对误差起到关键的控制作用\\
    如果$A$对称正定,则有
    \[\opcond_2(A)=\lambda_{\max}(A)/\lambda_{\min}(A)\]
\end{definition}
\begin{analysis}
考虑方程$A\vx=\vb$,在方程组右端有一个小扰动$\delta b$,则解$\vx$产生一个扰动$\delta\vx$,即
\[A(\vx+\delta\vx)=\vb+\delta\vb\]
于是有
\[\delta\vx=A^{-1}\delta\vb\]
两边取范数有
\[\norm{\delta\vx}\leq\norm{A^{-1}}\norm{\delta\vb}\]
在原式中取范数有
\[\norm{\vb}=\norm{A}\norm{\vx}\]
上面两条不等式相乘,整理得
\[\frac{\norm{\delta\vx}}{\norm{\vx}}\leq\norm{A}\norm{A^{-1}}\frac{\norm{\delta\vb}}{\norm{\vb}}\]
进而得到扰动变化与条件数相关
\end{analysis}

条件数满足以下四条性质
\begin{itemize}
    \item 对于任意$n$阶非奇异矩阵$A$,$\opcond(A)\geq 1$成立
    \item 对于任意$n$阶非奇异矩阵$A$及任意非零常数$c$,$\opcond(cA)=\opcond(A)$成立
    \item 对于任意正交矩阵$A$,有$\opcond_2(A)=1$
    \item 对于任意$n$阶非奇异矩阵$A$及任意$n$阶正交矩阵$P$,有$\opcond_2(PA)=\opcond_2(AP)=\opcond_2(A)$
\end{itemize}

一类十分典型的病态矩阵是希尔伯特矩阵$H_n$,其条件数$\opcond_2(H_n)\approx\ee^{3.5n}$,病态非常严重

\subsection{基本迭代法}
给定线性方程组$A\vx=\vb$,设$A=M-N$,其中$M$为非奇异矩阵,则原式变为
\[M\vx^{(k+1)}=N\vx^{(k)}+\vb\]
将$A$拆分成三个矩阵之和(只是将矩阵$A$元素分块而已)
\[A=D-L-U\]
对角线阵$D$、\textbf{负的}严格下三角阵$L$和严格上三角阵$U$
\[\begin{aligned}
    D&=\opdiag(a_{11},a_{22},\ldots,a_{nn})\\
    L&=-\bmat{0 & 0 & \cdots & 0\\ a_{21} & 0 & \cdots & 0\\\vdots & \ddots & \ddots & \vdots\\ a_{n1}&\cdots & a_{n,n-1} & 0}\\
    U&=-\bmat{0 & a_{12} & \cdots & a_{1n}\\\vdots & \ddots & \ddots & \vdots\\0 & \ddots & \ddots & a_{n-1,n}\\ 0& 0 & \cdots & 0}
\end{aligned}\]

\subsubsection{雅可比迭代法}
取$M=D$和$N=L+U$,即得雅可比迭代法
\[D\vx^{(k+1)}=(L+U)\vx^{(k)}+\vb\]
可以得到迭代格式
\[\iter{x_i}{k+1}=\frac{1}{a_{ii}}\lrp{-\sum_{j=1,j\ne i}^n a_{ij}\iter{x_j}{k}+b_i}\]

\begin{theorem}
    若$A$是对称正定矩阵,则雅可比迭代收敛的充要条件是$2D-A$也是对称正定矩阵
\end{theorem}

\subsubsection{高斯-赛德尔迭代法}
如果在雅可比迭代算法的第三步,将算出来的$\vx^{(k+1)}$的分量立即投入到下一个迭代方程,则得到Gauss-Seidel迭代法
\[D\vx^{(k+1)}=L\vx^{(k+1)}+U\vx^{(k)}+\vb\]
可以得到迭代格式
\[\iter{x_i}{k+1}=\frac{1}{a_{ii}}\lrp{-\sum_{j=1}^{i-1}a_{ij}\iter{x_j}{k+1}-\sum_{j=i+1}^na_{ij}\iter{x_j}{k}+b_i}\]

将原迭代格式
\[M\vx^{(k+1)}=N\vx^{(k)}+\vb\]
变为
\[\iter{\vx}{k+1}=B\iter{\vx}{k}+\vg\]
其中,$B=M^{-1}N$,$\vg=M^{-1}\vb$
\begin{theorem}
    对于上述迭代格式,给定任意初值$\iter{\vx}{0}$,有下列收敛结果和误差估计:
    \begin{enumerate}
        \item 迭代格式收敛的充要条件为谱半径$\rho(B)<1$
        \item 如果$\norm{B}<1$,则有估计
        \[\norm{\iter{\vx}{k}-\vx^\star}\leq\frac{\norm{B}^{k}}{1-\norm{B}}\norm{\iter{\vx}{1}-\iter{\vx}{0}}\]
        和
        \[\norm{\iter{\vx}{k}-\vx^\star}\leq\frac{\norm{B}}{1-\norm{B}}\norm{\iter{\vx}{k}-\iter{\vx}{k-1}}\]
    \end{enumerate}
\end{theorem}

\begin{definition}[不可约]
    一个每行每列仅有唯一的非零元$1$的方阵称为排列矩阵。
    设$A$是$n\geq 2$阶方阵,如果存在$n$阶排列矩阵$P$,使
    \[P^\T AP=\bmat{A_{11} & A_{12}\\\vzero & A_{22}}\]
    则称$A$为可约矩阵,否则为不可约阵\\
    简单来说,就是能否交换矩阵的行列使其变为上三角矩阵
\end{definition}
\begin{definition}[对角占优]
    设$A$是$n$阶方阵,若$A$满足
    \[|a_{ii}|\geq \sum_{j=1,j\ne i}^n|a_{ij}|,\;i=1,2,\ldots,n\]
    且其中至少有一个严格不等式成立,则称$A$是弱对角占优;若每一个不等式都是严格不等式,则称$A$是(行)严格对角占优
\end{definition}
\begin{theorem}
    若$A$是严格对角占优或不可约弱对角占优矩阵,则雅可比迭代和GS迭代都收敛
\end{theorem}

\subsubsection{超松弛(SOR)迭代法}
将GS迭代改写为
\[\begin{aligned}
    \vx^{(k+1)}&=D^{-1}(L\vx^{(k+1)}+U\vx^{(k)}+\vb)\\
    &=\vx^{(k)}+D^{-1}(L\vx^{(k+1)}+U\vx^{(k)}-D\vx^{(k)}+\vb)
\end{aligned}\]
等号右侧第二项为修正量,为获得更快的收敛速度,在其前面乘系数$\omega$,即得到逐次超松弛(SOR)迭代法,其中$\omega$为松弛因子
\[\begin{aligned}
    \vx^{(k+1)}&=\vx^{(k)}+\omega D^{-1}(L\vx^{(k+1)}+U\vx^{(k)}-D\vx^{(k)}+\vb)\\
    &=(D-\omega L)^{-1}[(1-\omega)D+\omega U]\vx^{(k)}+\omega(D-\omega L)^{-1}\vb\qquad\mbox{将$D$乘回左边再化简}
\end{aligned}\]
可以得到迭代格式
\[\iter{x_i}{k+1}=\iter{x_i}{k}+\frac{\omega}{a_{ii}}\lrp{-\sum_{j=1}^{i-1}a_{ij}\iter{x_j}{k+1}-\sum_{j=i}^n a_{ij}\iter{x_j}{k}+b_i}\]

记忆这些迭代格式只需理解清楚$L,U,D$的含义即可,然后看每一行($i$)的迭代,$L$是下三角,那么从第$1$列算到$i-1$列,其他类似。

\begin{theorem}
SOR迭代收敛的必要条件是$\omega\in(0,2)$
\end{theorem}
\begin{theorem}
设系数矩阵$A$对称正定\footnote{用顺序主子式全为正判定},则$0<\omega<2$时SOR迭代收敛
\end{theorem}

三种迭代法的迭代矩阵(注意到$U+A=D-L$)
\[\begin{aligned}
    B_J&=D^{-1}(L+U)=I-D^{-1}A\\
    B_G&=(D-L)^{-1}U=I-(D-L)^{-1}A\\
    B_{SOR}&=(D-\omega L)^{-1}[(1-\omega)D+\omega U]
\end{aligned}\]