% !TEX root = main.tex

\section{线性方程组直接解法}
\subsection{高斯消元法}
即逐行做\textbf{消元},变为上三角矩阵后\textbf{回代}

但由于浮点误差,高斯消元容易产生``大数吃小数''的现象,而达到错误的解(不稳定),如下式
\[\begin{cases}
    \eps x_1+x_2=1\\
    x_1+x_2=2
\end{cases}\]

为避免求解过程中不稳定的现象,可采用下列两种方法
\begin{itemize}
    \item 列主元高斯消去法:每次选择一列中最大的元素所在行,然后与当前行交换
    \item 全主元消去法:每次选择矩阵中最大元素所在行,与当前行交换
\end{itemize}

\subsection{三角分解}
\subsubsection{LU分解与LDU分解}
\begin{itemize}
\item 杜利脱尔(Doolittle)分解:由高斯消元法,$A=L_1^{-1}\cdots L_{n-1}^{-1}U=LU$,其中$L$为\textbf{单位}下三角矩阵,$U$为上三角
\[L=\bmat{1 & & & \\-m_{21} & 1 & & \\ \vdots & \ddots & \ddots & \\-m_{n1} & \cdots & -m_{n,n-1} & 1}\]
其中,$m_{ij}=-a_{ij}/a_{jj}$
\item 克洛脱(Crout)分解:$A=LU$,其中$L$为下三角,$U$为\textbf{单位}上三角
\item LDU分解:$A=LD\tilde{U}$,其中$L$为\textbf{单位}下三角,$D=\opdiag(u_{11},u_{22},\ldots,u_{nn})$,$\tilde{U}=D^{-1}U$为\textbf{单位}上三角
\end{itemize}

杜利脱尔分解要求对$A$高斯消去时,所有主元元素$a_{ii}^{(i)}$不为零
\begin{theorem}
    主元元素$a_{ii}^{(i)}$不为零的充要条件为$n$阶矩阵$A$的所有顺序主子式均不为$0$
\end{theorem}
\begin{theorem}
    若$A$的所有顺序主子式均不为$0$,则矩阵$A$的LU分解都具有唯一性
\end{theorem}

\subsubsection{乔列斯基分解}
当$A$为\textbf{对称正定}矩阵时,所有顺序主子式都大于$0$,因而存在唯一LU分解。
用LDU分解,$D$为非奇异的对角矩阵,由$A$的对称性有
\[A=LDU=U^\T D U^\T\]
又由分解的唯一性,有$L=D^\T$,得$A=LDL^\T$

设$D=\opdiag(d_1,\ldots,d_n),d_i\ne 0$,则$D$的对角元素均为正数。
记$D^{1/2}=\opdiag(\sqrt{d_1},\sqrt{d_2},\ldots,\sqrt{d_n})$,则
\[A=LDL^\T=LD^{1/2}D^{1/2}L^\T=(LD^{1/2})(LD^{1/2})^\T=GG^\T\]
式中$G$是对角元素均大于$0$的下三角矩阵,称该分解为正定矩阵$A$的乔列斯基(Choleskey)分解。
\[\begin{cases}
    \disp \sum_{j=1}^i g_{ij}^2=a_{ii}\\
    \disp \sum_{j=1}^i g_{ij}g_{kj}=a_{ki}
\end{cases}
\implies
\begin{cases}
    \disp g_{ii}=\sqrt{a_{ii}-\sum_{s=1}^{i-1}g_{is}^2}\\
    \disp g_{ki}=\lrp{a_{ki}-\sum_{s=1}^{i-1}g_{is}g_{ks}}\Big/g_{ii} & k=i+1,i+2,\ldots,n
\end{cases}\]
注意上式只不过是$A=G G^\T$的显式表达。

若正定矩阵已经有了乔列斯基分解$A=GG^\T$,则原线性方程组可转化为
\[\begin{cases}
    G\vy=\vb\\
    G^\T\vx=\vy
\end{cases}\]
从而得到计算公式
\[\begin{cases}
    \disp y_i=\lrp{b_i-\sum_{j=1}^{i-1}g_{ij}y_j}\Big/g_{ii},\qquad i=1,2,\ldots,n\\
    \disp x_i=\lrp{y_i-\sum_{j=i+1}^n g_{ji}x_j}\Big/g_{ii},\qquad i=n,n-1,\ldots,1
\end{cases}\]
该求解方法称之为\textbf{平方根法}。

\subsection{追赶法}
设$n$阶方程组$A\vx=\vd$,其中$A$为三对角方程
\[A=\bmat{b_1 & c_1 & & & \\a_2 & b_2 & c_2 & & \\ & \ddots & \ddots & \ddots & \\ & & a_{n-1} & b_{n-1} & c_{n-1}\\ & & & a_n & b_n},\;\vd=\bmat{d_1\\d_2\\\vdots\\d_{n-1}\\d_n}\]
对矩阵$A$做克洛脱分解,有
\[L=\bmat{l_1 & & & &\\v_2 & l_2 & & & \\ & \ddots & \ddots & & \\& & v_{n-1} & l_{n-1} &\\ & & & v_n & l_n},\;U=\bmat{1 & u_1 & & &\\ & 1 & u_2 & & \\ & & \ddots & \ddots & \\ & & & 1 & u_{n-1}\\ & & & & 1}\]
进而
\[A=LU=\bmat{l_1 & l_1u_1 & & & \\v_2 & v_2u_1+l_2 & l_2u_2 & & \\ & \ddots & \ddots & \ddots & \\ & & v_{n-1} & v_{n-1}u_{n-2}+l_{n-1} & l_{n-1}u_{n-1}\\ & & & v_n & v_nu_{n-1}+l_n}\]
做两次回代即可
\[\begin{cases}
L\vy=\vd\\
U\vx=\vy
\end{cases}\]

得到追赶法
\begin{enumerate}
    \item (追)对$i=1,2,\ldots,n-1$做
\[\begin{cases}
    a_i = v_i\\
    b_i = v_iu_{i-1}+l_i\\
    c_i = l_iu_i\\
    d_i = y_{i-1}a_i + y_il_i
\end{cases}
\implies
\begin{cases}
    l_i=b_i-a_iu_{i-1}\\
    y_i=(d_i-y_{i-1}a_i)/l_i\\
    u_i=c_i/l_i
\end{cases}\]
这里,置$u_0=y_0=a_1=0$
    \item (追)$l_n=b_n-a_nu_{n-1},y_n=(d_n-y_{n-1}a_n)/l_n$
    \item (赶)$x_n=y_n$
    \item (赶)对$i=n-1,\ldots,2,1$做
    \[x_i=y_i-u_ix_{i+1}\]
\end{enumerate}

为方便记忆,将$LU$两个矩阵整合
\[L=\bmat{l_1 & \boxed{u_1} & & &\\\boxed{v_2} & \boxed{l_2} & u_2 & & \\ & \ddots & \ddots & \ddots & \\& & v_{n-1} & l_{n-1} & u_{n-1}\\ & & & v_n & l_n}\]

\subsection{分块三角分解}
\[ L=\bmat{ I& \vzero\\ E& I}, U=\bmat{ F& G\\ \vzero& H}\]
且满足$ A= L U$,经计算有
\[ L=\bmat{ I &  \vzero\\ C  A^{-1}&  I}, U=\bmat{ A& B\\ \vzero& D- C A^{-1} B}\]
其中$ S= D- C A^{-1} B$为$ A$的舒尔(schur)补。