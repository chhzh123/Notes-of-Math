% !TEX root = main.tex

% http://mathworld.wolfram.com/DividedDifference.html
% http://mathfaculty.fullerton.edu/mathews/n2003/Web/NewtonPolyMod/NewtonPolyMod.html
\section{多项式插值}
关于函数表达式,有以下三个问题:
\begin{enumerate}
    \item 给定一系列$\{x_i,y_i\}$,给出函数过这些点,进而可以计算其他位置的值(插值方法)
    \item 计算函数$f$非常麻烦,找到$f$一个简单求值的合适逼近(连续函数逼近)
    \item 若$\{x_i,y_i\}$存在误差,找到公式近似表示这些数据(离散函数逼近)
\end{enumerate}

\begin{theorem}[唯一性]
$n$次插值多项式存在且唯一\\
注意:过$n+1$个点的插值多项式并不唯一,因过$n+1$个节点的插值多项式不一定是$n$次(如后面提到的埃尔米特多项式)
\end{theorem}
\begin{analysis}
设$n$次多项式$p(x)=a_0+a_1x+\cdots+a_nx^n$为函数$f(x)$在$[a,b]$上$n+1$个互异节点$x_i(i=0,1,\ldots,n)$上的插值多项式,则求$p(x)$的系数$a_i$相当于解线性方程组
\[\begin{cases}
a_0+a_1x_0+\cdots+a_nx_0^n=y_0\\
a_0+a_1x_1+\cdots+a_nx_1^n=y_1\\
\vdots\\
a_0+a_1x_n+\cdots+a_nx_n^n=y_n
\end{cases}\]
其系数行列式为范德蒙特(Vandermonde)行列式
\[V=\vmat{1 & x_0 & x_0^2 & \cdots & x_0^n\\1 & x_1 & x_1^2 & \cdots & x_1^n\\1 & x_2 & x_2^2 & \cdots & x_2^n\\\vdots & \vdots & \vdots & \ddots & \vdots\\1 & x_n & x_n^2 &\cdots & x_n^n}=\prod_{i=1}^n\prod_{j=0}^{i-1}(x_i-x_j)=\prod_{1\leq i<j\leq n}(x_i-x_j)\]
,因为$x_i$互不相同,故$V\ne 0$。
进而由克莱姆法则,$a_i$解存在且唯一,从而$p(x)$被唯一确定
\end{analysis}

由于$p(x)$为线性空间$P_n(x)$\footnote{次数小于等于$n$的代数多项式的集合}的一个点,因而其基底/插值基函数$\phi(x)$不唯一,有多种表示方法(不同的线性组合)
\[p(x)=a_0\phi_0(x)+a_1\phi_1(x)+\cdots+a_n\phi_n(x)\]

\subsection{拉格朗日插值}
\begin{definition}[拉格朗日插值基函数]
节点$x_k$上的插值基函数$l_k(x)$满足
\[l_k(x_k)=\begin{cases}1&i=k\\0&i\ne k\end{cases}\]
\end{definition}
\par 由定义知$l_k(x)$的零点为$x_i,i\ne k$,待定系数$A_k$
\[l_k(x)=A_k(x-x_0)(x-x_1)\cdots(x-x_{k-1})(x-x_{k+1})\cdots(x-x_n)\]
可求得$A_k$,进而得到插值基函数
\[l_k(x)=\prod_{\substack{i=0\\i\ne k}}^{n}\frac{x-x_i}{x_k-x_i}\]
因此拉格朗日插值函数为
\[L_n(x)=\sum_{k=0}^nf(x_k)\prod_{\substack{i=0\\i\ne k}}^{n}\frac{x-x_i}{x_k-x_i}\]
其中$L_1(x)$称为线性插值,$L_2(x)$称为抛物线插值

\begin{theorem}
    Lagrange插值基函数之和为$1$
\end{theorem}
\begin{analysis}
    取$f(x)=1$,对$f(x)$进行插值,由Lagrange插值函数的唯一性有
    \[\sum_{i=0}^nl_i(x)\equiv 1\]
\end{analysis}

\begin{theorem}[Lagrange函数的误差估计]
\[R_n(x)=\frac{f^{(n+1)}(\xi)}{(n+1)!}\prod_{i=0}^n(x-x_i),\;\xi=\xi(x)\in(a,b)\]
\end{theorem}
\begin{analysis}
    设$R_n(x)=f(x)-L_n(x)$,则在节点$x_i$上有
    \[R_n(x_i)=f(x_i)-L_n(x_i)=0,\qquad i=0,1,\ldots,n\]
    即$R_n(x)$在$[a,b]$上至少有$n+1$个零点,故有
    \[R_n(x)=K(x)(x-x_0)(x-x_1)\cdots(x-x_n)=K(x)\Pi(x)\]
    其中$K(x)$为待定函数,$\Pi(x)$为零点乘积

    引入辅助函数$\phi(t)=f(t)-L_n(t)-K(x)\Pi(t)$(误差-误差),视$x\ne x_i\in(a,b)$为固定点,则$\phi(t)$在$n+2$个点$x,x_0,x_1,\ldots,x_n$上取值为$0$。
    由罗尔定理,在$\phi(t)$任两个零点之间必存在一点$\eta$,使得$\phi'(\eta)=0$,于是$\phi'(x)$在$(a,b)$上至少有$n+1$个零点。
    对$\phi'(t),\phi''(t),\ldots,\phi^{(n)}(t)$反复使用罗尔定理,最后得到在$(a,b)$上至少存在一个点$\xi$,使得$\phi^{(n+1)}(\xi)=0$,而
    \[\phi^{(n+1)}(\xi)=f^{(n+1)}(\xi)-L_n^{(n+1)}(\xi)-K(x)\Pi^{(n+1)}(\xi)\]
    式中,$L_n(x)$是次数不超过$n$的多项式,$Pi(x)$是$n+1$次多项式,故$L_n^{(n+1)}(\xi)=0,\Pi^{(n+1)}(\xi)=(n+1)!$,于是求得
    \[K(x)=\frac{f^{n+1}(\xi)}{(n+1)!}\]
    代入$R_n(x)$即得证
\end{analysis}

若求区间内误差的最大值,有
\[\max_{a\leq x\leq b}|R_n(x)|\leq\frac{M_{n+1}}{(n+1)!}\max_{a\leq x\leq b}|\Pi(x)|\]
其中$\Pi(x)$常用AM-GM不等式

\subsection{牛顿插值}
\begin{definition}[牛顿插值基函数]
\[\begin{cases}
\varphi_0(x)=1\\
\varphi_j(x)=\displaystyle\prod_{i=0}^{j-1}(x-x_i),\;j=1,2,\ldots,n
\end{cases}\]
\end{definition}
\par 进而牛顿插值函数为
\[N_n(x)=a_0+a_1(x-x_0)+a_2(x-x_0)(x-x_1)+\cdots=a_0+\sum_{i=1}^na_i\prod_{j=0}^{i-1}(x-x_j)\]
实际上就是将$x_i$之后的项全部为$0$消掉了。
\par 由$f(x_k)=N_n(x_k)$,对比系数可得
\[\begin{aligned}
a_0 &=f(x_0)\\
a_1 &=\frac{f(x_1)-f(x_0)}{x_1-x_0}\\
a_2 &=\frac{\dfrac{f(x_2)-f(x_0)}{x_2-x_0}-\dfrac{f(x_1)-f(x_0)}{x_1-x_0}}{x_2-x_1}
\end{aligned}\]

\begin{definition}[差商]
一阶差商$f[x_i,x_k]=\dfrac{f(x_k)-f(x_i)}{x_k-x_i}$,二阶差商$f[x_i,x_j,x_k]=\dfrac{f[x_i,x_k]-f[x_i,x_j]}{x_k-x_j}$(一阶差商的差商),高阶差商为
\[f[x_0,x_1,\ldots,x_{k-1},x_k]=\frac{f[x_0,x_1,\ldots,x_{k-2},x_k]-f[x_0,x_1,\ldots,x_{k-2},x_{k-1}]}{x_k-x_{k-1}}\]
\end{definition}
类比导数的定义就可以很快记忆

差商有下列性质
\begin{enumerate}
    \item 差商可以表示为函数值的线性组合
    \[f[x_0,x_1,\ldots,x_{k-1},x_k]=\sum_{j=0}^k\frac{f(x_j)}{(x_j-x_0)\cdots(x_j-x_{j-1})(x_j-x_{j+1})\cdots(x_j-x_k)}\]
    \begin{analysis}
        用数学归纳法,$k=1$易证,假设$k=k$成立,证明$k=k+1$成立
        \[\begin{aligned}
            &\quad f[x_0,x_1,\ldots,x_{k-1},x_k,x_{k+1}]\\
            &=\frac{f[x_0,x_1,\ldots,x_{k-1},x_{k+1}]-f[x_0,x_1,\ldots,x_{k-1},x_k]}{x_{k+1}-x_{k}}\\
            &=\lrp{\sum_{\substack{j=0\\j\ne k}}^{k+1}\frac{f(x_j)}{(j;0,\ldots,j-1,j+1,\ldots,k-1,k+1)}
            -\sum_{j=0}^{k}\frac{f(x_j)}{(j;0,\ldots,j-1,j+1,\ldots,k)}}\Big/(x_{k+1}-x_k)\\
            &=\lrp{\sum_{j=0}^{k-1}\frac{f(x_j)(x_j-x_k)-f(x_j)(x_j-x_{k+1})}{(j;0,\ldots,j-1,j+1,\ldots,k)}
            +\frac{f(x_{k+1})}{(k+1;0,\ldots,k-1)}
            -\frac{f(x_k)}{(k;0,\ldots,k-1)}}\Big/(x_{k+1}-x_k)\\
            &=\sum_{j=0}^{k-1}\frac{f(x_j)}{(j;0,\ldots,j-1,j+1,\ldots,k)}
            +\frac{f(x_{k+1})}{(k+1;0,\ldots,k-1,k)}
            +\frac{f(x_k)}{(k;0,\ldots,k-1,k+1)}\\
            &=\sum_{j=0}^{k+1}\frac{f(x_j)}{(j;0,\ldots,j-1,j+1,\ldots,k,k+1)}
        \end{aligned}\]
    \end{analysis}
    \item 差商关于所含节点是对称的,即任意调换节点$x_i$和$x_j$的次序,值不变
    \[f[x_0,\ldots,x_i,\ldots,x_j,\ldots,x_k]=f[x_0,\ldots,x_j,\ldots,x_i,\ldots,x_k]\]
    \item 设$f(x)$在含有$x_0,x_1,\ldots,x_n$区间$(a,b)$上具有$n$阶导数,则在这区间上至少有一点$\xi$,使
    \[f[x_0,x_1,\ldots,x_n]=\frac{f^{(n)}(\xi)}{n!},\; \xi\in(a,b)\]
\end{enumerate}

由差商定义可得以下差商表(注:下面$f_i:=f(x_i)$)
\begin{center}
\begin{tabular}{ccccc}\hline
$x_k$ & $y_k$ & 一阶 & 二阶 & 三阶\\\hline
$x_0$ & \underline{$f_0$} & & &\\
$x_1$ & $f_1$ & \underline{$f[x_0,x_1]$} & &\\
$x_2$ & $f_2$ & $f[x_1,x_2]$ & \underline{$f[x_0,x_1,x_2]$} &\\
$x_3$ & $f_3$ & $f[x_2,x_3]$ & $f[x_1,x_2,x_3]$ & \underline{$f[x_0,x_1,x_2,x_3]$} \\\hline
\end{tabular}
\end{center}
并且有
\[\begin{aligned}
a_0&=f(x_0)\\
a_1&=f[x_0,x_1]\\
a_2&=f[x_0,x_1,x_2]=\frac{f[x_1,x_2]-f[x_1,x_0]}{x_2-x_0}\\
\vdots\\
a_n&=f[x_0,x_1,\ldots,x_n]
\end{aligned}\]
从而牛顿插值多项式为
\[N_n(x)=f(x_0)+f[x_0,x_1](x-x_0)+\cdots+f[x_0,x_1,\ldots,x_n](x-x_0)\cdots(x-x_{n-1})\]
且有递推式
\[N_{n+1}(t)=N_n(t)+f[x_0,x_1,\ldots,x_n,x](t-x_0)(t-x_1)\cdots(t-x_n)\]
可以看出,当增加一个节点时,牛顿插值只需增加一项,而拉格朗日插值需要全部重新算。

\begin{theorem}[Newton多项式的误差估计]
    \[R_n(x)=f[x_0,x_1,\ldots,x_n,x]\Pi(x)\]
\end{theorem}
\begin{analysis}
    $N_{n+1}(t)$是$\{(x_0,f(x_0)),\ldots,(x_n,f(x_n)),(x,f(x))\}$上的$n+1$次牛顿插值多项式,则在点$x$处一定满足插值条件
    \[f(x)=N_{n+1}(x)=N_n(x)+f[x_0,x_1,\ldots,x_n,x](x-x_0)\cdots(x-x_n)\]
    从而余项为
    \[f(x)-N_n(x)=f[x_0,x_1,\ldots,x_n,x]\Pi(x)\]
    又由插值多项式的唯一性,有$N_n(x)=L_n(x)$,故可得差商与导数之间的关系
    \[f[x_0,x_1,\ldots,x_k]=\frac{f^{(k)}(\xi)}{k!},\;\xi\in(a,b)\]
\end{analysis}

\begin{definition}[差分]
设已知函数$f(x)$在等距节点$x_i=x_0+ih(i=0,1,2,\ldots,n)$上的函数值为$f(x_i)=f_i$,其中$h>0$为步长,则称$\Delta f_i=f(x_i+h)-f(x_i)=f_{i+1}-f_i$为$f(x)$在$x_i$处步长为$h$的一阶向前差分。
称$\Delta^2 f_i=\Delta(\Delta f_i)=\Delta f_{i+1}-\Delta f_i$为函数$f(x)$在$x_i$处的二阶向前差分。
一般地$\Delta^n f_i=\Delta(\Delta^{n-1} f_i)$为函数$f(x)$在$x_i$处的$n$阶向前差分。
并规定,$\Delta^0 f_i=f_i$为零阶差分。
\end{definition}
\par 差分与差商的关系为
\[f[x_0,x_1,\ldots,x_k]=\frac{\Delta^k f_0}{k!h^k},\,k=1,2,\ldots,n\]

则牛顿插值公式的向前差分形式为
\[N_n(x)=N_n(x_0+th)=f_0+\frac{t}{1!}\Delta f_0+\frac{t(t-1)}{2!}\Delta^2 f_0+\cdots+\frac{t(t-1)\cdots(t-n+1)}{n!}\Delta^n f_0,\,t\in[0,1]\]
类似于泰勒展开。

\subsection{埃尔米特插值}
不仅要求插值点的函数值$H(x_i)$与原函数的值$f(x_i)$相同,还要求它们有公切线,即$H'(x_i)=f'(x_i)$。
若所有插值点都要满足上述条件,则为次数不超过$2n+1$的埃尔米特(Hermite)插值多项式
\[H_{2n+1}(x)=\sum_{k=0}^m(1-2l'_k(x_k)(x-x_k))l_k^2(x)f_k+\sum_{k=0}^n(x-x_k)l_k^2(x)f_k'\]
式中,$l_k(x)$为节点$\{x_i\}_{i=0}^n$上的拉格朗日基函数。
余项为
\[R_{2n+1}(x)=f(x)-H_{2n+1}(x)=\frac{f^{(2n+2)}(\xi)}{(2n+2)!}(\Pi(x))^2\]

\begin{example}
    若函数$f(x)$在$[a,b]$上有四阶连续导数,且已知函数$f(x)$在$[a,b]$的互异节点$x_0,x_1,x_2$上的函数值以及节点$x_0$的一阶导数值如下表所示
    \begin{center}
        \begin{tabular}{cccc}\hline
            $x$ & 0 & 1 & 2\\\hline
            $f(x)$ & 0 & 1 & 1\\
            $f'(x)$ & -3\\\hline
        \end{tabular}
    \end{center}
    求一个三次埃尔米特插值多项式$H(x)$,使其满足
    \[\begin{cases}
        H(x_i)=f(x_i) & i=0,1,2\\
        H'(x_0)=f'(x_0)
    \end{cases}\]
    并估计余项
\end{example}
\begin{analysis}
    先构造过$x_0,x_1,x_2$的拉格朗日插值函数
    \[L(x)=-\frac{1}{2}x^2+\frac{3}{2}x\]
    令$H(x)=L(x)+A(x-0)(x-1)(x-2)$,求导(用乘法法则)
    \[H'(0)=L'(0)+A(0-1)(0-2)\]
    解得$A=-9/4$,故
    \[H(x)=-\frac{9}{4}x^3+\frac{25}{4}x^2-3x\]
    记$\Pi(x)=(x-x_0)^2(x-x_1)(x-x_2)$,将余项表为
    \[R(x)=f(x)-H(x)=K(x)\Pi(x)\]
    引入构造函数$\phi(t)=f(t)-H(t)-K(x)\Pi(t)$,类似推理可得
    \[R(x)=\frac{f^{(4)}(\xi)}{4!}\Pi(x),\;\xi(x)\in(a,b)\]
\end{analysis}
\begin{example}
    若函数$f(x)$在$[a,b]$上有五阶连续导数,且已知函数$f(x)$在$[a,b]$的互异节点$x_0,x_1,x_2$上的函数值以及节点$x_0,x_1$的一阶导数值如下表所示
    \begin{center}
        \begin{tabular}{cccc}\hline
            $x$ & 0 & 1 & 2\\\hline
            $f(x)$ & 0 & 1 & 1\\
            $f'(x)$ & -3 & 9\\\hline
        \end{tabular}
    \end{center}
    求一个三次埃尔米特插值多项式$H(x)$,使其满足
    \[\begin{cases}
        H(x_i)=f(x_i) & i=0,1,2\\
        H'(x_0)=f'(x_0) & i=0,1
    \end{cases}\]
    并估计余项
\end{example}
\begin{analysis}
    构造差商表(有导数的点构造两份)
    \begin{center}
        \begin{tabular}{cccccc}\hline
            $x_k$ & $f(x_k)$ & 一阶 & 二阶 & 三阶 & 四阶\\\hline
            0 & 0\\
            0 & 0 & -3\\
            1 & 1 & 1 & 4\\
            1 & 1 & 9 & 8 & 4\\
            2 & 1 & 0 & -9 & 17/2 & -25/4\\\hline
        \end{tabular}
    \end{center}
    得到插值多项式
    \[H(x)=-\frac{25}{4}x^4+\frac{33}{2}x^3-\frac{25}{4}x^2-3x\]
\end{analysis}

\subsection{三次样条插值}
龙格(Runge)现象告诉我们并非插值多项式的次数越高,精度就越高,其源于等距分布的插值节点。
更好的方式是选择切比雪夫节点,如在$[-1,1]$上选择
\[x_i=\cos\lrs{\lrp{\frac{i}{n}\pi}},\qquad 0\leq i\leq n\]
任意区间$[a,b]$上$n+1$个切比雪夫节点可以通过线性变换得到
\[x_i=\frac{1}{2}(a+b)+\frac{1}{2}(b-a)\cos\lrs{\lrp{\frac{i}{n}\pi}},\qquad 0\leq i\leq n\]

另一种解决收敛性的方法则是分段低次插值,同时使插值函数具有一定的光滑性,故用样条(spline)插值的方式,每段用三次函数逼近,最终使得样条函数处处有二阶连续导数。
\begin{definition}[三次样条函数]
    若函数$s(x)\in C^2[a,b]$(区间$[a,b]$上具有二阶连续导数的函数全体),且在每个小区间$[x_j,x_{j+1}]$是三次多项式,其中$a=x_0<x_1<\cdots<x_n=b$是给定节点,则称$s(x)$是节点$x_0,x_1,\ldots,x_n$上的三次样条函数。
    若在节点$x_j$上给定函数值$y_j=f(x_j),j=0,1,\ldots,n$,并使
    \[s(x_j)=y_j,\qquad j=0,1,\ldots,n\]
    则称$s(x)$为$f(x)$在$[a,b]$上的三次样条插值函数
\end{definition}

三次样条条件
\begin{itemize}
    \item 插值条件:$s_k(x_j)=y_j,\quad j=k-1,k,\quad k=1,2,\ldots,n$,共$2n$个条件
    \item 端点导数条件:$\lim_{x\to x_k^-}s^{(p)}(x)=\lim_{x\to x^+_k}s^{(p)}(x),\quad p=1,2,\quad k=1,\ldots,n-1$,共$2n-2$个条件
    \item 边界条件:
    \begin{itemize}
        \item 第一类边界条件:$s'(x_0)=f_0',s'(x_n)=f_n'$,其中$f_0',f_n'$为给定值
        \item 第二类边界条件:$s''(x_0)=f_0'',s''(x_n)=f_n''$,其中$f_0'',f_n''$为给定值,若为$0$,则为\textbf{自然}边界条件
        \item 周期性条件:$\lim_{x\to x_0^+}s^{(p)}(x)=\lim_{x\to x_n^-}s^{(p)}(x),p=1,2$
    \end{itemize}
\end{itemize}

\[\begin{cases}
    \disp \lambda_j=\frac{h_{j+1}}{h_j+h_{j+1}}\\
    \disp \mu_j=1-\lambda_j=\frac{h_j}{h_j+h_{j+1}}\\
    \disp d_j=\frac{6}{h_j+h_{j+1}}\lrp{\frac{y_{j+1}-y_j}{h_{j+1}}-\frac{y_j-y_{j-1}}{h_j}}=6f[x_{j-1},x_j,x_{j+1}]
\end{cases}\]

三转角方法,用一阶导数求解;三弯矩方法,用二阶导数求解。
\[\bmat{2 & 1 & & & \\\mu_1 & 2 & \lambda_1 & & \\ & \mu_2 & 2 & \lambda_2 & \\ & & \ddots & \ddots & \ddots \\ & & & \mu_{n-1} & 2 & \lambda_{n-1}\\&&&&1&2}\bmat{M_0\\M_1\\M_2\\\vdots\\M_{n-1}\\M_n}=\bmat{d_0\\d_1\\d_2\\\vdots\\d_{n-1}\\d_n}\]
该矩阵严格对角占优,故有唯一解,可以用追赶法求解